%!TEX root = ../main.tex
% Resume/resume_fr.tex


Les enjeux énergétiques et environnementaux liés au réchauffement climatique amènent à
généraliser la sobriété énergétique des bâtiments neufs ainsi que la production locale
d’énergie à l’horizon $2020$. Ce travail de thèse se concentre sur le secteur de la maison
individuelle qui représente près de la moitié des logements neufs construits en France
pour un volume d’environ \si{200000}~unités par an.

\paragraph{} % (fold)
Le contexte de la maison individuelle à énergie positive \SI{100}{\percent} solaire consiste à
rechercher les compromis entre le niveau de performance du bâti qui détermine les besoins
en énergie et la capacité des équipements à valoriser l’énergie solaire pour d’une part
subvenir aux besoins en chaleur pour assurer le chauffage et la production d’eau chaude
sanitaire, et d’autre part produire l’électricité nécessaire à l’éclairage et aux autres
usages spécifiques (matériels électroménager, vidéo, etc.).

Après un examen des différents concepts de bâtiments à énergie positive, une analyse a été
menée pour identifier les solutions techniques de systèmes solaires combinés capables de
fournir le double service de production d’eau chaude et de chauffage. Un modèle détaillé a
été développé dans l’environnement Dymola et vérifié par inter-comparaison de modèles à
l’échelle des composants. Un algorithme de contrôle original a été mis au point pour
maximiser la performance globale du système. Une première étude paramétrique a montré que
ce système est capable dans certaines conditions de couvrir près de \SI{80}{\percent} des besoins en
chaleur de la maison étudiée. Néanmoins, son dimensionnement demeure complexe et la
recherche de compromis entre la sobriété de la maison et le dimensionnement des systèmes
solaires thermiques et photovoltaïques doit s’appuyer sur un algorithme d’optimisation
multi-objectifs adapté.

Un chapitre est donc consacré à l’élaboration d’un algorithme d’optimisation multi-
objectifs qui s’appuie sur la méthode des colonies d’abeilles virtuelles. Cette approche
s’est avérée particulièrement pertinente vis à vis du problème (paramètres discrets,
continus et qualitatifs) à caractère multi-objectifs (maximiser la valorisation du solaire
thermique pour le chauffage d’une part et pour la production d’eau chaude d’autre part,
minimiser la consommation d’énergie conventionnelle) et sous contrainte car seules les
solutions à bilan d’énergie positif sur l’année seront retenues. L’algorithme
d’optimisation développé ici a été confronté à une série de problèmes classiques et a
démontré sa capacité à construire l’ensemble des solutions avec un nombre relativement
faible d’évaluations du modèle.

\paragraph{} % (fold)
Le dernier chapitre présente deux applications de conception de maisons à énergie
positive. La première se situe en région bordelaise alors que la seconde est située à
proximité de Strasbourg. Ces deux conditions climatiques permettent de mettre en évidence
la capacité de l’algorithme d’optimisation à proposer un éventail de solutions optimales
présentant des compromis différents en termes de performance du bâti et de dimensionnement
des équipements solaires. Enfin, un outil d’aide à la décision permet d’explorer les
fronts optimaux pour dégager les solutions à retenir.

\vfill
\noindent\textbf{Mots clés~: } \keywordsFR
