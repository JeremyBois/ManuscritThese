%!TEX root = ../main.tex
% Resume/resume_eng.tex



With energy-related and environmental climate change challenges, energy sobriety and local
energy production are yet to become a mainstream practice for new buildings construction by
$2020$. This works focuses on single-family houses which in France represent half of new
buildings constructions with \SI{200000}{new units} new units each year.

\paragraph{} % (fold)
Near zero energy single-family houses with \SI{100}{\percent} solar energy consists on compromising
between performance of building envelope which defines energy needs and the ability
for equipments to value free solar energy. Hence solar energy must be able to cover
space heating and domestic hot water demands but also provide enough energy for
lightning and other specific uses such as domestic appliances.

After a literature review of near zero energy house concepts, an analysis was undertaken
to provide a clear view of solar combi-systems technical solutions with the ability to
provide enough energy for both needs: space heating and domestic hot water. Using Dymola
environment a detailed model was developed and its consistency was checked by
inter-comparison at component scale. An innovative control algorithm has been worked out to
maximize the solar system’s global performance. A first parametric study has shown that the system
was able to cover close to \SI{80}{\percent} of house heat requirement. However sizing of a solar
combi-system is a complex task and requires to find compromises between building sobriety, solar
thermal energy efficiency, and photovoltaics solar energy sizing. Because of the problem’s
complexity, a decision aid tool with an appropriate multi-criteria optimization algorithm
is required.

To that end a chapter is dedicated to the development of a multi-criteria optimization
algorithm based on artificial bee colony behavior. This approach has proved to be quite effective
to solve the problem and to handle continuous, discrete and qualitative decision variables.
Chosen solution was constrained to have a positive energy balance and must maximize solar
space heating and domestic fraction in a view to reduce total energy consumption.
A validation process has also been set up and the developed optimization algorithm
has proved its ability to solve standard problems with a fairly short number of evaluations.

\paragraph{} % (fold)
Adopted methodology was illustrated by two applications of the design phase of
a near zero energy detached house. First one is located at Bordeaux an second one
in Strasbourg. Selected climate conditions emphasize the ability of the proposed
approach to identify a wide range of optimal solutions showing differences within
the building's performance as well as the solar system sizing. Lastly a decision aid tool
allows to explore optimal front in a convenient way to shape adapted solutions.


\vfill
\noindent\textbf{Keywords~:} Solar combi-system, Nearly ZEB, Multi-criteria optimisation, Décision analysis, Modeling, Modelica
