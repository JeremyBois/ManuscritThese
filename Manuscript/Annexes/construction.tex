%!TEX root = ../main.tex
% Annexes/meta-modele.tex

\vspace{-1cm}
\noindent Les étapes de construction d’un bâtiment public sont définies à travers la loi
\href{https://www.legifrance.gouv.fr/affichTexte.do?cidTexte=JORFTEXT000000693683}{$85$-$704$ du $12$ juillet $1985$}~:
\begin{blockdescription}{Programmation}
    \item[Programmation] Aussi appelée phase de Faisabilité, elle fait principalement intervenir la
          maîtrise d’ouvrage qui établit le cahier des charges avec les tiers. Les besoins
          du client, l’objet de la construction, la faisabilité, et les appels d’offres
          sont réalisés.
    \item[Conception] Elle fait intervenir la \abr{MOA} et la maîtrise d’œuvre (\abr{MOE}).
          Des études technique et architecturales permettent de définir plus finement le cahier des charges.
          \begin{blockdescription}{ESQ}
              \item [\abr{ESQ}] L’ESQuisse ou AVant Projet vérifie la faisabilité technique du projet au regard
                    des différentes contraintes (économique, technique, réglementaire). Durant cette
                    phase sont définis~: les paramètres d’usages
                    (charges internes, scénarios), le cadre réglementaire (thermique, acoustique, mécanique),
                    les objectifs d’exemplarité à travers l’obtention de labels ($E+C-$,
                    \abr{BBCA}\dots) et les caractéristiques de l’implémentation
                    (données climatiques, étude du sol\dots).
              \item [\abr{APS}] L’Avant Projet Sommaire consiste à définir la géométrie générale
                    du projet à travers la définition des principaux volumes et les coûts prévisionnels.
              \item [\abr{APD}] L’Avant Projet Définitif permet d’affiner la géométrie du bâtiment
                    à partir des résultats des différentes études techniques. Les caractéristiques de
                    l’enveloppe et les systèmes techniques sont
                    arrêtés, les coûts prévisionnels découpés par lot.
              \item [\abr{PRO}] La phase de PROjet permet de finaliser l’étude et d’estimer
                     à la fois le coût des travaux et d’exploitation. Le planning des travaux
                     doit aussi être réalisé et le détail de chaque lot réalisé. Ces éléments
                     sont décrits à travers un dossier des Spécificités Techniques
                     détaillées (\abr{STD}) et un Dossier de Consultation des Entreprises (\abr{DCE}).
          \end{blockdescription}
    \item[Réalisation] Aussi appelée phase de construction, elle fait intervenir l’ensemble des acteurs
          du bâtiment, \abr{MOA}, \abr{MOE}, entreprises, tiers\dots\ Le cahier des charges, défini
          en phase de programmation puis affiné en phase de conception, est suivi. Cette
          phase comprend alors le suivi technique et financier du chantier mais aussi sa
          réception et mise en service.
\end{blockdescription}
