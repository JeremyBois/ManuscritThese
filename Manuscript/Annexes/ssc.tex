%!TEX root = ../main.tex
% Annexes/ssc.tex

\begin{landscape}

\begin{figure}
    \centering
    \includegraphics[height=0.83\textheight]{Ressources/Images/Modelisation/Captures/SSC_capture.pdf}
    \caption[Représentation de l’ensemble du \abr{SSC} et du bâtiment sous \textit{Dymola}]
            {Représentation de l’ensemble du \abr{SSC} (partie hydraulique et aéraulique)
             et du bâtiment sous \textit{Dymola}.}
             \label{fig:dymola_ssc_batiment}
\end{figure}

\begin{figure}
    \centering
    \includegraphics[height=0.95\textheight]{Ressources/Images/Modelisation/Captures/controleur_capture.pdf}
    \caption[Représentation de l’algorithme de contrôle sous \textit{Dymola}]
            {Représentation de l’algorithme de contrôle sous \textit{Dymola}. Chaque \abr{FSM}
             pointe vers un sous-modèle décrivant de manière détaillée le fonctionnement de l’algorithme.}
             \label{fig:dymola_ssc_controle}
\end{figure}

\end{landscape}


Description des modèles de la bibliothèque \fnref{https://github.com/lbl-srg/modelica-buildings}{\textit{Buildings}}
retenus pour chaque composant du système modélisé dans ce manuscrit~: \label{modele_solaires_list}
\begin{blockdescription}{Combles et Vide sanitaire}
    \item [Ballon] \textit{Fluid.Storage.StratifiedEnhancedInternalHex} qui modélise
                   un ballon avec stratification.
    \item [Échangeur interne] \textit{Fluid.Storage.BaseClasses.IndirectTankHeatExchanger}.
    \item [Vase d’expansion] \textit{Buildings.Fluid.Storage.ExpansionVessel}.
    \item [Pompes] \textit{Fluid.Movers.SpeedControlled\_Nrpm} permettant
                   de contrôler le indirectement le débit par modulation de la vitesse
                   de rotation normalisée de la pompe.
    \item [Ventilateurs] \textit{Fluid.Movers.FlowControlled\_m\_flow}.
                          permettant de contrôler le débit directement.
    \item [Parseur météo] \textit{BoundaryConditions.WeatherData.ReaderTMY3}.
                          Nécessite de transformer le fichier météo dans un format propre au modèle.
                          Un outil en \textit{Java} est disponible pour convertir un fichier de
                          type \textit{.epw} dans le format supporté \textit{.mos}.
    \item [Capteur solaire] \textit{Fluid.SolarCollectors.EN12975} implémentant
          l’équation quasi-dynamique décrite en \ref{par:approche_quasi_dynamique}.
    \item [Rayonnement diffus] \textit{BoundaryConditions.SolarIrradiation.DiffusePerez} permettant
          de calculé le rayonnement diffus ($G_{dif,\,inc}$) suivant le modèle développé par \textcite{Perez1990271}.
    \item [rayonnement direct] \textit{BoundaryConditions.SolarIrradiation.DirectTiltedSurface}
          permettant de calculer le rayonnement direct ($G_{dir,\,inc}$) selon \eqref{eq:rayonnement_surf_incline}.
    \item [Échangeur eau/air] \textit{Fluid.HeatExchangers.ConstantEffectiveness} qui
          modélise un échangeur avec une efficacité constante fixée à \SI{80}{\percent}.
    \item [Zone chauffée] \textit{Rooms.MixedAir} qui assume une zone où l’air est
          complètement mixé. Le modèle calcule séparément les échanges radiatifs
          et conductifs et permet de modéliser des fenêtres et des scénarios
          internes \parencite{Wetter2011}.
    \item [Combles et Vide sanitaire] Une version simplifiée de \textit{Rooms.MixedAir}
          ne permettant pas l’ajout de fenêtres ni d’apports internes.
    \item [Fenêtres] \textit{Rooms.Constructions.ConstructionWithWindow} séparant
          les apports radiatifs des apports convectifs nécessitant donc une description
          détaillée des composants de la fenêtre.
    \item [Vanne $\bm{3}$ voies] \textit{Fluid.Actuators.Valves.ThreeWayEqualPercentageLinear}
    \item [Vanne $\bm{2}$ voies] \textit{Fluid.Actuators.Valves.TwoWayEqualPercentage}
    \item [Canalisation extérieures] \textit{Fluid.FixedResistances.Pipe} permettant
          de discrétiser le volume de fluide afin de mieux évaluer les déperditions.
    \item [Puisage] À partir d’une loi de mélange, le volume d’eau du ballon nécessaire
          pour obtenir une température de puisage de \SI{40}{\celsius} est puisé
          par une pompe hors du ballon sanitaire.
\end{blockdescription}
\clearpage


\begin{figure}
    \centering
    \includegraphics[width=\textwidth, page=4]{Ressources/Images/Modelisation/circulateur.pdf}
    \caption[Fiche technique du circulateur \textit{Wilo-YonosECO 25/1-5 BMS}]
            {Fiche technique du circulateur \textit{Wilo-YonosECO 25/1-5 BMS}.}
    \label{fig:caracs_pompes}
\end{figure}

\begin{figure}
    \centering
    \includegraphics[width=\textwidth]{Ressources/Images/Modelisation/Capteurs/SkyPro/din_skyPro.pdf}
    \caption[Certification \textit{Solar Keymark} pour le capteur \textit{Kloben Sky Pro 12 CPC58}]
            {Certification \textit{Solar Keymark} pour le capteur \textit{Kloben Sky Pro 12 CPC58}.}
    \label{fig:caracs_skypro}
\end{figure}


\begin{figure}
    \centering
    \includegraphics[width=\textwidth]{Ressources/Images/Modelisation/Capteurs/IDMK/din_idmk.pdf}
    \caption[Certification \textit{Solar Keymark} pour le capteur \textit{Sonnenkraft IDMK-25AL}]
            {Certification \textit{Solar Keymark} pour le capteur \textit{Sonnenkraft IDMK-25AL}.}
    \label{fig:caracs_idmk}
\end{figure}


\begin{figure}
    \centering
    \includegraphics[width=\textwidth]{Ressources/Images/Modelisation/Capteurs/Eco/din_eco.pdf}
    \caption[Certification \textit{Solar Keymark} pour le capteur \textit{Dima Energy + Eco}]
            {Certification \textit{Solar Keymark} pour le capteur \textit{Dima Energy + Eco}.}
    \label{fig:caracs_eco}
\end{figure}


\begin{figure}
    \centering
    \includegraphics[width=\textwidth]{Ressources/Images/Modelisation/Capteurs/CPCStar/din_CPCStar.pdf}
    \caption[Certification \textit{Solar Keymark} pour le capteur \textit{Ritter CPC $14$ Star}]
            {Certification \textit{Solar Keymark} pour le capteur \textit{Ritter CPC $14$ Star}.}
    \label{fig:caracs_star}
\end{figure}
