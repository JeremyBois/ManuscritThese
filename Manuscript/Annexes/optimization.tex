%!TEX root = ../main.tex
% Annexes/optimization.tex

\begin{landscape}

\begin{figure}
% volume_collector_factorplot
    \centering
    \includegraphics[height=0.82\textheight]{Ressources/Images/EtudeDeCas/bordeaux_tanks_variations.pdf}
    \caption[Réparitition des solutions en fonction des sous-groupes de taille de
             ballon pour Bordeaux.]
            {Réparitition des solutions sur Bordeaux en fonction du volume des ballons
             pour $4$ indicateurs~: la $Conso_{app}$, le $F_{sav,\, ext}$,
             le $F_{sav}^{CH}$, et le $F_{sav}^{ECS}$.}
    \label{fig:tanks_variations_bordeaux}
\end{figure}

\begin{figure}
% volume_collector_factorplot
    \centering
    \includegraphics[height=0.82\textheight]{Ressources/Images/EtudeDeCas/strasbourg_tanks_variations_cum.pdf}
    \caption[Réparitition des solutions en fonction des sous-groupes de taille de
             ballon pour Strasbourg.]
             {Réparitition des solutions sur Strasbourg en fonction du volume des ballons
             pour $4$ indicateurs~: la $Conso_{app}$, le $F_{sav,\, ext}$,
             le $F_{sav}^{CH}$, et le $F_{sav}^{ECS}$. Représente les solutions cumulées
             de deux optimisations dont une n’autorisant qu’un faible volume pour les
             ballons.}
    \label{fig:tanks_variations_strasbourg}
\end{figure}

\begin{figure}
% volume_collector_factorplot
    \centering
    \includegraphics[height=0.82\textheight]{Ressources/Images/EtudeDeCas/strasbourg_smalltanks_variations.pdf}
    \caption[Réparitition des solutions en fonction de faibles volumes de ballons.]
            {Réparitition des solutions sur Strasbourg en fonction du volume des ballons ($100$, $150$,
             ou \SI{200}{\litre}) pour $4$ indicateurs~: la $Conso_{app}$, le
             $F_{sav,\, ext}$, le $F_{sav}^{CH}$, et le $F_{sav}^{ECS}$.}
    \label{fig:tanks_small_variations_strasbourg}
\end{figure}

\end{landscape}
