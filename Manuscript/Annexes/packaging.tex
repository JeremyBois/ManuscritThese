%!TEX root = ../main.tex
% Annexes/packaging.tex

\iunsure{Ajouter code source}

\blockquote{Combien de rectangles sans chevauchements ni rotations peut contenir le triangle $ABC$ ?}

\begin{figure}
    \centering
    \includegraphics{Ressources/Images/Demonstration/repartition_capteur.pdf}
    \caption[Représentation schématique du problème de \enquote{packaging}]
            {Représentation schématique du problème de \enquote{packaging} pour
             un triangle aïgue (gauche) et un triangle obtue (droite).
             \label{fig:rects_in_triangle}}
\end{figure}

Considérons un rectangle de dimensions $l \times h$ et un triangle $ABC$ (\figref{fig:rects_in_triangle}).
D’après la \textit{loi des cosinus (Théorême de Al Kashi)} il est possible d’écrire:
\begin{equation}
        \alpha = \arccos \left[\frac{AC^{2} + BC^{2} - AB^{2}}{2 \times AC \times BC}\right]
    \label{eq:loi_cosinus}
\end{equation}
\noindent On peut alors définir la hauteur du triangle $ABC$ en utilisant la trigonométrie et
\eqref{eq:loi_cosinus}:
\begin{equation}
        H = \sin (\arccos[\alpha]) \times BC%
          =\sin \left(\arccos \left[\frac{AC^{2} + BC^{2} - AB^{2}}{2 \times AC \times BC}\right]\,\right) \times BC
    \label{fig:height_triangle}
\end{equation}

\noindent En utilisant le \textit{théorême de Thalès} dans $ABC$ et A’BC on peut ainsi définir la longueur DE
en fonction de $H$.
\begin{equation}
    DE = \frac{BE \times AC}{BC} = \frac{(H - h) \times AC}{H}
\end{equation}

\noindent Le nombre total de rectangle sans chevauchement peut alors être définie suivant \eqref{eq:nb_rects_in_triangle}:
\begin{equation}
    N_{rect} = \sum_{n = 1}^{\floor*{\frac{H}{h}}} \, \floor*{\frac{(H - n \times h) \times AC}{l \times H}}
    \label{eq:nb_rects_in_triangle}
\end{equation}
