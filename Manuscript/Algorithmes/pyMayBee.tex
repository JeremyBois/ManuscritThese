%!TEX root = ../main.tex
% Algorithme/pyMayBee.tex

% Comments
\newcommand\commfont[1]{\footnotesize\ttfamily\textcolor{SolarizedBrCyan}{#1}}
\SetCommentSty{commfont}
\SetKwComment{AComment}{\#~}{}

% Functions
\SetKwFunction{ALevyFlight}{LevyFlight}
\SetKwFunction{ALevy}{Levy}

% Keywords
\SetKwData{ASources}{Sources}
\SetKwData{ASource}{Source}
\SetKwData{AButineuse}{Butineuse}
\SetKwData{AOuvriere}{Ouvrière}
\SetKwData{ANbrSources}{N}
\SetKwData{AOnlookers}{Ouvrières}
\SetKwData{AScouts}{Éclaireuses}
\SetKwData{AEmployed}{Butineuses}
\SetKwData{ABee}{Abeille}
\SetKwData{ABees}{Abeilles}
\SetKwData{AHive}{Essaim}
\SetKwData{AVariables}{Variables}
\SetKwData{AVariable}{Variable}
\SetKwData{ANbrVariables}{J}
\SetKwData{AArchive}{Archive}
\SetKwData{APopulation}{Population}
\SetKwData{ATirageA}{TirageA}
\SetKwData{ATirageB}{TirageB}
\SetKwData{ARatio}{Ratio}
\SetKwData{AMR}{MR}
\SetKwData{ATrial}{échec}
\SetKwData{AMaxTrial}{$Max_{échec}$}


% Phase d’initialisation
\begin{algorithm}[!htb]
  \SetAlgoVlined
  \DontPrintSemicolon
  \emph{Initialisation des \ASources sur l’ensemble de l’espace de décision}\;
  \For{$i \leftarrow 0$ \KwTo \ANbrSources}
  {
    \emph{Initialisation des variables de décision pour chaque source}\;
    \For{$j \leftarrow 0$ \KwTo \ANbrVariables}
    {
      \circled[b]{a} \emph{Initialisation de la position de la $\ASource_{i}$ de manière aléatoire}\;
      \Indp
      $x_{ij} = x_{j}^{min} + RandUniform(0, 1) \times (x_{j}^{max} - x_{j}^{min})$\;
      \Indm
      \BlankLine
      \circled[b]{b}  \emph{Génération de l’$\ABee_{i}$ dont la position est calculée suivant Définition~\ref{def:oblm}}\;
      \Indp
      $ \check{x_{ij}} = x_{j}^{min} + x_{j}^{max} - x_{ij}$\;
      \Indm
      \BlankLine
      \circled[b]{c} \emph{Évaluation de la $\ASource_{i}$ et de l’$\ABee_{i}$}\;
      \BlankLine
      avec $RandUniform$ un tirage aléatoire suivant une loi uniforme, et $x_{j}^{min}$, $x_{j}^{max}$
      respectivement le minimum et le maximum de la variable $j$\;
    }
    \If{$\ASource_{i}$ respecte toutes les contraintes}
    {
      $\AArchive \pluseq \ASource_{i}$ \AComment{On ajoute la source initial à l’archive}
    }
    \If{$\ABee_{i}$ respecte toutes les contraintes}
    {
      $\AArchive \pluseq \ABee_{i}$ \AComment{On ajoute la source opposée à l’archive}
    }
  }
  Mise à jour des \ASources d’après Algorithm~\ref{alg:maj_phase}\;
  \caption{Initialisation des sources par \abr{OBLM} (Définition~\ref{def:oblm}).}
  \label{alg:init_phase}
\end{algorithm}


% Phase des éclaireuses
\begin{algorithm}[!htb]
  \SetAlgoVlined
  \DontPrintSemicolon
  \AComment{Exploration par les \AScouts}
  \For{$i \leftarrow 0$ \KwTo \ANbrSources}
  {
    \If{$\ATrial_{i} > \AMaxTrial$ }
    {
      Génération de deux nouvelles positions suivant Algorithm~\ref{alg:init_phase}\;
    }
  }
  Mise à jour de la position des \ASources d’après Algorithm~\ref{alg:maj_phase}\;
  \caption{Phase des éclaireuses.}
  \label{alg:scout_phase}
\end{algorithm}


% Phase des ouvrières
\begin{algorithm}[!htb]
  \SetAlgoVlined
  \DontPrintSemicolon
  \AComment{Exploitation des sources par les \AOnlookers}
  \AComment{Plusieurs \AOnlookers peuvent modifier la même source}
  \For{$\AOuvriere \in \AOnlookers$}
    {
      Sélection aléatoire d’une \ASource $i$ selon la probabilité
      définie par l’équation \eqref{eq:attribution_prob_to_source}\;
      Sélection aléatoire d’une source $a$ ($a \neq i$) dans l’\AArchive\;
      \circled[b]{a} \emph{Génération d’une nouvelle position $\vec{x_{i}}'$ à partir de la
                         position $\vec{x_{i}}$ pour l’\AOuvriere }\;
      \For{$j \leftarrow 0$ \KwTo \ANbrVariables}
      {
      \begin{algomathdisplay}
        x_{ij}' =%
          \begin{cases}
            x_{ij}  + RandUniform(-1, 1)   \times \ (x_{ij} - x_{aj}) &\ \ATirageB < \AMR \\
            x_{ij}                                                    &\ sinon
          \end{cases}
      \end{algomathdisplay}
      }
      \BlankLine
      Avec \AMR est la probabilité de réaliser une modification (fixée à \num{0.2}) et
      \ATirageB un nombre aléatoire entre \num{0} et \num{1}\;
      \BlankLine
      \circled[b]{b} \emph{Évaluation des objectifs pour la nouvelle position $\vec{x_{i}}'$}\;
      \BlankLine
      \If{\AOuvriere respecte toutes les contraintes}
      {
        $\AArchive \pluseq \AOuvriere$ \AComment{Ajout de la solution trouvée par l’\AOuvriere à l’archive}
      }
    }

  Mise à jour de la position des \ASources qui ont été modifiées d’après Algorithme~\ref{alg:maj_phase}\;
  \caption{Phase des ouvrières.}
  \label{alg:onlooker_phase}
\end{algorithm}


% Maj des sources
\begin{algorithm}[!htb]
  \SetAlgoVlined
  \DontPrintSemicolon
  Récupérer le maximum et minimum pour chaque objectif et contraintes dans l’\AHive\;
  \For{$\ABee \in \ABees$}
  {
    Calcul du vecteur objectif pour l’\ABee ($\vec{F}'_{i}$) et pour sa \ASource $i$ ($\vec{F}_{i}$),
    respectivement aux positions $\vec{x_{i}}'$ et $\vec{x_{i}}$
    selon \eqref{eq:calc_modif_obj}\;
    \For{$m \leftarrow 1$ \KwTo M}
    {
      \begin{algomathdisplay}
      \begin{aligned}
      F_{im}' &= d_{m}'(\vec{x_{i}}) + p_{m}'(\vec{x_{i}}) \\
      F_{im}  &= d_{m}(\vec{x_{i}}) + p_{m}(\vec{x_{i}})   \\
      \end{aligned}
      \end{algomathdisplay}
    }
    \If{$\vec{x_{i}}' \succ \vec{x_{i}}$}
    {

      $\ASource_{i} \leftarrow \ABee$ \AComment{Mise à jour de la \ASource à partir de l’\ABee}

      $\ATrial_{i} \leftarrow 0$ \AComment{On réinitialise le nombre d’essais pour la \ASource $i$}
    }
    \Else
    {
      $\ATrial_{i} \pluseq 1$ \AComment{On incrémente le nombre d’essais pour la \ASource $i$}
    }
  }
  \caption{Mise à jour des \textbf{Sources} par les \textbf{Abeilles}}
  \label{alg:maj_phase}
\end{algorithm}


% Phase des butineuses
\begin{algorithm}[!htb]
  \SetAlgoVlined
  \DontPrintSemicolon
  \AComment{Exploration des sources par les \AEmployed}
  \For{$i \leftarrow 0$ \KwTo \ANbrSources}
  {
    Sélection aléatoire de $2$ sources~: $a$ ($a \neq i$) dans l’\AArchive, et $b$ ($b \neq i$) dans l’\AHive\;
    \BlankLine
    \circled[b]{a} \emph{Génération d’une nouvelle position $\vec{x_{i}}'$ à partir de la %
                       position $\vec{x_{i}}$ pour la \AButineuse $i$}\;
    \If{$\ATirageA < \ARatio $ }
      {
      \For{$j \leftarrow 0$ \KwTo \ANbrVariables}
      {
      \begin{algomathdisplay}
        x_{ij}' =%
          \begin{cases}
            \begin{aligned}
              x_{ij}  &+ \num{0.01} \times ~\ALevy  &\times (x_{ij} - x_{bj})  \\
                      &+ \num{0.01} \times |\ALevy|   &\times (x_{aj} - x_{ij})  \\
            \end{aligned} &\ \ATirageB < \AMR \\
            x_{ij}        &\ sinon
          \end{cases}
      \end{algomathdisplay}
      \ALevy est nombre aléatoire dans une distribution de Lévy (Définition~\ref{def:vol_levy})\;
      }
      }
    \Else
      {
      \For{$j \leftarrow 0$ \KwTo \ANbrVariables}
      {
      \begin{algomathdisplay}
        x_{ij}' =%
          \begin{cases}
            \begin{aligned}
              x_{ij}  &+ RandUniform(-1, 1)   &\times \ (x_{ij} - x_{bj})  \\
                      &+ RandUniform(0, 1)    &\times \ (x_{aj} - x_{ij})  \\
            \end{aligned} &\ \ATirageB < \AMR \\
            x_{ij}        &\ sinon
          \end{cases}
      \end{algomathdisplay}
      }
      }
      Où \ARatio est la probabilité de réaliser un vol de Lévy (fixée à \num{0.5}), et \AMR la probabilité
      de réaliser une modification (fixée à \num{0.3}). \ATirageA et \ATirageB étant des nombres aléatoires
      (entre \num{0} et \num{1}) tirés dans une distribution uniforme\;
      \BlankLine
    \circled[b]{b} \emph{Évaluation des objectifs pour la nouvelle position $\vec{x_{i}}'$}\;
    \If{$\AButineuse_{i}$ respecte toutes les contraintes}
    {
      \AComment{On ajoute la nouvelle source à l’archive}
      $\AArchive \pluseq \AButineuse_{i}$\;
    }
  }
%   \AComment{On ne conserve qu’une seule position par source}
  Mise à jour de la position des \ASources d’après Algorithme~\ref{alg:maj_phase}\;
  \caption{Phase des butineuses.}
  \label{alg:employed_phase}
\end{algorithm}
