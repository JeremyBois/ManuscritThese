%!TEX TS-program = PdfLaTeX
%!TEX encoding = UTF-8 Unicode


\documentclass[xcolor=x11names, compress, 11pt]{beamer}
% Options for doc
    % handout to hide navigation bar
    % compress to make pdf output smaller (not sure)


% \mode<presentation> {
%     \usepackage{pgfpages}
%     \setbeameroption{show notes on second screen=right}
% }

\mode<presentation>


%%%%%%%%%%%%%%%%%%%%%%%%%%%%%%%%%%% Packages %%%%%%%%%%%%%%%%%%%%%%%%%%%%%%%%%%%
\usepackage[OT1, T1]{fontenc}                              % .pdf Encoding spec
\usepackage[utf8]{inputenc}                                % .tex Encoding spec
\usepackage[greek, french]{babel}                          % language used

\usepackage{palatino}                                      % Layout and font model

\usepackage{graphicx}
\usepackage{hyperref}
\usepackage{changepage}                                    % Enable changing text margins
\usepackage{color, colortbl,transparent}                   % Add colors and opacity
\usepackage{siunitx}                                       % Unit package
\usepackage{booktabs, multirow}                            % Easy pro tables and multiple row
\usepackage{marvosym}                                      % Smileys


\usepackage{tikz,tcolorbox}


\hypersetup{
      pdfauthor   = {Jérémy Bois},%
      pdftitle    = {Soutenance Thèse - 20171009},%
      pdfsubject  = {Soutenance Thèse - 20171009},%
}




%%%%%%%%%%%%%%%%%%%%%%%%%%%%%%% Define new colors %%%%%%%%%%%%%%%%%%%%%%%%%%%%%%
\definecolor{Amaranth}{rgb}{0.9, 0.17, 0.31}
\definecolor{Asparagus}{rgb}{0.53, 0.66, 0.42}
\definecolor{Amber}{rgb}{1.0, 0.49, 0.0}
\definecolor{emphase}{rgb}{0.9, 0.17, 0.31}
\definecolor{BrownGreen}{HTML}{586E75}

\definecolor{SolarizedViolet}{HTML}{6c71c4}
\definecolor{SolarizedMagenta}{HTML}{d33682}
\definecolor{SolarizedBlue}{HTML}{268bd2}
\definecolor{SolarizedCyan}{HTML}{2aa198}
\definecolor{SolarizedGreen}{HTML}{859900}
\definecolor{SolarizedRed}{HTML}{dc322f}
\definecolor{BeeYellow}{HTML}{ffcb00}
\definecolor{SolarizedOrange}{HTML}{cb4b16}
\definecolor{SolarizedYellow}{HTML}{b58900}

\definecolor{SolarizedWhite}{HTML}{fdf6e3}
\definecolor{SolarizedBrWhite}{HTML}{eee8d5}
\definecolor{SolarizedBrCyan}{HTML}{93a1a1}
\definecolor{SolarizedBrBlue}{HTML}{839496}
\definecolor{SolarizedBrYellow}{HTML}{657b83}
\definecolor{SolarizedBrGreen}{HTML}{586e75}
\definecolor{SolarizedBlack}{HTML}{073642}
\definecolor{SolarizedBrBlack}{HTML}{002b36}
\definecolor{ClassicBlack}{HTML}{000000}



%%%%%%%%%%%%%%%%%%%%%%%%%%%%%%%%% Define border %%%%%%%%%%%%%%%%%%%%%%%%%%%%%%%%
\setlength\fboxsep{0pt}
\setlength\fboxrule{0.5pt}



%%%%%%%%%%%%%%%%%%%%%%%%%%%%%%% New columns type %%%%%%%%%%%%%%%%%%%%%%%%%%%%%%%
\newcolumntype{B}{>{\columncolor{Amaranth}}c}
\newcolumntype{G}{>{\columncolor{Asparagus}}c}
\newcolumntype{R}{>{\columncolor{Amber}}c}




\title{Outil d’Aide à la Décision pour la conception de maisons solaires à énergie positive}
\author{Jérémy Bois}
\date{2017/10/09}


%%%%%%%%%%%%%%%%%%%%%%%%%%%%%%%%% Beamer Layout %%%%%%%%%%%%%%%%%%%%%%%%%%%%%%%%
% Head with sections
\useoutertheme[subsection=false, shadow]{miniframes}
\useinnertheme{default}
\usefonttheme{professionalfonts}
\usefonttheme{serif}

\setbeamerfont{title like}{shape=\scshape}
\setbeamerfont{frametitle}{shape=\scshape}

\setbeamercolor{bgcolor}{fg=black,bg=blue}
\setbeamercolor{lower separation line head}{bg=SolarizedBlue}
\setbeamercolor{frametitle}{fg=SolarizedBlue,bg=SolarizedBrWhite}
\setbeamercolor{title}{fg=SolarizedBlue,bg=SolarizedBrWhite}
\setbeamercolor{normal text}{fg=ClassicBlack,bg=white}
\setbeamercolor{alerted text}{fg=SolarizedRed}
\setbeamercolor{example text}{fg=ClassicBlack}
\setbeamercolor{structure}{fg=ClassicBlack}
\setbeamercolor{palette tertiary}{fg=ClassicBlack,bg=black!10}
\setbeamercolor{palette quaternary}{fg=ClassicBlack,bg=black!10}
\setbeamercolor{footlinecolor}{fg=DeepSkyBlue4,bg=black!10}


% Remove navigation symbols
\setbeamertemplate{navigation symbols}{}

% Complete footline with small caps for subsection name
\setbeamertemplate{footline}{%
  \begin{beamercolorbox}[sep=1em,wd=\paperwidth,leftskip=0.5cm,rightskip=0.5cm]{footlinecolor}
    \tiny
    \hfill \textsc{\small\textbf{\insertsubsection}}\hfill \insertframenumber{} / \inserttotalframenumber
    \hfill
  \end{beamercolorbox}%
}

% Custom first page
\setbeamertemplate{title page}{%
  \vbox{}
  \begingroup
    \centering
    \begin{beamercolorbox}[sep=8pt,center,shadow=true,rounded=true]{title}
      \usebeamerfont{title}\inserttitle\par%
      \ifx\insertsubtitle\@empty%
      \else%
        \vskip0.25em%
        {\usebeamerfont{subtitle}\usebeamercolor[fg]{subtitle}\insertsubtitle\par}%
      \fi%
    \end{beamercolorbox}%
    \vskip1em\par
    \begin{beamercolorbox}[sep=8pt,center]{author}
      \usebeamerfont{author}\insertauthor
    \end{beamercolorbox}
    % \begin{beamercolorbox}[sep=8pt,center]{institute}
    %   \usebeamerfont{institute}\insertinstitute
    % \end{beamercolorbox}
    \begin{beamercolorbox}[sep=8pt,center]{date}
      \usebeamerfont{date}\insertdate
    \end{beamercolorbox}
  \endgroup
  \vfill
}




\AtBeginSection[]
{
 \begin{frame}[plain,noframenumbering]
  \vfill
  \centering
  \begin{beamercolorbox}[sep=8pt,center,shadow=true,rounded=true]{title}
    \usebeamerfont{title}\insertsectionhead\par%
  \end{beamercolorbox}
  \vfill
  \addtocounter{page}{-1}
  \end{frame}
}



\newcommand{\addsubtitle}[1]{%
\begin{beamercolorbox}[sep=2pt,center,shadow=true,rounded=true]{footlinecolor}
    #1\par%
\end{beamercolorbox}%
}


% % % % % % % % % % % % % % % % % % % % % % % % % % % % % % % % % % % % % % % %
\begin{document}
% % % % % % % % % % % % % % % % % % % % % % % % % % % % % % % % % % % % % % % %

\pagenumbering{Alph}
\begin{frame}[plain,noframenumbering]
\titlepage
\addtocounter{page}{-1}

\note{
Notes for the first frame here
}
\end{frame}
\pagenumbering{arabic}


% % Group can be used to declare local change
% \begingroup
% % \usebackgroundtemplate{\includegraphics[width=\paperwidth,height=\paperheight]{Includes/fond_title.jpg}}
% \begin{frame}[plain]
%     \title{\color{SolarizedBlue}\textsc{\textbf{}}}
%     %\subtitle{SUBTITLE}
%     \author{
%         \begin{columns}
%             \begin{column}{0.3\textwidth}
%                 \begin{flushleft}
%                     \includegraphics[scale=0.07]{Ressources/Images/Soutenance/logos/I2M_logo.jpg}
%                 \end{flushleft}
%             \end{column}%
%             \begin{column}{0.3\textwidth}
%                 \begin{center}
%                     \small
%                     Jérémy Bois\\Laurent Mora\\Etienne Wurtz\\
%                 \end{center}
%             \end{column}%
%             \begin{column}{0.3\textwidth}
%                 \begin{flushright}
%                     \includegraphics[scale=0.08]{Ressources/Images/Soutenance/logos/bordeaux_logo.jpg}
%                 \end{flushright}
%             \end{column}
%         \end{columns}
%         \vspace{1cm}
%         \begin{columns}
%         \begin{column}{0.3\textwidth}
%             \begin{flushleft}
%                 \includegraphics[scale=0.05]{Ressources/Images/Soutenance/logos/cea_logo.png}
%             \end{flushleft}
%         \end{column}
%         \begin{column}{0.3\textwidth}
%             \begin{center}
%                 \includegraphics[scale=0.22]{Ressources/Images/Soutenance/logos/comepos_logo.png}
%             \end{center}
%         \end{column}
%         \begin{column}{0.3\textwidth}
%             \begin{flushright}
%                 \includegraphics[scale=0.13]{Ressources/Images/Soutenance/logos/aquitaine_logo.png}
%             \end{flushright}
%         \end{column}
%         \end{columns}
%     }
%     \date{
%         \normalsize
%         2017/10/09
%     }
%     \titlepage
% \end{frame}
% \endgroup










% % ==============================================================================
% % ==============================================================================
\section{\scshape Contexte / Objectifs}
% -------------------------------------
% ------------------------------------------------------------------------------
\subsection{Contexte global}
\begin{frame}[c]
    \vfill
    \centering
    \includegraphics[height=0.5\textheight]{Ressources/Images/Environnement/evolution_effet_serre2.png}
    \vfill
    \includegraphics[width=\textwidth, clip=true, trim=0mm 0mm 0mm 150mm]{Ressources/Images/Environnement/elevation_temperature.jpg}
    \vfill
\end{frame}




% ------------------------------------------------------------------------------
\subsection{Contexte du bâtiment en France}
\begin{frame}[c]
    \vfill
    \begin{columns}
        \begin{column}{0.35\textwidth}
            \begin{itemize}
                \item \SI{200000}{chantiers}
            \end{itemize}
            \vskip2em
            \begin{itemize}
                \item[~] Sobriété
                \item[~] +
                \item[~] Efficacité
                \item[~] =
                \item[~] Réduction des coûts
            \end{itemize}
        \end{column}%
        \begin{column}{0.7\textwidth}
            \centering
            \includegraphics[height=0.45\textheight]{Ressources/Images/Environnement/evolution_energie_finale.png}
            \vskip1em
            \includegraphics[height=0.45\textheight]{Ressources/Images/Soutenance/Contexte_objectifs/reglementation.png}
        \end{column}%
    \end{columns}%
    \vfill
\end{frame}


% ------------------------------------------------------------------------------
\subsection{Conception d’un bâtiment}
\begin{frame}[c]
    \vfill
    \centering
    \includegraphics[height=0.4\textheight]{Ressources/Images/Soutenance/Contexte_objectifs/vie_batiment.png}
    \begin{columns}
        \begin{column}{0.5\textwidth}
            \begin{itemize}
                \item Choix de l’enveloppe
                \item Choix des équipements
                \item Recours aux EnR
            \end{itemize}
        \end{column}%
        \begin{column}{0.7\textwidth}
            \centering
            \includegraphics[height=0.5\textheight]{Ressources/Images/Soutenance/Contexte_objectifs/construction_batiment.png}
        \end{column}%
    \end{columns}%

    % \begin{itemize}
    %     \item construction / exploitation / démolition
    %     \item Programmation / Conception / réalisation
    %     \item Décrire les éléments pris en compte dans l’étude~:
    %     \begin{itemize}
    %         \item Bâti
    %         \item chauffage / ECS / charges internes / Auxiliaire / Ventilation
    %         \item Systèmes
    %         \item Recours aux EnR
    %         \item Cadre réglementaire
    %     \end{itemize}
    % \end{itemize}
    \vfill

    \note{RT 2012 impose des EnR ??}
\end{frame}


% ------------------------------------------------------------------------------
\subsection{La maison à énergie positive}
\begin{frame}[c]
    \begin{columns}
        \begin{column}{0.4\textwidth}
                \small
                Bilan~:
                \begin{itemize}
                    \scriptsize
                    \item Énergie finale
                    \item Énergie primaire
                    \item $CO_{2}$
                    \item Gain financier
                \end{itemize}
                \vskip3em
                 Cadre normatif Européen~:
                \begin{itemize}
                    \scriptsize
                    \item Sobriété
                    \item Efficacité
                    \item Énergie renouvelables
                    \item Bilan positif
                \end{itemize}
        \end{column}%
        \begin{column}{0.7\textwidth}
            \centering
            \includegraphics[height=0.55\textheight]{Ressources/Images/Environnement/bilan_nzeb.png}
            \vskip3em
            \includegraphics[height=0.25\textheight]{Ressources/Images/Environnement/hurdle_race.png}
        \end{column}%
    \end{columns}%
    \note{Pour respecter ces contraintes il faut faire preuve de sobriété sur le bâti ...}
\end{frame}



% ------------------------------------------------------------------------------
\subsection{L’énergie solaire}
\begin{frame}[c]
    \vfill
    \centering
    \includegraphics[height=0.6\textheight]{Ressources/Images/Environnement/evolution_enr.png}
    \begin{columns}
        \begin{column}{0.45\textwidth}
            \begin{center}
                \begin{itemize}
                    \item Progression du photovoltaïque
                    \item Progression solaire thermique en baisse
                \end{itemize}
            \end{center}
        \end{column}%
        \begin{column}{0.45\textwidth}
            \begin{center}
                \begin{itemize}
                    \item Impact politique important
                    \item Climat français propice
                \end{itemize}
            \end{center}
        \end{column}%
    \end{columns}%
    \vfill
    % http://www.travaux.com/dossier/construction-ecologique/12274/Baisse-des-tarifs-d-achat-du-photovoltaique-pour-2011.html
    % http://www.lefigaro.fr/impots/2010/09/06/05003-20100906ARTFIG00493-les-aides-au-photovoltaique-reduites-de-moitie.php
    \note{Chute du crédit d'impôt photovoltaïque en 2011 50 à 25\% + baisse du tarif de rachat 58 à 46 expliquant la chute de progression.}
\end{frame}


\begin{frame}[c]
    \vfill
    \centering
    \includegraphics[width=\textwidth]{Ressources/Images/Soutenance/Contexte_objectifs/repartition_installations.png}
    \begin{columns}
        \begin{column}{0.45\textwidth}
            \begin{center}
                \begin{itemize}
                    \item Solaire thermique peu développé en France
                    \item SSC sur le marché Européen
                \end{itemize}
            \end{center}
        \end{column}%
        \begin{column}{0.45\textwidth}
            \begin{center}
                \begin{itemize}
                    \item Majoritairement en maison individuelle
                    \item BEPOS avec SSC $< \SI{10}{\percent}$
                \end{itemize}
            \end{center}
        \end{column}%
    \end{columns}%
    \vfill
\end{frame}



% ------------------------------------------------------------------------------
\subsection{Objectifs / Démarche}
\begin{frame}[c]
    \begin{beamercolorbox}[sep=8pt,center,shadow=true,rounded=true]{frametitle}
      \usebeamerfont{frametitle} \footnotesize \textbf{Comment aider à la prise de décision dans la construction de maisons solaires à énergie positive ?}\par%
    \end{beamercolorbox}%
    \vfill
    \begin{columns}
        \begin{column}{0.2\textwidth}
            \vskip2em
            \addsubtitle{Verrous}
            \vskip7em
            \addsubtitle{Démarche}
        \end{column}%
        \begin{column}{0.9\textwidth}
            \begin{center}
                \begin{itemize}
                    \footnotesize
                    \item[--] Comment tenir compte du caractère multi-physiques et multi-critères ?
                    \item[--] Potentiel d’un système solaire combiné et d’une MEPOS ?
                    \item[--] Quels indicateurs retenir pour l’évaluation du SSC ?
                    \item[--] Quel équilibre entre performance sur l’enveloppe et efficacité des systèmes ?
                \end{itemize}
                \vskip2em
                \begin{itemize}
                    \footnotesize
                    \item[--] Modélisation puis évaluation du potentiel du SSC développé.
                    \item[--] Dimensionnement couplé de l’enveloppe du SSC et de la logique de contrôle.
                    \item[--] Génération d’un grand nombre d’alternatives.
                    \item[--] Aide au choix d’une solution interactivement.
                \end{itemize}
            \end{center}
        \end{column}%
    \end{columns}%
    \vfill
\end{frame}


% ------------------------------------------------------------------------------
\subsection{Sommaire}
\begin{frame}
\tableofcontents[hideallsubsections]
\addtocounter{page}{-1}
\end{frame}


















% % ==============================================================================
% % ==============================================================================
\section{\scshape Cas d’étude}
% ----------------------------



% ------------------------------------------------------------------------------
\subsection{Outil de Modélisation}
\begin{frame}[c]
    \vfill
    \begin{columns}
        \begin{column}{0.45\textwidth}
            \begin{center}
                \begin{itemize}
                    \item Multi-physiques
                    \item Équationnel
                    \item Acausal
                    \item Boîte blanche
                \end{itemize}
            \end{center}
            \includegraphics[width=0.5\textwidth]{Ressources/Images/Logos/modelica_logo.png}
            \includegraphics[width=0.5\textwidth]{Ressources/Images/Logos/dymola_logo.jpg}
        \end{column}%
        \begin{column}{0.45\textwidth}
            \centering
            \begin{center}
                \begin{itemize}
                    \item Large bibliothèque
                    \item Libre et ouverte
                \end{itemize}
            \end{center}
            \vskip2em
            \includegraphics[width=0.5\textwidth]{Ressources/Images/Logos/Berkeley_logo.jpg}
        \end{column}%
    \end{columns}%
    \vfill
    \note{Éprouvé dans d’autres domaines notamment le transport l’aéronautique ou les voitures.\\
          Itération importantes.
          Même logiciel pour le bâtiment et les systèmes}
\end{frame}




% ------------------------------------------------------------------------------
\subsection{Bâtiment étudié}
\begin{frame}[c]
    \vfill
    \centering
    Ajouter image du bâtiment \\
    \includegraphics[height=0.5\textheight]{Ressources/Images/Modelisation/maison.png}
    \vfill
\end{frame}



% ------------------------------------------------------------------------------
\subsection{Système étudié}
\begin{frame}[c]
    \vfill
    \includegraphics[height=\textheight, clip=true, trim=0mm 250mm 310mm 0mm]{Ressources/Images/Modelisation/Principe/air_modes.pdf}
    \vfill
\end{frame}
\begin{frame}[c]
    \vfill
    \includegraphics[height=\textheight, clip=true, trim=310mm 250mm 0mm 0mm]{Ressources/Images/Modelisation/Principe/air_modes.pdf}
    \vfill
\end{frame}
\begin{frame}[c]
    \vfill
    \includegraphics[height=\textheight, clip=true, trim=0mm 0mm 310mm 250mm]{Ressources/Images/Modelisation/Principe/air_modes.pdf}
    \vfill
\end{frame}
\begin{frame}[c]
    \vfill
    \includegraphics[height=\textheight, clip=true, trim=310mm 0mm 0mm 250mm]{Ressources/Images/Modelisation/Principe/air_modes.pdf}
    \vfill
\end{frame}



% ------------------------------------------------------------------------------
\subsection{Hétérogénéité des facteurs pris en compte}
\begin{frame}[c]
    \vfill
    \includegraphics[height=0.8\textheight]{Ressources/Images/Soutenance/CasEtude/facteurs.png}
    \vfill
    \begin{itemize}
        \item Fortement combinatoire
    \end{itemize}

\note{Changer tableau}
\end{frame}


% ------------------------------------------------------------------------------
\subsection{Objectifs et contraintes}
\begin{frame}[c]
    \vfill
    Objectifs~:
    \begin{itemize}
        \item Performance du SSC~: Sur le chauffage et sur l’ECS
        \item Production des capteurs PV
    \end{itemize}
    \vfill
    Contraintes~:
    \begin{itemize}
        \item Partage de la surface disponible en toiture 4 pans
        \item Solutions ayant un bilan positif
    \end{itemize}

\note{Ajouter plus de détail sur objectifs et contraintes voir une autre diapo}
\end{frame}










% % ==============================================================================
% % ==============================================================================
\section{\scshape Méthodologie}
% -------------------------------------



% ------------------------------------------------------------------------------
\subsection{Approches Existantes}
\begin{frame}[c]
    \vfill
     Incrémentale~: Approche actuelle dans le bâtiment par essais/erreurs
     Optimisation~: Approche valorisant temps machine
    \vfill

    \note{Ajouter des smiley pour comparer les approches}
\end{frame}



% ------------------------------------------------------------------------------
\subsection{Optimisation et décision}
\begin{frame}[c]
    \vfill
    \centering
    \includegraphics[width=0.8\textwidth]{Ressources/Images/Optimisation/meta_heuristique.pdf}
    \vfill
\end{frame}

\begin{frame}[c]
    \addsubtitle{La décision précède l’optimisation}
    \vfill
    \centering
    \includegraphics[width=0.8\textwidth]{Ressources/Images/Optimisation/meta_heuristique.pdf}
    \vfill

Ajouter cadre pour montrer de quoi je parle et plusieurs diapo~: décision --> optimisation (aggrégation)
\end{frame}

\begin{frame}[c]
    \addsubtitle{La décision précède l’optimisation}
    \vfill
    \centering
    \includegraphics[width=0.8\textwidth]{Ressources/Images/Optimisation/multi_to_mono.pdf}
    \vfill
\end{frame}

\begin{frame}[c]
    \addsubtitle{L’optimisation précède à la décision}
    \vfill
    \centering
    \includegraphics[width=0.8\textwidth]{Ressources/Images/Optimisation/meta_heuristique.pdf}
    \vfill

Ajouter cadre pour montrer de quoi je parle et plusieurs diapo~: optimisation --> décision (notion de pareto)
\end{frame}

\begin{frame}[c]
    \addsubtitle{L’optimisation précède à la décision}
    \vfill
    \centering
    \includegraphics[width=0.8\textwidth]{Ressources/Images/Optimisation/dominance.pdf}
    \vfill
\end{frame}


\begin{frame}[c]
    \addsubtitle{L’optimisation précède à la décision}
    \vfill
    \centering
    \begin{columns}
        \begin{column}{0.5\textwidth}
            \begin{center}
                \begin{itemize}
                    \item Variables non-continues
                    \item Combinatoire importante
                    \item Connaissances limitées
                \end{itemize}
            \end{center}
        \end{column}%
        \begin{column}{0.5\textwidth}
            \includegraphics[width=0.8\textwidth]{Ressources/Images/Optimisation/meta_heuristique.pdf}
        \end{column}%
    \end{columns}%
    \vfill

    \note{Refaire graphe pour gagner de la place}
\end{frame}




% ------------------------------------------------------------------------------
\subsection{Optimisation par Colonie d’Abeilles Virtuelles}
\begin{frame}[c]
    \vfill
    \centering
    \includegraphics[height=0.8\textheight]{Ressources/Images/Optimisation/ABC/comportement_abeilles.pdf}
    \vfill
\end{frame}


\begin{frame}[c]
    \vfill
    \centering
    \includegraphics[height=0.8\textheight]{Ressources/Images/Optimisation/ABC/algorithme_complet.pdf}
    \vfill
\end{frame}



% % ==============================================================================
% % ==============================================================================
\section{\scshape Application}
% --------------------------------------------


% ------------------------------------------------------------------------------
\subsection{Oeuvre}
\begin{frame}[c]
    \vfill
    \vfill
\end{frame}


















% % ==============================================================================
% % ==============================================================================
\section{\scshape Conclusions / Perspectives}
% ------------------------------------------


% ------------------------------------------------------------------------------
\subsection{Conclusions}
\begin{frame}[c]
    \vfill
    \vfill
\end{frame}


% ------------------------------------------------------------------------------
\subsection{Perspectives}
\begin{frame}[c]
    \vfill
    \vfill
\end{frame}






























% \subsection{Problématique}
% \begin{frame}[c]
%     \frametitle{Problématique}
%     \vfill
%     \large{\color{SolarizedGreen} Constat:}
%     \small
%     \begin{itemize}

%         % Présenter le sujet !!

%         \item Réchauffement climatique
%             \begin{itemize}
%                 \footnotesize
%     \small
%                 \item Nécessaire de réduire la consommation
%                 \item Vers des bâtiments à faible consommation
%             \end{itemize}
%         \item Climat favorable au solaire: 1112~\si{kWh/m^{2}/an}
%         \item Solaire thermique faiblement exploité (2\% des maisons individuelles)
%         \item De nombreux travaux existants
%         \item Etudes existantes trop peu détaillées
%             \begin{itemize}
%                 \footnotesize
%                 \item Données d’entrées mensuelles / modèles simplifiés
%                 \item Aucunes informations sur le comportement du système
%                 \item Algorithme de contrôle simplifié
%             \end{itemize}
%         \item Système et bâtiment dimensionnés indépendaments
%     \end{itemize}
%     \vfill
% \end{frame}

% \begin{frame}[c]
%     \vfill
%     \large{\color{SolarizedGreen} Verrous identifiés:}
%     \small
%     \begin{itemize}
%         \item Peu de développement utilisant des modèles détaillés de SSC$^{*}$
%         % \item Comment définir une maison faiblement consommatrice ?
%         \item Un dimensionnement couplé système/bâtiment, quel détail pour les composants ?
%         \item Pas ? Peu ? de travaux sur les intéractions système solaire~/~bâtiment
%         \item Le solaire thermique pour des maisons faiblement consommatrices ?
%         % \item Est-il possible de réaliser une maison 100\% solaire ?
%         \item Le couplage système~/~bâtiment permet-il ...
%             \begin{itemize}
%                 \footnotesize
%                 \item ... de mieux comprendre le fonctionnement du système ?
%                 \item ... d’améliorer la performance (pilotage + dimensionnement)?
%                 \item ... de remettre en question la sur-isolation ?
%             \end{itemize}
%         \item Forte cardinalité
%     \end{itemize}
%     \vfill

%     \tiny
%     $^{*}$SSC = Système solaire combiné
%     \vfill
% \end{frame}

% \subsection{Approche par optimisation}
% \begin{frame}[c]
%     % Choix de l’approche MEPOS (inspiration venant des NZEB et de PassivHaus)
%     % Plus de souplesse sur la performance de l’enveloppe
%     \frametitle{Approche par optimisation}
%     \vfill
%     \large{\color{SolarizedGreen}  1. ~Modelisation du SCC:}
%     \small
%     \begin{itemize}
%         \item Inspiration d’un modèle existant et innovant
%         \item Détail de l’algorithme de contrôle
%         \item Couplage système~/~bâtiment
%         \item Analyse détaillée de son comportement
%     \end{itemize}
%     \vfill
%     \large{\color{SolarizedGreen} 2. ~Évaluer les éléments caractéristiques d’un SSC:}
%     \small
%     \begin{itemize}
%         \item Quels sont les facteurs dimensionnants ?
%         \item Quels sont les indicateurs de performance ?
%         \item Quel méthode pour une optimisation couplée ?
%     \end{itemize}
%     \vfill
% \end{frame}

% % % ==============================================================================
% % % ==============================================================================
% \section{\scshape Décrire}
% % -------------------------------
% \subsection{Modélisation}
% \begin{frame}
%     \frametitle{Modélisation}
%     \vfill
%     \large{\color{SolarizedGreen} Contraintes:}
%     \small
%     \begin{itemize}
%         \item Évaluer la performance de système existant
%         \item Modélisation multi-physique
%         \item Nombreuses itérations / modifications / modèles
%         \item Souplesse sur la description des systèmes (pas de boîte noire)
%     \end{itemize}
%     \vfill
%     \large{\color{SolarizedGreen} Solution:}
%     \small
%     \begin{itemize}
%         \item Modelica~: Language object, équationnel et acausal
%         \item Dymola~: Interface graphique et post-traitement
%         \item DymoSim~: Intégrateur multi-solveur
%     \end{itemize}
%     \vfill
%     \center
%     \begin{columns}
%         \begin{column}{0.3\textwidth}
%             \includegraphics[height=8mm]{Ressources/Images/Soutenance/logos/modelica_logo.png}
%         \end{column}
%         \begin{column}{0.3\textwidth}
%             \includegraphics[height=8mm]{Ressources/Images/Soutenance/logos/dassault_logo.jpg}
%         \end{column}
%         \begin{column}{0.3\textwidth}
%             \includegraphics[height=8mm]{Ressources/Images/Soutenance/logos/dymola_logo.jpg}
%         \end{column}
%     \end{columns}
%     \vfill
%     ~
% \end{frame}


% \subsection{Mode de chauffage}
% \begin{frame}
%     \frametitle{Quel modèle ?}
%     % Vérification != Validation
%     % Vérification capteur, vérification consommations, vérification comportement,
%     % vérification facteurs caractéristiques annuels (SolisArt)

%     \large{\color{SolarizedGreen} Contrôler la fiabilité de l’approche par modélisation}
%     \includegraphics[scale=0.35]{Includes/Modelisation/eau_hydraulique.pdf}
% \end{frame}

% \begin{frame}[c]
%     % Présenter vecteur Air
%     \vfill
%     \center
%     \Large
%     \begin{columns}
%         \begin{column}{0.3\textwidth}
%             Réactivité
%         \end{column}
%         \begin{column}{0.3\textwidth}
%             Simplicité
%         \end{column}
%         \begin{column}{0.3\textwidth}
%             Vecteur Air
%         \end{column}
%     \end{columns}
%     \vfill
%     \includegraphics[scale=0.34]{Includes/Modelisation/air_hydraulique.pdf}
%     \vfill
% \end{frame}

% % -------------------------------
% \subsection{Quelle performance}
% \begin{frame}[c]
%     \frametitle{Performances Annuelles}
%     \includegraphics[scale=0.2]{Includes/Modelisation/annuel_bordeaux.png}
% \end{frame}

% \begin{frame}[c]
%     \frametitle{Que retenir ?}
%     \vfill
%     \begin{itemize}
%         \item Construction de multiples modèles de SSC
%         \vfill
%         \item Capacité à évaluer les performances du modèle
%         \vfill
%         \item Validation du modèle de capteur
%         \vfill
%         \item Confiance dans le comportement du système à travers~:
%             \begin{itemize}
%                 \item L’expertise de SolisArt
%                 \item L’obtention des performances évaluées sur site
%             \end{itemize}
%     \end{itemize}
%     \vfill
% \end{frame}


% % % ==============================================================================
% % % ==============================================================================
% \section{\scshape Comprendre}
% % --------------------------
% \subsection{Analyse paramétrique}
% \begin{frame}[c]
%     \frametitle{Objectifs}
%     \large{\color{SolarizedGreen} Approche paramétrique}
%     \begin{itemize}
%         \item Évaluer l’impact des modifications
%             \begin{itemize}
%                 \item Variation des scénarios (consigne, puisage, apports internes, ...)
%                 \item Variation des conditions climatiques
%             \end{itemize}
%         \vfill
%         \item Définir les facteurs influents
%         \vfill
%         \item Définir les indicateurs importants
%         \vfill
%         \item Definir les scénarios pour les occupants
%     \end{itemize}
%     \vfill
% \end{frame}

% \begin{frame}[c]
%     \frametitle{Échantillon de l’analyse paramétrique}
%     \begin{adjustwidth}{-2.3em}{-2em}
%         \includegraphics[scale=0.2]{Includes/Identification/variations.pdf}
%     \end{adjustwidth}
% \end{frame}

% \begin{frame}[c]
%     \frametitle{Description du modèle}
%     \vfill
%     \begin{itemize}
%         \item Nombre de variable initiales~: {\color{emphase}57 211}
%         \item Nombre d’équations après réduction~: {\color{emphase} 17 966}
%     \end{itemize}
%     \vfill
%     \begin{adjustwidth}{-2.3em}{0pt}
%         \includegraphics[scale=0.22]{Includes/Identification/modele.png}
%     \end{adjustwidth}
%     \vfill
% \end{frame}

% \begin{frame}[c]
%     \frametitle{Niveau de détail du modèle: Bâtiment}
%     % DIFFERENCE ENERGY PLUS ET DYMOLA
%     \large{\color{SolarizedGreen} Modèle monozone:}
%     \small
%     \begin{itemize}
%         \item Simulation de Octobre à Mai
%         \item Consommation identiques au modèle multi-zone
%             \begin{itemize}
%                 \scriptsize
%                 \item Energy Plus: {\color{emphase}1190~\si{kWh}}
%                 \item Modelica: {\color{emphase}1114~\si{kWh}}
%             \end{itemize}
%         \item Le confort d’été n’est pas évalué
%     \end{itemize}
%     \vfill
%     \center
%     \small
%     \begin{columns}
%         \begin{column}{0.7\textwidth}
%             \includegraphics[scale=0.15]{Includes/Identification/consoElectrique.PNG}
%         \end{column}
%         \begin{column}{0.3\textwidth}
%             \vfill
%             Annuelle $\sim$ 3/4h
%             \vfill
%             Réduite $\sim$ 1h
%             \vfill
%         \end{column}
%     \end{columns}
%     ~
% \end{frame}

% \begin{frame}[c]
%     \frametitle{Niveau de détail du modèle: Contrôle}
%     \includegraphics[scale=0.19]{Includes/Identification/control_air_curve.pdf}
%     % Expliquer l’apport d’un modèle détaillé sur la compréhension
%     % et l’impact sur la performance
% \end{frame}


% \begin{frame}[c]
%     \frametitle{Que retenir ?}
%     \begin{itemize}
%         \item Impact important des occupants
%         \item Identification d’un problème multi-objectif
%         \item Réduction de la cardinalité en fixant les scénarios
%     \end{itemize}
%     \vfill
%     \center
%     \includegraphics[scale=0.25]{Includes/Identification/repartition_chauffage_consosElec.png}
%     % Problème fortement dépendant des scénarios (impact important des occupants)
%     % Problème fortement dépendant de la météo
%     % Processus d’optimisation limité au choix en amont
%     % Aide au choix des facteurs et indicateurs pour l’analyse de sensibilité
%     % Aide à la détermination d’objectifs pour l’optimisation
%     \vfill
% \end{frame}


% % % ==============================================================================
% % % ==============================================================================
% \section{\scshape Identifier}
% % ----------------------------------------
% \subsection{Approche retenues}
% \begin{frame}[c]
%     \frametitle{Analyse de sensibilité}
%     \large{\color{SolarizedGreen} Méthode de Morris:}
%     \small
%     \begin{itemize}
%         \item Méthode OAT (variation d’un unique paramètre à la fois)
%         \item Globale et qualitative
%     \end{itemize}
%     % Décrire rapidement Morris (Méthode OAT d’analyse de sensibilité globale qualitative)
%     % Présenter les facteurs / les indicateurs
%     \vfill
%     \scriptsize
%     \begin{tabular}{lp{6cm}r}
%         \toprule
%         Nom                      & Signification                                        & Unité    \\
%         \midrule
%         Chauffage solaire        & Quantité d’énergie solaire transmise à la maison     & \si{kWh} \\
%         Chauffage electrique     & Quantité d’énergie électrique pour le chauffage      & \si{kWh} \\
%         Chauffage solaire actif  & Part solaire direct                                  & \si{kWh} \\
%         Chauffage solaire passif & Part solaire indirect                                & \si{kWh} \\
%         BallonECS solaire        & Quantité d’énergie solaire apportée au ballon ECS    & \si{kWh} \\
%         BallonECS electrique     & Quantité d’énergie électrique apportée au ballon ECS & \si{kWh} \\
%         Consommation appoint     & Quantité totale de l’appoint (pompes comprises)      & \si{kWh} \\
%         Production solaire       & Énergie récupérée au niveau des capteurs             & \si{kWh} \\
%         Production solaire utile & Production moins les pertes en ligne                 & \si{kWh} \\
%         Couverture solaire       & Part solaire                                         & \%       \\
%         Couverture chauffage     & Part solaire sur le chauffage                        & \%       \\
%         Couverture ECS           & Part solaire sur l’ECS                               & \%       \\
%         Consommation chauffage   & Chauffage solaire + électrique                       & \si{kWh} \\
%         Consommation Électrique  & Consommation appoint + charges électriques           & \si{kWh} \\
%         Couverture Électrique    & Production solaire / Consommation électrique         & \%       \\
%         \bottomrule
%     \end{tabular}
%     \vfill
% \end{frame}

% \begin{frame}[c]
%     \frametitle{Sensibilité des indicateurs}
%     \vspace{-1cm}
%     \begin{adjustwidth}{-4.5em}{-2em}
%         \includegraphics[scale=0.42]{Includes/Identification/indicators_table.pdf}
%     \end{adjustwidth}

% \end{frame}


% \begin{frame}[c]
%     \frametitle{Que retenir ?}
%     \vfill
%     \large{\color{SolarizedGreen} Choix des indicateurs:}
%     \small
%     \begin{itemize}
%         \item Couverture solaire sur le chauffage
%         \item Couverture solaire sur l’ECS
%         \item Consommation de l’appoint
%         \item Contrainte de bilan énergétique positif
%     \end{itemize}
%     \vfill
%     \large{\color{SolarizedGreen} En cours:}
%     \small
%     \begin{itemize}
%         \item Analyse de Morris sur l’ensemble des indicateurs
%         \item Augmentation de la plage de variation de la performance de la maison
%         \item Prise en compte des contraintes de la maison IGC
%     \end{itemize}
%     \vfill
% \end{frame}

% % % ==============================================================================
% % % ==============================================================================
% \section{\scshape Caractériser}
% % -------------------------------
% \subsection{Optimisation multi-objectifs}
% \begin{frame}[c]
%     \frametitle{Comment ?}
%     \center
%     \normalsize
%     \vspace{-0.6cm}
%     \large{\color{SolarizedGreen} Problème complexe multi-objectif et sous contraintes}
%     \vfill
%     \small
%     \begin{columns}
%         \begin{column}{0.3\textwidth}
%             Connaissances limitées
%         \end{column}
%         \begin{column}{0.3\textwidth}
%             Combinatoire
%         \end{column}
%         \begin{column}{0.3\textwidth}
%             Temps de simulation important
%         \end{column}
%     \end{columns}
%     \vfill
%     \includegraphics[scale=0.19]{Includes/Identification/Classification_optimisations_approchee.png}
%     \vfill
% \end{frame}

% \begin{frame}[c]
%     \frametitle{Construction d’une méthode adaptée}
%     % Décrire un meta-heuristique
%     \large{\color{SolarizedGreen} Inspiré du comportement des abeilles mélifères}
%     \center
%     \begin{adjustwidth}{-4.5em}{0em}
%         \begin{columns}
%             \begin{column}{0.4\textwidth}
%                 \includegraphics[scale=0.22]{Includes/Caracterisation/algorithme_complet.png}
%             \end{column}
%             \begin{column}{0.6\textwidth}
%                 \small
%                 \begin{itemize}
%                     \item Un générateur de solution (ABC)
%                 \end{itemize}
%                     \vfill
%                 \begin{itemize}
%                     \item Une méthode d’archivage ($\epsilon$-dominance)
%                 \end{itemize}
%                     \vfill
%                 \begin{itemize}
%                     \item Une marche aléatoire (Marche de Lévy)
%                 \end{itemize}
%                     \vfill
%                 \begin{itemize}
%                     \item Une méthode d’initialisation (Opposite Based Learning)
%                 \end{itemize}
%                     \vfill
%                 \begin{itemize}
%                     \item Une methode de prise en compte des contraintes (Pénalisation dynamique)
%                 \end{itemize}
%             \end{column}
%         \end{columns}
%     \end{adjustwidth}
% \end{frame}

% \begin{frame}[c]
%     \large{\color{SolarizedGreen} Artificial Bee Colony}
%     % DESCRIPTION GÉNÉRATEUR IMAGE ABC
%     \vfill
%     \center
%     \includegraphics[scale=0.15]{Includes/Caracterisation/BeeDance.png}
%     \vfill
% \end{frame}

% \begin{frame}[c]
%     \large{\color{SolarizedGreen} Archivage par $\epsilon$-dominance}
%     % DESCRIPTION ARCHIVE IMAGE ARCHIVE
%     \vfill
%     \center
%     \includegraphics[scale=0.5]{Includes/Caracterisation/selection_boxes.png}
%     \vfill
% \end{frame}


% % % ==============================================================================
% % % ==============================================================================
% \section{\scshape Bilan}
% % -------------------------------
% \subsection{Conclusions et perspectives}
% \begin{frame}[c]
%     \frametitle{Conclusions et perspectives}
%     \large{\color{SolarizedGreen} Conclusions:}
%     \vfill
%     \small
%     \begin{itemize}
%         \item Modélisation de multiples systèmes solaires
%         \item Sélection du vecteur Air pour l’étude de cas
%         \item Identification de l’importance des occupants pour une maison passive
%         \item Réduction des degrés de liberté pour réduire la cardinalité du problème
%         \item Identification des facteurs influents
%         \item Identification des objectifs pour l’optimisation
%         \item Définition des contraintes
%         \item Proposition d’une methode d’optimisation adaptée au problème
%     \end{itemize}
%     \vfill
% \end{frame}

% \begin{frame}[c]
%     \vfill
%     \large{\color{SolarizedGreen} Perspectives:}
%     \vfill
%     \small
%     \begin{itemize}
%         \item Finalisation de l’étude de sensibilité
%         \vfill
%         \item Prise en compte du facteur coût économique dans l’optimisation
%         \vfill
%         \item Réalisation de l’optimisation multi-objectif
%         \vfill
%         \item Optimisation prévue fin Décembre
%     \end{itemize}
%     \vfill
% \end{frame}

% % % % ==============================================================================
% % % % ==============================================================================
% % \section{\scshape Analyse de Morris}
% % % ---------------------------------
% % \subsection{Paramètres}
% % \begin{frame}[c]
% %     \frametitle{Paramètres}
% %     \vfill
% %     \begin{tabular}{llc}
% %         \toprule
% %         Nom                & Signification              & Valeur         \\
% %         \midrule
% %         k                  & Nombre de variable         & 22             \\
% %         \\
% %         N                  & Nombre de pré-trajectoires & 100            \\
% %         \\
% %         p                  & Nombre de niveaux          & 4              \\
% %         \\
% %         r                  & Nombre de trajectoires     & 10             \\
% %         \\
% %         $\Delta$           & Pas de discrétisation      & $\frac{2}{3}$  \\
% %         \\
% %         Saut               & Le saut                    & 2              \\
% %         \\
% %         $r \times (k + 1)$ & Nombre de simulation       & 230            \\
% %         \bottomrule
% %     \end{tabular}
% %     \vfill
% % \end{frame}


% % \subsection{Facteurs}
% % \begin{frame}[c]
% %     \frametitle{Facteurs de l’enveloppe}
% %     \vfill
% %     \begin{tabular}{lccr}
% %         \toprule
% %         Nom                        & Borne mini & Borne max & Unité            \\
% %         \midrule
% %         Résistance plancher       & 7.5         & 10        & \si{m^{2}.K / W} \\
% %         \\
% %         Résistance plafond        & 7.5         & 10        & \si{m^{2}.K / W} \\
% %         \\
% %         Résistance mur            & 4.5         & 6.5       & \si{m^{2}.K / W} \\
% %         \\
% %         Surface vitrée Sud        & 8.62        & 12.9      & \si{m^{2}}       \\
% %         \\
% %         Surface vitrée Nord       & 0.46        & 0.684     & \si{m^{2}}       \\
% %         \\
% %         Surface vitrée Est        & 10.08       & 15.12     & \si{m^{2}}       \\
% %         \\
% %         Surface vitrée Ouest      & 5.18        & 7.78      & \si{m^{2}}       \\
% %         \\
% %         Facteur solaire           & 0.38        & 0.63      &                  \\
% %         \bottomrule
% %     \end{tabular}
% %     \vfill
% % \end{frame}

% % \begin{frame}[c]
% %     \frametitle{Facteurs du système}
% %     \vfill
% %     \begin{tabular}{lccr}
% %         \toprule
% %         Nom                            & Borne mini  & Borne max & Unité               \\
% %         \midrule
% %         Volume ballon ECS              & 0.1         & 0.5       & \si{m^{3}}          \\
% %         Résistance ballon ECS          & 7           & 10        & \si{m^{2}.K / W}    \\
% %         Volume ballon Stockage         & 0.1         & 0.5       & \si{m^{3}}          \\
% %         Résistance ballon Stockage     & 7           & 10        & \si{m^{2}.K / W}    \\
% %         Inclinaison capteur            & 15          & 60        & \si{\degree}        \\
% %         Orientation capteur            & Ouest (-90) & Est (90)  & \si{\degree}        \\
% %         Nombre capteurs                & 2           & 5         &                     \\
% %         Rendement optique              & 0.63        & 0.84      &                     \\
% %         a1                             & 0.9249      & 3.796     & \si{W/m^{2}.K}      \\
% %         a2                             & 0.00069     & 0.013     & \si{W/m^{2}.K^{2}}  \\
% %         Pente                          & -5.103      & -0.975    & \si{W/(m^{2}.K)}    \\
% %         Épaisseur isolant canalisation & 0.013       & 0.04      & \si{m}              \\
% %         \midrule
% %         Consigne solaire               & 19          & 25        & \si{\degreeCelsius} \\
% %         Delta solaire minimum          & 5           & 15        & \si{\degreeCelsius} \\
% %         \bottomrule
% %     \end{tabular}
% %     \vfill
% % \end{frame}


% % \subsection{Indicateurs}
% % \begin{frame}[c]
% %     \frametitle{Indicateurs}
% %     \vfill
% %     \footnotesize
% %     \begin{tabular}{lp{6cm}r}
% %         \toprule
% %         Nom                      & Signification                                        & Unité    \\
% %         \midrule
% %         Chauffage solaire        & Quantité d’énergie solaire pour le chauffage         & \si{kWh} \\
% %         Chauffage electrique     & Quantité d’énergie électrique pour le chauffage      & \si{kWh} \\
% %         Chauffage solaire actif  & Part solaire direct                                  & \si{kWh} \\
% %         Chauffage solaire passif & Part solaire indirect                                & \si{kWh} \\
% %         BallonECS solaire        & Quantité d’énergie solaire apportée au ballon ECS    & \si{kWh} \\
% %         BallonECS electrique     & Quantité d’énergie électrique apportée au ballon ECS & \si{kWh} \\
% %         Consommation electrique  & Quantité totale électrique                           & \si{kWh} \\
% %         Production solaire       & Quantité totale solaire                              & \si{kWh} \\
% %         Production solaire utile & Quantité totale solaire moins les pertes en ligne    & \si{kWh} \\
% %         Couverture solaire       & Part solaire                                         & \%       \\
% %         Couverture chauffage     & Part solaire sur le chauffage                        & \%       \\
% %         Couverture ECS           & Part solaire sur l’ECS                               & \%       \\
% %         \bottomrule
% %     \end{tabular}
% %     \vfill
% % \end{frame}



% % % % % % % % % % % % % % % % % % % % % % % % % % % % % % % % % % % % % % % %
\end{document}
% % % % % % % % % % % % % % % % % % % % % % % % % % % % % % % % % % % % % % % %



% % ==============================================================================
% % ==============================================================================
