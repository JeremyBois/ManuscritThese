%!TEX root = ../main.tex

Ce travail n’aurait pas été possible sans le soutient d’un certain nombre de personnes.
Je tiens donc à les remercier\dots\\

Je commence donc par mes encadrants qui m’ont proposé un sujet passionnant et
offert une grande liberté dans la manière de l’aborder~:
\begin{itemize}
  \item \textbf{Laurent Mora}~: Suite à mon stage de fin d’études \textbf{Laurent} m’a proposé
        de réaliser une thèse. Alors que ma formation ne m’y destinait pas,
        il a su m’apporter la confiance nécessaire pour entreprendre
        cette aventure.
        Il a aussi tout au long de ces travaux su alimenter ma réflexion grâce à de
        nombreuses interrogations pertinentes et une vision globale de la portée
        du sujet mais aussi du domaine du bâtiment.
        Je le remercie aussi pour m’avoir permis de réaliser des heures d’enseignements
        à l’\abr{IUT} Génie Civil en parallèle de mes travaux.

  \item \textbf{Etienne Wurtz}~: Tout comme Laurent, j’ai rencontré \textbf{Etienne} durant mon stage
        de Master~2. Il a su amorcer de nombreux débats sur la manière d’aborder
        le sujet et a toujours veillé à s’assurer que je me force à prendre plus de
        recul sur mon sujet.
\end{itemize}

\paragraph{} % (fold)
Je souhaite également remercier les membres du jury en commençant par les rapporteurs,
\textbf{Bruno Peuportier} et \textbf{Michel Pons}, pour la relecture attentive et l’analyse du fond comme
de la forme de ce manuscrit. Merci aussi à \textbf{Patrice Joubert} d’avoir accepté de présider
le jury de thèse et \textbf{Giles Rusaouën} d’être venu assister et débattre autour de mon travail.

Bien sûr ces travaux ont nécessité l’acquisition de nombreuses compétences. Je souhaite donc
remercier \textbf{Stéphanie Decker} pour m’avoir accordé une partie de son temps pour me présenter son
approche d’optimisation.
Je remercie aussi l’entreprise \textbf{SolisArt} pour les nombreux échanges
autour du système solaire combiné qui ont permis d’amorcer la réflexion pour ces travaux.
De même le travail réalisé dans le cadre du projet \textbf{COMEPOS} avec l’entreprise \textbf{IGC} m’ont
permis d’avoir avoir un regard extérieur nécessaire sur mes avancements.

\paragraph{} % (fold)
Sans surprise je remercie l’ensemble du personnel du laboratoire avec lesquelles j’ai
eu l’occasion de discuter et en particulier \textbf{Muriel} pour sa disponibilité et
réactivité malgré mes demandes de dernières minutes (je suppose que l’on apprend de
ces mentors\dots).

\paragraph{} % (fold)
Ces travaux m’ont aussi permis de rencontrer de nombreuses personnes d’origines et
parcours divers. Étant resté près de \num{4} ans dans les locaux j’ai eu l’occasion
de voir passer de nombreuses personnes dans le bureau. Merci à \textbf{Tan} (en particulier pour
la découverte de la nourriture vietnamienne), \textbf{Miguel} (j’espère
ne pas avoir était trop chiant sur la climatisation\dots), \textbf{Marita} (\enquote{La patilla del cambio~!}),
\textbf{Tom} (j’attends toujours ma viande séchée\dots), \textbf{Yibao} (profite bien du vélo~!). J’ai aussi rencontré
ma première chérie (oui \textbf{José} je parle bien de toi si tu lis cette page~!) et réussi
à infiltrer la communauté Corse (j’espère que tu vas m’inviter chez toi un de ces jours
\textbf{Anto}~!). Je remercie bien sûr aussi \textbf{Stéphanie}, \textbf{YingYing},
\textbf{Yohann}, \textbf{Guilio}, \dots

\paragraph{} % (fold)
La thèse a aussi été l’occasion de rencontrer des personnes qui me sont aujourd’hui très proches.
J’ai aussi eu le plaisir de rencontrer \textbf{Hugo} qui m’a transmis une partie
de sa passion pour \st{la montagne} les Pyrénées. Il m’a aussi fait l’honneur
de me choisir comme Parrain pour sa formidable (l’adjectif ne m’engage pas dans le futur)
petite fille.
J’ai eu la chance (ton mot préféré\dots) de rencontrer ma véritable petite chérie (la seconde
du coup\dots), \textbf{Maimouna}. Je suis heureux de toujours être à ses cotés aujourd’hui encore.
Je pense qu’ils ont été à eux deux le pilier sur lequel je pouvais me reposer lors
des moments difficiles.

\paragraph{} % (fold)
Je remercie bien sûr la salle d’escalade (\textbf{Rock’Altitude} si tu me lis\dots) pour
m’avoir permis de vider mon esprit et de faire des rencontres éphémères sur les différents
blocs. Aussi merci à \textbf{Youpi} pour sa joie de vivre et son insouciance.

\paragraph{} % (fold)
Je n’oublie pas \textbf{mes parents} et \textbf{grands-parents} qui m’ont soutenu
et qui m’ont permis de me construire et de devenir ce que je suis aujourd’hui
(merci pour les patisseries et les bons petits plats\dots).

\paragraph{} % (fold)
Je finis ce chapitre avec \textbf{Maimouna} qui m’a supporté professionnellement
comme personnellement d’une manière incroyable et avec qui je continue ma petite vie\dots

