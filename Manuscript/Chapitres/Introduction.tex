%!TEX root = ../main.tex

Les effets du réchauffement climatiques sont aujourd’hui facilement observables et sont
tels qu’ils bouleversent le fonctionnement de la planète. L’énergie considérée comme
infinie durant les \num{30} glorieuses s’est avérée être limitée et la sur-utilisation des
énergies fossiles nocive pour la faune comme la flore. Des mesures sont alors prises au
niveau international et national afin de contenir le réchauffement climatique et éviter un
dérèglement plus important. Dans le bâtiment, les mesures visent principalement la
diminution de la production de $CO_{2}$, et des dépenses énergétiques, grâce à
l’amélioration de la qualité de l’enveloppe et de l’efficacité énergétique. Le bâtiment
est en effet le premier consommateur d’énergie en France et apparaît donc comme un élément
clé pour la réussite de la transition énergétique. Bien que la rénovation soit le
principal levier pour permettre la réduction des consommations du parc Français, son
amélioration passe par le développement des bâtiments neufs et donc des maisons
individuelles qui représentent $\nicefrac{3}{8}$ des consommations en énergie primaire. La
maison individuelle doit ainsi être innovante énergétiquement afin de proposer des
solutions économiquement viables et performantes pour la rénovation et ainsi réduire, le
coût des rénovations futures. Cette progression vers un développement toujours plus
exemplaire est encadré par la réglementation thermique entrée en vigueur dès $1974$ qui
fixe des contraintes énergétiques de plus en plus fortes amenant au développement de
systèmes et matériaux toujours plus performants. Aujourd’hui les nouveaux bâtiments à
l’horizon $2020$ devront être à énergie positive afin de répondre aux problématiques
climatiques de manière efficace. Il est cependant nécessaire de préparer le secteur aux
mesures futures par l’intermédiaire de labels et/ou de projets démonstrateurs.

\paragraph{} % (fold)
La conception d’un bâtiment est aujourd’hui le résultat d’un processus itératif guidé par
l’expérience. Un bâtiment de référence est proposé et une étude paramétrique est réalisée
afin de trouver une solution appropriée. Cependant l’amélioration de l’efficacité des
bâtiments entraîne l’explosion de la combinatoire des solutions, alors que parallèlement,
le temps disponible pour chaque appels d’offres reste identique. La conception d’une
maison à énergie positive est donc un problème multi-critères où il est essentiel de
trouver le bon équilibre entre qualité de l’enveloppe, coût de construction, et efficacité
des équipements~: un processus itératif n’est donc pas adapté. Il existe de plus des
besoins non compressibles, comme la production d’eau chaude sanitaire, ou l’augmentation
croissante des besoins domestiques tel que l’électroménager. Ainsi, en plus du respect du
principe de sobriété, il est nécessaire de substituer les énergies respectueuses de
l’environnement tel que le solaire, aux énergies fossiles et nucléaires, afin de réduire la
part anthropique responsable du réchauffement climatique. L’énergie solaire est en effet
disponible de manière presque illimitée et inonde chaque jour l’ensemble du globe
terrestre. Elle apparaît ainsi comme une candidate de choix pour répondre à la fois aux
besoins électriques et thermiques. Alors que le photovoltaïque est en plein expansion, le
solaire thermique subit une tendance négative notamment à cause d’un manque de
compétitivité économique et à un encombrement important qui freine son développement. Ces
travaux proposent donc le développement simultané du bâtiment et des systèmes solaires
responsables de la couverture de l’ensemble des usages des occupants, afin d’évaluer
le potentiel du solaire appliqué au maisons très performantes.
% Il est en effet important dans un bâtiment très performant de tenir compte des
% interactions existantes entre occupants, performance de l’enveloppe, et efficacité des
% systèmes afin de pouvoir proposer la solution la plus adaptée.
Une méthodologie multi-critères
et multi-objectifs d’aide à la conception de maisons solaires à énergie positives est
ainsi proposée où~: le temps machine est assujetti aux tâches répétitives, et le temps humain
libéré pour les tâches cognitives. En effet avec le développement de la puissance de
calcul des ordinateurs, une approche automatisée est aujourd’hui pertinente en plus
de permettre d’améliorer l’exploration du domaine de définition.
% De plus,
% l’approche retenue introduit une préférence uniquement en aval de l’optimisation multi-
% objectifs, après l’exploration de l’ensemble de l’espace de décision.
% Le caractère exploratoire permet de potentiellement trouver de nouvelles alternatives plus
% adaptées aux maisons à énergie positives. L’ajout de préférence se traduit elle par la
% prise en compte des contraintes propres et par l’ajout de l’expérience pour aider à la
% décision finale.
L’objectif de ces travaux est ainsi double~: valoriser le temps de
cerveau tout en proposant des solutions optimales à travers la mise en place d’une
méthodologie adaptée, et, évaluer la faisabilité d’une maison à énergie positive couplée à
un système solaire thermique.

\paragraph{} % (fold)
Il est ainsi dans un premier temps abordé plus en détail le contexte du réchauffement
climatique et des différentes initiatives tant au niveau normatif que incitatif. La place
de l’énergie solaire dans le bouquet énergétique et les différentes avancées technologies
comme réglementaires sont ensuite discutées afin de pouvoir proposée une méthodologie
adaptée. Le second chapitre introduit les outils utilisés et le cas d’étude retenue à
travers la description des scénarios retenues et du fonctionnement du système solaire. Une
étude paramétrique est aussi réalisée afin d’évaluer le potentiel du couplage entre un
système solaire et une maison à énergie positive. Fort de l’expérience acquise à travers
les deux premiers chapitres, le troisième introduit la méthodologie d’aide à la décision
retenue pour la conception de maison solaires à énergie positive. Il y est notamment
introduit des moyens permettant de réduire la complexité du problème, une approche
d’optimisation permettant l’exploration du domaine de faisabilité, ou encore un outil
interactif pour simplifier l’introduction des préférences pour le choix final. Le
quatrième et dernier chapitre illustre la mise en place de la méthodologie d’aide à la
décision développée à travers deux applications de conception de maisons à énergie
positive. Ces deux conditions climatiques permettent de mettre en évidence la capacité de
l’approche à proposer un large éventail de solutions optimales présentant des compromis
différents en termes de performance du bâti et de dimensionnement des équipements
solaires. Enfin l’aide à la décision est illustrée par l’introduction de la préférence
permettant de dégager des solutions à retenir.
