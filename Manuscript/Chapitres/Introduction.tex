%!TEX root = ../main.tex


Les effets du réchauffement climatique sont aujourd’hui facilement observables et sont
tels qu’ils bouleversent le fonctionnement de la planète. L’énergie considérée comme
infinie durant les \num{30} glorieuses s’est avérée être limitée et la sur-utilisation des
énergies fossiles comme nocive pour la faune et la flore. Des mesures sont alors prises au
niveau international et national afin de contenir le réchauffement climatique et éviter un
dérèglement plus important. Dans le bâtiment, les mesures visent principalement la
diminution de la production de $CO_{2}$, et des dépenses énergétiques, grâce à
l’amélioration de la qualité de l’enveloppe et de l’efficacité énergétique. Le bâtiment
est en effet le premier consommateur d’énergie en France et apparaît donc comme un élément
clé pour la réussite de la transition énergétique. Avec près de \num{200000} nouveaux
chantiers terminés par an, la conception de maisons individuelles performantes et
respectueuses de l’environnement, est indispensable pour réduire la facture énergétique et
le coût des rénovations futures. Cette progression vers un développement toujours plus
exemplaire est encadrée par la réglementation thermique entrée en vigueur dès $1974$ qui
fixe des contraintes énergétiques de plus en plus fortes. Ces contraintes
se traduisent par le développement de systèmes et matériaux toujours plus performants
mais aussi par à la prise en compte de plus en plus de paramètres.

\paragraph{} % (fold)
La conception d’un bâtiment est aujourd’hui le résultat d’un processus itératif guidé par
l’expérience. Un bâtiment de référence est proposé et une étude paramétrique est réalisée
afin de trouver une solution appropriée. Cependant l’amélioration de l’efficacité des
bâtiments entraîne l’explosion de la combinatoire des solutions, alors que parallèlement,
le temps disponible pour chaque appel d’offres reste identique. La conception d’une
maison à énergie positive est donc un problème multi-critères où il est essentiel de
trouver le bon équilibre entre qualité de l’enveloppe, coût de construction, et efficacité
des équipements~: un processus itératif n’est donc pas adapté. Il existe de plus des
besoins non compressibles, comme la production d’eau chaude sanitaire, ou l’augmentation
croissante des besoins domestiques tel que l’électroménager. Ainsi, en plus du respect du
principe de sobriété, il est nécessaire de substituer les énergies respectueuses de
l’environnement tel que le solaire, aux énergies fossiles et nucléaires, afin de réduire la
part anthropique responsable du réchauffement climatique. L’énergie solaire est en effet
disponible de manière presque illimitée et inonde chaque jour l’ensemble du globe
terrestre. Elle apparaît ainsi comme une candidate de choix pour répondre à la fois aux
besoins électriques et thermiques. Alors que le photovoltaïque est en plein expansion, le
solaire thermique subit une tendance négative notamment à cause d’un manque de
compétitivité économique et à un encombrement important qui freine son développement.

À l’horizon $2020$, la future réglementation imposera que les nouveaux bâtiments soient
tous à énergie positive et le concepteur devra alors tenir compte de l’explosion de la
combinatoire pour faire son choix. À travers ce document, il est donc proposé une approche
tenant compte du principe de sobriété, où la majeure partie des besoins énergétiques de la
maison individuelle sont couverts par un système solaire combiné. Contrairement aux
approches actuellement utilisées, le bâtiment, les systèmes solaires (photovoltaïque et
thermique), et la logique de contrôle, sont considérés comme un ensemble cohérent~: le
bâtiment et les systèmes solaire sont dimensionnés simultanément et fonctionnent de
concert. La méthodologie développée est donc globale, multi-critères, et multi-objectifs
et permet d’accompagner la prise de décision des concepteurs lors de la construction d’une
maison à énergie positive solaire. L’objectif de ces travaux est ainsi triple~: valoriser
le temps de cerveau à travers un processus automatisé, améliorer l’exploratoire afin de
proposer une large palette de solutions optimales pour la prise de décision, et, évaluer
la faisabilité d’une maison à énergie positive couplée à un système solaire thermique.


\paragraph{} % (fold)
Dans un premier temps abordé plus en détail le contexte du réchauffement
climatique et des différentes initiatives tant au niveau normatif que incitatif. La place
de l’énergie solaire dans le bouquet énergétique et les différentes avancées technologiques
comme réglementaires sont ensuite discutées afin de pouvoir proposée une méthodologie
adaptée. Le second chapitre introduit les outils utilisés et le cas d’étude retenue à
travers la description des scénarios retenus et du fonctionnement du système solaire. Une
étude paramétrique est aussi réalisée afin d’évaluer le potentiel du couplage entre un
système solaire et une maison à énergie positive. Fort de l’expérience acquise à travers
les deux premiers chapitres, le troisième introduit la méthodologie d’aide à la décision
retenue pour la conception de maisons solaires à énergie positive. Il y est notamment
introduit des moyens permettant de réduire la complexité du problème, une approche
d’optimisation permettant l’exploration du domaine de faisabilité, ou encore un outil
interactif pour simplifier l’introduction des préférences pour le choix final. Le
quatrième et dernier chapitre illustre la mise en place de la méthodologie d’aide à la
décision développée à travers deux applications de conception de maisons à énergie
positive. Ces deux conditions climatiques permettent de mettre en évidence la capacité de
l’approche à proposer un large éventail de solutions optimales présentant des compromis
différents en termes de performance du bâti et de dimensionnement des équipements
solaires. Enfin l’aide à la décision est illustrée par l’introduction de la préférence
permettant de dégager des solutions à retenir.
