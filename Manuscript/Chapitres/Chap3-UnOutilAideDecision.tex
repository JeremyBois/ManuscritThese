% Chapitres/Chap3-UnOutilAideDecision

\section{Description générale} % (fold)
\label{sec:description_generale}
-----------------------------------
Définition:
 - Formulation
 - Algorithmes existants
 - Application

% section description_gene rale (end)



\section{Multi-critère} % (fold)
\label{sec:multi_critere}
-----------------------------------
Description des méthodologies:
 - Formulation
 - Optimisation puis décision
 - Décision puis optimisation
 - Combinée

Description des méthodes existantes:
 - Graphe des possibilités
 - Exactes
 - Approchées

Les type de variables:
 - Quantitative
 - Qualitative (combinatoire, continue)

Approches Pareto:
 - Comparaisons de solution
 - Front de Pareto
 - Critère de performance (Hyper-volume, répartition, convergence, rapidité)
 - Les méthode de sélection de Front de Pareto
 - les métriques utilisable pour évaluer la qualité de ce front
% section multi_critere (end)



\section{Analyse de sensibilité} % (fold)
\label{sec:analyse_de_sensibilite}
-----------------------------------
Réduction du nombre de variables par l’étude de sensibilité:
 - Méthodes locales
 - Méthodes globales
 - Screening (Morris)

Nécessiter de limiter le nombre de variables à évaluer pour réduire la cardinalité
du problème.
% section analyse_de_sensibilite (end)

