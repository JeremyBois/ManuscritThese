% Chapitres\Chap2-Modelisation.tex

\section{Choix de modélisation} % (fold)
\label{sec:choix_de_modelisation}
% -----------------------------------
% Décrire les outils utilisés:
%  - Modelica et les bibliothèques
%  - Dymola et les solveurs
\NOTE{Décrire les outils et pourquoi ils ont été choisis}
% section choix_de_modelisation (end)


\section{Description du bâtiment} % (fold)
\label{sec:description_du_batiment}
% -----------------------------------
\subsection{Description des systèmes} % (fold)
\label{sub:description_des_systemes}
% Quel niveau de détail:
%  - Le niveau choisie et pourquoi ?
%  - Le choix d’un modèle existant, pourquoi ?
\NOTE{Décrire pourquoi avoir fait le choix d’une modélisation détaillée}
\NOTE{Décrire pourquoi avoir fait le choix de modéliser un système existant}

% Détail des systèmes modélisés:
%  - Liste des systèmes et leur origine
%  - Fonctionnement de chaque système
%  - Explication de la partie algorithmie plus en détail
\NOTE{Décrire la partie système et algorithmique}
\NOTE{Présentation schématique des différents systèmes modélisé et leur fonctionnement}
% subsection description_des_systemes (end)
% section description_du_batiment (end)

\subsection{Description de l’enveloppe} % (fold)
\label{sub:description_de_l_enveloppe}
\NOTE{Décrire globalement le bâtiment}
% subsection description_de_l_enveloppe (end)


\section{Approche monozone} % (fold)
\label{sec:approche_monozone}
% Analyse paramétrique / premiers résultats:
%  - Identification de l’impact de certains paramètres
%  - Résultats de la comparaison mono-zone, multi-zone
\NOTE{Décrire les limitations de cette approches grâce au premiers résultats}
% section approche_monozone (end)


\section{Approche multizone} % (fold)
\label{sec:approche_multizone}
% Outils nécessaires:
%  - Energy Plus
%  - FMU
%  - Python
%  - Sketchup
%  - Open Studio
\NOTE{Décrire les outils nécessaire pour l’approche multi-zone}
\NOTE{Décrire la mise en place du modèle}
\NOTE{Décrire les résultats et comparer avec le cas monozone}
% section approche_multizone (end)


\section{Validation expérimentale} % (fold)
\label{sec:validation_experimentale}
% -----------------------------------
% Validation expérimentale des modèles:
%  - Description de l’expérimentation
%  - Mise en place
%  - Vérification du modèle
\NOTE{Décrire l’utilité de valider le modèle numérique par des expérimentations}
\NOTE{Détailler la mise en place de l’expérimentation à échelle 1}
% section validation_experimentale (end)
