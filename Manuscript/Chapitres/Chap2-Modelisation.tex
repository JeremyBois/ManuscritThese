
\section{Les outils} % (fold)
\label{sec:les_outils}
-----------------------------------
Décrire les outils utilisés:

 - Modelica
 - Dymola

Pourquoi ces outils ?
% section les_outils (end)


\section{Description des systèmes} % (fold)
\label{sec:description_des_systemes}
-----------------------------------
Partie système
Quel niveau de détail:

 - Le niveau choisie et pourquoi ?
 - Le choix d’un modèle existant, pourquoi ?


Détail des systèmes modélisés:

 - Liste des systèmes et leur origine
 - Fonctionnement de chaque système
 - Explication de la partie algorithmie plus en détail
% section description_des_systemes (end)


\section{Description du bâtiment} % (fold)
\label{sec:description_du_batiment}
-----------------------------------
\subsection{Approche monozone} % (fold)
\label{sub:approche_monozone}
Décrire cette approche et ses limites.
% subsection approche_monozone (end)

\subsection{Approche multizone} % (fold)
\label{sub:approche_multizone}
Outils nécessaires:

 - Energy Plus
 - FMU
 - Python
 - Sketchup
 - Open Studio

Décrire les outils puis les modifications que cela engendre au niveau de la modélisation
des systèmes.
% subsection approche_multizone (end)
% section description_du_batiment (end)


\section{Premiers résultats} % (fold)
\label{sec:premiers_resultats}
-----------------------------------
Analyse paramétrique / premiers résultats:

 - Identification de l’impact de certains paramètres
 - Résultats de la comparaison mono-zone, multi-zone

% section premiers_resultats (end)


\section{Validation expérimentale} % (fold)
\label{sec:validation_experimentale}
-----------------------------------
Validation expérimentale des modèles:

 - Description de l’expérimentation
 - Mise en place
 - Vérification du modèle
% section validation_experimentale (end)
