%!TEX root = ../main.tex
% Chapitres\Chap1-ContexteTravaux.tex



% ..............................................................................
% ..............................................................................
\section{L’Homme au centre d’un trouble climatologique} % (fold)
\label{sec:l_homme_au_centre_d_un_trouble_climatologique}
% ------------------------------------------------------------------------------
\subsection{Vers une prise de conscience collective} % (fold)
\label{sub:vers_une_prise_de_conscience_collective}
~
\itodo{Décrire l’histoire de la prise de conscience}
La prise de conscience de l’impact de l’humain sur son environnement intervient vers
la fin du 20ème siècle. Il est alors mis en exergue les nombreux problème dont les
générations futures devront gérer~: destruction de la couche d’ozone, gestion des
déchets, bouleversement climatiques, pollution de l’eau, atteintes à la bio-diversité ...
En 1970, le club de Rome commandite le rapport Meadows, publié en
1972 sous le nom \enquote{The Limits to Growth} traduit en français par
\enquote{Halte à la croissance ?} (\mtodo{Ref Rapport Meadows}).
Pour la première fois, la possibilité d’une pénurie des ressources énergétiques est
envisagé par des chercheurs du MIT à travers plusieurs scénarios plus ou moins
catastrophiques. Ce rapport fut la première pierre nécessaire à la construction d’une
prise de conscience commune et sera ensuite mis à jour à plusieurs reprises.
En 1987 un nouveau rapport cette fois commandité par
les Nations Unies, \enquote{Our Common Future} traduit par \enquote{Notre avenir à tous}
plus connus sous le nom de rapport Bruntland (\mtodo{Ref Rapport Bruntland}). Contrairement
au rapport Meadows qui présente des scénarios de ce qui pourrait arriver si rien ne change,
ce document pose les bases pour un développement équitable mettant en avant
la protection des ressources naturelles mais aussi de l’équité face aux ressources
entre les individus. Le terme de \enquote{sustainable development} traduit en français
par \enquote{Développement durable}, aujourd’hui largement repris, est introduit~:
\blockquote{
    Un développement qui répond aux besoins des générations présentes sans
    compromettre la capacité des générations futures de répondre aux leurs.
}
\ftodo{Ajouter représentation du développement durable}

\href{http://unfccc.int/essential_background/convention/status_of_ratification/items/2631.php}{CCNUCC}\\
Au début porté par des militants, la conscience collective amène à la prise en compte de
l’environnement dans les textes réglementaires avec notamment le protocole de Montreal
(1987) mettant en place des restrictions afin de protéger la couche d’ozone. Peu de temps
après, au 3ème Sommet de la Terre se tenant à Rio en 1992, la Convention-cadre des Nations
unies sur les changements climatiques (CCNUCC) est adoptée et est aujourd’hui ratifiée par 197
pays montrant l’importance des question du développement durable sur le plan international.
Elle développe un plan d’action pour le 21ème siècle connu sous le nom d’Agenda 21
mais aussi 27 principes pour sa mise en œuvre. Les domaines traités sont très variés
et couvrent par exemple la pauvreté, la gestion des ressources notamment en eau, des déchets,
mais aussi la pollution ou l’agriculture...

La 3ème Conférence des Parties (COP, Conference Of Parties) à Kyoto (1997) entrée
en vigueur début 2005, complète la convention par un accord visant à réduire les
émissions de gaz à effet de serre (GES).

\href{http://www.automatesintelligents.com/echanges/2006/nov/rapportstern.html}{Stern bilan}\\
Pour la première fois, en 2006, le rapport Stern (\mtodo{Ajouter ref Stern}) commandé par
le gouvernement anglo-saxon décrit sur le plan économique l’impact d’une inaction
sur le problème du réchauffement climatique. Précédemment porté par des scientifiques,
la sonnette d’alarme est ici tiré par un économiste mettant en avant les risques
tant sur le plan humain que économique. Le réchauffement impacte en effet directement
les composantes essentielles de notre mode de vie~: l’accès à la santé, à la nourriture,
l’eau ... Afin de solutionner la problématique, il est suggérer de mettre en place
la coopération technique en augmentant l’effort de recherche sur le sujet mais aussi
sur la mise en place de moyens techniques mais aussi d’équilibrer les dépenses
entre les pays développés et en voie de développement. Il met ainsi en avant la nécessité
pour les pays riches qui sont les principaux responsables du réchauffement climatique
d’aider un développement respectueux de l’environnement pour les pays plus pauvres.
Une coopération internationale est ainsi indispensable afin de répondre à la problématique
de manière efficace.
En France, les loi Grenelles I (2007) et II (2010) marquent le premier pas vers
une démarche responsable sur les transport, la santé, l’énergie ou encore la préservation
de la bio-diversité comme notamment avec la labellisation de l’agriculture BIO ou
la réduction de la précarité énergétiques des foyers.

\href{http://www.ecologique-solidaire.gouv.fr/loi-transition-energetique-croissance-verte}{LTECV}
\href{https://www.legifrance.gouv.fr/affichTexte.do?cidTexte=LEGITEXT000031742863&dateTexte=20170606}{LTECV}\\
En 2015, l’adoption de la Loi pour la Transition Énergétique et la Croissance
Verte (LTECV) fixe plusieurs objectifs tant sur la consommation énergétique
que sur les émissions de GES afin de préparer à la sortie des énergies fossiles
et créer de l’emploi pour une population en augmentation. Il y est notamment inscrit
l’objectif d’une réduction de 40\% de l’émission de GES entre 1990 et 2030 et la
réalisation du facteur 4 pour 2050. Sur l’énergie, une réduction de la consommation
en énergie finale et primaire pour 2030 est aussi détaillée tant sur le plan économique
que logistique avec respectivement des objectifs de -20\% (-50\% en 2050) et de -30\%.
De plus des mesures propres à notre bouquet énergétiques sont aussi décrites comme
la réduction de la dépendance de la France à l’énergie nucléaire.

\href{https://www.edf.fr/groupe-edf/espaces-dedies/l-energie-de-a-a-z/tout-sur-l-energie/produire-de-l-electricite/le-nucleaire-en-chiffres}{EDF}\\
La mesure engage l’état à réduite à 50\% la part de production électrique d’origine nucléaire
qui est en 2014 de 77\%.


Au niveau international, la 21ème COP (2016) a permise d’aller encore plus loin avec l’acception de la mise en place
de mesures pour la réduction des émissions (GES) avec notamment l’objectif
de contenir le réchauffement climatique en dessous des 2°C (réalisation du plan facteur 4) et ainsi permettre
d’atteindre la \enquote{neutralité carbone} en compensant les GES dans la seconde moitié
du siècle. Contrairement au précédent accord (Accords de Kyoto), le parti-pris est ici
la transparence entre les différents signataires. En effet chaque signataire à l’obligation de rendre
compte régulièrement des objectifs et évolutions réalisées et ces informations seront
rendues publiques afin d’inciter à l’exemplarité. Bien que l’accord soit symboliquement
fort et un pas en avant vers un développement plus durable, aucunes prises de mesures
n’est cependant obligatoires. De plus la sobriété énergétique n’est pas mentionnée dans l’accord
principe pourtant clé du scénario NégaWatt.

~\\ \itodo{Décrire Négawatt}
L’association NégaWatt qui développe depuis 2003 un scénario pour la transition énergétique
à travers 3 mesures principales (\mtodo{ref figure}) met clairement
en avant la nécessité de la sobriété énergétique, la nécessité de repenser la manière
dont nous consommons l’énergie en réduisant le gaspillage. La dernière mise à jour
du scénario (2017) permet d’atteindre une couverture 100% renouvelable grâce à la
biomasse, puis l’éolien et le photovoltaïque et la fermeture en 2035 du dernier
réacteur nucléaire arrivant à plus de 40 ans d’années de fonctionnement.
https://fr.wikipedia.org/wiki/Afterres2050
Au niveau de l’agriculture et donc de l’accès à la nourriture, le scénario NégaWatt
est complété par le scénario Afterres2050 développé par l’ONG Solagro. Celui-ci s’inspire
des avancées déjà proposées à travers les loi Grenelle et du scénario NégaWatt et montre
que le respect notamment du facteur 4 pourrait permettre de couvrir les besoins
alimentaires de l’ensemble des français en mettant en avant une fois de plus la
sobriété comme élément clé avec notamment la réduction de la consommation de viande.

\ftodo{Ajouter représentation de la transition par négaWatt}

\itodo{Parler du projet DeserTec}
\itodo{Impact à différentes échelles}
% subsection vers_une_prise_de_conscience_collective (end)



% ------------------------------------------------------------------------------
\subsection{Comprendre l’évolution} % (fold)
\label{sub:comprendre_l_evolution}

\subsubsection{Un bagage nécessaire} % (fold)
\label{ssub:un_bagage_necessaire}
~
\itodo{Décrire rapports GIEC et évolution}
Les décision politiques étant étroitement lié aux résultats scientifiques sur le
sujet, une organisation indépendante a été créé en 1988 afin de fournir une base
scientifique solide. Cette organisation connu sous le nom de
Groupe d'experts Intergouvernemental sur l'Évolution du Climat (GIEC), regroupe
des experts en climatologie dont le but est d’apporter un regard critique sur
des travaux provenant de différents groupes de recherche scientifiques, techniques,
et socio-économiques. Il a pour but la réalisation sans parti-pris de l’expertise
des risques liés au réchauffement climatique et de faire re-sortir les informations
et découvertes faisant consensus dans la communauté scientifique. Finalement, grâce à ces
publications périodiques, les connaissances accumulées sur le sujet permettent de
sensibiliser le public non expert, élément indispensable pour que des engagements
politiques soit pris. Ainsi, il est nécessaire dans un premier temps d’expliquer les principaux termes
nécessaire à la compréhension de la situation climatique avant d’en présenter les chiffres
clés.

\href{https://fr.wikipedia.org/wiki/%C3%89nergie_(physique)}{Énergie}
L’énergie qui tient une place centrale dans la problématique est ici utilisé non dans
son sens morale mais dans son sens physique. L’énergie en physique est la mesure
de la capacité d’un système à produire de la chaleur, modifier un état, ou encore
entrainer un mouvement. Dans notre problématique il est principalement considéré
deux types d’énergie~: l’énergie dite primaire et l’énergie dite finale.
L’énergie primaire correspond à l’énergie prélevée à l’environnement alors que l’énergie
finale est l’énergie après soustraction des pertes de stockage, de transport, et de
transformation. L’énergie finale correspond ainsi à l’énergie consommée par l’utilisateur
finale et ne tient ainsi pas compte des moyens utilisés pour sa production.

\ttodo{Ajouter les facteurs de conversion en France}

D’un point de vue environnemental, l’augmentation des émissions de GES et l’élévation
de la température de la planète représente une menace importante
Le premier facteur fait intervenir la notion d’effet de serre, un phénomène naturel
engendré par la présence de divers composés gazeux dans l’atmosphère.
L’atmosphère ayant une faible réflexion et absorption pour le rayonnement solaire
(visible, proche UV et IR), un grande partie arrive à la surface de la terre. Une
partie est alors absorbée alors qu’une autre est renvoyé vers l’espace. Dû à cet
échauffement, le sol émet un rayonnement dans le domaine de l’IR lointain. Certain
composant gazeux étant faiblement transparents à ce domaine d’émission, une partie
de l’énergie est absorbée au lieu d’être rejeté dans l’espace~: c’est le mécanisme
de l’effet de serre.
La température moyenne sur l’ensemble de la surface du globe est ainsi le résultat
du bilan radiatif entre le rayonnement solaire incident et le rayonnement infrarouge.
N’ayant pas d’influence sur la vapeur d’eau présente dans l’atmosphère
(55\% de la contribution à l’effet de serre), le CO2 joue alors un rôle essentiel
dans l’étude de l’impact de l’homme et des conséquences à moyen et long terme.

\ftodo{Pourcentage de la contribution des émissions anthropiques}

% Sans l’impact de l’être humain, le bilan permet à la terre d’atteindre une température
% moyenne de \SI{15}{\celsius} au lieu de \SI{-18}{\celsius}.


% subsubsection un_bagage_necessaire (end)

\subsubsection{Évolution des facteurs environnementaux} % (fold)
\label{ssub:evolution_des_facteurs_environnementaux}
~
\itodo{Évolution de la température et des énergies}
Actuellement les mesures montre un forçage radiatif excédentaire entrainant une
lente augmentation de la température du sol. En 2003, le GIEC estimait ce forçage
à \SI{2.3}{\watt\per\metre\squared} dont \SI{1.6}{\watt\per\metre\squared} imputable
aux émissions anthropiques. Bien que nous soyons en période de réchauffement
(causes externes astronomiques), l’activité humaine semble être responsable d’une
augmentation anormalement rapide.
\itodo{Ajouter résultats GIEC plus récent sur GES, voir *GEA 2012*}


\itodo{Mettre en avant l’accélération des consommations / GES / population}
\ftodo{Ajouter évolution CO2 selon GIEC}
Cet accroissement est lié à deux grand facteurs. Premièrement, l’augmentation de la robotisation
et le développement très rapide depuis les années 90 de l’industrialisation à fortement
augmenter la consommation d’énergie fossiles. Deuxièmement, l’accroissement très important
de la population en quelques années entrainant une explosion de la demande énergétique
(\mtodo{GEA 2012}). Cette énergie est aujourd’hui a 80\% produite grâce à des sources
carbonées, le pétrole, le charbon. Il est ainsi clair que des mesures doivent être
prises afin de limiter la consommation en énergie primaire. De plus une alternative
respectueuse de l’environnement, les énergies renouvelables, réduisant les émissions de
GES est indispensable pour permettre d’inverser la tendance.

\itodo{Le coût des énergies renouvelables}
\href{http://decrypterlenergie.org/les-energies-renouvelables-coutent-elles-trop-cher}{coût ENR}\\
C’est dans cette optique que les mesures décrites ci-dessus ont été prise, mais
il demeure encore aujourd’hui une


% subsubsection evolution_des_facteurs_environnementaux (end)
% subsection comprendre_l_evolution (end)


% ------------------------------------------------------------------------------
\subsection{L’impact du secteur du bâtiment} % (fold)
\label{sub:l_impact_du_secteur_du_bâtiment}
~
\ftodo{Émission des GES par secteur en France}

\itodo{Loi Grenelle I et II}
En france la loi Grenelle I (2007) puis II (2010) venant compléter la première, porte l’engagement national
sur l’environnement en définissant un cadre d’action afin de répondre à l’urgence écologique.
Elle est notamment porteuse de l’initiative d’un programme accéléré de rénovation thermique du
parc des bâtiments existants afin de réduire l’émission de GES mais aussi la précarité énergétique
des foyers.

% subsection l_impact_du_secteur_du_bâtiment (end)Le bâtiment~: un élément clé
% section l_homme_au_centre_d_un_trouble_climatologique (end)




% ..............................................................................
% ..............................................................................
\section{Vers le bâtiment à énergie positive} % (fold)
\label{sec:vers_le_batiment_a_energie_positive}
% ------------------------------------------------------------------------------
\subsection{Le bâtiment passif} % (fold)
\label{sub:le_batiment_passif}
% - - - - - - - - - - - - - - - - - - - - - - - - - - - - - - - - - - - - - - -
\subsubsection{Une volonté de réduire l’impact de l’homme} % (fold)
\label{ssub:une_volonte_de_reduire_l_impact_de_l_homme}
~
\itodo{Réglementation jusqu’à RT 2005 pour introduire BBC}
Le développement de la performance des bâtiment a ainsi débuté avec la prise de
conscience de l’impact de l’Homme sur son environnement, et surtout avec la prise
de conscience des conséquence néfastes en résultant.

% subsubsection une_volonte_de_reduire_l_impact_de_l_homme (end)


% - - - - - - - - - - - - - - - - - - - - - - - - - - - - - - - - - - - - - - -
\subsubsection{Le cas français} % (fold)
\label{ssub:le_cas_francais}
~
\itodo{Décrire BBC et BBC + pour la préparation de la réglementation}
\itodo{Maisons test, validation des mesures, validation de la faisabilité}

% subsubsection le_cas_francais (end)


% - - - - - - - - - - - - - - - - - - - - - - - - - - - - - - - - - - - - - - -
\subsubsection{Les autres approches européennes} % (fold)
\label{ssub:les_autres_approches_europeennes}
~
\itodo{Décrire les autres approches}
Minergie, Passivhaus

% subsubsection les_autres_approches_europeennes (end)
% subsection le_batiment_passif (end)



% ------------------------------------------------------------------------------
\subsection{Le bâtiment à énergie positive} % (fold)
\label{sub:le_batiment_a_energie_positive}
% - - - - - - - - - - - - - - - - - - - - - - - - - - - - - - - - - - - - - - -
\subsubsection{Description générale du concept} % (fold)
\label{ssub:description_generale_du_concept}
~
\itodo{Décrire le principe et pouquoi}
\itodo{Introduire les limites et les éléments floues}
% subsubsection description_generale_du_concept (end)


% - - - - - - - - - - - - - - - - - - - - - - - - - - - - - - - - - - - - - - -
\subsubsection{La définition d’un cadre international} % (fold)
\label{ssub:la_definition_d_un_cadre_international}
~
\itodo{Décrire les travaux de l’annexe 52 et de l’IEA}
% subsubsection la_definition_d_un_cadre_international (end)


% - - - - - - - - - - - - - - - - - - - - - - - - - - - - - - - - - - - - - - -
\subsubsection{L’approche européenne} % (fold)
\label{ssub:l_approche_europeenne}
~
\itodo{Décrire les travaux du CEN}
\href{http://tinyurl.com/y7uqfe8r}{XPAIR Réseau}

% subsubsection l_approche_europeenne (end)

\subsubsection{L’évolution de la réglementation thermique en France} % (fold)
\label{ssub:l_evolution_de_la_reglementation_thermique_en_france}
~
\itodo{Décrire les nouveaux label annonciateur de la nouvelle réglementation}
\href{http://tinyurl.com/yb9u6adg}{label E+C-}
% subsubsection l_evolution_de_la_reglementation_thermique_en_france (end)
% subsection le_batiment_a_energie_positive (end)
% section vers_le_batiment_a_energie_positive (end)





% ..............................................................................
% ..............................................................................
\section{La maison solaire} % (fold)
\label{sec:la_maison_solaire}
% ------------------------------------------------------------------------------
\subsection{L’énergie solaire} % (fold)
\label{sub:l_energie_solaire}
Avant de détailler les avancées réalisées dans le domaine du solaire et plus particulièrement
dans le solaire thermique appliqué au bâtiment, il est nécessaire de décrire pourquoi
et comment cette énergie est récupéré et surtout les avantages et inconvénients.

\itodo{Radiation solaire}
L’énergie solaire provient sans surprise du soleil mais dans le cas de l’étude
de système solaire terrestre, il est important de faire la distinction entre
le rayonnement au niveau de l’atmosphère (partie supérieure) et le rayonnement
au sol dit terrestre ou au niveau de la mer. C’est ce dernier qui nous intéresse et plus particulièrement le proche
Infrarouge (IR), le visible, et le proche Ultra Violet (UV) dont la longueur
d’onde est majoritairement entre 300 à 2500nm (rayonnement terrestre).
Le soleil est sphère gazeuse dont la température effective est de \SI{5777}{\kelvin}.
On entend par température effective, la température d’un corps noir émettant la même quantité
de rayonnement électromagnétique et représente ici une approximation de la température
de surface. En effet la température intérieure est estimée entre \num{8} à \SI{40e-6}{\kelvin}
(\ref{Duffie1980}). Dans ces travaux, seul le rayonnement radiatif au niveau de la
photosphère nous intéresse.
% subsection l_energie_solaire (end)

% ------------------------------------------------------------------------------
\subsection{D’hier à aujourd’hui} % (fold)
\label{sub:d_hier_a_aujourd_hui}
~
\itodo{Décrire historique photovoltaïque et thermique, surtout thermique}
% subsection d_hier_a_aujourd_hui (end)




\itodo{Le débit passant dans le collecteur p.490}
Avant 1979, pour une installation domestique et afin de couvrir les besoins
en $ECS$, un débit de l’ordre de \SI{0.015}{kg\per(\metre\squared\period\second)}
est couramment retenue. En \mtodo{Van Koopen 1979} montre que un débit réduit
permet d’améliorer la stratification. Partant de ce constat, \mtodo{Wuestling 1985}
observe le lien entre débit, stratification, et [fraction solaire]{Ajouter définition}.
Il montre alors que lorsque la stratification est importante, l’utilisation d’un débit
réduit (\num{0.002} à \SI{0.007}{kg\per(\metre\squared\period\second)}) permet
d’obtenir une fraction solaire plus importante d’un tiers comparé à un ballon dont
l’eau est fortement mixé. Comme le souligne \mtodo{Hollands and Lightstone} en pratique
un ballon n’est jamais fortement stratifié. Il est cependant possible par son dimensionnement
d’augmenter sa stratification, en évitant par exemple que le volume de stockage soit
complètement vidé de son énergie en matinée. Ainsi en plus d’impacter positivement
la performance du système le coût de la pompe est réduit. Plus tard (\mtodo{Norton and Probert, 1986}),
cherche à évaluer le comportement des système dont le débit est créé naturellement
par le différentielle de température implémenté par la suite dans TRNSYS. Les études
suivantes (\mtodo{Tabor 1969, Gordon abd Zarmi 1981}) montre de plus que un unique
passage du fluide à travers l’echangeur du ballon amène à la même performance que
l’utilisation d’un débit plus important impliquant plusieurs passages. Comme explicité
ci-avant, la stratification permet d’améliorer la fraction solaire et autant d’énergie est
alors transmis dans les deux cas. Il est cependant nécessaire de relativiser ces
résultats car ils font l’hyphothèse forte que le puisage est nul durant cette période.



% ------------------------------------------------------------------------------
\subsection{Le solaire thermique en maison individuelle} % (fold)
\label{sub:le_solaire_thermique_en_maison_individuelle}
~
\itodo{Décrire les travaux existant sur le solaire et les avancées}
% subsection le_solaire_thermique_en_maison_individuelle (end)




% ------------------------------------------------------------------------------
\subsection{Le choix d’une approche par modélisation} % (fold)
\label{sub:le_choix_d_une_approche_par_modelisation}
\subsubsection{Les avantages de la modélisation} % (fold)
\label{ssub:les_avantages_de_la_modelisation}
~
\itodo{Comparer expérimental et modélisation}
\itodo{Pourquoi modelisation nécessaire en amont}
\itodo{Décrire les outils existants et les limites}
% subsubsection les_avantages_de_la_modelisation (end)



% - - - - - - - - - - - - - - - - - - - - - - - - - - - - - - - - - - - - - - -
\subsubsection{La maison solaire~: un problème multi-objectif} % (fold)
\label{ssub:la_maison_solaire_un_probleme_multi_objectif}
~
\itodo{Décrire le problème et les solutions auquels ces travaux cherchent à répondre.}
% subsubsection la_maison_solaire_un_probleme_multi_objectif (end)
% subsection le_choix_d_une_approche_par_modelisation (end)
% section la_maison_solaire (end)




























% \itodo{Ajouter citations, voir article BS 2017}
% \itodo{Ajouter description de l’état de l’art des algos Voir Article}
% De nombreuses études ont déjà cherché à évaluer la performance d’un système solaire
% combiné à l’aide de méthodes plus ou moins détaillées. L’approche la plus répandue utilise
% les besoins mensuels de la maison générés grâce à un modèle du bâtiment simplifié
% \parencite{Raffenel2009657,Martinopoulos2014130}. Cette approche néglige le comportement dynamique du système
% solaire comme du bâtiment. D’autres approches utilisent un modèle de bâtiment plus
% détaillé (TrnSys, Energy Plus) permettant d’évaluer plus précisément la couverture du
% système \parencite{Glembin2012601}. Ces approches permettent de tenir compte de l’évolution dynamique du système
% solaire combiné (SSC) mais négligent les interactions entre le bâtiment et les systèmes. Le
% bâtiment est seulement considéré comme une consommation (chauffage et production d’ECS) et
% le système solaire comme une source potentielle répondant à cette demande. Cette approche
% est aussi celle retenue pour la Task 26 \parencite{Task262003} au cours de laquelle de nombreux modèles
% de SSC ont été développés puis validés.

% Dans ces travaux, une approche détaillée de l’algorithme de contrôle et des systèmes est retenue. En
% effet, plusieurs études ont mis en évidence l’importance de la modélisation du contrôle
% sur la performance d’un système solaire combiné \parencite{Kicsiny20123489,Huang20123278}.
% Afin de présenter des résultats pouvant être obtenues sur des bâtiment réels pour la
% prochaine réglementation thermique (\mtodo{Ajouter citation}{RT 2020}), un algorithme existant et innovant
% a été utilisé. Celui-ci est modifié et adapté pour répondre aux contraintes du projet~: obtenir
% un bâtiment réactif en utilisant l’énergie solaire.
% L’originalité de l’approche réside principalement dans l’évaluation couplée du système
% solaire combiné et du bâtiment. Dans cette optique les interactions bâtiment / systèmes et
% systèmes / bâtiment sont pris en compte et évaluées.

% Dans un premier temps des outils retenues pour la modélisation et l’analyse des résultats.
% Dans un second temps le modèle SSC développée ainsi que le bâtiment et les scénarios, sont
% discutés. Finalement une étude paramétrique est réalisée. Elle permettra de mieux
% comprendre les interactions existantes entre bâtiment et système et de fixer les scénarios
% qui seront retenues pour l’aide à la décision. De plus les indicateurs nécessaires à la
% caractérisation d’un SSC seront identifiés.


% % ..............................................................................
% % ..............................................................................
% \section{Le solaire thermique appliqué au bâtiment} % (fold)
% \label{sec:le_solaire_thermique_applique_au_batiment}

% % ------------------------------------------------------------------------------
% \subsection{Contexte énergétique} % (fold)
% \label{sub:contexte_energetique}
% \itodo{État de l’art sur performance des bâtiments, danger réchauffement, augmentation
%        de la population}
% \itodo{Parler de ce que fait NegaWatt notament avec le projet <Europe,Territoires>.\\
%        \url{http://www.negawatt.org/telechargement/Docs/160615_Rapport-final_Europe-territoire_Phase1.pdf}\\
%        \url{http://www.negawatt.org/telechargement/Docs/160324_Synthese_Etude_Europe-territoires_Phase1.pdf}}
% % subsection contexte_énergétique (end)

% % ------------------------------------------------------------------------------
% \subsection{Le concept MEPOS} % (fold)
% \label{sub:le_concept_mepos}
% \itodo{Description des approches existantes: Passivhaus, NZEB, MEPOS}
% \itodo{Décrire l’initiative de COMEPOS pour le label MEPOS}
% \itodo{Décrire pourquoi ce choix (car il a apprit des anciens labels)}
% % subsection le_concept_mepos (end)

% % ------------------------------------------------------------------------------
% \subsection{Le solaire thermique pour une production énergétique respectueuse} % (fold)
% \label{sub:le_solaire_thermique_pour_une_production_energetique_respectueuse}
% \itodo{État de l’art sur les applications du solaire thermique.}
% \itodo{Montrer que le solaire thermique n’a pas la côte dans la construction à énergie positive.
%        Problème de coût, de confiance, ou de performance ?}
% \itodo{Montrer que l’innovation passe par le neuf avant d’être intégré à la rénovation.}
% % subsection le_solaire_thermique_pour_une_production_énergétique_respectueuse (end)
% % section le_solaire_thermique_appliqué_au_bâtiment (end)



% % ..............................................................................
% % ..............................................................................
% \section{Une approche par optimisation} % (fold)
% \label{sec:une_approche_par_optimisation}

% % ------------------------------------------------------------------------------
% \subsection{Approches explorées} % (fold)
% \label{sub:approches_explorees}
% \itodo{Décrire les différentes approches déjà explorées}
% \itodo{Augmentation surface capteur, sur-isolation, mon approche couplée pour un
%        compromis entre coût/surface capteur/isolation}
% % subsection approches_explorées (end)

% % ------------------------------------------------------------------------------
% \subsection{L’optimisation d’une maison solaire: un problème multi-critère} % (fold)
% \label{sub:l_optimisation_d_une_maison_solaire_un_probleme_multi_critere}
% \itodo{Décrire brièvement les methodes existantes}
% \itodo{Décrire les outils nécessaires (sensibilité, opimisation, aide à la décision)}
% % subsection l_optimisation_d_une_maison_solaire_un_problème_multi_critère (end)
% % section une_approche_par_optimisation (end)


% % ..............................................................................
% % ..............................................................................
% \section{Le choix d’un modèle de système solaire couplé au bâtiment} % (fold)
% \label{sec:le_choix_d_un_modele_de_systeme_solaire_couple_au_batiment}
% % ------------------------------------------------------------------------------
% \subsection{Les modèles existants} % (fold)
% \label{sub:les_modeles_existants}
% \itodo{Décrire le choix de l’approche par modélisation}
% Il existe plusieurs moyens permettant d’évaluer un système énergétique. Le premier
% consiste à reproduire expérimentalement le système et son environnement. Cependant
% ce processus est couteux , spécialement lorsque on essayer d’évaluer un système à l’échelle
% du bâtiment. On est de plus contraint par les conditions extérieures que l’on ne contrôle
% pas. Ainsi ces deux raisons font qu’il est compliqué et couteux de chercher à dimensionner
% expérimentalement un système. Pour réduire le coût de la recherche et explorer plus de
% variation il est nécessaire de pouvoir contrôler les conditions limites et de pouvoir
% itérer rapidement entre différents compositions/régulations. La modélisation entre alors
% en jeu proposant un contrôle complet des systèmes, de leur régulation, et des conditions
% limites. Le système n’étant pas physique il est alors possible de faire varier n’importe
% quel paramètre simplement afin d’évaluer son importance, son impact, ...
% La modélisation est donc l’outil de choix pour réaliser une étude de faisabilité, un
% dimensionnement, une optimisation.

% \itodo{État de l’art des modèles numériques}
% \itodo{Détail des modèles existant, contrôle, couplages}
% \itodo{Montrer que ces modèles sont très génériques et souvent très simplifiés.
%        De plus l’algorithme de contrôle est non évalué/optimisé.}
% % subsection les_modèles_existants (end)

% % ------------------------------------------------------------------------------
% \subsection{Un modèle solaire couplé au bâtiment} % (fold)
% \label{sub:un_modele_solaire_couple_au_bâtiment}
% \itodo{Décrire le choix de l’approche par modélisation}


% \itodo{Décrire les outils utilisés: Modelica et les bibliothèques, Dymola et les solveurs}
% Il existe de nombreux langages, logiciels pour réaliser des simulations plus ou moins complexes.
% La première choses à définir est donc le niveau de précision que l’on souhaite pour son modèle ou
% pour les différentes parties du modèle. Notre cas d’étude se place à l’échelle du bâtiment mais
% l’on souhaite aussi conserver un contrôle important sur la gestion des équipements.
% Ensuite il est nécessaire de déterminer le niveau d’accessibilité que l’on souhaite avoir
% sur chaque composant du système. Le modèle \textbf{boîte noire} ne pourrait pas correspondre à notre
% demande. Celui-ci ne nous offre pas le liberté de comprendre comment évolue chaque
% composants du système et nous empêche d’explorer/modifier le code.
% Pour la même raison un modèle \textbf{boîte grise} nous limite dans l’accès à certaines
% partie du code et donc à la compréhension interne du fonctionnement du système.
% Il est alors nécessaire d’utiliser un modèle \textbf{boîte blanche} garantissant
% un contrôle total sur chaque partie du système et permettant d’évaluer le comportement
% au niveau global mais aussi composant par composant.
% Nous avons ainsi opté pour le langage Modelica et la plateforme de développement
% Dymola (Dynamics Modeling Laboratory).


% \itodo{Modelica description}
% \mtodo{Ajouter référence}{Modelica} est un langage de programmation libre et ouvert développé pour répondre aux
% contraintes de la modélisation multi-physique. Il a été pensé pour être intuitif
% et offre une approche équationnelle et orienté objet au développeur.
% L’approche objet est très intuitive et permet d’encapsuler un ensemble de données
% et d’offrir des interface pour accéder à ces données. On peut alors composer de nouveaux
% objets grâce à des références vers d’autre objets (composition) ou en héritant
% du comportement d’un objet pour lui ajouter une spécialisation (héritage).
% Enfin le langage est acausal permettant d’itérer entre différentes formulation
% d’un problème facilement. Un système acausal récupère l’ensemble des variables qu’il
% connait et défini les inconnus à partir de celles-ci. L’ordre d’écriture des équations
% n’est donc plus importante et modifier un système n’oblige pas à re-écrire complètement
% les équations pour isoler la variable que l’on cherche à déterminer. Prenons l’exemple
% de la formule $U = R \times I$. Un problème causal requiert de connaître $R$ et $I$
% pour trouver $U$.L’équation doit être re-écrite si on cherche à trouver $R$ ou $I$.
% Si on utilise une approche acausal alors cette équation a une solution si on a deux des
% trois inconnus ($U et R$ ou $R et I$, ...) sans avoir à modifier l’équation. On peut ainsi utiliser
% la même formulation pour peut importe les variables connues.
% On peut ainsi voir que le développement sous Modelica permet de rapidement proto-typer
% des systèmes complexes.


% \itodo{Dymola description}
% Dymola est une suite de logiciels développée par \mtodo{Ajouter référence}{Dassault System}
% permettant d’ajouter de nombreuses fonctionnalités.
% La première étant l’interface graphique permettant de connecter différentes portion
% de code de manière plus intuitive. Il ajoute aussi un débogueur puissant permettant
% de trouver rapidement la portion de code qui pose problème, un outil pour faire du
% refactoring. Enfin il offre un outil puissant pour compiler, initialiser et intégrer
% le modèle avec un large choix d’intégrateurs avec le programme \mtodo{Ajouter référence}{Dymosim}.
% Il propose aussi du support pour la parallélisation, les FMU, et, un outil
% puissant pour faire du traitement des données durant et après les simulations. Enfin
% Dymola propose grâce à des scripts d’accéder à l’ensemble des fonctionnalités
% comme lancer une simulation, modifier un modèle, exporter un modèle, ...
% Dymola est ainsi un outil puissant pour accélérer le développement de modèle Modelica.


% \itodo{Couplage Dymola + Modelica description}
% Ces deux outils offrent alors de nombreux avantages. On contrôle chaque partie
% du système, le code est réutilisable, on peut choisir le détail de chaque partie
% du modèle, et on peut le coupler avec d’autres logiciels si besoin est.

% Le langage Modelica étant largement utilisé, de nombreuses bibliothèques open source
% on été développées dont la liste peut être trouvée sur le site officiel de l’association Modelica
% (\url{https://www.modelica.org/libraries}). Dans notre étude nous avons utilisé
% la bibliothèque \mtodo{Ajouter référence}{Buildings} qui est développé par le
% Laboratoire National Lawrence Berkeley (LBNL). C’est une bibliothèque libre et ouverte
% orienté pour le secteur du bâtiment offrant de nombreux modèles de base.


% \itodo{Pourquoi une modélisation détaillée d’un système existant}
% Le but de cette étude est d’évaluer le potentiel de couverture d’un système solaire,
% il apparaît donc important de pouvoir évaluer dans le détail son comportement au niveau
% de chaque élément. On va chercher à comprendre comment évolue la température au sein
% de la maison mais aussi des ballons, des capteurs, ...
% Il est aussi important de pouvoir modifier facilement les différents éléments composant
% le système comme les temporisations, les consignes, la taille des différents équipements, ...
% L’étude ne s’intéresse en effet pas seulement à la performance finale du système mais
% aux raisons et limitations qui ont pour conséquence ce résultat.


% \itodo{Récapituler les besoins de l’étude}
% Si on résume on a donc besoin de contrôler chaque composant pour évaluer si il se
% comporte comme on l’entends. Il est aussi nécessaire de pouvoir facilement de manière
% intuitive les sous-modèles sans affecter le modèle principal. On veut de plus pouvoir
% évaluer le système au niveau du bâtiment et il est donc nécessaire de réaliser des
% simulation sur une échelle de temps importante (de l’ordre de l’année).


% \itodo{Conclure sur le choix de Modelica et Dymola}
% Le choix du couple Modelica + Dymola est donc le résultat d’une recherche d’un outil
% répondant à nos contraintes. On veut avoir un contrôle complet de chaque composant, leur
% physique comme leur régulation et évaluer au niveau bâtiment la performance de celui-ci.
% Ces deux outils nous permet d’évaluer/modifier/comprendre efficacement le système
% sur différentes échelles tout en encourageant le processus itératif de cette étude.



% \itodo{Montrer que aujourd’hui peu de travail a été fait sur l’optimisation couplée
%        et que par conséquent ce sujet est innovant dans son approche du problème}
% \itodo{Introduire la partie suivante}
% % subsection un_modèle_solaire_couplé_au_bâtiment (end)
% % section le_choix_d_un_modèle_de_système_solaire_couplé_au_bâtiment (end)
