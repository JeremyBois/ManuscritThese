%!TEX root = ../main.tex
% Chapitres\Chap1-ContexteTravaux.tex



% ...........................................................................
% ...........................................................................
\section{L’Homme au centre d’un trouble climatologique} % (fold)
\label{sec:l_homme_au_centre_d_un_trouble_climatologique}
% ------------------------------------------------------------------------------
\subsection{Vers une prise de conscience collective} % (fold)
\label{sub:vers_une_prise_de_conscience_collective}
~
\itodo{Décrire l’histoire de la prise de conscience}
La prise de conscience de l’impact de l’humain sur son environnement intervient vers
la fin du $20ème$ siècle. Il est alors mis en exergue les nombreux problème dont les
générations futures devront gérer~: destruction de la couche d’ozone, gestion des
déchets, bouleversement climatiques, pollution de l’eau, atteintes à la bio-diversité\dots
En $1970$, le club de Rome commandite le rapport Meadows, publié en
$1972$ sous le nom \enquote{The Limits to Growth} traduit en français par
\enquote{Halte à la croissance ?} (\mtodo{Ref Rapport Meadows}).
Pour la première fois, la possibilité d’une pénurie des ressources énergétiques est
envisagé par des chercheurs du MIT à travers plusieurs scénarios plus ou moins
catastrophiques. Ce rapport fut la première pierre nécessaire à la construction d’une
prise de conscience commune et sera ensuite mis à jour à plusieurs reprises.
En $1987$ un nouveau rapport cette fois commandité par
les Nations Unies, \enquote{Our Common Future} traduit par \enquote{Notre avenir à tous}
plus connus sous le nom de rapport Bruntland (\mtodo{Ref Rapport Bruntland}). Contrairement
au rapport Meadows qui présente des scénarios de ce qui pourrait arriver si rien ne change,
ce document pose les bases pour un développement équitable mettant en avant
la protection des ressources naturelles mais aussi de l’équité face aux ressources
entre les individus. Le terme de \enquote{sustainable development} traduit en français
par \enquote{Développement durable}, aujourd’hui largement repris, est introduit~:
\blockquote{
    Un développement qui répond aux besoins des générations présentes sans
    compromettre la capacité des générations futures de répondre aux leurs.
}
\ftodo{Ajouter représentation du développement durable}

\href{http://unfccc.int/essential_background/convention/status_of_ratification/items/2631.php}{CCNUCC}\\
Au début porté par des militants, la conscience collective amène à la prise en compte de
l’environnement dans les textes réglementaires avec notamment le protocole de Montreal
($1987$) mettant en place des restrictions afin de protéger la couche d’ozone. Peu de temps
après, au 3ème Sommet de la Terre se tenant à Rio en $1992$, la Convention-cadre des Nations
unies sur les changements climatiques (CCNUCC) est adoptée et est aujourd’hui ratifiée par 197
pays montrant l’importance des question du développement durable sur le plan international.
Elle développe un plan d’action pour le $21ème$ siècle connu sous le nom d’Agenda $21$
mais aussi $27$ principes pour sa mise en œuvre. Les domaines traités sont très variés
et couvrent par exemple la pauvreté, la gestion des ressources notamment en eau, des déchets,
mais aussi la pollution ou l’agriculture...

La 3ème Conférence des Parties (COP, Conference Of Parties) à Kyoto ($1997$) entrée
en vigueur début $2005$, complète la convention par un accord visant à réduire les
émissions de gaz à effet de serre (\acro{GES}).

\href{http://www.automatesintelligents.com/echanges/2006/nov/rapportstern.html}{Stern bilan}\\
Pour la première fois, en $2006$, le rapport Stern (\mtodo{Ajouter ref Stern}) commandé par
le gouvernement anglo-saxon décrit sur le plan économique l’impact d’une inaction
sur le problème du réchauffement climatique. Précédemment porté par des scientifiques,
la sonnette d’alarme est ici tiré par un économiste mettant en avant les risques
tant sur le plan humain que économique. Le réchauffement impacte en effet directement
les composantes essentielles de notre mode de vie~: l’accès à la santé, à la nourriture,
l’eau\dots Afin de solutionner la problématique, il est suggérer de mettre en place
la coopération technique en augmentant l’effort de recherche sur le sujet mais aussi
sur la mise en place de moyens techniques mais aussi d’équilibrer les dépenses
entre les pays développés et en voie de développement. Il met ainsi en avant la nécessité
pour les pays riches qui sont les principaux responsables du réchauffement climatique
d’aider un développement respectueux de l’environnement pour les pays plus pauvres.
Une coopération internationale est ainsi indispensable afin de répondre à la problématique
de manière efficace.
Au niveau européen, la réunion de mars $2007$ entre gouvernements et chefs de l’état
de l’Union Européenne (\acro{EU}) définissent des objectifs globaux pour $2020$. Ces objectifs
sont au cœur de la directive 20/20/20 (\mtodo{Ajouter ref}) et engage l’Europe à réduire
de \SI{20}{\percent} des \acro{GES} (par rapport à $1990$), à améliorer de \SI{20}{\percent} l’efficacité énergétique et
la part des énergies renouvelables sur la consommation finale.
En France, l’engagement \enquote{facteur 4} en $2003$ et la directive européenne sont
repris dans les objectifs des lois Grenelle I et II ($2009$). L’enjeu pour la France est
alors à la fois sur le plan sociétal, politique, et économique et traduit une volonté de décisions sur le
long terme même si la question du nucléaire n’est pas encore à l’ordre du jour. Ces lois
marquent le premier pas vers une démarche responsable sur les transport, la santé, l’énergie ou encore la préservation
de la bio-diversité comme notamment avec la labellisation de l’agriculture BIO ou
la réduction de la précarité énergétiques des foyers.

\href{http://www.ecologique-solidaire.gouv.fr/loi-transition-energetique-croissance-verte}{LTECV}
\href{https://www.legifrance.gouv.fr/affichTexte.do?cidTexte=LEGITEXT000031742863&dateTexte=20170606}{LTECV}\\
En $2015$, l’adoption de la Loi pour la Transition Énergétique et la Croissance
Verte (LTECV) fixe plusieurs objectifs tant sur la consommation énergétique
que sur les émissions de \acro{GES} afin de préparer à la sortie des énergies fossiles
et créer de l’emploi pour une population en augmentation. Il y est notamment inscrit
l’objectif d’une réduction de 40\% de l’émission de \acro{GES} entre 1990 et 2030 et la
réalisation du facteur 4 pour $2050$. Sur l’énergie, une réduction de la consommation
en énergie finale et primaire pour $2030$ est aussi détaillée tant sur le plan économique
que logistique avec respectivement des objectifs de -20\% (-50\% en 2050) et de -30\%.
De plus des mesures propres à notre bouquet énergétiques sont aussi décrites comme
la réduction de la dépendance de la France à l’énergie nucléaire.

\href{https://www.edf.fr/groupe-edf/espaces-dedies/l-energie-de-a-a-z/tout-sur-l-energie/produire-de-l-electricite/le-nucleaire-en-chiffres}{EDF}\\
La mesure engage l’état à réduite à 50\% la part de production électrique d’origine nucléaire
qui est en 2014 de 77\%.


Au niveau international, la 21ème COP (2016) a permise d’aller encore plus loin avec l’acception de la mise en place
de mesures pour la réduction des émissions (\acro{GES}) avec notamment l’objectif
de contenir le réchauffement climatique en dessous des 2°C (réalisation du plan facteur 4) et ainsi permettre
d’atteindre la \enquote{neutralité carbone} en compensant les \acro{GES} dans la seconde moitié
du siècle. Contrairement au précédent accord (Accords de Kyoto), le parti-pris est ici
la transparence entre les différents signataires. En effet chaque signataire à l’obligation de rendre
compte régulièrement des objectifs et évolutions réalisées et ces informations seront
rendues publiques afin d’inciter à l’exemplarité. Bien que l’accord soit symboliquement
fort et un pas en avant vers un développement plus durable, aucunes prises de mesures
n’est cependant obligatoires. De plus la sobriété énergétique n’est pas mentionnée dans l’accord
principe pourtant clé du scénario NégaWatt \parencite{Salomon2012}.

~\\ \itodo{Décrire Négawatt}
L’association NégaWatt qui développe depuis $2003$ un scénario pour la transition énergétique
à travers 3 mesures principales (\mtodo{ref figure}) met clairement
en avant la nécessité de la sobriété énergétique, la nécessité de repenser la manière
dont nous consommons l’énergie en réduisant le gaspillage. La dernière mise à jour
du scénario ($2017$) permet d’atteindre une couverture 100\% renouvelable grâce à la
biomasse, puis l’éolien et le photovoltaïque et la fermeture en 2035 du dernier
réacteur nucléaire arrivant à plus de 40 ans d’années de fonctionnement.
\href{https://fr.wikipedia.org/wiki/Afterres2050}{Afterres2050}
Au niveau de l’agriculture et donc de l’accès à la nourriture, le scénario NégaWatt
est complété par le scénario \acro{Afterres2050} développé par l’\acro{ONG} Solagro. Celui-ci s’inspire
des avancées déjà proposées à travers les loi Grenelle et du scénario NégaWatt et montre
que le respect notamment du facteur 4 pourrait permettre de couvrir les besoins
alimentaires de l’ensemble des français en mettant en avant une fois de plus la
sobriété comme élément clé avec notamment la réduction de la consommation de viande.

\ftodo{Ajouter représentation de la transition par négaWatt}

\itodo{Parler du projet DeserTec}
\itodo{Impact à différentes échelles}
% subsection vers_une_prise_de_conscience_collective (end)



% ------------------------------------------------------------------------------
\subsection{Comprendre l’évolution} % (fold)
\label{sub:comprendre_l_evolution}

\subsubsection{Un bagage nécessaire} % (fold)
\label{ssub:un_bagage_necessaire}
~
\itodo{Décrire rapports GIEC et évolution}
Les décision politiques étant étroitement lié aux résultats scientifiques sur le
sujet, une organisation indépendante a été créé en $1988$ afin de fournir une base
scientifique solide. Cette organisation connu sous le nom de
Groupe d'experts Intergouvernemental sur l'Évolution du Climat (\acro{GIEC}), regroupe
des experts en climatologie dont le but est d’apporter un regard critique sur
des travaux provenant de différents groupes de recherche scientifiques, techniques,
et socio-économiques. Il a pour but la réalisation sans parti-pris de l’expertise
des risques liés au réchauffement climatique et de faire re-sortir les informations
et découvertes faisant consensus dans la communauté scientifique. Finalement, grâce à ces
publications périodiques, les connaissances accumulées sur le sujet permettent de
sensibiliser le public non expert, élément indispensable pour que des engagements
politiques soit pris. Ainsi, il est nécessaire dans un premier temps d’expliquer les principaux termes
nécessaire à la compréhension de la situation climatique avant d’en présenter les chiffres
clés.

\href{https://fr.wikipedia.org/wiki/%C3%89nergie_(physique)}{Énergie}
L’énergie qui tient une place centrale dans la problématique est ici utilisé non dans
son sens morale mais dans son sens physique. L’énergie en physique est la mesure
de la capacité d’un système à produire de la chaleur, modifier un état, ou encore
entraîner un mouvement. Dans notre problématique il est principalement considéré
deux types d’énergie~: l’énergie dite primaire et l’énergie dite finale.
L’énergie primaire correspond à l’énergie prélevée à l’environnement alors que l’énergie
finale est l’énergie après soustraction des pertes de stockage, de transport, et de
transformation. L’énergie finale correspond ainsi à l’énergie consommée par l’utilisateur
finale et ne tient ainsi pas compte des moyens utilisés pour sa production.

\ttodo{Ajouter les facteurs de conversion en France}

D’un point de vue environnemental, l’augmentation des émissions de GES et l’élévation
de la température de la planète représente une menace importante
Le premier facteur fait intervenir la notion d’effet de serre, un phénomène naturel
engendré par la présence de divers composés gazeux dans l’atmosphère.
L’atmosphère ayant une faible réflexion et absorption pour le rayonnement solaire
(visible, proche \acro{UV} et \acro{IR}), un grande partie arrive à la surface de la terre. Une
partie est alors absorbée alors qu’une autre est renvoyé vers l’espace. Dû à cet
échauffement, le sol émet un rayonnement dans le domaine de l’IR lointain. Certain
composant gazeux étant faiblement transparents à ce domaine d’émission, une partie
de l’énergie est absorbée au lieu d’être rejeté dans l’espace~: c’est le mécanisme
de l’effet de serre.
La température moyenne sur l’ensemble de la surface du globe est ainsi le résultat
du bilan radiatif entre le rayonnement solaire incident et le rayonnement infrarouge.
N’ayant pas d’influence sur la vapeur d’eau présente dans l’atmosphère
(55\% de la contribution à l’effet de serre), le $CO_{2}$ joue alors un rôle essentiel
dans l’étude de l’impact de l’homme et des conséquences à moyen et long terme.

\ftodo{Pourcentage de la contribution des émissions anthropiques}

% Sans l’impact de l’être humain, le bilan permet à la terre d’atteindre une température
% moyenne de \SI{15}{\celsius} au lieu de \SI{-18}{\celsius}.


% subsubsection un_bagage_necessaire (end)

\subsubsection{Évolution des facteurs environnementaux} % (fold)
\label{ssub:evolution_des_facteurs_environnementaux}
~
\itodo{Évolution de la température et des énergies}
% International Energy Agency (IEA), Energy Technologies Perspectives:
% Scenarios and Strategies to 2050,OECD/IEA, 2008, Paris.
Actuellement les mesures montre un forçage radiatif excédentaire entraînant une
lente augmentation de la température du sol. En $2003$, le GIEC estimait ce forçage
à \SI{2.3}{\watt\per\metre\squared} dont \SI{1.6}{\watt\per\metre\squared} imputable
aux émissions anthropiques. Bien que nous soyons en période de réchauffement
(causes externes astronomiques), l’activité humaine semble être responsable d’une
augmentation anormalement rapide.
\itodo{Ajouter résultats GIEC plus récent sur GES, voir *GEA 2012*}


\itodo{Mettre en avant l’accélération des consommations / GES / population}
\ftodo{Ajouter évolution CO2 selon GIEC}
Cet accroissement est lié à deux grand facteurs. Premièrement, l’augmentation de la robotisation
et le développement très rapide depuis les années $90$ de l’industrialisation à fortement
augmenter la consommation d’énergie fossiles. Deuxièmement, l’accroissement très important
de la population en quelques années entraînant une explosion de la demande énergétique
(\mtodo{GEA 2012}). Cette énergie est aujourd’hui a \SI{80}{\percent} produite grâce à des sources
carbonées, le pétrole, le charbon. Il est ainsi clair que des mesures doivent être
prises afin de limiter la consommation en énergie primaire. De plus une alternative
respectueuse de l’environnement, les énergies renouvelables, réduisant les émissions de
GES est indispensable pour permettre d’inverser la tendance.

\itodo{Le coût des énergies renouvelables}
\href{http://decrypterlenergie.org/les-energies-renouvelables-coutent-elles-trop-cher}{coût ENR}\\
C’est dans cette optique que les mesures décrites ci-dessus ont été prise, mais
il demeure encore aujourd’hui une

% subsubsection evolution_des_facteurs_environnementaux (end)
% subsection comprendre_l_evolution (end)
% section l_homme_au_centre_d_un_trouble_climatologique (end)





% ..............................................................................
% ..............................................................................
\section{Vers le bâtiment à énergie positive} % (fold)
\label{sec:vers_le_batiment_a_energie_positive}
% ------------------------------------------------------------------------------
\subsection{Un long chemin déjà parcouru} % (fold)
\label{sub:un_long_chemin_deja_parcouru}
~
\itodo{Réglementation jusqu’à \acro{RT\,$2005$} pour introduire BBC}
À la suite de la seconde guerre mondiale, durant la période des 30 glorieuses, la France
et d’autres pays développés profitent d’une croissance forte tant au niveau économique,
que démographique (baby-boom). Cette expansion est portée par un accès aisé aux énergies
et particulièrement aux énergies fossiles. L’abondance de ressources énergétique traduit
alors un développement industriel important et dans le bâtiment par des
constructions énergivores et une perte du savoir faire ancien. C’est suite au premier choc
pétrolier de $1973$ (\mtodo{ref}), que la France met en place en 1974 la première
Réglementation Thermique (RT) afin de faire face aux dépenses énergétiques importantes
dans le secteur du bâtiment. En effet l’énergie qui hier semblait inépuisable apparaît
alors comme une ressource limitée et la sécurité de l’approvisionnement devient une
préoccupation nationale. D’après l’ADEME, la consommation annuelle en énergie finale est
alors de l’ordre de \SI{370}{kWh\per\metre\squared} dans le secteur du résidentiel. La
réglementation fixe alors un objectif de réduction de la consommation de
\SI{25}{\percent}. Une méthodologie de calcul commune est alors établie et
un contrôle régulier des système de chauffage comme de climatisation imposée.
Cette réglementation évolue ensuite en $1982$, $1989$, $2000$, et $2006$ avec la \acro{RT\,2005} porté par
les différents acteurs du secteur du bâtiment. Cette dernière marque de nombreuses avancées
réglementaire avec par exemple l’ajout d’un bâtiment de référence, la construction bioclimatique
avec l’intégration des énergies renouvelables dans la méthode de calcul ou encore l’évaluation
du confort d’été.
Aujourd’hui la \acro{RT\,$2012$} en vigueur impose
des contraintes fortes sur la qualité énergétique des bâtiment au niveau de l’enveloppe
avec le \acro{$B_{bio}$} et sur les consommations avec le \acro{$C_{ep}$\,max}. Cette forte accélération
est le résultat de la prise de conscience par la scène politique et économique de l’impact de
l’homme sur le climat.

\itodo{n ZEB et impact bâtiment}
Représentant plus de \SI{40}{\percent} de la consommation en énergie
finale en Europe (\mtodo{Ajouter ref}), le secteur du bâtiment a un rôle déterminant à
jouer dans le respect des engagements politiques. Dans une approche volontariste,
les états européens adoptent ainsi en $2002$, la directive sur la performance énergétique des bâtiments
(\acro{2002/91/EC} \mtodo{Ajouter ref}) qui fut à l’origine
d’avancées significatives pour tous les états membres avec notamment la mise en place
des diagnostics de performance énergétique (\acro{DPE}), des valeurs seuils et une méthodologie commune
pour évaluer la performance énergétique des bâtiments, le contrôle régulier des chaudières (\SI{+10}{kW})
et un effort de normalisation commune par le Comité Européen de Normalisation (\acro{CEN}).
Cette directive a été mis à jour en $2010$ (\acro{2010/31/EU} \textcite{EPBD2010}) et introduit le concept
de \enquote{Nearly Zero Energy Buildings} (\textit{nearly}\,\acro{ZEB}) incitant chaque état membre à mettre
en œuvre une politique permettant d’atteindre pour tous les bâtiments neufs une consommation
énergétique en énergie primaire \enquote{quasi nulle} en $2018$ pour les bâtiments
publics et en $2020$ pour le privé (article $2$ et $9$).
Dans l’optique proposée de vision globale
ces engagements ont été traduit par la plupart des pays membres dans leur réglementation
thermique propre comme la France à travers les disposition des lois Grenelle traduit dans
la \acro{RT\,$2005$} puis la \acro{RT\,$2012$}.
Ces mesures incitatrices sont aussi à l’origine de la publication dans le journal officiel
en mai $2007$ de l’arrêté encadrant l’obtention d’un label \acro{HPE} qui permet de valoriser
un bâtiment obtenant un niveau de performance globale supérieur à la réglementation en
vigueur. Le label comporte cinq niveaux dont le plus performant, le label Bâtiment à
Basse Consommation (\href{https://www.effinergie.org/web/index.php/les-labels-effinergie/bbc-effinergie}{\acro{BBC-$2005$}})
est inspiré du label Suisse Minergie. Pour la première fois en France, un
label d’excellence énergétique est proposé imposant une consommation en énergie primaire
sur le chauffage, la production d’\acro{ECS}, le refroidissement, et l’éclairage à
\SI{50}{kWh\per\metre\squared} (modulé en fonction des conditions géographiques).
Utilisé comme un ban d’essai par les maîtres d’ouvrages, le label à permis au professionnels
de s’adapter aux dispositions du label qui seront repris de manière similaire par la
\acro{RT\,$2012$}. En effet le passage à la \acro{RT\,$2012$} impose aux bâtiments un \acro{$C_{ep}$\,max} réduit
de moitié en plus de devoir tenir compte de l’éclairage et des auxiliaires alors que
dans la \acro{RT\,$2005$}, le \acro{$C_{ep}$\,max} est définie uniquement pour le chauffage, la production
d’\acro{ECS}, et le refroidissement.
Les travaux se concentre aujourd’hui sur la préparation de la \acro{RT\,2020}, qui devra
permettre d’honorer les engagements européens et se rapprocher des objectifs internationaux
pour $2050$.


\ftodo{Émission des GES par secteur en France}

\iunsure{Parler des label environnementaux}
% subsection un_long_chemin_deja_parcouru (end)



% ------------------------------------------------------------------------------
\subsection{Description générale du concept} % (fold)
\label{sub:description_generale_du_concept}
~
\itodo{Décrire le principe et pourquoi}
\itodo{Introduire les limites et les éléments floues}
\href{Further development of EU energy efficiency policies}{http://tinyurl.com/yafjydu7}


Bien qu’il n’existe pas actuellement de cadre réglementaire en France,
la directive européenne \acro{2010/31/EU} \parencite{EPBD2010} décrit un cadre commun pour les pays
de l’Union Européenne. Elle transcrit la volonté des États membres à construire pour $2020$
uniquement des bâtiments consommant \enquote{quasiment rien} (\textit{nearly}\,\acro{ZEB}) comme explicité ci-avant.
À partir de cette définition générale, chaque pays de l’\acro{UE} doit ainsi développer sa
propre réglementation. La France a pris l’initiative de développer des Bâtiment à
Énergie POSitives (\acro{BEPOS}) dans le cadre de sa propre réglementation. Communément,
ce terme indique que le bâtiment produit plus qu’il ne consomme, définition
auquel la directive ajoute les points suivants~:
\begin{itemize}
    \item Les bilans énergétiques doivent être réalisés en énergie primaire
    \item La consommation doit être couverte par des sources d’énergies renouvelables
    \item La production doit être réalisé localement
\end{itemize}
Bien que le concept soit mieux définis, il reste fortement interprétable,
sans fournir d’indications concrètes permettant d’atteindre cet objectif.

Pour cette raison, de nombreuses définition ont été développées au sein du globe,
si bien que la commission européenne a mandaté une expertise permettant de classer ou
tout du moins d’identifier les différences \parencite{ECOFYS2013}.
En effet, chaque pays porte ces propres
convictions, sa propre culture du bâtiment et de la construction, sa propre politique,
son propre mix énergétique, ses propres contraintes d’accès à l’énergie, et bien sûr
son climat.
Au final le rapport suggère l’élaboration de près de $75$ définitions différentes
et soulève de nombreux questionnement sur la bonne manière de procéder.
Il est alors nécessaire de confronter ces approches à la réalité du domaine du bâtiment
afin de pouvoir proposer des solutions techniques et technologiques innovantes mais
atteignable pour $2020$. Dans cette optique, il est nécessaire de s’appuyer sur les
les retours d’expériences et l’analyse de bâtiment performant. Comme
l’a été le label \acro{BBC\,$2005$} en France pour la \acro{RT\,$2005$}, de nombreux projets pilotes
induit par la mise en place de labels voient le jour.
% subsection description_generale_du_concept (end)



% ------------------------------------------------------------------------------
\subsection{Le cas Français} % (fold)
\label{sub:le_cas_francais}
% - - - - - - - - - - - - - - - - - - - - - - - - - - - - - - - - - - - - - - -
\subsubsection{Les labels français} % (fold)
\label{ssub:les_labels_francais}
\itodo{Maisons test, validation des mesures, validation de la faisabilité}
En France, le principe appliqué est celui du gagnant gagnant. Afin d’encourager
la construction de maisons performantes des aides sont proposées, aux futurs propriétaires
dans le privée, et aux collectivités dans le public (\mtodo{décret du $28/06/16$ et arrêté du $12/10/16$}).
La conduite des travaux permet en retour aux entreprises de valoriser un savoir faire, et
les retours techniques, technologiques, économiques, et logistiques résultant amènent
le secteur du bâtiment à se réorganiser tout en aidant à l’écriture comme la mise en place
de la future réglementation, la \acro{RT\,2020}.

\href{http://www.certivea.fr/offres/label-effinergie-2017}{\textit{Effinergie} 2017}

L’association \textit{Éffinergie} propose ainsi en $2013$, deux nouveaux labels~:
\href{https://www.effinergie.org/web/index.php/les-labels-effinergie/le-label-effinergie-plus}{\textit{Effinergie +}}
et \href{https://www.effinergie.org/web/index.php/les-labels-effinergie/bepos-effinergie}{\acro{BEPOS}-\textit{Éffinergie}}
permettant de valoriser une performance énergétique du bâtiment
supérieure au cadre réglementaire. Dans les deux cas la valeur maximale du \acro{$B_{bio}$},
du \acro{$C_{ep}$\,max}, de l’étanchéité à l’air ou de l’efficacité des équipements sont renforcées.
Il est aussi nécessaire d’évaluer les consommations dues aux équipements internes
et de faire un suivi des consommations du bâtiment à l’utilisateur.
Le label \acro{BEPOS}-\textit{Éffinergie} va plus loin que le label \textit{Éffinergie\,+} en rendant
obligatoire l’évaluation en énergie grise sur le bâtiment mais aussi ces occupants
(\href{https://www.effinergie.org/web/index.php/effinergie-ecomobilite}{Éco-mobilité}).
Finalement la production locale doit permettre de compenser la consommation amenant à
un bilan en énergie primaire positif (modulé suivant les règles de la \acro{RT\,$2012$}
pour tenir compte de la diversité).


\itodo{Décrire E+C-}
Plus récemment, en novembre $2016$, l’état ouvre la période d’expérimentation des \acro{BEPOS} à
bas carbone avec le programme Objectifs Bâtiments Énergie Carbone (\acro{OBEC}) sur la base d’un
nouveau référentiel~: le label \acro{E+/C-} \parencite{Ministere2016}. Ce nouveau label permet de classer suivant
respectivement $4$ et $2$ niveaux la performance énergétique et le taux d’émission en $CO_{2}$
des bâtiments (un niveau plus important indique une performance plus importante). L’objectif
cible est la préparation de la \acro{RT\,$2012$} à travers trois sous-objectifs~:
\begin{itemize}
    \item sensibiliser les filières du bâtiment
    \item améliorer la compétence des acteurs du bâtiment
    \item alimenter les bases de données économiques, énergétiques, et environnementales
\end{itemize}
Au niveau énergétique (partie \acro{E+}), l’obtention des deux premiers niveaux implique l’amélioration
du bâti et des systèmes. Le niveau trois nécessite en plus l’utilisation d’un mode
de production locale en $EnR$ pour compenser la consommation du bâtiment. Enfin le dernier
niveau indique que le bâtiment a un bilan négatif ou nul et qu’il contribue à la
production en $EnR$ du quartier. Pour la partie \acro{C-} le respect du label nécessite
de ne pas dépasser une valeur seuil sur respectivement le cycle de vie du bâtiment
(\acro{$Eges_{max}$}) et plus spécifiquement pour les matériaux de construction et équipements (\acro{$Eges_{PCE, max}$}).

\itodo{BEPOS 2017}
Basé sur ce nouveau label, l’association \textit{Éffinergie} propose, trois nouveaux labels~:
\href{https://www.effinergie.org/web/index.php/les-labels-effinergie/le-label-bbc-effinergie-2017}{\acro{BBC} \textit{Éffinergie} 2017},
\href{https://www.effinergie.org/web/index.php/les-labels-effinergie/le-label-bepos-bepos-effinergie-2017}{\acro{BEPOS} \textit{Éffinergie} 2017},
\href{https://www.effinergie.org/web/index.php/les-labels-effinergie/le-label-bepos-bepos-effinergie-2017}{\acro{BEPOS}\,+ \textit{Éffinergie} 2017}
qui impliquent respectivement le respect à minima d’un niveau 2, 3, et 4 sur le critère \acro{E+} et
de 1 sur le critère \acro{C-}. Ces labels ajoute en plus, la prise en compte du confort des occupants, et la sobriété
et l’efficacité énergétique. De par leur exigences, ces nouveaux labels s’inscrivent dans une
démarche plus globale en tenant compte de la performance énergétique mais aussi de l’émission
des \acro{GES}, du confort des occupants et de la sobriété qui est pour rappel l’élément clé
du scénario $NégaWatt$.

\itodo{Label BBCA}
Concernant la partie \acro{C-} ces labels n’exigent cependant pas une performance supplémentaires
et se contentent même du premier niveau. C’est un autre label, le label bas carbone de l’association
\href{http://www.certivea.fr/offres/label-bbca-batiment-bas-carbone}{\acro{BBCA}} qui permet de
certifier son bâtiment comme étant très performant sur les émissions de \acro{GES}
dont le niveau 2 sur le critère \acro{C-} du label $E+C-$ est un pré-requis. L’évaluation
se fait suivant quatre grandes étapes et un système de point permet d’évaluer le niveau
du bâtiment d’où découle un classement qualitatif~: \acro{BBCA}, \acro{BBCA} Performance, et \acro{BBCA} Excellence.
L’indicateur évalue à travers ce processus les réductions de \acro{GES} (construction et utilisation),
l’économie circulaire, et l’utilisation de matériaux bio-sourcés.

\itodo{Bilan et ajouter ref}
La France a ainsi été très active tant au niveau réglementaire que incitatif et propose
aujourd’hui des certifications permettant à un propriétaire ou un maître d’ouvrage
de valoriser un choix réfléchie tant sur le plan énergétique, environnemental, que
social ou sociétale. Les données recueillies permettent de plus de préparer la future réglementation
qui est l’élément clé d’un respect des engagements sur le climat. Finalement l’expérience
acquise à travers ces expérimentations permettra de mieux définir les contraintes
économiques et techniques, condition nécessaire pour appliquer au niveau national les nouvelles
directives.
% subsubsection les_labels_francais (end)


% - - - - - - - - - - - - - - - - - - - - - - - - - - - - - - - - - - - - - - -
\subsubsection{Les labels étrangers décernés en France} % (fold)
\label{ssub:les_labels_etrangers_decernes_en_france}

\iunsure{Décrire les autres approches}
Minergie, Passivhaus
% subsubsection les_labels_etrangers_decernes_en_france (end)
% subsection le_cas_francais (end)




% ------------------------------------------------------------------------------
\subsection{La complexité de la définition d’un cadre international} % (fold)
\label{sub:la_definition_d_un_cadre_international}
\itodo{Décrire les travaux de l’annexe 52 et de l’IEA}
Au niveau international la \acro{BEPOS} connu sous l’acronyme \acro{ZEB} a fait l’objet d’un
travail important et novateur.
Le concept est né en réponse aux problématiques climatiques et la nécessite
de repenser la manière de construire les bâtiments qui représente aux États Unies
\SI{40}{\percent} de la consommation en énergie primaire et plus de \SI{70}{\percent}
de la consommation électrique \parencite{Torcellini2006a}. Il
restait alors à définir un concept commun et fournir les éléments aidant à sa mise en
place. En effet même si l’idée fait consensus, il restait encore à définir les indicateurs
considérés ou encore l’adéquation entre besoins et demande.
Il est aussi nécessaire de définir les frontières, comme par exemple le terme \enquote{locale},
la période considérée pour son évaluation, ou les flux considérés. Ces questions sont traitées
dans le cadre de la tâche 40 de l’\acro{IEA} (\enquote{Net Zero Energy Solar Buildings}) et de l’annexe $52$ de
l’\acro{ECBCS} (\enquote{Towards Net Zero Energy Solar Buildings}) portée par l’Agence Internationale
de l’Énergie (\acro{AIE}). Les travaux se focalisent
sur la définition d’une \acro{BEPOS} connectée (\acro{Net\,ZEB}, Net Zero Energy Building) où
la production renouvelable alimentant le réseau permet d’équilibrer les consommations du bâtiment.
À l’horizon 2011, les nombreux travaux et maisons de démonstration permettent d’alimenter
les différentes bases de données et de nombreuses méthodes de calcul sont identifiées
\parencite{Marszal2011971}.


% - - - - - - - - - - - - - - - - - - - - - - - - - - - - - - - - - - - - - - -
\subsubsection{Le choix des indicateurs} % (fold)
\label{ssub:le_choix_des_indicateurs}
Afin de pouvoir évaluer le caractère positif d’un bâtiment, il est dans un premier
temps nécessaire de définir un système d’unité, un indicateur suivant lequel
s’appuyer. \textcite{Torcellini2006} propose quatre indicateurs afin de convenir
à plusieurs profils~:
\begin{itemize}
    \item \enquote{Net Zero Site Energy}~: La production renouvelable sur site
          doit pouvoir compenser la consommation finale du bâtiment.
    \item \enquote{Net Zero Source Energy}~: La production renouvelable transmise au réseau
          doit permettre de compenser la consommation primaire du bâtiment.
    \item \enquote{Net Zero Energy Costs}~: Le gain d’argent engendré par la vente
           de la production locale compense au minimum la consommation acheté.
    \item \enquote{Net Zero Energy Emissions}~: La part renouvelable doit être assez
           importante pour couvrir les émissions engendrés par la consommation non
           renouvelable tel que les énergies fossiles.
\end{itemize}
L’auteur met aussi en avant les avantages et limites de chaque approches. Les
deux première approches considèrent l’énergie comme indicateur principale à la différence
près que la seconde tient compte de l’énergie totale nécessaire, l’énergie primaire, pour fournir l’énergie consommée
au bâtiment, l’énergie finale, considérée dans la première approche. Le choix
de l’énergie primaire est plus international et le calcul selon les émissions
est plus complexe car il faut définir quels critères retenir parmi les nombreux critères
disponibles~: émissions de \acro{GES}, énergie grise, déchets radioactifs\dots
L’utilisation de l’énergie finale ou du coût est lui plus simple à mettre en place
car elle ne tient pas compte des facteurs de conversions propres aux bouquets énergétiques de chaque pays.
L’approche utilisant l’énergie primaire ressort de la comparaison comme étant l’indicateur le
plus utilisé mais certaines approches cumulent les indicateurs comme le traduit en France les différents
labels décrits ci-avant.


% - - - - - - - - - - - - - - - - - - - - - - - - - - - - - - - - - - - - - - -
\subsubsection{La méthodologie de calcul} % (fold)
\label{ssub:la_methodologie_de_calcul}
Comme il a été vu, le bilan du bâtiment est réalisé afin d’obtenir à minima un équilibre
entre l’énergie exportée vers le réseau (production) et l’énergie importée par le
bâtiment (consommation).
Les coefficient utilisés pour réaliser ce bilan sont souvent considérés
symétriques car il est considéré que l’énergie exportée vers le réseau représente
autant d’énergie non renouvelable évité. Cette affirmation n’est pourtant valide
uniquement lorsque l’énergie exportée ne nécessite pas la consommation d’énergie
supplémentaire comme le souligne \textcite{Sartori2012220}. En effet l’ajout par exemple d’un système
photovoltaïque entraine une consommation d’énergie grise pour son installation et
son entretien. L’auteur décrit une approche asymétrique par pondération et explique que
sa mise en place résulte d’une volonté politico-économique. Par exemple afin
de favoriser l’adoption des énergies nouvelles, l’énergie produite sur site
est revendu plus chère que celle achetée.
L’approche la plus courante cherche à évaluer le bilan énergétique sur une fréquence annuelle
même si certaines approches calculent les indicateurs sur la durée de vie complète
du bâtiment afin de tenir compte de son vrai impact environnemental.
\textcite{Voss201146} met par exemple en exergue l’importance de la part des énergies grises
qui tient entre \SI{20}{\percent} et \SI{30}{\percent} pour une durée de vie du bâtiment
considéré à \SI{80}{ans}.

\textcite{Sartori2012220} propose trois type de bilan (\figref{fig:bilan_zeb})~:
\begin{enumerate}
    \item charge / production (\enquote{load generation balance})
    \item importation / exportation (\enquote{import/export balance})
    \item mensuel (\enquote{monthly net balance})
\end{enumerate}
Le bilan charge / production est définie sur une base annuelle
et tient compte des interactions entre le bâtiment et le réseau mais nécessite pour
ce faire un pas de simulation faible afin d’estimer l’autoconsommation.
Le choix d’un pas de temps plus réduit induit l’estimation plus précise du comportement
des occupants à travers les consommations électro-domestiques, le puisage en \acro{ECS}\dots
Dans la seconde approche, équilibre importation / exportation, les énergies respectives
de chaque coté du bilan sont dans un premier temps pondérées et sommées avec leurs coefficients
respectifs. Ainsi il est donc considéré que l’ensemble de la production et exportée vers le réseau
et à l’inverse que l’ensemble de la consommation est fournie par le réseau. La seconde approche est ainsi plus
simple à mettre en place, expliquant son adoption par un plus grand nombre de méthodologies.
La première approche peut cependant être retenue expérimentalement sur les maisons pilotes
et donner de précieuses informations. Finalement le dernier bilan proposé est mensuel et
permet de mieux évaluer l’équilibre ou déséquilibre existant entre la production et la
consommation et soulève la question de l’adéquation entre production et consommation
qui est discuté après avoir établis un cadre clair des frontières du bilan.
\begin{figure}
    \centering
    \ftodo{Sartori type de bilan}
    % \includegraphics{Ressources/Images/Modelisation/composant_vs_bloc.png}
    \caption{Représentation schématiques de trois méthodes utilisées pour faire
             le bilan d’un \acro{Net\,ZEB} selon \textcite{Sartori2012220}.}
    \label{fig:bilan_zeb}
\end{figure}
% subsubsection la_methodologie_de_calcul (end)

\paragraph{Les données météorologiques} % (fold)
\label{par:les_donnees_meteorologiques}
\textcite{Sartori2010} met aussi en avant la question des données météorologiques
et en particulier l’utilisation de fichier typiques construits sur la connaissance
du passé. \textcite{Robert2012150} montre qu’il est possible grâce à une technique de morphing
\parencite{Belcher200549} de générer des fichiers typiques ou des séries d’années. Ces
données météorologiques tiennent alors compte de la connaissance acquise sur l’évolution
climatique et raison pour laquelle l’auteur suggère leur utilisation.
% paragraph les_donnees_meteorologiques (end)
% subsubsection le_choix_des_indicateurs (end)


% - - - - - - - - - - - - - - - - - - - - - - - - - - - - - - - - - - - - - - -
\subsubsection{Les frontières considérées} % (fold)
\label{ssub:les_frontieres_considerees}
\paragraph{Frontières physiques} % (fold)
\label{par:frontières_physiques}
La définition d’une \acro{ZEB} implique un bilan positif grâce à une production dite
\enquote{locale} dont le périmètre physique reste à définir.
\textcite{Torcellini2006} propose pour la définition d’une \acro{ZEB}
de favoriser dans un premier temps les solutions permettant de réduire les consommations
tels que l’isolation ou l’efficacité des systèmes. Dans un second temps, les solutions favorisant
la production d’énergie renouvelable locale (\enquote{On-Site}), et enfin
l’utilisation de sources extérieures (\enquote{Off-Site}). Au niveau du site, une
distinction est fait entre la production \emph{sur} le bâtiment et \emph{au} niveau de la
parcelle détenue par le propriétaire avec une préférence pour le premier.
Au niveau des sources extérieures, l’utilisation des site extérieurs afin de se
fournir en bio-énergies est favorisée. Le recours au réseau apparaît comme la dernière option.
\textcite{Marszal2010} propose une représentation graphique permettant de clairement
identifié le périmètre physique sans pour autant mettre en avant un quelconque
classement de préférence. Il ajoute aussi une distinction supplémentaire au niveau
des ressources extérieures au site en séparant l’investissement du propriétaire dans
un site extérieure pour produire de l’énergie renouvelable et le recours au réseau.
% paragraph frontières_physiques (end)

\paragraph{Les usages~:} % (fold)
\label{par:les_usages}
La définition d’une \acro{BEPOS} met ainsi clairement en avant l’utilisation d’énergies
renouvelables afin de compenser les consommations sur site. Cette approche exclue
cependant les systèmes de cogénération qui sont pourtant des solutions qui permettent
de maximiser l’utilisation des énergies non renouvelables \parencite{Sartori2010}.
\textcite{Marszal2011971} met aussi en exergue la diversité observée sur les usages considérés pour le calcul
des indicateurs. Les approches les plus anciennes ne considèrent que les principaux usages~: chauffage
et production d’\acro{ECS} alors que d’autres plus récentes tiennent aussi compte de l’éclairage,
du refroidissement, et des auxiliaires. La diminution des usages courants met en effet
sur le devant de la scène les autres usages auparavant négligé. D’autres approches
associe aussi au bilan les occupants à travers les consommations électro-domestiques.
La prise en compte de ces nouveaux usages nécessite d’estimer les consommations des occupants
(éclairage, électro-domestique) mais aussi de tenir compte d’un niveau de confort (climatisation).
\itodo{Compléter avec Sartori 2012}
% paragraph les_usages (end)
% subsubsection les_frontieres_considerees (end)


% - - - - - - - - - - - - - - - - - - - - - - - - - - - - - - - - - - - - - - -
\subsubsection{Adéquation temporelle avec le réseau~:} % (fold)
\label{ssub:adequation_temporelle_avec_le_réseau}
\itodo{Stress réseau et données météo}
Un \acro{Net\,ZEB} est part définition connecté au réseau. De plus il a été mis en évidence
que l’obtention d’un bilan positif au niveau annuel ne permet pas d’assurer la
symétrie des échanges sur toute l’année. Plus la fréquence du bilan positif est importante,
plus l’asymétrie entre production et consommation est visible. En effet, la production
d’énergie renouvelable est en majorité assurée par une production photovoltaïque
dont les pics de production en période estivale ne sont pas en adéquation avec la forte demande
énergétique en période hivernale. Le difficulté réside ainsi à gérer les pointes
de consommations mais aussi les excédents de production.
Cette mutation dans le bouquet énergétique amorcé par l’arrivée de nouvelles énergies comme
l’éolien ou le photovoltaïque a imposé le développement des \enquote{Smart Grid}
ou réseaux intelligents. Cette nouvelle infrastructure doit ainsi permettre de répondre
au problèmes d’intermittence des principales sources d’énergies renouvelables actuellement
utilisées pour générer de l’électricité. Des indicateurs ont alors été développés
afin d’évaluer l’adéquation entre bâtiment et réseau.
\textcite{Voss2010} fait la différence entre les interactions au niveau du bâtiment
(\enquote{load maching}) ou entre le bâtiment et le réseau (\enquote{grid interaction}).
La première famille d’indicateurs permet d’évaluer le niveau de concomitance entre la
demande du bâtiment et production sur site. Le second quand à lui permet d’évaluer
la correspondance temporelle entre exportation et besoins au niveau du réseau.
Une référence complète des différents indicateurs trouvées dans la littérature est
proposé par \textcite{Salom2011} puis intégrés dans les travaux de annexe 52 \parencite{Salom2014}.
% subsubsection adequation_temporelle_avec_le_réseau (end)


% - - - - - - - - - - - - - - - - - - - - - - - - - - - - - - - - - - - - - - -
\subsubsection{Mesures et vérifications~:} % (fold)
\label{ssub:mesures_et_verifications}
\itodo{Sartori 2012}
Dans le cadre de la sous tâche \acro{A} \parencite{Noris2013}, le monitoring apparaît comme indispensable
afin de vérifier par la mesure la performance du bâtiment en la confrontant aux
résultats de simulations. Il est recommandé à minima de vérifier le respect du bilan
positif qui est le concept central des \acro{BEPOS}. Bien que ne faisant pas parti de
la définition du concept, une \acro{Net\,ZEB} doit avant tout être agréable à vivre. Dans cette
optique le confort intérieur des occupants doit être analysé à travers une étude de
la Qualité de l’Air Intérieur (\acro{QAI}) permettant de s’assurer que la performance du bâtiment n’est pas
obtenue en sacrifiant le bien être des occupants. Un \acro{Net\,ZEB} étant par définition connecté,
l’adéquation du bâtiment avec le réseau fait aussi partis des principales recommandations.
Finalement, le rapport présente une procédure standardisé mettant en avant différentes échelles
de suivie du bâtiment ainsi qu’un outil d’analyse facilitant la comparaison des divers projets entre eux.
% subsubsection mesures_et_verifications (end)
% subsection la_definition_d_un_cadre_international (end)


% ------------------------------------------------------------------------------
\subsection{L’approche européenne} % (fold)
\label{ssub:l_approche_europeenne}
~
% énergie primaire retenue
\itodo{Décrire les travaux du CEN}
\href{http://tinyurl.com/y7uqfe8r}{XPAIR Réseau}
Au niveau européen, les exigences pour $2020$ sont décrite à travers la directive
\acro{2010/31/EU} \parencite{EPBD2010} qui esquisse la direction générale à prendre pour
parvenir aux \textit{nearly}\,\acro{ZEB}. Afin de faciliter l’application de la directive, la commission
européenne a mandaté le Comité Européen de Normalisation (\acro{CEN})
pour la rédaction d’un cadre normatif~: \acro{prEN\,15603} \mtodo{Ajouter ref}. Ces travaux s’inscrivent
dans dans un projet de normalisation européenne pour le développement d’une définition
commune, flexible, et claire de la \textit{nearly}\,\acro{ZEB} afin d’inciter les professionnels au développement
de solutions innovantes et adaptées. De plus le cadre de la définition doit être
suffisamment générique afin de permettre sa traduction au niveau national par les différents
états membres. Ainsi une description précise et détaillée est retenue pour le choix des indicateurs,
des frontières considérées, ainsi que sur l’adéquation temporelle avec le réseau \parencite{Zirngibl2014}.


% - - - - - - - - - - - - - - - - - - - - - - - - - - - - - - - - - - - - - - -
\subsubsection{Les frontière considérées} % (fold)
\label{ssub:les_frontière_considérées}
La directive décrit que la part de la production considérée doit être sur site (\enquote{on-site})
ou à proximité (\enquote{nearby}). Les travaux du \acro{CEN} ajoute une description plus précise
du périmètre en explicitant ces deux termes. La production sur site considère uniquement
le terrain où est localisé le bâtiment, alors que la production à proximité intègre
les sources locales jusqu’à l’échelle du quartier permettant de mutualiser les productions
en énergies renouvelables.
\iunsure{Revoir traduction}
La condition requise étant d’avoir une connexion dédiée et un équipement spécifique
caractérisant ce lien. Des coefficient de conversion en énergie primaire seront
alors calculés afin de l’expliciter sur le plan énergétique.​
% subsubsection les_frontière_considérées (end)



% - - - - - - - - - - - - - - - - - - - - - - - - - - - - - - - - - - - - - - -
\subsubsection{La méthodologie retenue} % (fold)
\label{ssub:la_methodologie_retenue}
Un \textit{nearly}\,\acro{ZEB} est un bâtiment dont les consommations sur les cinq usages
(chauffage, \acro{ECS}, éclairage, auxiliaires, et refroidissement)
sont très faibles et doivent être couverts par une production locale en énergie renouvelable.
Le \acro{CEN} propose une méthodologie sur quatre axes permettant de couvrir de manière
précise la définition point par point.

Le premier axe permet l’évaluation des besoins à travers l’analyse
de la qualité de l’enveloppe et du partitionnement du bâtiment~: isolation, inertie, qualité de l’air,
et conception bioclimatique. L’approche est ainsi similaire au calcul du \acro{$B_{bio}$} décrit dans
la \acro{RT\,$2012$} présenté dans la section précédente.

Le second axe permet d’évaluer la performance du bâtiment et de ses systèmes à travers les
cinq usages. Un total en énergie primaire est retenu afin de tenir compte de l’ensemble
des pertes dues à l’acheminement, la transformation, et le stockage de manière équitable
pour toutes les énergies. À cette étape les

Le troisième axe est définie afin d’évaluer la proportion d’énergie renouvelable sur site en séparant
le calcul pour chaque vecteur énergétique. L’indicateur proposé est le Renewable Energy Ratio
(\acro{RER}) définie comme le rapport entre la consommation en énergie primaire et la consommation
en énergie primaire non renouvelable. Chaque usage étant séparé, la production des
capteurs \acro{PV} par exemple ne peut donc pas compenser le chauffage couvert par une chaudière
au gaz.

Finalement le dernier axe vérifie la performance globale du bâtiment en réalisant
le bilan entre les énergies importées et exportées. Le bilan étant réalisé en énergie
primaire, la consommation finale de chaque vecteur énergétique est pondérée par un
coefficient.

La méthodologie décrit ainsi de manière précise les attentes sur les besoins, les
consommations, la part des énergies renouvelables, et sur la balance énergétique globale
illustrée à travers la \figref{fig:attribution_nZEB}.
Finalement, le non-respect d’un des indicateurs respectifs de chaque axes disqualifie le bâtiment
qui ne peut donc pas accéder au titre de \textit{nearly}\,\acro{ZEB}.

\begin{figure}
    \centering
    \ftodo{Les étapes décries par le CEN}
    % \includegraphics{Ressources/Images/Modelisation/composant_vs_bloc.png}
    \caption{Description des quatre axes amenant à l’attribution du titre de
             \textit{nearly}\,\acro{ZEB} d’après \textcite{Zirngibl2014}.}
    \label{fig:attribution_nZEB}
\end{figure}
% subsubsection la_methodologie_retenue (end)


% - - - - - - - - - - - - - - - - - - - - - - - - - - - - - - - - - - - - - - -
\subsubsection{Adéquation temporelle avec le réseau} % (fold)
\label{ssub:adequation_temporelle_avec_le_reseau}

% subsubsection adequation_temporelle_avec_le_reseau (end)
% subsection l_approche_europeenne (end)



% ------------------------------------------------------------------------------
\subsection{Bilan sur la maison à énergie positive} % (fold)
\label{sub:bilan_sur_la_BEPOS}
\itodo{Ajouter récapitulatif}
% subsection bilan_sur_la_BEPOS (end)
% section vers_le_batiment_a_energie_positive (end)





% ..............................................................................
% ..............................................................................
\section{La maison solaire à énergie positive} % (fold)
\label{sec:la_maison_solaire_a_energie_positive}
% irradiation / ensoleillement  / insolation = énergie / m2
% irradiance / rayonnement = densité de flux énergétique == densité de la puissance rayonnée = puissance / m2
% ------------------------------------------------------------------------------
\subsection{Du soleil à la Terre} % (fold)
\label{sub:du_soleil_a_la_terre}
Avant de détailler les avancées réalisées dans le domaine du solaire et plus particulièrement
dans le solaire thermique appliqué au bâtiment, il est nécessaire de décrire pourquoi
et comment cette énergie est récupéré et surtout les avantages et inconvénients.

% - - - - - - - - - - - - - - - - - - - - - - - - - - - - - - - - - - - - - - -
\subsubsection{L’énergie solaire} % (fold)
\label{ssub:l_energie_solaire}
\itodo{Radiation solaire}
L’énergie solaire provient sans surprise du soleil qui est sphère gazeuse
dont  la température intérieure est estimée entre \num{8} à \SI{40e-6}{\kelvin}
\parencite{Duffie1980}. Dans ces travaux, seul le rayonnement radiatif au niveau de la
photosphère nous intéresse dont la température effective est de \SI{5777}{\kelvin}. On
entend par température effective, la température d’un corps noir émettant la même quantité
de rayonnement électromagnétique et représente ici une approximation de la température de
surface du soleil.
Le soleil se trouve à \SI{1.495e11}{\metre} et la densité de la
puissance rayonnée reçue au niveau de la surface extérieure de l’atmosphère terrestre, la
constante solaire ($G_{cs}$) estimée pour la première fois à \num{1228} par \mtodo{Claude Pouillet, 1838}.
Aujourd’hui grâce au satellites équipés de radiomètres, elle est estimée
à \SI{1360.8 +- 0.5}{\watt\per\metre\squared} \parencite{Kopp2011}. Après
pénétration dans l’atmosphère une partie est absorbée, dévié, ou reflété et on distingue
la part directe ($G_{dir}$) de la part diffuse ($G_{dif}$.) L’irradiance directe est la part de la puissance
rayonnée non dispersé au passage dans l’atmosphère et est souvent exprimée pour une surface perpendiculaire
a la direction de propagation ($G_{dir,\,nor}$). L’irradiance diffuse est au contraire
déviée lors de son passage dans l’atmosphère et est souvent exprimée à l’horizontale
($G_{dif}$), la direction de propagation n’étant plus unique. Finalement, la somme des
parts directes et diffuses forment l’irradiance globale et est le plus souvent exprimé à
l’horizontale (\acro{G}). Dans l’optique de l’évaluation d’un système solaire, il est
aussi courant de décrire l’énergie solaire reçue par unité de surface~: l’irradiation.
Lorsque uniquement le spectre solaire est considéré,
, soit les longueurs d’ondes allant de \num{0.3} à \SI{3}{\micro\metre}, les termes
d’ensoleillement ou d’insolation (\acro{I}) sont plus courants.
Dans le reste du document, sauf indication explicite, l’ensoleillement (\acro{I}) et le rayonnement (\acro{G})
sont globaux et pour une surface horizontale.

\itodo{Le soleil dans l’espace et le temps}
\itodo{$\beta \alpha \omega$ sont pris pour le vol de Lévy.}
La Terre tournant autour du soleil, le temps standard (UTC, Temps Universel Coordonné)
et plus particulièrement le temps solaire ou Temps Universel (UT1) intervient dans
toute les relations permettant de calculer la position du soleil.
En plus de tourner autour du soleil, la terre tourne sur elle-même et n’est pas
parfaitement circulaire. Il est ainsi nécessaire afin de déterminer la position
du soleil d’introduire un certain nombre de paramètres~:
\begin{itemize}
    \item Le jour de l’année, compris entre 1 et 365 inclus
    \item La déclinaison ($\delta$)~: angle entre la droite reliant la terre et le soleil
          (centre) et le plan équatorial variant entre \num{-23.45} et \SI{23.45}{\degree}
          (positive au Nord)
    \item Latitude ($\phi$)~: position Nord-Sud d’un point par rapport à l’équateur
    \item La longitude (\acro{L})~: position Est-Ouest d’un point par rapport au méridien de Greenwich.
    \item L’angle horaire ($\omega$)~: L’écart par rapport au méridien local, négatif le
          matin et positif en après-midi.
    \item L’angle d’incidence ($\theta$)~: écart angulaire entre le segment reliant le soleil
          à une surface et la normale à cette même surface.
    \item L’angle zénithal ($\theta_{sol}$)~: L’angle d’incidence pour une surface horizontale
    \item L’altitude solaire ($\alpha_{sol}$)~: écart angulaire entre une surface horizontale et le segment reliant
          le soleil à cette même surface. C’est l’angle complémentaire
          de l’angle zénithal.
    \item Azimut solaire ($\gamma_{sol}$)~: Écart angulaire entre la projection du segment reliant
          le soleil et une surface et le Sud (positif dans le sens horaire)
    \item Azimut de la surface ($\gamma$)~: Déviation par rapport au Sud de la projection
          de la surface sur l’horizontale ($\SI{-180}{\degree} \leq \gamma \leq \SI{180}{\degree}$). Pour une surface orienté Sud~: $\gamma=0$.
    \item Inclinaison de la surface ($\beta$)~: Angle formé entre la surface et l’horizontale.
          Si $\beta > \SI{90}{\degree}$ alors la surface est dos au soleil.
    \item L’angle de profil ($\alpha_{p}$)~: angle résultant de la projection de $\alpha_{sol}$ sur un plan perpendiculaire
          à la surface considérée.
\end{itemize}
\ftodo{Ajouter schéma avec les angles}
\iunsure{Décrire les équations utilisés pour les approximer}
\mtodo{Cooper 1969 et Spencer 1971} proposent diverses équations permettant d’approximer
les différents angles solaires ($\omega$, $\theta$, $\gamma_{sol}$, $\delta$, $\alpha_{sol}$).
Ainsi afin de calculer la position du soleil il est alors uniquement nécessaire de connaître la latitude, la longitude,
l’orientation de la surface ($\gamma$) ainsi que l’heure locale. Ces équations simplifiées
couplées à la géométrie des bâtiment permettent alors de calculer le masque créé par les obstructions
tel qu’une avancée de toiture, un arbre, une montagn\dots grâce des relations trigonométriques simples.
Couplée au données d’ensoleillement estimées il est de plus possible d’apprécier l’ensoleillement
sur une surface inclinée~: un capteur solaire thermique, un vitrage...
Il est aussi possible par exemple grâce à \href{http://pysolar.org/}{Pysolar} de calculer
de manière très \href{http://docs.pysolar.org/en/latest/#validation}{précise} la
position du soleil pour n’importe quel position du globe.
% subsubsection l_energie_solaire (end)


% - - - - - - - - - - - - - - - - - - - - - - - - - - - - - - - - - - - - - - -
\subsubsection{Les données météorologiques} % (fold)
\label{ssub:les_donnees_meteorologiques}
\itodo{Mesurer l’irradiation}
Maintenant qu’il est possible d’estimer la position du soleil et que le vocabulaire
décrivant l’énergie solaire est introduit, il est temps de comprendre comment
le rayonnement terrestre est mesurée. Plusieurs instruments existent, les deux principaux
étant le pyranomètre et le pyrhéliomètre qui font tout les deux partie de la famille
des radiomètres car ils permettent de mesurer l’intensité du flux solaire.
Le pyrhéliomètre est inventé en 1825 par Herschel \parencite{Kutz2013} même si l’invention
ne prendra ce nom que avec la construction du pyrhéliomètre de Pouillet en 1837 \parencite{Boer1985}.
Il mesure le $G_{dir,\,nor}$ grâce à un cône de visibilité très
restreint même si une faible part du rayonnement issue du ciel est aussi captée.
Les rayons solaires pénètre dans un collimateur afin d’être dirigé vers une cavité.
Une thermopile est alors utilisée afin de transformer l’énergie thermique en énergie électrique.
Il peut être noté que pour fonctionner correctement, l’instrument doit toujours être orienté
en direction du soleil et donc monté sur un traqueur solaire.
Le premier pyranomètre sphérique est quand à lui inventé en $1836$ par Belloni \parencite{Boer1985}.
Contrairement au pyrhéliomètre, il permet de mesurer le rayonnement hémisphérique total
comprenant direct et diffus à l’horizontal (\acro{G}) et ne demande pas de dispositif de
suivi du soleil. Le rayonnement est aussi calculé grâce à une thermopile mais un
dispositif basé sur des cellules photovoltaïques existent aussi bien que moins précis
à cause d’une réponse non uniforme sur le spectre solaire. Grâce à un anneau de protection,
il est aussi possible de supprimer la part directe ($G_{dir,\,hor}$) limitant la mesure
à la part diffuse ($G_{dif,\,hor}$). Ainsi en couplant les résultats de ces deux instruments
au équations permettant le calcul de l’angle solaire, il est d’évaluer précisément le
rayonnement solaire sur une surface inclinée.


\itodo{Les fichiers météos~: différence entre réelles et typiques}
Afin d’estimer la performance d’un système solaire il est nécessaire d’avoir des
données d’ensoleillement. La première approche consiste à utiliser des données réelles
du passé. Ces données peuvent alors être disponibles à plusieurs fréquences et
une distinction est nécessaire entre les données récupérées par des stations météos et les
données récupérer à partir de données satellites. Avec les données issues de stations météo
le rayonnement mesurée est directement le rayonnement
terrestre et comme il a été vu, il est possible de capturer le direct,
le diffus, ou bien le rayonnement global. Néanmoins, l’installation d’instruments de mesures est nécessaire et
la position géographique retenue doit être suffisamment proche du site étudié.
En France, ces données peuvent être obtenues à partir des stations météos de
Météo France ou bien plus récemment avec l’organisme WhiteBoxTechnologie \mtodo{Ajouter ref}
Les données satellites permettent de récupérer les irradiations à n’importe quel
point du globe et tiennent compte des spécificité du lieu comme le dénivelé. Cependant
le rayonnement au sol issue des données satellites et calculé à partir d’images couplées
à un algorithme de conversion dépend du type de fichier météo.
\iunsure{Décrire les différents types existant}
La SOlar radiation DAta (\acro{SODA}) est la principale source de donnée regroupant les
services de 4 pays~: la France, les États-Unis, la Suisse, et l’Italie.

La seconde approche consiste à utiliser des méthodes statistiques sur un échantillon
importante de données ($\pm 30$) afin de créer un fichier météo dit typique. Ce fichier
météo est alors constituer de différents blocs (souvent des mois entiers) de données
sélectionnées sur l’ensemble de l’échantillon.

\itodo{Ajouter description faite sur rapport IGC}

Il est aujourd’hui couramment utilisé des fichiers météos
ayant un pas de temps horaire mais certains travaux montre des progrès dans l’estimation
d’un profil réaliste à partir de données mensuels et/ou des méthodes statistiques
(\mtodo{Ref mois vers heure}).
% subsubsection les_donnees_meteorologiques (end)
% subsection du_soleil_a_la_terre (end)



% ------------------------------------------------------------------------------
\subsection{Valoriser l’énergie solaire} % (fold)
\label{sub:valoriser_l_energie_solaire}
Nous savons maintenant d’où vient l’énergie solaire et comment elle est mesurée
au niveau du sol. Il a de plus été vu que l’énergie solaire après son passage dans
l’atmosphère doit être considéré suivant une part directe et une part diffuse.
Cette partie s’intéresse ainsi à décrire les moyens existants permettant de récupérer
cette énergie ou du moins une partie et plus particulièrement les applications
dites de solaire thermique qui est au cœur du sujet de ces travaux. Comme il a été
vu l’énergie solaire est utilisé depuis très longtemps afin de fournir soit de l’électricité,
soit de l’énergie sous forme de chaleur.
Le système le plus courant dans le domaine du bâtiment est appelé panneau solaire, ou capteur solaire.
On distingue trois grandes familles de capteurs solaires~:
\begin{itemize}
    \item Le capteur plan
    \item Le capteur sous-vide
    \item Le capteur à concentration
\end{itemize}


% - - - - - - - - - - - - - - - - - - - - - - - - - - - - - - - - - - - - - - -
\subsubsection{Le capteur plan} % (fold)
\label{ssub:le_capteur_plan}
C’est le système le plus ancien et aujourd’hui encore le plus économique. Parmi les
capteurs plan, une distinction peut être faite entre les capteurs sans vitrages
et les capteurs vitrées.

\paragraph{Le capteur non-vitré~:} % (fold)
\label{par:le_capteur_non_vitre}
Un capteur solaire sans vitrage est composé d’un tube en plastique noir laissant
passer un fluide caloporteur. Le rayonnement solaire incident est en partie absorbée
par le fluide par conduction. L’énergie restante est perdue soit par convection soit
par rayonnement (\figref{fig:schema_capteur_plan}, gauche). C’est le système le plus simple
et le plus économique mais les pertes importantes par convection font que ce système
n’est pas très performant. N’ayant pas de protection, ni d’isolation il est en
effet très sensible au vent et à la température extérieure.
Pour ces raisons, il est principalement utilisé pour
des opérations annexes tel que le chauffage d’une piscine en été.
% paragraph le_capteur_non_vitre (end)

\paragraph{Le capteur vitré~:} % (fold)
\label{par:le_capteur_vitre}
L’invention du premier capteur vitré \mtodo{Trouver ref} a permis de réduire de manière
importante les pertes thermiques par convection et par rayonnement infrarouge.
De manière schématique, un capteur vitrée est composé d’une surface transparente, d’un absorbeur,
de tubes, et d’un isolant thermique (\figref{fig:schema_capteur_plan}, droite).

\begin{figure}
    \centering
    \ftodo{Capteur plan non et vitrés}
    % \includegraphics{Ressources/Images/Modelisation/composant_vs_bloc.png}
    \caption{Description schématique d’un capteur plan, non vitré (gauche), et vitrée (droite)
             d’après.}
    \label{fig:schema_capteur_plan}
\end{figure}

Un vitrage est utilisé pour la surface transparente afin de laisser passer la
majorité du rayonnement solaire (visible et Proche InfraRouge (\acro{PIR})) grâce à ces propriétés
optiques.
L’absorbeur comme son nom l’indique doit absorber au maximum l’énergie solaire et
émettre le moins possible dans l’infrarouge. En effet d’après la loi du déplacement
de Wien, à température ambiante les objets émettent spontanément dans l’InfraRouge
Moyen (\acro{MIR}) entre \SIrange{3}{50}{\micro\metre}.
L’absorbeur est ainsi de couleur noire (\SI{\leq 95}{\percent}) afin d’absorber un maximum dans le visible
et subit un traitement chimique (chrome ou de céramique à base d’oxydes
métalliques) afin d’obtenir une surface faiblement émissive (\SI{\leq 12}{\percent}).
Le vitrage peut lui aussi être faiblement émissif afin de laisser passer le rayonnement
solaire mais réfléchir le \acro{MIR}. La combinaison des deux permet alors de récupérer
la majorité du spectre solaire au niveau de l’absorbeur tout en limitant les
pertes grâce aux inter-réflexions. Les tubes contiennent le fluide caloporteur
et le cuivre est le matériaux le plus utilisé car il est simple à travailler et
à une conductivité thermique importante ($\lambda = \SI{390}{\watt\per(\metre\period\kelvin)}$).
Finalement l’isolant permet de réduire les pertes par conduction en face arrière et
sur les cotés.
% paragraph le_capteur_vitre (end)
% subsubsection le_capteur_plan (end)


% - - - - - - - - - - - - - - - - - - - - - - - - - - - - - - - - - - - - - - -
\subsubsection{Le capteur sous-vide} % (fold)
\label{ssub:le_capteur_sous_vide}
À la différence du capteur plan, le capteur sous-vide réduit les déperditions
au niveau de l’absorbeur en remplaçant l’air par du vide supprimant les échanges
convectifs. On distingue quatre types~: le capteur à circulation directe,
le capteur à caloduc, le capteur à effet \enquote{Thermos}, et le capteur à
réflecteur intégré. Pour le capteur à circulation direct, l’absorbeur est une structure
plan à ailettes et la circulation de l’eau est assurée par un système de tube en \acro{U}.
L’absorbeur de chaque tube est contrairement aux autres modèles, orientable permettant
de les utiliser en façades. Le capteur à caloduc utilise le principe de la vaporisation
de l’eau déminéralisé dans le caloduc lui-même à l’intérieur du tube. L’eau en se vaporisant
remonte jusqu’à la tête du caloduc et s’y condense au contact du fluide solaire.
Le capteur à effet \enquote{Thermos} est lui composé de deux tubes concentriques,
le tube intérieur étant l’absorbeur. Le vide est réalisé entre les deux tubes et
l’échange de chaleur se fait soit par un système de caloduc, soit directement par
circulation du fluide dans le tube intérieur. Afin d’améliorer la part solaire reçu,
un réflecteur peut être ajouté soit sur la paroi intérieur du tube extérieur (\mtodo{ref Schott-Rohglas})
soit à l’extérieur (\mtodo{ref \acro{CPC}}).

\begin{figure}
    \centering
    \ftodo{Image capteurs vitrés sous vide}
    % \includegraphics{Ressources/Images/Modelisation/composant_vs_bloc.png}
    \caption{Exemple de capteurs solaires thermique avec~: (a), (b), (c), (d)}
    \label{fig:image_panneaux_solaires}
\end{figure}
% subsubsection le_capteur_sous_vide (end)


% - - - - - - - - - - - - - - - - - - - - - - - - - - - - - - - - - - - - - - -
\subsubsection{Le capteur à concentration} % (fold)
\label{ssub:le_capteur_a_concentration}
\iunsure{Ajouter des exemples}
Ces capteurs cherchent à concentrer en un point focal le rayonnement direct
grâce à un système de de réflecteur ou de lentille. Grâce à l’énergie concentré
la température atteinte sur ce point est comprise entre \SIrange{400}{1500}{\celsius}
en fonction des systèmes. Contrairement aux capteurs précédent, les hautes températures
obtenues permettent de fabriquer de l’électricité soit par conversion directe, soit par
détente. Dans cette famille, il est considéré, les capteurs
cylindro-paraboliques, les capteurs paraboliques, les centrales à tour, et les
fours solaires.
Les capteurs cylindro-paraboliques utilise des réflecteur cylindrique orientable
sur un axe (pour suivre le soleil) afin de concentrer le rayonnement direct sur un
foyer linéaire ou passe un fluide (souvent de l’huile). La vapeur produite est envoyé
vers une turbine pour fabriquer de l’électricité par détente.
Les capteurs paraboliques utilise un système de suivi du soleil sur deux axes
et un réflecteur en forme de paraboloïde de révolution (\mtodo{Trouver ref}).
L’électricité est produite par conversion directe via un moteur Stirling.
Ayant aussi un système de suivi sur deux axes, la centrale à tour peut suivre parfaitement
le soleil afin de capter un maximum du rayonnement direct. Il comporte un miroir
plan ou légèrement concave et permet de monter jusqu’à \SI{1500}{\celsius}. L’électricité
est fabriqué soit à l’aide d’une turbine et d’un fluide intermédiaire, soit par
production directe. Finalement le dernier dispositif est le four solaire
composé d’un miroir concave concentrant le rayonnement en son foyer. Deux applications
peuvent être notées. La première et la production d’électricité
comme par exemple avec le four solaire français mondialement connu
d’\href{http://www.promes.cnrs.fr/index.php?page=historique}{Odeillo}
qui permet d’atteindre une puissance d’\SI{1}{\mega\watt}. Il utilise un ensemble
de miroirs orientables (héliostats) dirigeant les rayons vers un miroir concave géant
concentrant les rayons vers le foyer. La deuxième application et la cuisson avec
des appareils plus ou moins rudimentaires.
\itodo{Ajouter exemple d’applications concentrique pas dans le bâtiment}
% subsubsection le_capteur_a_concentration (end)
% subsection valoriser_l_energie_solaire (end)


% ------------------------------------------------------------------------------
\subsection{D’hier à aujourd’hui} % (fold)
\label{sub:d_hier_a_aujourd_hui}
% - - - - - - - - - - - - - - - - - - - - - - - - - - - - - - - - - - - - - - -
\subsubsection{Les chiffres clés} % (fold)
\label{ssub:les_chiffres_cles}
\itodo{Évolution du domaine dans le bâtiment avec des chiffres}
\itodo{**Mauthner2016** et **Weiss2017** pour l’évolution du solaire.}

\itodo{Tendance}
L’\acro{IEA} actualise tout les ans son rapport sur l’évolution mondiale du solaire thermique
dans le cadre de son programme \enquote{Solar Heating and Cooling} (\acro{SHC}). La dernière
édition en date estiment que la couverture des données représente \SI{95}{\percent} de
marché mondial \parencite{Weiss2017}. Ainsi fin 2016, près de \SI{456}{\giga\watt_{th}}
sont installés et une évolution de \SI{7.4}{\percent} est enregistré depuis le début du
siècle. Cette évolution est portée par un fort développement des structures industrielles
($\geq \SI{350}{\kilo\watt_{th}}$ ou $\geq \SI{500}{\metre\squared}$). La plus grande
installation à ce jour étant au Danemark avec près de \SI{110}{\mega\watt_{th}} pour
\SI{156694}{\metre\squared} de capteurs plans représentant \SI{30}{\percent} de la surface
installée courant $2016$. Historiquement, la puissance maximale double tout les ans depuis
2013~:
\begin{description}[align=left]
    \item [2013] Marstal au Danemark avec \SI{\sim 20000}{\metre\squared}
    \item [2014] Chili avec \SI{\sim 39000}{\metre\squared}
    \item [2015] Vojens au Danemark avec \SI{\sim 70000}{\metre\squared}
    \item [2016] Silkeborg  au Danemark avec \SI{\sim 156000}{\metre\squared}
\end{description}
Parmi les $66$ pays analysés, le marché est majoritairement situé en Chine
(\SI{309.5}{\giga\watt_{th}}) et plus modestement en l’Europe (\SI{49.2}{\giga\watt_{th}})
qui à eux deux représentent \SI{82.3}{\percent} de la puissance thermique installée.
Ramené au nombre d’habitant, il peut être observé que le marché se développe plutôt bien
en Europe avec l’Autriche (\SI{421}{\kilo\watt_{th}\per(1000 hab.)}), la Grèce
(\SI{287}{\kilo\watt_{th}\per(1000 hab.)}), ou Allemagne
(\SI{164}{\kilo\watt_{th}\per(1000 hab.)}) contre \SI{226}{\kilo\watt_{th}\per(1000 hab.)}
pour la Chine. En France, le solaire thermique n’est que peu développé avec seulement
\SI{22.03}{\kilo\watt_{th}\per(1000 hab.)} (hors \acro{DOM}) alors que l’ensoleillement
(\SI{1112}{\kilo\watt\hour\per\squared\metre}) est en moyenne plus important que
pour l’Allemagne (\SI{1091}{\kilo\watt\hour\per\squared\metre}).
Le solaire a ainsi permis en $2016$ d’éviter au niveau mondial, l’émission de
\SI{130}{\mega\tonne} de $CO_{2}$ et \SI{40.3}{\mega\tonne} de produits pétroliers et
respectivement \SI{362}{\tonne} et \SI{112}{\mega\tonne} en France.

Malgré certains progrès, la surface installée décroit de \SI{14}{\percent} en
$2015$ par rapport à $2014$ et la tendance semble continuer en $2016$. Au niveau
économique, le marché stagne en Europe et est décroissant dans le reste du monde à
l’exception de l’Afrique Sub-Saharienne et de l’Asie (Chine non-inclus).
L’énergie solaire thermique assume donc une perte d’intérêt depuis le début du siècle
(\figref{fig:tendances_enr}). La tendance enregistrée par le solaire thermique est ainsi
depuis 2010 en décroissance nette en comparaison avec l’éolien et le photovoltaïque. De
plus, l’éolien produit ainsi plus que le solaire thermique à partir de $2016$ et bien que
le photovoltaïque soit encore derrière, il profite d’une tendance à la hausse depuis
$2014$.

\begin{figure}
    \centering
    \ftodo{Puissance en fonction du type d’énergie p.14 \parencite{Weiss2017}}
    % \includegraphics{Ressources/Images/Modelisation/composant_vs_bloc.png}
    \caption{Évolution tendancielle et en puissance installée cumulée de la répartition
             du solaire thermique, de l’éolien, et du photovoltaïque d’après
             \textcite{Weiss2017}}
    \label{fig:tendances_enr}
\end{figure}
\href{http://www.sunwindenergy.com/content/solar-process-heat-surprisingly-popular}
L’association Allemande (\textsf{BSW-Solar}, German Solar Industry Association)
a publié une carte des installations industrielles mondiales et met en avant les
principaux facteurs responsables du développement de ce secteur \parencite{Augsten2017}.
Le premier facteur est le prix des énergies fossiles, suivit par les considérations
politiques des états imposant une part plus importante d’énergie renouvelable. Arrivée
en second en $2013$, les aides financières sont relégués à la troisième place avec
cette nouvelle étude en $2016$ (\mtodo{Ajouter ref}). Les industrielles
mettent aussi en avant le scepticisme des consommateurs vis à vis du solaire thermique
alors que \SI{70}{\percent} des groupes interrogés estiment que le marché est déjà
compétitif. Le marché entre de plus en compétition avec les installations \acro{PV} pour
lesquels les industrielles s’accordent majoritairement sur leur compétitivité.

\iunsure{Ajouter évolution Européenne du solaire}

\itodo{Applications}
Au niveau mondial $108$ millions de systèmes sont référencés en $2015$, dont $nicefrac{3}{4}$
sont des systèmes à thermosiphon.
Au niveau des applications, la production d’\acro{ECS} est prioritaire avec \SI{63}{\percent}
dans des maisons individuelles et \SI{28}{\percent} sur des bâtiments plus importants comme
les hôpitaux ou les écoles. La tendance actuelle est cependant inverse avec une forte
décroissance en $2015$ de la part de la production d’\acro{ECS} en maison individuelles
et une forte croissance dans les autres bâtiments. Ainsi les dernières données montre que
la répartition est respectivement de \SI{41}{\percent} et \SI{51}{\percent} en maison
individuelle et pour les autres bâtiments.
Les \acro{SSC} restent très minoritaire avec seulement \SI{2}{\percent}.
Finalement \SI{6}{\percent} sont utilisés pour le chauffage des piscines et que les \SI{1}{\percent}
restants alimentent des réseaux de chaleur, des processus industriels, ou encore des systèmes de
refroidissement.
Au niveau européen le marché est principalement tourné vers les capteurs plans avec \SI{72.3}{\percent}
des parts et les systèmes comportent pour \SI{61}{\percent} des pompes.
Ces résultats s’explique par l’impact très important de la Chine sur le bilan mondial faisant
des capteurs sous vide, la technologie la plus utilisée.
La part des système de production d’\acro{ECS} en maison individuelle est la même que au niveau mondial,
cependant les grandes installations ne représente plus que \SI{12}{\percent}. Alors que
dans la plupart des autres pays du monde, le \acro{SSC} représente une part infime, il
représente en Europe \SI{19}{\percent} des systèmes installés même si la tendance est à
la baisse~: \SI{17}{\percent} en $2014$ \parencite{Mauthner2016} et \SI{16}{\percent} en
$2015$ \parencite{Weiss2017}.

\itodo{Conclure sur l’évolution}
Il a été présenté dans un premier temps les types de capteurs existants puis les
principaux  systèmes actuellement en opération afin de de produire soit de l’énergie
thermique, soit de l’énergie électrique. Les applications sont nombreuses mais subissent
une tendance très négative en comparaison avec les autres énergies renouvelables.
Le secteur du solaire thermique est ainsi au niveau mondial en décroissance forte
et stagne sur le continent Européen. Bien que des applications de \acro{SSC} existent
au niveau Européen, la tendance est aussi à la baisse. Ces tendances traduisent
une perte de confiance dans cette technologie déjà observé par les industrielles.
% subsubsection les_chiffres_cles (end)


% - - - - - - - - - - - - - - - - - - - - - - - - - - - - - - - - - - - - - - -
\subsubsection{Évolution du cadre normatif} % (fold)
\label{ssub:evolution_du_cadre_normatif}
\iunsure{Utile ou pas vu que je reprends un peu dans le second avec capteurs}
% subsubsection evolution_du_cadre_normatif (end)


% - - - - - - - - - - - - - - - - - - - - - - - - - - - - - - - - - - - - - - -
\subsubsection{Les dispositifs existants} % (fold)
\label{ssub:les_dispositifs_existants}
\itodo{Les dispositifs existants}
Non compatible avec des installations traditionnelles de chauffage ou de production
d’\acro{ECS} les capteurs à concentration sont en majorité utilisés pour des applications
industrielles comme décrit ci-avant. Dans le bâtiment des systèmes de cogénération
se développent cependant en couplant un moteur à vapeur avec des capteurs à concentration
cylindro-paraboliques \mtodo{Ajouter ref}.

\itodo{Intégration au bâti, ...}
% subsubsection les_dispositifs_existants (end)
% subsection d_hier_a_aujourd_hui (end)



% ------------------------------------------------------------------------------
\subsection{Les avancées dans le domaine du solaire thermique} % (fold)
\label{sub:les_avancées_dans_le_domaine_du_solaire_thermique}
\itodo{Décrire le problème et les solutions auquels ces travaux cherchent à répondre.}
\itodo{Les travaux / technologies existants}
% Solar house design : From solar building design to Net Zero Energy Buildings performance insights of an office building (pdf)
% PV vs Thermal : Economic and energy analysis of three solar assisted heat pump systems in near zero energy buildings
% Detailed energy simulations of a net-zero energy triplex in Montreal
% Domestic hot water production in a net zero energy triplex in Montreal

\itodo{Ce qui est en cours}
\href{http://www.iea-shc.org/tasks-current}{IEA}\\
\href{http://task53.iea-shc.org/publications}{Task 53}\\
\href{http://task54.iea-shc.org/publications}{Task 54}\\

\itodo{Le débit passant dans le collecteur p.490}
Avant 1979, pour une installation domestique et afin de couvrir les besoins
en \acro{ECS}, un débit de l’ordre de \SI{0.015}{kg\per(\metre\squared\period\second)}
est couramment retenue. En \mtodo{Van Koopen 1979} montre que un débit réduit
permet d’améliorer la stratification. Partant de ce constat, \mtodo{Wuestling 1985}
observe le lien entre débit, stratification, et [fraction solaire]{Ajouter définition}.
Il montre alors que lorsque la stratification est importante, l’utilisation d’un débit
réduit (\num{0.002} à \SI{0.007}{kg\per(\metre\squared\period\second)}) permet
d’obtenir une fraction solaire plus importante d’un tiers comparé à un ballon dont
l’eau est fortement mixé. Comme le souligne \mtodo{Hollands and Lightstone} en pratique
un ballon n’est jamais fortement stratifié. Il est cependant possible par son dimensionnement
d’augmenter sa stratification, en évitant par exemple que le volume de stockage soit
complètement vidé de son énergie en matinée. Ainsi en plus d’impacter positivement
la performance du système le coût de la pompe est réduit. Plus tard (\mtodo{Norton and Probert, 1986}),
cherche à évaluer le comportement des système dont le débit est créé naturellement
par le différentielle de température implémenté par la suite dans TRNSYS. Les études
suivantes (\mtodo{Tabor 1969, Gordon abd Zarmi 1981}) montre de plus que un unique
passage du fluide à travers l’échangeur du ballon amène à la même performance que
l’utilisation d’un débit plus important impliquant plusieurs passages. Comme explicité
ci-avant, la stratification permet d’améliorer la fraction solaire et autant d’énergie est
alors transmis dans les deux cas. Il est cependant nécessaire de relativiser ces
résultats car ils font l’hypothèse forte que le puisage est nul durant cette période.

\itodo{Réglementation et calcul du rendement}
\itodo{Niveau critique de rayonnement voir p.266}
\itodo{Énergie utile / différentiel d’énergie utile}
\itodo{La disposition orientation inclinaison des capteurs}
\itodo{Ce qui reste à faire}
% subsection les_avancées_dans_le_domaine_du_solaire_thermique (end)


% ------------------------------------------------------------------------------
\subsection{Conception de maisons solaires à énergie positive} % (fold)
\label{sub:conception_de_maisons_solaires_a_energie_positive}
% - - - - - - - - - - - - - - - - - - - - - - - - - - - - - - - - - - - - - - -
\subsubsection{Le processus de conception dans le bâtiment} % (fold)
\label{ssub:le_processus_de_conception_dans_le_batiment}
La étapes de construction d’un bâtiment public sont définies à travers la loi
\href{https://www.legifrance.gouv.fr/affichTexte.do?cidTexte=JORFTEXT000000693683}{$85$-$704$ du $12$ juillet $1985$}
Le texte décrit la construction d’un bâtiment à travers trois principales étapes
souvent repris dans le secteur privé~:
\begin{description}[align=left]
    \item [Programmation]
          Aussi appelée phase de Faisabilité, elle fait principalement intervenir la
          maîtrise d’ouvrage qui établit le cahier des charges avec les tiers. Les besoins
          du client, l’objet de la construction, la faisabilité, et les appels d’offres
          sont réalisés.
    \item [Conception]
          Elle fait intervenir la \acro{MOA} et la maîtrise d’œuvre (\acro{MOE}). Des études technique \\
          Elle fait intervenir la \acro{MOA} et la maîtrise d’œuvre (\acro{MOE}). Des études technique
          et architecturales permettent de définir plus finement le cahier des charges
          définit durant la phase amont.
          permet
    \item [Réalisation]
          Aussi appelée phase de construction, elle fait intervenir l’ensemble des acteurs
          du bâtiment, \acro{MOA}, \acro{MOE}, entreprises, tiers\dots Le cahier des charges définit
          en phase de programmation puis affiné en phase de conception est suivi. Cette
          phase comprend alors le suivi technique et financier du chantier mais aussi sa
          réception et mise en service.
\end{description}

La phase de conception est elle définie par quatre grandes sous-tâches~:
\begin{description}
    \item [\textsf{ESQ}] L’ESQuisse ou AVant Projet (\textsf{AVP}) vérifie la faisabilité technique du projet au regard
          des différentes contraintes (économique, technique, réglementaire). Durant cette
          phase sont définis~: les paramètres d’usages
          (charges internes, scénarios), le cadre réglementaire (thermique, acoustique, mécanique),
          les objectifs d’exemplarité à travers l’obtention de labels ($E+C-$,
          \acro{BBC} Éffinergie 2017, \acro{BBCA}\dots) et les caractéristiques de l’implémentation
          (données climatiques, étude du sol\dots).
    \item [\textsf{APS}] L’Avant Projet Sommaire consiste à définir la géométrie générale
          du projet à travers la définition des principaux volumes, l’aspect extérieur,
          et les coûts prévisionnels.
    \item [\textsf{APD}] L’Avant Projet Définitif permet d’affiner la géométrie du bâtiment
          à partir des résultats des différentes études techniques. La géométrie
          les caractéristiques de l’enveloppe, les systèmes techniques sont
          arrêtés et les coût prévisionnels découpés par lot.
    \item [\textsf{PRO}] La phase de PROjet permet de finaliser l’étude et d’estimer
           à la fois le coût des travaux et d’exploitation. La planning des travaux
           doit aussi être réalisé et le détail de chaque lot réalisé. Ces éléments
           sont décrits à travers un dossier des Spécificités Techniques
           détaillées (\acro{STD}) et un Dossier de Consultation des Entreprises (\acro{DCE}).
\end{description}

Ces travaux s’inscrivent donc dans la phase de conception en proposant une méthodologie
permettant l’optimisation de la construction de maisons solaires positives à énergie solaire.
De manière plus détaillée ce travail intervient au cours de trois tâches~: \acro{ESQ}, \acro{APS}
et \acro{APD}. Le chapitre $2$ se place ainsi durant les deux premières tâches
et permettra de préciser les scénarios d’usages tel que le profil de
puisage retenu mais aussi les différentes caractéristiques d’implémentation et les
spécificités souhaitées. Les chapitres $3$ et $4$ quand à eux s’inscrivent dans la phase
d’\acro{APD} où le choix technique est affiné à travers un processus d’optimisation.

\iunsure{Décrire ici sensibilité et méta-modèles}
% subsubsection le_processus_de_conception_dans_le_batiment (end)


% - - - - - - - - - - - - - - - - - - - - - - - - - - - - - - - - - - - - - - -
\subsubsection{Un problème multi-critère et multi-objectif} % (fold)
\label{ssub:un_probleme_multi_critere_et_multi_objectif}

\paragraph{Critique de l’approche actuelle dans le bâtiment} % (fold)
\label{par:critique_de_l_approche_actuelle_dans_le_batiment}
La construction d’un bâtiment est à l’origine guidé par le bon sens et l’expérience
acquise dans l’optique de satisfaire aux besoins immédiat du client. La prise
de conscience d’une quantité de ressource limité avec le premier choc pétrolier
a comme expliqué en introduction changé la vision de la construction et les diverses
réglementation visent à toujours plus d’exemplarité. Le bâtiment d’aujourd’hui doit
ainsi être placé dans une vision plus large.
Parallèlement le développement de l’informatique a permis le développement d’outils
techniques et économiques marquant une évolution dans le processus de construction.
Au début réservé à une élite, sa démocratisation a permis son application dans le secteur
énergétique et est aujourd’hui largement utilisé par les bureaux d’études mais
les habitudes ont survécu à ce changement de moyens~: la phase de conception est toujours
guidé par l’expérience. L’accès à la simulation a cependant permis d’améliorer
l’exploration de solutions et de vérifier certaines suppositions plus rapidement
à travers une approche par essai / erreur. À partir d’un cas de référence définie
par l’expérience un ensemble de variations sont introduites à l’aide d’itération
successive. Ce processus est couteux en temps et ne permet pas de valoriser la puissance
de calcul disponible aujourd’hui. De plus, les contraintes économiques amènent les bureaux
d’études à accélérer leur études et à produire dans l’urgence à cause d’une méthodologie
basée sur le temps humain.
Les enjeux étant de plus en plus complexe, il convient de valoriser au maximum le
temps machine afin d’aider les acteurs du bâtiment dans l’analyse d’options techniques
de plus en plus importantes.
En effet la miniaturisation, la finesse de gravure,
et l’augmentation du nombre de transistor par processeur a permis une croissance
exponentielle de la puissance de calcul. L’arrivée des processeurs multi-cœur à encore
repoussée les limites de calcul, si bien que le développement d’une méthodologie basée
sur le temps machine est tout à fait pertinente.
% paragraph critique_de_l_approche_actuelle_dans_le_batiment (end)

\paragraph{Approche proposée} % (fold)
\label{par:approche_proposee}
La construction d’un bâtiment est par définition multi-critère car faisant intervenir
un nombre conséquent de paramètres que ce soit sur l’enveloppe, les systèmes, la logique
de contrôle, le climat, ou bien le comportement des occupants.
La construction d’une maison solaire à énergie positive est de plus multi-objectif
en cherchant conjointement à améliorer la part solaire sur le chauffage, l’\acro{ECS},
à obtenir un bilan positif. Ce bilan positif peut être obtenu soit en réduisant au
maximum la consommation, soit en compensant l’énergie consommée sur place par une production locale.
Dans le cas d’une maison solaire la production locale peut être assurée par une production
photovoltaïque. Cette production est en effet la seule manière de couvrir les besoins
spécifiques des occupants d’origine électrique tels que l’éclairage ou les consommations
électroménagères. À l’opposée réduire la consommation sur site peut être réalisé
soit en optimisant l’enveloppe pour minimiser les déperditions, soit en utilisant
le solaire thermique afin de couvrir au maximum les besoins en \acro{ECS} et en chauffage
de la maison.
Comme décrit dans la directive européenne \parencite{EPBD2010} et explicité à travers
la norme $prEN\,15603$ (\mtodo{Ajouter ref}) le bâtiment à énergie positive européen
doit avoir une enveloppe performante (objectif $1$), des systèmes performants (objectif $2$),
une part renouvelable sur site et par usage importante (objectif $3$), et assurer un
bilan énergétique sur les cinq usages nul.
Le respect de chaque objectif étant péremptoire, il convient donc de faire un effort
sur chaque aspect du bâtiment. Ces objectifs indique aussi clairement que la part
d’énergie renouvelable consommée sur site devra couvrir une majeure partie des
consommations faisant de la performance du système solaire thermique, une priorité.

Ces travaux proposent dans cette optique une méthodologie d’optimisation
multi-objectif et multi-critère pour aider en particulier le développement de
maisons solaires à énergie positives. La performance de l’enveloppe sera évaluée
conjointement à la performance du système allant à contre courant des pratiques
actuelles de la \acro{RT\,$2012$} qui favorise dans un premier temps une étude de l’enveloppe
(respect du \acro{$B_{bio}$\,max}) et dans un second temps la performance des systèmes
(respect du \acro{$C_{ep}$\,max}). Ce choix est motivé par le troisième objectif du cadre
normatif en construction au niveau européen. Ces travaux amènent ainsi à repenser
les caractéristiques de l’enveloppe comme étant complémentaire à la performance du système
et soulève la question de la pertinence de la course à la sur-isolation en amont
de l’évaluation des systèmes énergétiques mis en place.
Les travaux actuels sur les \acro{SSC} relègue au second plan la partie algorithmique
Ces travaux s’inscrivent aussi dans une démarche visant à valoriser l’énergie solaire
par le développement d’une logique de contrôle minutieuse et précise. Bien que
l’algorithme de fonctionnement soit décrit dans la littérature au niveau des installation
industrielles, les applications dans le bâtiment et plus particulièrement dans les
maisons individuelles ne sont pas légions. Ces travaux tiendront aussi compte
des spécificités de chaque climat et décriront de manière conjointe les différences
observées. Ces travaux visent finalement à définir les compromis existants entre
enveloppe, performance du système thermique, et bilan énergétique positif mais aussi
au compromis existant pour chaque climat entre production \acro{ECS} et chauffage.

Dans un premier temps, le choix des outils est introduit à travers une étude comparative.
Suit une description du bâtiment de référence et des développements réalisés sur le
\acro{SSC} à travers une présentation du système technique et une étude rétrospective
détaillé de la logique de contrôle implémentée.
Ce second chapitre introduit ainsi les paramètres et scénarios d’usage considérés dans le reste de l’étude. Il
permet aussi d’évaluer a priori la performance pouvant être espérée.

Le troisième chapitre décrit l’approche générale retenue et décrit les méthodes complémentaire
d’aide à la décision existantes afin de justifier la méthodologie retenue. De plus
une analyse circonstanciée des méthodes d’optimisation est réalisé afin de justifier
le choix de la méthode retenue (\figref{fig:plan_schématique}).

Finalement le quatrième chapitre explicite la méthodologie d’optimisation retenue
à travers deux cas d’étude pour des conditions climatiques distinctes en se basant
sur les résultats obtenues au chapitre deux.

\begin{figure}
    \centering
    \ftodo{Description schématique du projet de thèse par chapitre}
    % \includegraphics{Ressources/Images/Modelisation/composant_vs_bloc.png}
    % \caption{}
    \label{fig:plan_schématique}
\end{figure}
% paragraph approche_proposee (end)
% subsubsection un_probleme_multi_critere_et_multi_objectif (end)
% subsection conception_de_maisons_solaires_a_energie_positive (end)
% section la_maison_solaire_a_energie_positive (end)



























% \itodo{Ajouter citations, voir article BS 2017}
% \itodo{Ajouter description de l’état de l’art des algos Voir Article}
% De nombreuses études ont déjà cherché à évaluer la performance d’un système solaire
% combiné à l’aide de méthodes plus ou moins détaillées. L’approche la plus répandue utilise
% les besoins mensuels de la maison générés grâce à un modèle du bâtiment simplifié
% \parencite{Raffenel2009657,Martinopoulos2014130}. Cette approche néglige le comportement dynamique du système
% solaire comme du bâtiment. D’autres approches utilisent un modèle de bâtiment plus
% détaillé (TrnSys, Energy Plus) permettant d’évaluer plus précisément la couverture du
% système \parencite{Glembin2012601}. Ces approches permettent de tenir compte de l’évolution dynamique du système
% solaire combiné (SSC) mais négligent les interactions entre le bâtiment et les systèmes. Le
% bâtiment est seulement considéré comme une consommation (chauffage et production d’ECS) et
% le système solaire comme une source potentielle répondant à cette demande. Cette approche
% est aussi celle retenue pour la Task 26 \parencite{Task262003} au cours de laquelle de nombreux modèles
% de SSC ont été développés puis validés.

% Dans ces travaux, une approche détaillée de l’algorithme de contrôle et des systèmes est retenue. En
% effet, plusieurs études ont mis en évidence l’importance de la modélisation du contrôle
% sur la performance d’un système solaire combiné \parencite{Kicsiny20123489,Huang20123278}.
% Afin de présenter des résultats pouvant être obtenues sur des bâtiment réels pour la
% prochaine réglementation thermique (\mtodo{Ajouter citation}{\acro{RT\,2020}}), un algorithme existant et innovant
% a été utilisé. Celui-ci est modifié et adapté pour répondre aux contraintes du projet~: obtenir
% un bâtiment réactif en utilisant l’énergie solaire.
% L’originalité de l’approche réside principalement dans l’évaluation couplée du système
% solaire combiné et du bâtiment. Dans cette optique les interactions bâtiment / systèmes et
% systèmes / bâtiment sont pris en compte et évaluées.

% Dans un premier temps des outils retenues pour la modélisation et l’analyse des résultats.
% Dans un second temps le modèle SSC développée ainsi que le bâtiment et les scénarios, sont
% discutés. Finalement une étude paramétrique est réalisée. Elle permettra de mieux
% comprendre les interactions existantes entre bâtiment et système et de fixer les scénarios
% qui seront retenues pour l’aide à la décision. De plus les indicateurs nécessaires à la
% caractérisation d’un SSC seront identifiés.


% % ..............................................................................
% % ..............................................................................
% \section{Le solaire thermique appliqué au bâtiment} % (fold)
% \label{sec:le_solaire_thermique_applique_au_batiment}

% % ------------------------------------------------------------------------------
% \subsection{Contexte énergétique} % (fold)
% \label{sub:contexte_energetique}
% \itodo{État de l’art sur performance des bâtiments, danger réchauffement, augmentation
%        de la population}
% \itodo{Parler de ce que fait NegaWatt notament avec le projet <Europe,Territoires>.\\
%        \url{http://www.negawatt.org/telechargement/Docs/160615_Rapport-final_Europe-territoire_Phase1.pdf}\\
%        \url{http://www.negawatt.org/telechargement/Docs/160324_Synthese_Etude_Europe-territoires_Phase1.pdf}}
% % subsection contexte_énergétique (end)

% % ------------------------------------------------------------------------------
% \subsection{Le concept MEPOS} % (fold)
% \label{sub:le_concept_mepos}
% \itodo{Description des approches existantes: Passivhaus, NZEB, MEPOS}
% \itodo{Décrire l’initiative de COMEPOS pour le label MEPOS}
% \itodo{Décrire pourquoi ce choix (car il a apprit des anciens labels)}
% % subsection le_concept_mepos (end)

% % ------------------------------------------------------------------------------
% \subsection{Le solaire thermique pour une production énergétique respectueuse} % (fold)
% \label{sub:le_solaire_thermique_pour_une_production_energetique_respectueuse}
% \itodo{État de l’art sur les applications du solaire thermique.}
% \itodo{Montrer que le solaire thermique n’a pas la côte dans la construction à énergie positive.
%        Problème de coût, de confiance, ou de performance ?}
% \itodo{Montrer que l’innovation passe par le neuf avant d’être intégré à la rénovation.}
% % subsection le_solaire_thermique_pour_une_production_énergétique_respectueuse (end)
% % section le_solaire_thermique_appliqué_au_bâtiment (end)



% % ..............................................................................
% % ..............................................................................
% \section{Une approche par optimisation} % (fold)
% \label{sec:une_approche_par_optimisation}

% % ------------------------------------------------------------------------------
% \subsection{Approches explorées} % (fold)
% \label{sub:approches_explorees}
% \itodo{Décrire les différentes approches déjà explorées}
% \itodo{Augmentation surface capteur, sur-isolation, mon approche couplée pour un
%        compromis entre coût/surface capteur/isolation}
% % subsection approches_explorées (end)

% % ------------------------------------------------------------------------------
% \subsection{L’optimisation d’une maison solaire: un problème multi-critère} % (fold)
% \label{sub:l_optimisation_d_une_maison_solaire_un_probleme_multi_critere}
% \itodo{Décrire brièvement les methodes existantes}
% \itodo{Décrire les outils nécessaires (sensibilité, opimisation, aide à la décision)}
% % subsection l_optimisation_d_une_maison_solaire_un_problème_multi_critère (end)
% % section une_approche_par_optimisation (end)


% % ..............................................................................
% % ..............................................................................
% \section{Le choix d’un modèle de système solaire couplé au bâtiment} % (fold)
% \label{sec:le_choix_d_un_modele_de_systeme_solaire_couple_au_batiment}
% % ------------------------------------------------------------------------------
% \subsection{Les modèles existants} % (fold)
% \label{sub:les_modeles_existants}
% \itodo{Décrire le choix de l’approche par modélisation}
% Il existe plusieurs moyens permettant d’évaluer un système énergétique. Le premier
% consiste à reproduire expérimentalement le système et son environnement. Cependant
% ce processus est couteux , spécialement lorsque on essayer d’évaluer un système à l’échelle
% du bâtiment. On est de plus contraint par les conditions extérieures que l’on ne contrôle
% pas. Ainsi ces deux raisons font qu’il est compliqué et couteux de chercher à dimensionner
% expérimentalement un système. Pour réduire le coût de la recherche et explorer plus de
% variation il est nécessaire de pouvoir contrôler les conditions limites et de pouvoir
% itérer rapidement entre différents compositions/régulations. La modélisation entre alors
% en jeu proposant un contrôle complet des systèmes, de leur régulation, et des conditions
% limites. Le système n’étant pas physique il est alors possible de faire varier n’importe
% quel paramètre simplement afin d’évaluer son importance, son impact,\dots
% La modélisation est donc l’outil de choix pour réaliser une étude de faisabilité, un
% dimensionnement, une optimisation.

% \itodo{État de l’art des modèles numériques}
% \itodo{Détail des modèles existant, contrôle, couplages}
% \itodo{Montrer que ces modèles sont très génériques et souvent très simplifiés.
%        De plus l’algorithme de contrôle est non évalué/optimisé.}
% % subsection les_modèles_existants (end)

% % ------------------------------------------------------------------------------
% \subsection{Un modèle solaire couplé au bâtiment} % (fold)
% \label{sub:un_modele_solaire_couple_au_bâtiment}
% \itodo{Décrire le choix de l’approche par modélisation}


% \itodo{Décrire les outils utilisés: Modelica et les bibliothèques, Dymola et les solveurs}
% Il existe de nombreux langages, logiciels pour réaliser des simulations plus ou moins complexes.
% La première choses à définir est donc le niveau de précision que l’on souhaite pour son modèle ou
% pour les différentes parties du modèle. Notre cas d’étude se place à l’échelle du bâtiment mais
% l’on souhaite aussi conserver un contrôle important sur la gestion des équipements.
% Ensuite il est nécessaire de déterminer le niveau d’accessibilité que l’on souhaite avoir
% sur chaque composant du système. Le modèle \textbf{boîte noire} ne pourrait pas correspondre à notre
% demande. Celui-ci ne nous offre pas le liberté de comprendre comment évolue chaque
% composants du système et nous empêche d’explorer/modifier le code.
% Pour la même raison un modèle \textbf{boîte grise} nous limite dans l’accès à certaines
% partie du code et donc à la compréhension interne du fonctionnement du système.
% Il est alors nécessaire d’utiliser un modèle \textbf{boîte blanche} garantissant
% un contrôle total sur chaque partie du système et permettant d’évaluer le comportement
% au niveau global mais aussi composant par composant.
% Nous avons ainsi opté pour le langage Modelica et la plateforme de développement
% Dymola (Dynamics Modeling Laboratory).


% \itodo{Modelica description}
% \mtodo{Ajouter référence}{Modelica} est un langage de programmation libre et ouvert développé pour répondre aux
% contraintes de la modélisation multi-physique. Il a été pensé pour être intuitif
% et offre une approche équationnelle et orienté objet au développeur.
% L’approche objet est très intuitive et permet d’encapsuler un ensemble de données
% et d’offrir des interface pour accéder à ces données. On peut alors composer de nouveaux
% objets grâce à des références vers d’autre objets (composition) ou en héritant
% du comportement d’un objet pour lui ajouter une spécialisation (héritage).
% Enfin le langage est acausal permettant d’itérer entre différentes formulation
% d’un problème facilement. Un système acausal récupère l’ensemble des variables qu’il
% connait et défini les inconnus à partir de celles-ci. L’ordre d’écriture des équations
% n’est donc plus importante et modifier un système n’oblige pas à re-écrire complètement
% les équations pour isoler la variable que l’on cherche à déterminer. Prenons l’exemple
% de la formule $U = R \times I$. Un problème causal requiert de connaître \acro{R} et \acro{I}
% pour trouver \acro{U}.L’équation doit être re-écrite si on cherche à trouver \acro{R} ou \acro{I}.
% Si on utilise une approche acausal alors cette équation a une solution si on a deux des
% trois inconnus ($U et R$ ou $R et I$, ...) sans avoir à modifier l’équation. On peut ainsi utiliser
% la même formulation pour peut importe les variables connues.
% On peut ainsi voir que le développement sous Modelica permet de rapidement proto-typer
% des systèmes complexes.


% \itodo{Dymola description}
% Dymola est une suite de logiciels développée par \mtodo{Ajouter référence}{Dassault System}
% permettant d’ajouter de nombreuses fonctionnalités.
% La première étant l’interface graphique permettant de connecter différentes portion
% de code de manière plus intuitive. Il ajoute aussi un débogueur puissant permettant
% de trouver rapidement la portion de code qui pose problème, un outil pour faire du
% refactoring. Enfin il offre un outil puissant pour compiler, initialiser et intégrer
% le modèle avec un large choix d’intégrateurs avec le programme \mtodo{Ajouter référence}{Dymosim}.
% Il propose aussi du support pour la parallélisation, les FMU, et, un outil
% puissant pour faire du traitement des données durant et après les simulations. Enfin
% Dymola propose grâce à des scripts d’accéder à l’ensemble des fonctionnalités
% comme lancer une simulation, modifier un modèle, exporter un modèle, ...
% Dymola est ainsi un outil puissant pour accélérer le développement de modèle Modelica.


% \itodo{Couplage Dymola + Modelica description}
% Ces deux outils offrent alors de nombreux avantages. On contrôle chaque partie
% du système, le code est réutilisable, on peut choisir le détail de chaque partie
% du modèle, et on peut le coupler avec d’autres logiciels si besoin est.

% Le langage Modelica étant largement utilisé, de nombreuses bibliothèques open source
% on été développées dont la liste peut être trouvée sur le site officiel de l’association Modelica
% (\url{https://www.modelica.org/libraries}). Dans notre étude nous avons utilisé
% la bibliothèque \mtodo{Ajouter référence}{Buildings} qui est développé par le
% Laboratoire National Lawrence Berkeley (LBNL). C’est une bibliothèque libre et ouverte
% orienté pour le secteur du bâtiment offrant de nombreux modèles de base.


% \itodo{Pourquoi une modélisation détaillée d’un système existant}
% Le but de cette étude est d’évaluer le potentiel de couverture d’un système solaire,
% il apparaît donc important de pouvoir évaluer dans le détail son comportement au niveau
% de chaque élément. On va chercher à comprendre comment évolue la température au sein
% de la maison mais aussi des ballons, des capteurs, ...
% Il est aussi important de pouvoir modifier facilement les différents éléments composant
% le système comme les temporisations, les consignes, la taille des différents équipements, ...
% L’étude ne s’intéresse en effet pas seulement à la performance finale du système mais
% aux raisons et limitations qui ont pour conséquence ce résultat.


% \itodo{Récapituler les besoins de l’étude}
% Si on résume on a donc besoin de contrôler chaque composant pour évaluer si il se
% comporte comme on l’entends. Il est aussi nécessaire de pouvoir facilement de manière
% intuitive les sous-modèles sans affecter le modèle principal. On veut de plus pouvoir
% évaluer le système au niveau du bâtiment et il est donc nécessaire de réaliser des
% simulation sur une échelle de temps importante (de l’ordre de l’année).


% \itodo{Conclure sur le choix de Modelica et Dymola}
% Le choix du couple Modelica + Dymola est donc le résultat d’une recherche d’un outil
% répondant à nos contraintes. On veut avoir un contrôle complet de chaque composant, leur
% physique comme leur régulation et évaluer au niveau bâtiment la performance de celui-ci.
% Ces deux outils nous permet d’évaluer/modifier/comprendre efficacement le système
% sur différentes échelles tout en encourageant le processus itératif de cette étude.



% \itodo{Montrer que aujourd’hui peu de travail a été fait sur l’optimisation couplée
%        et que par conséquent ce sujet est innovant dans son approche du problème}
% \itodo{Introduire la partie suivante}
% % subsection un_modèle_solaire_couplé_au_bâtiment (end)
% % section le_choix_d_un_modèle_de_système_solaire_couplé_au_bâtiment (end)
