

\section{Contexte énergétique} % (fold)
\label{sec:contexte_energetique}
-----------------------------------
État de l’art sur:

 - Évolution de la performance des bâtiments
 - Danger du réchauffement climatique
 - Augmentation de la population
% section contexte_energetique (end)


\section{Systèmes solaires et performance} % (fold)
\label{sec:systemes_solaires_et_performance}
-----------------------------------
État de l’art sur les modélisation solaires:

 - Le détail des modèles existants
 - Le détail du contrôle des modèles existants
 - Les travaux déjà réalisés sur le couplage bâtiment/systèmes


Montrer que ces modèles sont très génériques et souvent très simplifiés.
De plus l’algorithme de contrôle est non évalué/optimisé.

Montrer que aujourd’hui peu de travail a été fait sur l’optimisation couplée
et que par conséquent ce sujet est innovant dans son approche du problème
% section systemes_solaires_et_performance (end)


\section{Le bâtiment à énergie positive} % (fold)
\label{sec:le_batiment_a_energie_positive}
-----------------------------------
Description des approches:

 - Passivhaus
 - NZEB
 - MEPOS

Sélection d’une approche: MEPOS

Montrer que le solaire thermique n’a pas la côte dans la construction à énergie positive.
Problème de coût, de confiance, ou de performance ?

Montrer que l’innovation passe par le neuf avant d’être intégré à la rénovation.
% section le_batiment_a_energie_positive (end)
