% Chapitres\Chap1-ContexteTravaux.tex




% ..............................................................................
% ..............................................................................
\section{Contexte énergétique} % (fold)
\label{sec:contexte_energetique}
\itodo{État de l’art sur performance des bâtiments, danger réchauffement, augmentation
       de la population}
% section contexte_energetique (end)




% ..............................................................................
% ..............................................................................
\section{Le bâtiment à énergie positive} % (fold)
\label{sec:le_batiment_a_energie_positive}
\itodo{Description des approches existantes: Passivhaus, NZEB, MEPOS}
\itodo{Contexte et choix de MEPOS, pourquoi ?}
\itodo{Montrer que le solaire thermique n’a pas la côte dans la construction à énergie positive.
       Problème de coût, de confiance, ou de performance ?}
\itodo{Montrer que l’innovation passe par le neuf avant d’être intégré à la rénovation.}
% section le_batiment_a_energie_positive (end)



% ..............................................................................
% ..............................................................................
\section{Le solaire appliqué au bâtiment} % (fold)
\label{sec:le_solaire_appliqué_au_batiment}
\itodo{État de l’art sur la modélisation de système solaires}

\itodo{Détail des modèles existant, contrôle, couplages}
\itodo{Montrer que ces modèles sont très génériques et souvent très simplifiés.
       De plus l’algorithme de contrôle est non évalué/optimisé.}
\itodo{Montrer que aujourd’hui peu de travail a été fait sur l’optimisation couplée
       et que par conséquent ce sujet est innovant dans son approche du problème}
% section le_solaire_appliqué_au_batiment (end)




