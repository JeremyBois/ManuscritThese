%!TEX root = ../main.tex
% Chapitres\Chap1-ContexteTravaux.tex



Ce chapitre décrit les objectifs et le contexte dans lequel ces travaux s’inscrivent. En
effet le développement de maisons à énergie positive est l’aboutissement d’une volonté
politique nécessaire pour répondre aux enjeux climatiques et économiques. Le solaire est
lui, une énergie quasi-infinie pouvant se substituer aux énergies finies telles que le
nucléaire ou les sources fossiles. Afin de proposer une méthodologie appropriée, il est
dans un premier temps exposé le contexte climatique, l’importance du secteur du bâtiment,
et plus particulièrement la place de la maison individuelle. Suit une description de la
notion de maison à énergie positive et de la place et des avancées du solaire thermique et
plus particulièrement des systèmes solaires combinés. Finalement en s’appuyant sur les
travaux existants et les contraintes propres au contexte, il est proposé une méthodologie
d’aide à la conception de maisons solaires à énergie positive adaptée.
\clearpage



% % ...........................................................................
% % ...........................................................................
\section{Objectifs} % (fold)
\label{sec:objectifs}
Le bâtiment est le premier consommateur d’énergie (\defref{def:energie}) en France et est
donc un élément clé pour la réussite de la transition énergétique. Les nouveaux bâtiments
doivent ainsi être performants et respectueux de l’environnement pour réduire la facture
énergétique et le coût des rénovations futures. En effet, l’amélioration de la conception
des bâtiments neufs se traduit par une amélioration du futur parc immobilier et donc par
un allègement des coûts. Avec $\nicefrac{3}{4}$ des consommations en énergie primaire, les
ménages dont plus de \SI{50}{\percent} sont des maisons individuelles se doivent d’être
exemplaires. Le concept d’énergie positive n’a cependant pas de
définition unique et son cadre d’application n’est pas encore clairement établi même si
des initiatives nationales et internationales permettent d’y voir plus clair. Ces
initiatives cherchent à répondre à la problématique du réchauffement climatique grâce à
des mesures d’efficacité, de sobriété ou encore grâce aux énergies alternatives.

\begin{Def}[Énergie]\label{def:energie}
L’énergie en physique est la mesure de la capacité d’un système à produire
de la chaleur, modifier un état, ou encore entraîner un mouvement. Dans notre
problématique une distinction est nécessaire entre l’énergie dite \textbf{primaire} et
l’énergie dite \textbf{finale}. L’énergie primaire correspond à l’énergie prélevée à
l’environnement alors que l’énergie finale est l’énergie primaire après soustraction des
pertes de stockage, de transport, et de transformation. L’énergie finale correspond ainsi
à l’énergie consommée par l’utilisateur sans tenir compte des consommations engendrées pour sa
production à l’opposée de l’énergie primaire. Le bouquet énergétique étant propre à chaque
pays, les facteurs de conversion le sont eux aussi. Pour la France,
les coefficients souvent admis sont définis en annexe (\tabref{tab:coef_finale_primaire}).
\end{Def}

Parmi les solutions disponibles, l’énergie solaire apparaît comme un candidat de choix
pour répondre aux problématiques de réchauffement climatique et à la forte croissance de la
population. C’est une énergie propre, elle ne rejette pas de $CO_{2}$ dans
l’atmosphère et permet de réduire la dépendance aux énergies fossiles et nucléaires. Un
système solaire permet ainsi de répondre aux besoins d’une population grandissante sans
pour autant polluer davantage. Abondante et inépuisable l’énergie solaire permet selon la
technologie retenue de couvrir des besoins électriques (électroménager\dots) ou thermiques
(chauffage\dots). Cependant bien qu’inépuisable l’énergie est uniquement disponible durant
la période diurne et les systèmes doivent alors permettre de valoriser au mieux cette
énergie en tenant compte de cette contrainte forte. Alors que le photovoltaïque profite
d’une croissance forte avec l’augmentation des charges électriques, les systèmes solaires
combinés (\abr{SSC}) restent très minoritaires et sont principalement implémentés dans la
rénovation.

Un \abr{SSC} est un système permettant de couvrir la production d’Eau Chaude Sanitaire
(\abr{ECS}) et les besoins en chauffage d’un bâtiment grâce l’énergie solaire. L’énergie
solaire est récupérée par des capteurs puis, soit utilisée directement, soit stockée dans
des ballons pour couvrir des besoins en période nocturne. Cependant ces systèmes
comportent certains inconvénients responsables en partie de l’impopularité de la
technologie. Ils nécessitent par exemple le recours à des ballons de taille importante
qui augmentent le coût d’investissement. De plus bien souvent, la place disponible pour
les équipements est réduite et le système doit être installé dans la zone chauffée. Ces
choix techniques sont le résultat d’un dimensionnement basé uniquement sur l’expérience
acquise à travers les installations plus anciennes. Pourtant les contraintes propres à la
conception d’une Maison à Énergie POSitive (\abr{MEPOS}) ne sont pas les mêmes que celle
de constructions plus anciennes. Les bâtiments sont en effet aujourd’hui bien mieux
isolés, les charges internes augmentent et deviennent majoritaires, les besoins de
chauffage sont bien moins important, mais les besoins en \abr{ECS} restent eux
sensiblement les mêmes.

Il est ainsi nécessaire de repenser la conception de \abr{SSC} pour les Bâtiments à
Énergie POSitive (\abr{BEPOS}). Se reposer uniquement sur l’expérience acquise apparaît
comme un frein au développement de la technologie. Ces travaux cherchent ainsi à évaluer
le potentiel d’un \abr{SSC} adapté aux \abr{BEPOS}. Inspiré par la norme européenne
\abr{PREN\,$15603$} actuellement en rédaction, une méthodologie d’aide à la décision est
proposée afin d’aider à leurs conception. L’originalité de ces travaux réside
principalement dans la prise en compte concomitante de la performance de l’enveloppe et du
\abr{SSC} permettant d’identifier des combinaisons pertinentes entre les équipements du
\abr{SSC}, la logique de contrôle, et l’enveloppe du bâti. Dans cette optique, une
modélisation détaillée du système et de l’algorithme de contrôle est requis afin de
pouvoir mieux comprendre les interactions existantes entre les deux. Le parti pris est le
suivant~: les contraintes techniques sont propres à chaque site, chaque industriel, mais
aussi à chaque client dont les attentes varient fortement. Partant de ce constat, il
apparaît plus intéressant de proposer un ensemble de solutions techniquement différentes
mais répondant toutes aux contraintes de la \abr{MEPOS} à travers une approche
multi-objectifs.

\paragraph{} % (fold)
Ce chapitre décrit le contexte global dans lequel se place ces travaux. Il est dans un
premier temps présenté les enjeux liés au réchauffement climatique et à la pénurie des
ressources pour une population en forte croissance. Puis il est introduit les initiatives
à l’origine de la prise de conscience collective de l’impact destructeur lié à la sur-
utilisation des énergies fossiles. Ce sont en effet ces actions qui sont à l’origine des
textes réglementaires qui encadrent la rénovation et la conception des bâtiments. Cette
volonté politique se traduit par le développement de bâtiment toujours plus performants
sur le plan énergétique comme environnemental et aujourd’hui par la construction de
\abr{BEPOS}. La seconde partie de ce chapitre présente les différents travaux et
initiatives à l’origine de ce projet. Elle met notamment en avant la complexité inhérente
à la définition d’une base réglementaire internationale commune pourtant nécessaire pour
répondre de manière durable aux problématiques énergétiques et environnementales. La
troisième partie invite le lecteur à découvrir l’énergie solaire, comment elle est
mesurée, récupérée, et les difficultés liées à son évaluation en particulier à cause des
différences géométriques propres à chaque capteur. Il existe en effet de nombreuses
technologies en solaire thermique permettant de valoriser soit en chaleur, soit en
électricité, l’énergie solaire. Suit une description de l’évaluation de l’évolution du
solaire thermique à différentes échelles. Finalement, les différents travaux de la
littérature dans le domaine du solaire thermique et plus particulièrement pour les
\abr{SSC} sont discutés. Ces avancées sont utilisées pour proposer un \abr{SSC} et une
méthodologie adaptée à la conception de \abr{MEPOS} solaires dont la démarche
est discutée dans la dernière partie de ce chapitre.
% section objectifs (end)



% ...........................................................................
% ...........................................................................
\section{L’Homme au centre d’un trouble climatologique} % (fold)
\label{sec:l_homme_au_centre_d_un_trouble_climatologique}
Comme explicité ci-avant l’energie solaire est une alternative pertinente aux énergies
non-renouvelables et apparaît donc comme pertinente pour répondre aux enjeux globaux liés au
réchauffement climatique. Cette première section s’intéresse au contexte climatiques de manière
globale et présente les différentes initiatives à l’origine de la prise de
conscience de l’importance politique, économique, et environnemental d’un développement
raisonné. La dernière partie discute plus en détail les actions françaises à l’échelle
nationale, européenne, et internationale.

% ------------------------------------------------------------------------------
\subsection{Le réchauffement climatique~: origines et conséquences} % (fold)
\label{sub:le_rechauffement_climatique_origines_et_consequences}
Auparavant les différentes tâches étaient principalement réalisées soit en utilisant les
éléments (vent, eau, feu) soit en utilisant la force animale (bœufs, chevaux\dots). Les
révolutions industrielles marquent l’émergence des machines, la société passe alors d’une
dominante artisanale et agraire à une société commerciale et industrielle affectant aussi
bien l’agriculture que la politique ou l’économie. Dès le début des années $90$ le
pétrole devient une source énergétique stratégique et est
activement utilisé dans l’industrie. La tendance s’accélère après la seconde guerre
mondiale, durant la période des $30$ glorieuses, où la France et d’autres pays développés
profitent d’une croissance forte tant au niveau économique, que démographique (baby-boom).
Cette expansion est portée par un accès aisé aux énergies et particulièrement aux énergies
fossiles principaux vecteur de développement de nos civilisations occidentales. Durant
cette période marquée par le développement du capitalisme, une énergie bon marché
considérée comme inépuisable, et un accroissement important de la population, la demande
énergétique explose. Afin de pouvoir comparer les différentes énergies entre elles, une
unité commune est retenue~: la tonne équivalent pétrole (\abr{tep}) ou tonnes of oil
équivalent (\abr{toe}) dans sa version anglophone qui équivaut au pouvoir calorifique
d’une tonne de pétrole, soit \SI{41.868}{\giga\joule}.
À titre de comparaison, l’énergie primaire consommée en $1973$ était de
\SI{6101}{\mega tep} dont \SI{87}{\percent} en énergie fossile alors qu’elle est en $2014$
à \SI{13699}{\mega tep} pour une part de près de \SI{81}{\percent} en énergie fossile
(\figref{fig:energy_fraction}). Cette énergie reste ainsi encore aujourd’hui fortement
majoritaire même si la consommation du pétrole diminue.

\begin{figure}
    \centering
    \includegraphics[width=0.7\textwidth]{Ressources/Images/Environnement/repartition_energy.png}
    \caption[Répartition des sources de production en énergie primaire pour $1973$ et $2014$]
            {Répartition des sources de production en énergie primaire pour
             $1973$ et $2014$ d’après \textcite{IEA2016} (\enquote{other} comprend
             la géothermie, le solaire, et l’éolien).}
    \label{fig:energy_fraction}
\end{figure}


Une fraction de cette énergie est utilisée pour le chauffage, la voiture, ou encore pour
nos équipements électriques de plus en plus nombreux~: \SI{+41}{\percent} entre $2010$ et
$2013$ en France \textcite{ADEME2015}. L’autre fraction moins visible représente
l’énergie utilisée pour la production, le transport ou le stockage des
biens de consommation de la vie courante. La sur-exploitation de ces énergies montre
cependant des limites et impacte de manière importante l’environnement.

Les effets du réchauffement climatique sont aujourd’hui facilement observables et sont
tels qu’ils bouleversent le fonctionnement de la planète. Bien que nous soyons en période
de réchauffement (causes externes astronomiques) l’impact de l’être humain est aujourd’hui
reconnu. La sur-utilisation des énergies fossiles couplée à une forte évolution
démographique se traduit par de trop fortes émissions en Gaz dits \enquote{à Effet de
Serre (\defref{def:effet_serre})} (\abr{GES}). Actuellement les mesures montrent un forçage
radiatif excédentaire entraînant une lente augmentation de la température du sol
terrestre. Entre $1750$ et $2005$, ce forçage est estimé à
\SI{2.63}{\watt\per\metre\squared} dont \SI{1.6}{\watt\per\metre\squared} imputable aux
émissions anthropiques (\cite{Myhre2013}, tableau $8.6$). La température moyenne sur
l’ensemble de la surface du globe étant le résultat du bilan radiatif entre le rayonnement
solaire incident et le rayonnement infrarouge terrestre émis, elle augmente lorsque la
part du rayonnement terrestre émis puis absorbé par l’atmosphère augmente.
L’étude des émissions anthropiques peut être assimilée à l’étude des émissions néfastes
additionnelles aux émissions naturelles. La vapeur d’eau qui contribue à \SI{55}{\percent}
à l’effet de serre n’est donc pas considéré l’Homme n’a pas un impact direct sur sa
concentration dans l’atmosphère. Dans des conditions normales l’effet de serre est en
effet positif et permet de maintenir une température correcte pour le développement de la
biodiversité~: en son absence la température moyenne de la terre serait de \SI{-18}{\celsius}.
En conséquence, l’effet de serre additionnel est en majorité dû à la concentration de
$CO_{2}$ avec \SI{56.6}{\percent}, mais aussi au méthane ($CH_{4}$), au protoxyde d’azote
($N_{2}O$) et à l’ozone (O3). \figref{fig:evolution_effet_serre} met en évidence l’accélération
des consommations à partir de $1950$ qui se traduit par une forte augmentation de la
concentration de $CO_{2}$ dans l’atmosphère.

\begin{Def}[Effet de serre]\label{def:effet_serre}
C’est un phénomène naturel engendré par la présence de divers composés
gazeux dans l’atmosphère. L’atmosphère ayant un faible indice de réflexion et d’absorption
pour le rayonnement solaire (visible, proche \abr{UV} et proche \abr{IR}), une grande
partie du rayonnement est transmise à la surface de la terre. Une fraction est alors
absorbée alors qu’une autre est réfléchie et repart dans l’espace. La part absorbée
engendre un échauffement, le sol émet alors un rayonnement dans le domaine de l’\abr{IR}
lointain. Certains composants gazeux de l’atmosphère étant faiblement transparents pour ce
domaine d’émission, une partie de l’énergie est absorbée au lieu d’être rejetée dans
l’espace~: c’est le mécanisme d’effet de serre.
\end{Def}

\begin{figure}
    \centering
    \includegraphics[width=0.8\textwidth]{Ressources/Images/Environnement/evolution_effet_serre2.png}
    \caption[Évolution de la concentration anthropique mondiale en $CO_{2}$]
            {Évolution de la concentration anthropique mondiale en $CO_{2}$ au cours des derniers siècles
             \parencite{IPCC2014}.}
    \label{fig:evolution_effet_serre}
\end{figure}

L’activité humaine depuis la révolution industrielle amène aujourd’hui à un taux de
concentration en \abr{GES} jamais atteint depuis près de \SI{800000}{ans}
\parencite{IPCC2014}. Bien que non imputables en totalité au réchauffement climatique, les
événements extrêmes attribués au changement climatique sont de plus en plus nombreux et
touchent l’ensemble des écosystèmes (\cite{IPCC2014}, Figure SPM.4). L’augmentation des
occurrences de canicules, des fortes précipitations, ou l’élévation du niveau de la mer
font notamment partie des conséquences directes (\figref{fig:evolution_climat}). Lorsque
seules les émissions naturelles sont considérées, les résultats de simulations mettent
clairement en évidence une stagnation de la température moyenne. À l’inverse, les
observations et simulations qui tiennent compte de la part anthropique montrent que
l’énergie stockée dans les océans (\abr{OHC}, Ocean Heat Content), la température des
océans comme de la terre, ou encore, la fonte des glaces en Arctique sont imputables à
l’être humain. Des mesures doivent ainsi être prises afin de limiter la consommation en
énergie primaire et les émissions en \abr{GES}. Le mode de développement actuel de l’être
humain (transport, bâtiment, population, robotisation\dots) est aussi responsable de
nombreux autres facteurs environnementaux aggravants, comme l’acidification des sols, la
pollution de l’air, la réduction de la biodiversité\dots\ \parencite{Biermann2016341}.
L’énergie étant cependant indispensable, seule une alternative respectueuse de
l’environnement, les énergies renouvelables, permettront d’inverser la tendance.

\begin{figure}
    \centering
    \includegraphics[width=.75\textwidth]{Ressources/Images/Environnement/elevation_temperature.jpg}
    \caption[Mise en évidence des effets anthropiques globaux et à l’échelle de la région]
            {Mise en évidence des effets anthropiques globaux et à l’échelle de la région
             à travers l’évolution des température des océans et des terres, des précipitations,
             de l’énergie stockée dans les océans (\abr{OHC}, Ocean Heat Content), et
             de la surface de mer gelée. Les lignes rouges et bleues foncées représentent l’évolution moyenne
             des différentes simulations respectivement avec et sans influence humaine. Les zones
             ombrée en bleu et rose représentent les intervalles de confiance pour les différentes
             simulations. Finalement les lignes de teintes grises décrivent les observations
             de différents travaux \parencite{AchutaRao2013}.}
    \label{fig:evolution_climat}
\end{figure}
% subsection le_rechauffement_climatique_origines_et_consequences (end)

% ------------------------------------------------------------------------------
\subsection{Vers une prise de conscience collective} % (fold)
\label{sub:vers_une_prise_de_conscience_collective}
En $1970$, le club de Rome commandite le rapport Meadows \enquote{Halte à la croissance ?}
\parencite{Meadows1972}. Pour la première fois, la possibilité d’une pénurie des
ressources énergétiques est envisagée par des chercheurs du \abr{MIT} à travers plusieurs
scénarios plus ou moins catastrophiques. Ce rapport fut la première pierre nécessaire à la
construction d’une prise de conscience commune de l’impact nocif de l’Homme sur l’environnement
et des problèmes qu’il engendre~: destruction de la couche d’ozone,
déchets, bouleversement climatiques, pollution de l’eau, atteintes à la bio-
diversité\dots\ En $1987$ un nouveau rapport cette fois commandité par les Nations Unies,
\enquote{Notre avenir à tous}, plus connu sous le nom de rapport \textit{Bruntland} \parencite{Brundtland1987}.
Contrairement au rapport \textit{Meadows} qui présente des scénarios de ce qui pourrait
arriver si rien ne change, ce document pose les bases pour un développement équitable
mettant en avant la protection des ressources naturelles mais aussi de l’équité des
individus face aux ressources. Le terme \enquote{Développement durable}
(\figref{fig:developpement_durable}, aujourd’hui largement repris, est introduit~:

\blockquote{
    Un développement qui répond aux besoins du présent sans
    compromettre la capacité des générations futures à répondre aux leurs.
}

\begin{figure}
    \centering
    \begin{subfigure}[b]{0.55\textwidth}
        \centering
        \includegraphics[width=0.75\textwidth]{Ressources/Images/Environnement/developpement_durable.png}
        \caption{}
        \label{fig:developpement_durable}
    \end{subfigure}
    \quad
    \begin{subfigure}[b]{0.4\textwidth}
        \centering
        \includegraphics[width=0.75\textwidth]{Ressources/Images/Environnement/triptyque_negaWatt.png}
        \caption{}
        \label{fig:negawatt_axes}
    \end{subfigure}
    \caption[Principe du développement durable et du scénario négaWatt]
             {Principe du développement durable \protect\footnotemark (a). Principe du
              scénario \textit{négaWatt} (b) \parencite{Salomon2012}.}
    \label{fig:developpement_durable_negawatt}
\end{figure}
\footnotetext{\url{http://www.cdcmortainais.fr}}

Les décisions politiques étant étroitement liées aux résultats scientifiques,
une organisation indépendante a été créé en $1988$ par l’Organisme des Nations
Unies (\abr{ONU}) afin de fournir une base scientifique solide. Cette organisation connue sous le nom de
Groupe d'experts Intergouvernemental sur l'Évolution du Climat (\abr{GIEC}), regroupe
des experts en climatologie dont le but est d’apporter un regard critique sur
les travaux provenant de différents groupes de recherche scientifiques, techniques,
et socio-économiques. Il a pour but, la réalisation sans parti-pris de l’expertise
des risques liés au réchauffement climatique et de faire re-sortir les informations
et découvertes faisant consensus dans la communauté scientifique. Grâce à ces
publications périodiques, les connaissances accumulées sur le sujet permettent de
sensibiliser le public non expert, élément indispensable pour que des engagements
politiques soient pris.

Peu de temps après, au $3^{ème}$ \textit{Sommet de la Terre} se tenant à Rio en $1992$, la
Convention-Cadre des Nations Unies sur les Changements Climatiques
(\fnref{http://unfccc.int/essential_background/convention/status_of_ratification/items/2631.php}{\abr{CCNUCC}})
est adoptée. Cette convention développe un plan d’action pour le $21^{ème}$ siècle,
l’\textit{Agenda $21$} mais aussi $27$ principes pour sa mise en œuvre. Les domaines
traités sont très variés et couvrent par exemple la pauvreté, la gestion des ressources
notamment en eau, des déchets, mais aussi la pollution ou l’agriculture\dots\  Elle est
ensuite complétée par un accord visant à réduire les émissions de gaz à effet de serre
(\abr{GES}) lors de la $3^{ème}$ Conférence des Parties (\abr{COP}, Conference Of Parties) à
Kyoto ($1997$) avec une entrée en vigueur début $2005$. La \abr{CCNUCC} est aujourd’hui
ratifiée par $197$ pays montrant l’intérêt des questions du développement durable sur le
plan international. Pour la première fois, en $2006$, le \textit{rapport Stern}
\parencite{Stern2006} commandé par le gouvernement britannique décrit l’impact
économique d’une inaction face aux problèmes du réchauffement climatique.
Précédemment mis en avant par des scientifiques, la sonnette d’alarme est ici tirée par un
économiste. Le
réchauffement impacte en effet directement les composantes essentielles de notre mode de
vie~: l’accès à la santé, à la nourriture ou encore à l’eau. Il est alors suggéré un effort
de recherche et une coopération technique au niveau international afin de répondre à la
problématique de manière efficace. Il est aussi mis en avant la nécessité pour les pays
riches, principaux responsables du réchauffement climatique, d’aider au développement des
pays plus pauvres dans des conditions respectueuses de l’environnement.

Dès $2007$, gouvernements et chefs d’états de l’Union Européenne
(\abr{EU}) définissent des objectifs globaux pour $2020$. Ces objectifs sont au cœur du
\fnref{http://www.assemblee-nationale.fr/13/europe/rap- info/i1260.asp}{paquet climat-énergie}
et engagent l’Europe à réduire de \SI{20}{\percent} (par rapport à $1990$) ces
émissions de \abr{GES}, à améliorer de \SI{20}{\percent} l’efficacité énergétique, et à
couvrir \SI{20}{\percent} de la consommation finale par des énergies renouvelables
(\abr{EnR}). Au niveau international, la $21ème$ \abr{COP} ($2016$) a permis d’aller
encore plus loin avec l’acception de la mise en place de mesures pour la réduction des
émissions de \abr{GES} avec notamment l’objectif de contenir le réchauffement climatique
en dessous des \SI{2}{\degree} (réalisation du plan facteur $4$) À terme, le respect de la trajectoire
permettra d’atteindre la \enquote{neutralité carbone} en compensant les émissions de
\abr{GES} dans la seconde moitié du siècle. Contrairement au précédent accord
(\textit{Accords de Kyoto}), le parti-pris est ici d’imposer une transparence entre les
différents signataires. Chaque signataire doit ainsi rendre compte régulièrement des
objectifs et évolutions réalisées, informations qui seront disponibles publiquement afin
d’inciter à l’exemplarité. Ancrées d’une forte symbolique et marquant un pas important vers
un développement plus durable, aucune prise de mesure n’est cependant obligatoire et
la sobriété énergétique n’est pas mentionnée.


\subsubsection{Le cas français} % (fold)
\label{ssub:le_cas_francais}
En France, l’engagement \enquote{facteur 4} ($2003$) et le paquet climat-énergie défini
au niveau européen sont repris dans les objectifs des lois \textit{Grenelle I et II}
\parencite{Grenelle2010}. Ce texte traduit pour la France une volonté de décisions sur le
long terme à la fois sur le plan sociétal, politique, et économique, même si la question
du nucléaire n’est pas encore à l’ordre du jour. Ces lois marquent le premier pas vers une
démarche responsable sur les transports, la santé, l’énergie ou encore la préservation de
la bio-diversité comme notamment la labellisation de l’agriculture \emph{BIO} ou la
réduction de la précarité énergétique des foyers. Ces mesures s’appuient notamment sur le
plan de transition énergétique proposé par l’association \textit{négaWatt}
\parencite{Salomon2012}.

\textit{NégaWatt} est une association française qui développe depuis $2003$ un scénario de
transition énergétique permettant à l’horizon $2050$ de couvrir les besoins énergétiques à
\SI{100}{\percent} grâce aux énergies renouvelables~: biomasse, éolien, solaire \parencite{negaWatt2017}. La
trajectoire proposée suit trois axes principaux (\figref{fig:negawatt_axes})~: plus de
sobriété, améliorer l’efficacité énergétique, promouvoir les énergies renouvelables.
L’association regroupe dans cet objectif, des experts du domaine de l’énergie, des
économistes, des sociologues\dots, et tient compte des données statistiques sur des
domaines multiples comme l’évolution démographique, la consommation des foyers, le coût
des énergies\dots\ La sobriété élément clé du scénario invite à repenser son mode de vie et
propose ainsi à la fois une transition énergétique mais aussi comportemental basé sur un
principe simple à l’origine du mot \textit{négawatt}~:
\blockquote{L’énergie qui pollue le moins est celle que l’on ne consomme pas.}
Parmi les nombreuses mesures concrètes proposées, il peut être cité, l’amélioration des transports en
commun, la réduction des dépenses électriques inutiles, l’isolation du bâti existant,
l’économie circulaire, ou encore un programme de sortie progressive du nucléaire pour
$2035$.

Depuis $2012$, au niveau de l’agriculture et donc de l’accès à la nourriture, le scénario \textit{négaWatt}
est complété par le scénario \textit{Afterres2050} développé par l’\abr{ONG} \textit{Solagro}. Ce
scénario s’intéresse principalement à la gestion des forêts, la biodiversité, ou encore
les terres agricoles afin de nourrir de manière durable la population française.
Le scénario met notamment en avant le fait que le respect du facteur $4$ couplé à une alimentation
et une agriculture plus raisonnée permettrait de couvrir les besoins alimentaires
de l’ensemble des français. S’appuyant sur le principe de sobriété, le scénario
indique aussi qu’une réduction par deux de la consommation de viande est
nécessaire afin d’une part de réduire les émissions de gaz à effet de serre mais
aussi afin de libérer des terres pour la production de biomasse.
Le travail de ces associations est rapidement reconnu et s’inscrit en $2015$ dans le cadre de
de la Loi pour la Transition Énergétique et la Croissance Verte
(\fnref{https://www.legifrance.gouv.fr/affichTexte.do?cidTexte=JORFTEXT000031044385}{\abr{LTECV}})
comme un des scénarios de maîtrise de l’énergie au coté du scénario de l’\textit{ADEME}
basé sur l’efficacité, du scénario \textit{ANCRE} basé sur la diversité des énergies, et du scénario \textit{Négatep}
basé sur la dé-carbonisation.
La loi décrit ainsi de nombreux engagements visant à réduire l’impact de l’homme sur le
climat, créer de l’emploi pour une population grandissante, mais aussi réduire la précarité
avec un plan de rénovation du patrimoine bâti. Des engagement forts sont ainsi pris au
niveau énergétique et environnemental pour $2030$~:
\begin{itemize}
    \item réduire de \SI{40}{\percent} (par rapport à $1990$) les émissions des gaz a
          effet de serre et par quatre à l’horizon $2050$.
    \item réduire la consommation en énergie finale de \SI{20}{\percent} (par rapport à $2012$)
          et de \SI{50}{\percent} pour $2050$.
    \item réduire de \SI{30}{\percent} la consommation en énergie primaire.
    \item porter à \SI{32}{\percent} la part des \abr{EnR} sur la consommation finale brute.
    \item réduire la dépendance à l’énergie nucléaire pour la production d’électricité à
          \SI{50}{\percent} évaluée à \fnref{http://tinyurl.com/y7ujs7dn}{\SI{77}{\percent}} en $2014$
\end{itemize}

La définition du contexte général met en exergue la nécessité de réduire la consommation
mondiale et de couvrir les consommations restantes avec des énergies renouvelables. Dans
cet optique, ces travaux cherche à développer une méthodologie permettant d’aider au
développement de bâtiments où la consommation en énergie non-renouvelable est minimale.
L’énergie solaire est substituée aux énergies non-renouvelables pour couvrir les besoins
d’un bâtiment à faible consommation grâce à un \abr{SSC} et à une production
photovoltaïque permettant de couvrir les besoins thermiques et électriques du bâtiment. La
méthodologie s’appuie sur le concept de \abr{BEPOS} et en particulier sur le cadre
normatif européen en construction où le principe de sobriété est mis en avant. Le concept
de \abr{BEPOS} étant complexe et sa définition est le résultat de nombreux travaux au
niveau international, européen, et national décrits dans la section suivante.
% subsubsection le_cas_francais (end)
% subsection vers_une_prise_de_conscience_collective (end)
% section l_homme_au_centre_d_un_trouble_climatologique (end)





% ..............................................................................
% ..............................................................................
\section{Vers le bâtiment à énergie positive} % (fold)
\label{sec:vers_le_batiment_a_energie_positive}
Représentant la majeure partie des consommations en énergie dans le monde, le secteur
du bâtiment a un rôle déterminant à jouer dans le respect des engagements politiques.
Dans cette optique la réglementation thermique s’est renforcée au fur et mesure des années.
En $2020$, la réglementation imposera le concept de \abr{BEPOS} pour tous
nouvelles constructions.
Pour accompagner les concepteurs dans cette tâche difficile, ces travaux proposent le
développement d’une méthodologie d’aide à la décision en s’appuyant sur le concept de
\abr{BEPOS} et particulièrement sur le cadre européen (\abr{PREN\,$15603$}) qui sera
traduit au niveau national dans la future réglementation. Les nouveaux bâtiment devront
avoir une enveloppe performante, produire autant d’énergie primaire que consommée sur
site (bilan positif), et une part importante des consommations devra être couverte par des énergies
renouvelables produites \emph{localement}. Ainsi couplée à une enveloppe performante, le
solaire est une solution pertinente~: les besoins thermiques pouvant être couverts par un
\abr{SSC} et les besoins électriques par un système photovoltaïque.


Cette section introduit ainsi l’évolution réglementaire à l’origine du concept de
\abr{BEPOS}.Suit un descriptif des travaux existants au niveau international, européens,
et français.


% ------------------------------------------------------------------------------
\subsection{Un long chemin déjà parcouru} % (fold)
\label{sub:un_long_chemin_deja_parcouru}
En France selon \textcite{ADEME2015}, le secteur du bâtiment est responsable à
\SI{45}{\percent} des consommations en énergie finale devant les transport
(\SI{33}{\percent}) et l’industrie (\SI{19}{\percent})
(\figref{fig:evolution_energy_finale}). C’est de plus le troisième secteur le plus
polluant avec \SI{20}{\percent} des émissions en $CO_{2}$(eq) derrière
l’industrie (\SI{30}{\percent}) et les transports (\SI{28}{\percent}).

\begin{figure}
    \centering
    \begin{subfigure}[b]{0.45\textwidth}
        \centering
        \includegraphics[width=\textwidth]{Ressources/Images/Environnement/evolution_energie_finale.png}
        \caption{}
        \label{fig:evolution_energy_finale}
    \end{subfigure}
    \quad
    \begin{subfigure}[b]{0.45\textwidth}
        \centering
        \includegraphics[width=\textwidth]{Ressources/Images/Environnement/repartition_energy_primaire.png}
        \caption{}
        \label{fig:repartition_conso_primaire}
    \end{subfigure}
    \caption[Description du secteur énergétique français]
             {Évolution de la consommation française en énergie finale (a) et
              répartition de la consommation française en énergie primaire en $2012$ (b)
              d’après \textcite{ADEME2015}.}
    \label{fig:energy_france}
\end{figure}

C’est suite au premier choc pétrolier de $1973$ que la France met en place en $1974$ la première
Réglementation Thermique (\abr{RT}) afin de faire face aux dépenses énergétiques importantes
dans le secteur du bâtiment. L’énergie qui hier semblait inépuisable apparaît
alors comme une ressource limitée et la sécurité de l’approvisionnement devient une
préoccupation nationale. D’après l’\abr{ADEME}, la consommation annuelle en énergie finale est
alors de l’ordre de \SI{370}{kWh\per\metre\squared} dans le secteur du résidentiel et la
réglementation fixe un objectif de réduction de la consommation de
\SI{25}{\percent}. Une méthodologie de calcul commune est aussi mise en place et
un contrôle régulier des systèmes de chauffage comme de climatisation imposée.
Cette réglementation évolue ensuite en $1982$, $1989$, puis $2000$ portée par
les différents acteurs du secteur du bâtiment.

Dans une approche volontariste, les états européens adoptent en $2002$, la directive sur la
performance énergétique des bâtiments (\abr{2002/91/EC}, \cite{EPBD2002}) qui fut à
l’origine d’avancées significatives pour tous les états membres avec notamment la mise en
place des diagnostics de performance énergétique (\abr{DPE}), des valeurs seuils et une
méthodologie commune pour évaluer la performance énergétique des bâtiments, le contrôle
régulier des chaudières (\SI{+10}{kW}) et un effort de normalisation commun par le Comité
Européen de Normalisation (\abr{CEN}).
Ces engagements ont été traduits par la plupart des pays membres dans leur \abr{RT}
propre comme la France à travers les
disposition des lois Grenelle et les \abr{RT}\,$2005$.
Mise en application en $2006$, elle marque de nombreuses avancées
réglementaires notamment avec l’introduction du concept de bâtiment de référence. Chaque
bâtiment est ainsi évalué par rapport à la référence et des gardes-fous sont mis en place
afin d’éviter les incohérences énergétiques.
Ces mesures incitatrices sont aussi à l’origine de la publication dans le journal officiel
en mai $2007$ de l’arrêté encadrant l’obtention d’un label \abr{HPE} qui permet de valoriser
un bâtiment obtenant un niveau de performance globale supérieur à la \abr{RT}.
Le label comporte cinq niveaux dont le plus performant, le label Bâtiment à
Basse Consommation (\fnref{https://www.effinergie.org/web/index.php/les-labels-effinergie/bbc-effinergie}{\abr{BBC}-$2005$})
est inspiré du label Suisse \textit{Minergie}. Pour la première fois en France, un
label d’excellence énergétique est proposé et impose une consommation en énergie primaire
sur le chauffage, la production d’\abr{ECS}, le refroidissement, et l’éclairage à
\SI{50}{kWh\per\metre\squared} (modulé en fonction des conditions géographiques).
La mise en place de ce label a permis aux professionnels de s’adapter aux contraintes
énergétiques fortes qui seront reprises dans la réglementation actuellement en vigueur, la
\abr{RT}\,$2012$.
En effet le passage à la \abr{RT\,$2012$} impose aux bâtiments une consommation maximale (\abr{$C_{ep}$\,max})
réduite de moitié. De plus le nouveau seuil tient aussi compte de l’éclairage et des auxiliaires alors
que dans la \abr{RT\,$2005$}, le \abr{$C_{ep}$\,max} est défini uniquement pour le
chauffage, la production d’\abr{ECS}, et le refroidissement. Les travaux se concentrent
aujourd’hui sur la préparation de la \abr{RT\,2020}, qui devra permettre d’honorer les
engagements européens et se rapprocher des objectifs internationaux pour $2030$ et $2050$.

\subsubsection{La place de la maison individuelle} % (fold)
\label{ssub:la_place_de_la_maison_individuelle}
Si on considère uniquement le parc résidentiel, la consommation moyenne
est de \SI{186}{\kWh_{ep}\per\metre\squared} en $2012$ dont \SI{67}{\percent} due
au chauffage. Les ménages représentent ainsi \SI{25}{\percent} des consommations
en énergie primaire (\figref{fig:repartition_conso_primaire}). Finalement malgré
une réduction de la consommation moyenne des logements jusqu’en $2006$, la
consommation augmente entre $2002$ et $2012$. Il est donc important
de faire des économies d’énergie dans le bâtiment et en particulier dans le résidentiel
afin d’une part de respecter les engagements mais aussi réduire la facture énergétique.
D’après l’Institut National de la Statistique et des Études Économiques (\abr{INSEE})
la maison individuelle représente en $2016$ \SI{56}{\percent} des \SI{35}{millions}
de logements en France.
Rénover le patrimoine existant est la mesure la plus importante afin de réduire
les consommations en énergie fossile, réduire notre dépendance à l’énergie nucléaire,
et limiter les émissions de \abr{GES}. Il est cependant important de continuer à
innover dans le neuf et par extension dans la maison individuelle.
Malgré une diminution en faveur des habitats collectifs depuis $2008$, la maison individuelle
représente en effet près de \num{200000} nouveaux chantiers terminés par an, dont $\nicefrac{3}{4}$
de maisons individuelles pures \parencite{Caicedo2015}.
Ainsi les nouvelles constructions de maisons individuelles se doivent d’êtres innovantes
énergétiquement afin de réduire le coût des rénovations futures. C’est dans cette optique que le concept de \abr{BEPOS} a été
développé.
% subsubsection la_place_de_la_maison_individuelle (end)
% subsection un_long_chemin_deja_parcouru (end)



% ------------------------------------------------------------------------------
\subsection{Description générale du concept de \abr{BEPOS}} % (fold)
\label{sub:description_generale_du_concept_de_bepos}
Bien qu’il n’existe pas encore de cadre réglementaire en France,
la directive européenne \abr{2010/31/EU} \parencite{EPBD2010} (mise à jour de \textcite{EPBD2002})
décrit un programme commun pour les pays de l’\abr{EU}.
Elle introduit le concept \enquote{nearly Zero Energy Buildings}
(\textit{nearly}\,\abr{ZEB}) incitant chaque état membre à mettre en
œuvre une politique permettant d’atteindre pour tous les bâtiments neufs une consommation
énergétique en énergie primaire \enquote{quasi nulle} en $2018$ pour les bâtiments publics
et en $2020$ pour le privé (article $2$ et $9$).

En France cette initiative prend le nom de Bâtiment à
Énergie POSitive (\abr{BEPOS}), indiquant que le bâtiment produit plus qu’il ne consomme~; définition
à laquelle la directive ajoute les contraintes suivantes~:
\begin{itemize}
    \item Les bilans énergétiques doivent être réalisés en énergie primaire
    \item La consommation doit être couverte par des sources d’énergies renouvelables
    \item La production doit être réalisée localement
\end{itemize}
Le concept reste ainsi fortement interprétable et aucune indication concrète
permettant d’atteindre ces objectifs n’est pour le moment fournie.
Pour cette raison, de nombreuses définitions ont été développées au sein du globe,
si bien que la commission européenne a mandaté une expertise permettant de classer ou
tout du moins d’identifier les différences \parencite{ECOFYS2013}. En effet, chaque pays
porte ses propres convictions, sa propre culture du bâtiment et de la construction, sa
propre politique, son propre mix énergétique, ses propres contraintes d’accès à l’énergie,
et bien sûr son propre climat.
Au final le rapport suggère l’élaboration de près de $75$ définitions différentes et
soulève de nombreux questionnements sur la bonne manière de procéder. Afin de pouvoir
proposer des solutions techniques et technologiques innovantes mais atteignable pour
$2020$ des initiatives au niveau national, européen, et international voient le jour.
Ces initiatives s’appuient en partie sur les retours d’expériences et l’analyse de bâtiments
performants et permettent de définir un cadre international.
% subsection description_generale_du_concept (end)



% ------------------------------------------------------------------------------
\subsection{La complexité de la définition d’un cadre international} % (fold)
\label{sub:la_definition_d_un_cadre_international}
Au niveau international le \abr{BEPOS} connu sous l’acronyme \abr{ZEB} a fait l’objet d’un
travail important et novateur. Le concept est né en réponse aux problématiques climatiques
et la nécessité de repenser la manière de construire les bâtiments qui représente aux
États-Unies \SI{40}{\percent} de la consommation en énergie primaire et plus de
\SI{70}{\percent} de la consommation électrique \parencite{Torcellini2006a}. Il restait
alors à définir un concept commun et fournir les éléments aidant à sa mise en place. En
effet même si l’idée fait consensus, il restait encore à définir les indicateurs à retenir
ou encore l’adéquation entre besoins et demande. Il est aussi nécessaire de définir les
frontières, comme par exemple le terme \enquote{locale}, la période considérée pour son
évaluation, ou les flux considérés. Ces questions sont traitées dans le cadre de la
tâche\,40 (\enquote{Net Zero Energy Solar Buildings}) et de l’annexe $52$ de l’\abr{ECBCS}
(\enquote{Towards Net Zero Energy Solar Buildings}) portées par l’Agence Internationale de
l’Énergie (\abr{IEA}) \parencite{Athienitis2015}. Les travaux se focalisent sur la
définition d’un \abr{BEPOS} connecté (\textit{Net}\,\abr{ZEB}, Net Zero Energy Building)
où la production renouvelable alimentant le réseau permet d’équilibrer les consommations
du bâtiment. À l’horizon $2011$, les nombreux travaux et maisons de démonstration
permettent d’alimenter les différentes bases de données et de nombreuses méthodes sont
identifiées \parencite{Marszal2011971}.


% - - - - - - - - - - - - - - - - - - - - - - - - - - - - - - - - - - - - - - -
\subsubsection{Le choix des indicateurs} % (fold)
\label{ssub:le_choix_des_indicateurs}
Afin de pouvoir évaluer le caractère positif d’un bâtiment, il est dans un premier
temps nécessaire de définir un système d’unité, un indicateur suivant lequel
s’appuyer. \textcite{Torcellini2006} propose quatre indicateurs permettant
d’approcher de manière différente le concept de \textit{Net}\,\abr{ZEB}~:
\begin{blockdescription}{<< Net Zero Energy Emissions >>~:}
    \item[\enquote{Net Zero Site Energy}~:] La production renouvelable sur site
          doit pouvoir compenser la consommation finale du bâtiment.
    \item[\enquote{Net Zero Source Energy}~:] La production renouvelable transmise au réseau
          doit permettre de compenser la consommation primaire du bâtiment.
    \item[\enquote{Net Zero Energy Costs}~:] Le gain d’argent engendré par la vente
           de la production locale compense au minimum la consommation acheté.
    \item[\enquote{Net Zero Energy Emissions}~:] La part renouvelable doit être assez
           importante pour couvrir les émissions engendrés par la consommation non
           renouvelable telles que les énergies fossiles.
\end{blockdescription}
L’auteur met aussi en avant les avantages et limites de chaque approche. Le choix de
l’énergie primaire permet d’avoir une cohérence international mais apporte une complexité
liée aux facteurs de conversion propres à chaque bouquet énergétique. Le calcul selon les
émissions est plus complexe car il faut définir quels critères retenir parmi les nombreux
critères disponibles~: émissions de \abr{GES}, énergie grise, déchets radioactifs
\dots\ L’utilisation de l’énergie finale ou du coût permet de simplifier sa mise en place
mais met au même niveau l’ensemble des moyens de production. L’approche utilisant
l’énergie primaire ressort de la comparaison comme étant l’indicateur le plus utilisé mais
certaines approches cumulent les indicateurs comme le traduit en France les différents
labels décrits ci-avant.
% subsubsection le_choix_des_indicateurs (end)


% - - - - - - - - - - - - - - - - - - - - - - - - - - - - - - - - - - - - - - -
\subsubsection{La méthodologie de calcul} % (fold)
\label{ssub:la_methodologie_de_calcul}
Le bilan du bâtiment sur l’indicateur retenu doit permettre d’obtenir à minima un
équilibre entre l’énergie exportée vers le réseau (production) et l’énergie importée par
le bâtiment (consommation). Les coefficients utilisés pour réaliser ce bilan sont souvent
considérés symétriques, l’énergie exportée vers le réseau représente autant d’énergie non
renouvelable évitée qui aurait été nécessaire pour sa production. Cette affirmation est
cependant uniquement valide lorsque l’énergie exportée ne nécessite pas la consommation
d’énergie supplémentaire \parencite{Sartori2012220}. En effet l’ajout par exemple d’un
système photovoltaïque entraîne une consommation d’énergie grise pour son installation et
son entretien. L’auteur décrit une approche asymétrique par pondération et explique que sa
mise en place résulte d’une volonté politico-économique. Par exemple, en France, afin de favoriser
l’adoption des énergies nouvelles, l’énergie produite sur site est revendue plus chère que
celle achetée.
L’approche la plus courante cherche à évaluer le bilan énergétique sur une fréquence annuelle
même si certaines approches calculent les indicateurs sur la durée de vie complète
du bâtiment afin de tenir compte de son vrai impact environnemental \parencite{Voss201146}.
% \textcite{Voss201146} mettent par exemple en exergue l’importance de la part des énergies grises
% qui tient entre \SI{20}{\percent} et \SI{30}{\percent} pour une durée de vie du bâtiment
% fixée à \SI{80}{ans}.

\textcite{Sartori2012220} proposent trois types de bilan (\figref{fig:bilan_zeb})~:
\begin{itemize}
    \item charge / production (\enquote{load generation balance})
    \item importation / exportation (\enquote{import/export balance})
    \item mensuel (\enquote{monthly net balance})
\end{itemize}

Le bilan \textit{charge / production} est défini sur une base annuelle et tient compte des
interactions entre le bâtiment et le réseau mais nécessite pour ce faire un pas de
simulation faible afin d’estimer l’autoconsommation. Le choix d’un pas de temps plus
réduit induit l’estimation plus précise du comportement des occupants à travers les
consommations électro-domestiques ou le puisage en \abr{ECS}. Dans la seconde approche,
équilibre \textit{importation / exportation}, les énergies respectives de chaque coté du
bilan sont dans un premier temps pondérées par leurs coefficients respectifs puis sommées.
Ainsi, l’ensemble de la production est exportée vers le réseau et l’ensemble de la consommation est fournie par le réseau. La seconde
approche est ainsi plus simple à mettre en place, expliquant son adoption par un plus
grand nombre de méthodologies. La première approche peut cependant être retenue
expérimentalement sur les maisons pilotes et donner de précieuses informations. Finalement
le dernier bilan proposé est \textit{mensuel} et permet de mieux évaluer l’équilibre ou
le déséquilibre existant entre la production et la consommation soulevant la question de
l’adéquation entre production et consommation qui est discutée après avoir établi un cadre
clair des frontières du bilan.
\begin{figure}
    \centering
    \includegraphics[width=0.8\textwidth]{Ressources/Images/Environnement/bilan_nzeb.png}
    \caption[Représentation schématique du bilan d’un \textit{Net}\,\abr{ZEB}]
            {Représentation schématique de trois méthodes utilisées pour faire
             le bilan d’un \textit{Net}\,\abr{ZEB} selon \textcite{Sartori2012220}.}
    \label{fig:bilan_zeb}
\end{figure}

\textcite{Sartori2010} mettent aussi en avant la question des données météorologiques et
en particulier l’utilisation de fichiers typiques construits sur la connaissance du passé.
\textcite{Robert2012150} montrent qu’il est possible grâce à une technique de morphing
\parencite{Belcher200549} de générer des fichiers tenant compte des prévisions climatiques
afin de simuler le temps de demain et non celui d’hier.

% subsubsection la_methodologie_de_calcul (end)


% - - - - - - - - - - - - - - - - - - - - - - - - - - - - - - - - - - - - - - -
\subsubsection{Les frontières considérées} % (fold)
\label{ssub:les_frontieres_considerees}
\paragraph{Frontières physiques} % (fold)
\label{par:frontières_physiques}
La définition d’un \abr{ZEB} implique un bilan positif grâce à une production dite
\enquote{locale} dont le périmètre physique reste à définir.
\textcite{Torcellini2006} proposent pour la définition d’un \abr{ZEB} de favoriser dans un
premier temps les solutions permettant de réduire les consommations comme l’isolation ou
l’efficacité des systèmes. Dans un second temps, les solutions favorisant
la production d’énergie renouvelable locale (\enquote{On-Site}), et enfin
l’utilisation de sources extérieures (\enquote{Off-Site}). Au niveau du site, une
distinction est faite entre la production \emph{sur} le bâtiment et \emph{au} niveau de la
parcelle détenue par le propriétaire avec une préférence pour le premier.
Au niveau des sources extérieures, l’utilisation de la bio-énergie est favorisée et
le recours au réseau apparaît comme la dernière option.
\textcite{Marszal2010} proposent une représentation graphique où est identifié le
périmètre physique sans pour autant mettre en avant un quelconque classement de
préférence. Il ajoute aussi une distinction supplémentaire au niveau des ressources
extérieures au site en séparant l’investissement du propriétaire dans un site extérieur
pour produire de l’énergie renouvelable, et le recours au réseau.
% paragraph frontières_physiques (end)

\paragraph{Les usages~:} % (fold)
\label{par:les_usages}
La définition d’un \abr{BEPOS} met en avant l’utilisation d’énergies renouvelables afin de
compenser les consommations sur site. Uniquement chercher à obtenir un bilan positif
apparaît cependant comme une solution limitée excluant les systèmes de cogénération qui
sont pourtant des solutions qui permettent de maximiser l’utilisation des énergies non
renouvelables \parencite{Sartori2010}.
\textcite{Marszal2011971} mettent aussi en exergue la diversité observée sur les usages considérés pour le calcul
des indicateurs. Les approches les plus anciennes ne considèrent que les principaux usages~: chauffage
et production d’\abr{ECS} alors que d’autres plus récentes tiennent aussi compte de l’éclairage,
du refroidissement, et des auxiliaires. La diminution des usages courants met en aussi
sur le devant de la scène les autres usages auparavant négligés comme les consommations électro-domestiques.
% paragraph les_usages (end)
% subsubsection les_frontieres_considerees (end)


% - - - - - - - - - - - - - - - - - - - - - - - - - - - - - - - - - - - - - - -
\subsubsection{Adéquation temporelle avec le réseau~:} % (fold)
\label{ssub:adequation_temporelle_avec_le_réseau}
Un \textit{Net}\,\abr{ZEB} est par définition connecté au réseau, cependant, l’obtention
d’un bilan positif au niveau annuel ne permet pas d’assurer la symétrie des échanges sur
toute l’année. En effet, la production d’énergie renouvelable est en majorité assurée par
une production photovoltaïque dont les pics de production en période estivale ne sont pas
en adéquation avec la forte demande énergétique en période hivernale. La difficulté réside
ainsi dans la gestion des pointes de consommation (hiver), des excédents de production
(été), et de la non-continuité de la production renouvelable (éolien, photovoltaïque).
Cette mutation dans le bouquet énergétique a donc imposé le développement des
\enquote{Smart Grid} ou réseaux intelligents afin de répondre aux problèmes d’intermittence
des principales sources d’énergies renouvelables actuellement utilisées pour générer de
l’électricité. Des indicateurs ont alors été développés afin d’évaluer l’adéquation entre
le bâtiment et le réseau. \textcite{Voss2010} font la différence entre les interactions au
niveau du bâtiment (\enquote{load maching}) ou entre le bâtiment et le réseau
(\enquote{grid interaction}). La première famille d’indicateurs permet d’évaluer le niveau
de concomitance entre la demande du bâtiment et la production sur site. La seconde permet
d’évaluer la correspondance temporelle entre exportation du bâtiment et besoins au niveau
du réseau. Une référence complète des différents indicateurs trouvés dans la littérature
est proposée par \textcite{Salom2011} puis intégrée dans les travaux de l’annexe $52$
\parencite{Salom2014}.
% subsubsection adequation_temporelle_avec_le_réseau (end)


% - - - - - - - - - - - - - - - - - - - - - - - - - - - - - - - - - - - - - - -
\subsubsection{Mesures et vérifications~:} % (fold)
\label{ssub:mesures_et_verifications}
Dans le cadre de la sous tâche \abr{A} \parencite{Noris2013}, le monitoring apparaît comme indispensable
afin de vérifier par la mesure la performance du bâtiment en la confrontant aux
résultats de simulation. Il est recommandé à minima de vérifier le respect du bilan
positif qui est le concept central des \abr{BEPOS}. Bien que ne faisant pas partie de
la définition du concept, un \textit{Net}\,\abr{ZEB} doit avant tout être agréable à vivre. Dans cette
optique le confort intérieur des occupants doit être analysé à travers une étude de
Qualité de l’Air Intérieur (\abr{QAI}) permettant de s’assurer que la performance du bâtiment n’est pas
obtenue en sacrifiant le bien être des occupants. Un \textit{Net}\,\abr{ZEB} étant par définition connecté,
l’adéquation du bâtiment avec le réseau est aussi une des principales recommandations.
Finalement, le rapport présente une procédure standardisée mettant en avant différentes échelles
de suivi du bâtiment ainsi qu’un outil d’analyse facilitant la comparaison des divers projets entre eux.
% subsubsection mesures_et_verifications (end)
% subsection la_definition_d_un_cadre_international (end)


% ------------------------------------------------------------------------------
\subsection{L’approche européenne} % (fold)
\label{ssub:l_approche_europeenne}
Au niveau européen, les exigences pour $2020$ sont décrites à travers la directive
\abr{2010/31/EU} \parencite{EPBD2010} qui esquisse la direction générale à prendre pour
parvenir aux \textit{nearly}\,\abr{ZEB}. Afin de faciliter l’application de la directive,
la commission européenne a mandaté le Comité Européen de Normalisation (\abr{CEN}) pour la
rédaction d’un cadre normatif~: \abr{PREN\,$15603$} \parencite{CEN2013}. La norme décrit
une définition flexible, et claire de la \textit{nearly}\,\abr{ZEB} afin d’inciter les
professionnels au développement de solutions innovantes mais doit aussi être suffisamment
générique afin de permettre sa traduction au niveau national par les différents états
membres \parencite{Zirngibl2014}.


% - - - - - - - - - - - - - - - - - - - - - - - - - - - - - - - - - - - - - - -
\subsubsection{Les frontière considérées} % (fold)
\label{ssub:les_frontière_considérées}
La directive \abr{2010/31/EU} considère que la part de la production considérée dans le
calcul du bilan énergétique doit être sur site
(\enquote{on-site}) ou à proximité (\enquote{nearby}). Les travaux du \abr{CEN} ajoutent une
description plus précise du périmètre en explicitant ces deux termes. La production sur
site considère uniquement le terrain où est localisé le bâtiment, alors que la production
à proximité intègre les sources locales à l’échelle du quartier autorisant ainsi la mutualisation
des productions en énergies renouvelables. La condition requise étant d’avoir
une connexion dédiée et un équipement spécifique caractérisant ce lien afin de pouvoir
calculer des coefficients de conversion en énergie primaire pour expliciter ce lien au
niveau énergétique.
% subsubsection les_frontière_considérées (end)



% - - - - - - - - - - - - - - - - - - - - - - - - - - - - - - - - - - - - - - -
\subsubsection{La méthodologie retenue} % (fold)
\label{ssub:la_methodologie_retenue}
Un \textit{nearly}\,\abr{ZEB} est un bâtiment dont les consommations sur les cinq usages
(chauffage, \abr{ECS}, éclairage, auxiliaires, et refroidissement)
sont très faibles et doivent être couverts par une production locale en énergie renouvelable.
Le \abr{CEN} propose une méthodologie sur quatre axes permettant de couvrir de manière
précise la définition point par point.

Le premier axe permet l’évaluation des besoins à travers l’analyse
de la qualité de l’enveloppe et du partitionnement du bâtiment~: isolation, inertie, qualité de l’air,
et conception bioclimatique. L’approche est ainsi similaire au calcul du \abr{$B_{bio}$} décrit dans
la \abr{RT\,$2012$}.
Le second axe permet d’évaluer la performance du bâtiment et de ses systèmes à travers les
cinq usages. Un total en énergie primaire est retenu afin de tenir compte de l’ensemble
des pertes dues à l’acheminement, la transformation, et le stockage de manière équitable
pour toutes les énergies.
Le troisième axe est défini afin d’évaluer la proportion d’énergie renouvelable sur site en séparant
le calcul pour chaque vecteur énergétique. L’indicateur proposé est le \enquote{Renewable Energy Ratio}
(\abr{RER}) défini comme le rapport entre la consommation en énergie primaire et la consommation
en énergie primaire non renouvelable. Chaque usage étant séparé, la production des
capteurs \abr{PV} par exemple ne peut donc pas compenser le chauffage couvert par une chaudière
au gaz.
Finalement le dernier axe vérifie la performance globale du bâtiment en réalisant
le bilan entre les énergies importées et exportées. Le bilan étant réalisé en énergie
primaire, la consommation finale de chaque vecteur énergétique est pondérée par un
coefficient.
La méthodologie décrit ainsi de manière précise les attentes sur les besoins, les
consommations, la part des énergies renouvelables, et sur la balance énergétique globale
illustrée à travers la \figref{fig:attribution_nZEB}.
Enfin, le non-respect d’un des indicateurs respectifs de chaque axe disqualifie le bâtiment
qui ne peut donc pas accéder au titre de \textit{nearly}\,\abr{ZEB}.

\begin{figure}
    \centering
    \includegraphics[width=0.8\textwidth]{Ressources/Images/Environnement/hurdle_race.png}
    \caption[Description des quatre axes amenant à l’attribution du titre de
             \textit{nearly}\,\abr{ZEB}]
            {Description des quatre axes amenant à l’attribution du titre de
             \textit{nearly}\,\abr{ZEB} d’après \textcite{Zirngibl2014}.}
    \label{fig:attribution_nZEB}
\end{figure}
% subsubsection la_methodologie_retenue (end)
% subsection l_approche_europeenne (end)


% ------------------------------------------------------------------------------
\subsection{Les initiatives françaises} % (fold)
\label{sub:les_initiatives_francaises}
En France, le principe appliqué est celui du gagnant gagnant. Afin d’encourager la
construction de maisons performantes des aides sont proposées, aux futurs propriétaires
dans le privé, et aux collectivités dans le public. La conduite des travaux permet en retour aux entreprises de valoriser un
savoir faire et développer une expertise. Les retours techniques, technologiques,
économiques, et logistiques résultant amènent le secteur du bâtiment à se réorganiser tout
en aidant à la préparation de la future réglementation, la \abr{RT\,2020}.

L’association \fnref{https://www.effinergie.org/web/index.php/les-labels-effinergie/}{\textit{Éffinergie}}
propose ainsi en $2013$, deux nouveaux labels~: \textit{Effinergie\,+} et
\abr{BEPOS}-\textit{Éffinergie} permettant de valoriser une performance énergétique du
bâtiment supérieure au cadre réglementaire. Dans les deux cas la valeur maximale du
\abr{$B_{bio}$}, du \abr{$C_{ep}$\,max}, de l’étanchéité à l’air ou de l’efficacité des
équipements sont renforcés. Il est aussi nécessaire d’évaluer les consommations dues aux
équipements internes et de faire un suivi des consommations du bâtiment à l’utilisateur.
Le label \abr{BEPOS}-\textit{Éffinergie} va plus loin que le label \textit{Éffinergie\,+}
en rendant obligatoire l’évaluation en énergie grise sur le bâtiment mais aussi sur le
comportement des occupants (\fnref{https://www.effinergie.org/web/index.php/effinergie-ecomobilite}{Éco-mobilité}).
Finalement la production locale doit permettre de compenser la consommation amenant à un
bilan en énergie primaire positif (modulé suivant les règles de la \abr{RT\,$2012$} pour
tenir compte de la diversité des conditions françaises).

Plus récemment, en novembre $2016$, l’état ouvre la période d’expérimentation des
\abr{BEPOS} à bas carbone avec le programme Objectifs Bâtiments Énergie Carbone
(\abr{OBEC}) sur la base d’un nouveau référentiel~: le label \abr{E+/C-}
\parencite{Ministere2016}. Il permet de classer suivant quatre niveaux la performance énergétique,
et suivant deux niveaux le taux d’émission en $CO_{2}$ des bâtiments.
L’objectif cible est la préparation de la \abr{RT\,$2012$} à travers trois sous-objectifs~:
\begin{itemize}
    \item sensibiliser les filières du bâtiment
    \item améliorer la compétence des acteurs du bâtiment
    \item alimenter les bases de données économiques, énergétiques, et environnementales
\end{itemize}
Au niveau énergétique (partie \abr{E+}), l’obtention des deux premiers niveaux implique
l’amélioration du bâti et des systèmes. Le niveau trois nécessite en plus l’utilisation
d’un mode de production locale en $EnR$ pour compenser la consommation du bâtiment. Enfin
le dernier niveau indique que le bâtiment a un bilan négatif ou nul et qu’il contribue à
la production en $EnR$ du quartier. Pour la partie \abr{C-} le respect du label nécessite
de ne pas dépasser une valeur seuil sur respectivement le cycle de vie du bâtiment
(\abr{$Eges_{max}$}) et plus spécifiquement pour les matériaux de construction et
équipements (\abr{$Eges_{PCE, max}$}).

L’association \textit{Éffinergie} propose, trois nouveaux labels sur la base du référenciel
\abr{E+/C-}~: \abr{BBC} \textit{Éffinergie}\,$2017$, \abr{BEPOS} \textit{Éffinergie}\,$2017$,
\abr{BEPOS}\,+ \textit{Éffinergie}\,$2017$, qui impliquent respectivement le respect à
minima d’un niveau $2$, $3$, et $4$ sur le critère \abr{E+} et de $1$ sur le critère \abr{C-}.
De par leurs exigences, ces nouveaux labels s’inscrivent dans une
démarche plus globale en tenant compte de la performance énergétique mais aussi de
l’émission des \abr{GES}, du confort des occupants et de la sobriété qui est pour rappel
l’élément clé du scénario $négaWatt$.
Concernant la partie \abr{C-} ces labels n’exigent cependant pas une performance
supplémentaire et se contentent même du premier niveau.

Afin de certifier son bâtiment comme étant très peu émetteur de \abr{GES}, le label bas carbone de l’association
\fnref{http://www.certivea.fr/offres/label-bbca-batiment-bas-carbone}{\abr{BBCA}}
peut être demandé. Le pré-requis pour l’évaluation du bâtiment est l’obtention
du niveau $2$ sur le critère \abr{C-} du référenciel $E+C-$. L’évaluation se fait
suivant quatre grandes étapes et un système de points permet d’évaluer le niveau du
bâtiment d’où découle un classement qualitatif~: \abr{BBCA}, \abr{BBCA}
\textit{Performance}, et \abr{BBCA} \textit{Excellence}. L’indicateur évalue à travers ce
processus les réductions de \abr{GES} (construction et utilisation), l’économie
circulaire, et l’utilisation de matériaux bio-sourcés.

Finalement à travers le projet de recherche \abr{COMEPOS} il est testé la faisabilité
technique comme économique de technologies innovantes afin de mieux maitriser les surcoûts
de construction d’une \abr{MEPOS}. \abr{COMEPOS} est un projet visant à traduire
le concept de \abr{BEPOS} pour la maison individuelle regroupant à la fois des
universitaires, des industriels, et des constructeurs immobiliers travaillant de
concert. Le projet a pour vocation de permettre l’accélération de la mise en place sur le
marché des \abr{MEPOS} en montrant leur faisabilité à travers $25$ démonstrateurs
innovants répartis dans toute la France. Chaque maison propose une conception différent,
que ce soit sur la géométrie ou l’enveloppe du bâti, les systèmes retenus, ou encore la
clientèle ciblée. Le projet est organisé en six lots~:
\begin{enumerate}
  \item Identification des verrous et proposition d’un concept de \abr{BEPOS}.
  \item Choix des équipements et composants innovants à mettre en place dans chaque
        démonstrateurs.
  \item Évaluation de la faisabilité à travers des simulations dynamique du comportement
        du bâtiment et une analyse du cycle de vie.
  \item Construction des démonstrateurs.
  \item Suivi de la performance en temps réel de chaque démonstrateur~: analyse du pilotage,
        retour d’expérience des occupants.
  \item Valorisation et suivi du projet.
\end{enumerate}
L’objectif principal du projet est la garantie de performance en confrontant les résultats
obtenus par la simulation à ceux obtenus par monitoring lors de l’occupation des
démonstrateurs. L’expérience acquise permettra ainsi aux constructeurs de mieux anticiper
les besoins et contraintes propre à la construction de futurs \abr{MEPOS} à l’horizon
$2020$.

\paragraph{} % (fold)
La France a ainsi été très active tant au niveau réglementaire que incitatif et propose
aujourd’hui des certifications permettant à un propriétaire ou un maître d’ouvrage de
valoriser un choix réfléchi tant sur le plan énergétique, que environnemental, social ou
sociétal. Les données recueillies permettront ainsi de mieux préparer les acteurs
du bâtiment aux contraintes de la future réglementation nécessaire pour respecter les engagements sur le climat.
% subsection les_initiatives_francaises (end)


% ------------------------------------------------------------------------------
\subsection{Bilan sur la maison à énergie positive} % (fold)
\label{sub:bilan_sur_la_BEPOS}
Le \abr{BEPOS} pourtant simple dans l’idée nécessite la prise en compte de nombreux
facteurs~: le choix du ou des indicateurs évalués, l’intermittence des énergies, les
usages pris en compte, la période et les limites physiques considérées, le type de bilan,
et finalement le processus de vérification. Les divers travaux internationaux ont permis
de mieux comprendre le processus de construction d’un \abr{BEPOS} en proposant une
synthèse des travaux effectués. De plus le nombre important de projets innovants ou de
démonstrations de faisabilité permet d’identifier forces et faiblesses de chaque
approche. Au niveau européen, un processus de normalisation est en cours (\abr{PREN\,$15603$}).
Suffisamment générique mais précise, cette
norme sera appliquée au niveau national dans le respect des spécificités de chaque pays.
C’est dans cette optique que l’état français ouvre la période d’expérimentation des
\abr{BEPOS} à bas carbone (programme \abr{OBEC}) sur la base du nouveau référentiel~: le label \abr{E+/C-}.
Hier un concept, les travaux au niveau international ont permis de décrire un cadre commun
et de vérifier la viabilité de l’approche. Les travaux en cours au niveau national comme
le projet \abr{COMEPOS} devront eux permettre de préparer les professionnels du bâtiment
et de mettre en exergue les forces et faiblesses de notre bouquet énergétique.


La conception d’une \abr{MEPOS} est donc complexe et fait intervenir de nombreux
paramètres que ce soit au niveau des systèmes, de l’enveloppe, ou même des critères
propres à la définition retenue. La forte combinatoire résultant de ce constat nécessite
le développement d’une méthodologie adaptée. De plus afin de respecter les nouvelles
directives imposées par la norme européenne, une source locale d’énergie est nécessaire.
Celle-ci exclue l’énergie exportée vers le réseau et chaque vecteur énergétique doit être
évalué séparément. Ainsi la production locale revendu au réseau, par exemple par
l’intermédiaire des capteurs
\abr{PV} ne sera pas comptabilisée. De la même manière, les consommations en énergie
fossile (chaudières) ne pourront pas être compensés par une production
\abr{PV} même si le choix a été fait d’une installation en
auto-consommation car les deux vecteurs énergétiques diffèrent. Le solaire thermique et
par extension les \abr{SSC} apparaissent donc comme une solution pertinente pour permettre
l’obtention d’un bilan énergétique positif mais aussi assurer qu’une part suffisante en
énergie renouvelable soit produite localement. C’est dans cette optique qu’il est proposé
à travers ces travaux une méthodologie d’aide à décision pour la conception de \abr{MEPOS}
solaires. Bien que le \abr{SSC} soient en théorie une solution adaptée, il reste cependant
nécessaire d’évaluer le potentiel d’un couplage entre un \abr{SSC} et une \abr{MEPOS}

Ainsi, la section suivante introduit l’énergie solaire et les technologies existantes mais
aussi les limites et difficultés liées à la caractérisation et l’évaluation
de tels systèmes. Il est aussi décrit les systèmes d’équations utilisés dans le reste de
cette thèse pour calculer le rendement d’un capteur solaire thermique, ou encore, les
travaux dont le \abr{SSC} développé dans ce manuscrit s’est inspiré.
% subsection bilan_sur_la_BEPOS (end)
% section vers_le_batiment_a_energie_positive (end)





% ..............................................................................
% ..............................................................................
% irradiation / ensoleillement  / insolation = énergie / m2
% irradiance / rayonnement = densité de flux énergétique == densité de la puissance rayonnée = puissance / m2
\section{Les systèmes solaires} % (fold)
\label{sec:les_systemes_solaires}
Le solaire et plus particulièrement le solaire thermique est
une solution pertinente pour réduire les consommations d’une \abr{MEPOS} et permettre
de s’assurer qu’une part importante locale soit couverte par des énergies renouvelables.
L’avantage principal d’un système solaire thermique actif, est sa simplicité. Il est en
effet uniquement nécessaire d’avoir un capteur solaire, une pompe et un volume de stockage. C’est pour
cette raison que ces systèmes sont utilisés depuis longtemps avec plus ou moins
d’efficacité. Bien que relativement simple à mettre en place en comparaison avec les
autres alternatives, les \abr{SSC} ne sont très populaires. Le défaut principal de ces
systèmes est en effet la place que prennent les équipements et le coût d’investissement
important nécessaire. Ces deux critères sont liés, un système nécessitant un nombre
important de capteurs ou un volume de ballon plus important coutera plus cher et prendra
plus de place. Ces deux facteurs représentent les principaux verrous qui freinent le
développement de la technologie. Il est donc nécessaire de simplifier et surtout réduire
le volume des ballons tout en conservant un système solaire performant, un des objectifs
de ces travaux.

Cependant avant d’introduire le \abr{SSC} retenu et d’expliquer les raisons derrière ce
choix, il apparaît important comme pour le \abr{BEPOS} de présenter le contexte général et
les différents travaux existants. En effet ce sont ces travaux qui ont permis le
développement du \abr{SSC} modélisé dans ces travaux afin de caractériser le potentiel
d’\abr{MEPOS} solaire. Cette section présente donc dans un premier
temps l’énergie solaire et son potentiel au niveau de la surface terrestre. Suit une
description des limites et difficultés inhérentes à la caractérisation et au
dimensionnement d’un système solaire thermique comme photovoltaïque. Après une rapide
description du solaire thermique et de sa place au niveau mondial et européen, un état
de l’art présente les principales avancées réalisées notamment pour le développement des \abr{SSC}.


% ------------------------------------------------------------------------------
\subsection{Du soleil à la Terre} % (fold)
\label{sub:du_soleil_a_la_terre}
Avant de décrire les différents types de capteurs existant en solaire thermique,
il est intéressant dans un premier temps de présenter le potentiel au niveau de
la surface terrestre et les moyens permettant de le mesurer. En effet que ce soit
pour de l’expérimental ou de la modélisation, il est toujours nécessaire de connaître
a minima le rayonnement incident sur les capteurs. Dans le cas expérimental, cette
information est la plupart du temps mesurée sur site afin d’être la plus précise
possible. Dans le cas de la modélisation, il est nécessaire d’avoir recours à des
fichiers météorologiques qui peuvent soit représenter une année complète, soit des
fragments de mois qui forment le climat caractéristique du site étudié. Ces fichiers
sont indispensables pour pouvoir caractériser le potentiel d’une installation solaire
et donc la performance d’un \abr{SSC} même si comme illustré ci-après ils ne comportent
pas que des avantages.

Ainsi cette partie présente le potentiel solaire au niveau terrestre, puis décrit les
moyens disponibles pour le mesurer. Il est ensuite introduit les différents fichiers
météorologiques et leurs avantages et inconvénients. Cependant la plupart des maisons
individuelles sont construites avec des pans de toiture inclinés que ce soit pour
l’esthétique ou pour favoriser un l’écoulement pluvial ou encore améliorer la résistance à
diverses conditions climatiques. Il est donc aussi introduit une méthode de calcul
simplifiée permettant d’obtenir le potentiel solaire sur une surface inclinée en fonction
du temps universel.


% - - - - - - - - - - - - - - - - - - - - - - - - - - - - - - - - - - - - - - -
\subsubsection{L’énergie solaire} % (fold)
\label{ssub:l_energie_solaire}
L’énergie solaire provient sans surprise du soleil qui est une sphère gazeuse dont la
température intérieure est estimée entre \num{8} à \SI{40e6}{\kelvin}
\parencite{Duffie1980}. Dans ces travaux, seul le rayonnement radiatif au niveau de la
photosphère nous intéresse dont la température effective est de \SI{5777}{\kelvin}. On
entend par température effective, la température d’un corps noir émettant la même quantité
de rayonnement électromagnétique qui représente ici une approximation de la température de
surface du soleil. Le soleil se trouve à \SI{1.495e11}{\metre} de la Terre et la densité
de la puissance rayonnée reçue au niveau de la surface extérieure de l’atmosphère
terrestre, la constante solaire ($G_{cs}$) est estimée pour la première fois à \SI{1228}{\watt\per\metre\squared}
par Claude Pouillet en $1838$ (\cite{Coulson2012}, page $17$). Aujourd’hui grâce aux satellites équipés de radiomètres,
elle est estimée à \SI{1360.8 +- 0.5}{\watt\per\metre\squared} \parencite{Kopp2011}. Après
pénétration dans l’atmosphère une partie est absorbée, déviée, ou réfléchie. Il est alors
considéré une part directe ($G_{dir}$) et une part diffuse ($G_{dif}$). L’irradiance
directe est la part de la puissance rayonnée non dispersée au passage dans l’atmosphère et
est souvent exprimée pour une surface perpendiculaire à la direction de propagation
($G_{dir,\,nor}$). L’irradiance diffuse est au contraire déviée lors de son passage dans
l’atmosphère et est souvent exprimée à l’horizontale ($G_{dif}$), la direction de
propagation n’étant plus unique. Finalement, la somme des parts directes et diffuses
forment l’irradiance globale exprimée à l’horizontale (\abr{G}). Dans l’optique de
l’évaluation d’un système solaire, il est aussi courant de décrire l’énergie solaire reçue
par unité de surface~: l’\textit{irradiation}. Dans le cas spécifique où uniquement le spectre solaire est considéré,
\SIrange{0.3}{3}{\micro\metre}, les termes d’\textit{ensoleillement} ou d’\textit{insolation} (\abr{I}) sont
cependant plus courants.
Le gisement solaire (\figref{fig:gisement_solaire}) est fortement dépendant de la latitude du lieu et une majeure partie
de la France a un climat propice avec une irradiation globale annuelle de \SI{1112}{kWh\per\metre\squared}.

\begin{figure}
    \centering
    \includegraphics[width=0.75\textwidth]{Ressources/Images/Solaire/irradiation_europe.png}
    \caption[Irradiation globale horizontale en Europe]
            {Irradiation globale horizontale ($I$) en Europe (\fnref{http://solargis.com/}{Solargis}).}
    \label{fig:gisement_solaire}
\end{figure}
% subsubsection l_energie_solaire (end)


% - - - - - - - - - - - - - - - - - - - - - - - - - - - - - - - - - - - - - - -
\subsubsection{Les données météorologiques} % (fold)
\label{ssub:les_donnees_meteorologiques}
\paragraph{Mesurer l’irradiation~:} % (fold)
\label{par:mesurer_l_irradiation_}
L’énergie solaire est une énergie abondante et inépuisable, il apparaît donc opportun
de chercher à l’utiliser pour nos propres applications. Pour ce faire, il est nécessaire
d’estimer la quantité d’énergie disponible afin d’évaluer le potentiel solaire d’un site.
Plusieurs instruments existent, les deux principaux étant le pyranomètre et le
pyrhéliomètre qui font tous les deux partie de la famille des radiomètres~: ils permettent
de mesurer l’intensité du flux solaire. Le pyrhéliomètre est inventé en $1825$ par
\textit{Herschel} \parencite{Kutz2013} même si l’invention ne prendra ce nom que avec la
construction du pyrhéliomètre de \textit{Pouillet} \parencite{Boer1985}. Il permet de
mesurer le $G_{dir,\,nor}$ grâce à un cône de visibilité très restreint même si une faible
part du rayonnement issue du ciel est aussi captée (\figref{fig:schema_pyrheliometre}).
Les rayons solaires pénètrent dans un collimateur afin d’être dirigé vers une cavité. Une
thermopile est alors utilisée afin de transformer l’énergie thermique en énergie
électrique. Il peut être noté que pour fonctionner correctement, l’instrument doit
toujours être orienté en direction du soleil et donc monté sur un traqueur solaire. Le
premier pyranomètre sphérique est quant à lui inventé en $1836$ par \textit{Belloni}
\parencite{Boer1985}. Contrairement au pyrhéliomètre, il permet de mesurer le rayonnement
hémisphérique total comprenant les parts directes et diffuses (\abr{G}) et ne nécessite donc pas
de dispositif de suivi du soleil. Le rayonnement est aussi calculé grâce à une thermopile
mais un dispositif basé sur des cellules photovoltaïques existe aussi bien que moins
précis à cause d’une réponse non uniforme sur le spectre solaire. Grâce à un écran de
protection, il est aussi possible d’obtenir uniquement la part diffuse ($G_{dif,\,hor}$)~: l’intégration
est faite sur l’ensemble de l’hémisphère à l’exception de l’angle solide caché par
l’écran.

\begin{figure}
    \centering
    \begin{subfigure}[b]{0.45\textwidth}
        \centering
        \includegraphics[width=0.6\textwidth]{Ressources/Images/Solaire/pyrheliometre.png}
        \caption{}
        \label{fig:schema_pyrheliometre}
    \end{subfigure}
    \quad
    \begin{subfigure}[b]{0.45\textwidth}
        \centering
        \includegraphics[width=0.6\textwidth]{Ressources/Images/Solaire/pyranometre.jpg}
        \caption{}
        \label{fig:schema_pyranometre}
    \end{subfigure}
    \caption[Exemple de pyrhéliomètre et de pyranomètre avec écran]
             {Exemple de pyrhéliomètre (a), et de pyranomètre avec écran (b).}
    \label{fig:image_mesure_rayonnement}
\end{figure}
% paragraph mesurer_l_irradiation_ (end)

\paragraph{Les fichiers météorologiques} % (fold)
\label{par:les_fichiers_meteorologiques}
Afin d’estimer la performance d’un système solaire il est essentiel d’avoir des données
d’ensoleillement. Les données météorologiques varient cependant fortement d’une année à
l’autre et il est donc nécessaire de simuler un nombre important d’années afin d’obtenir
le comportement moyen du système considéré. Afin de limiter la durée de simulation une
autre approche est couramment retenue, l’utilisation de fichiers \textit{typiques}
(\abr{TMY}, Typical Meteorological Year). Le rayonnement solaire est cependant
très variable d’une année à l’autre et la construction de fichiers types est par
conséquent fortement dépendante de la période considérée (\figref{fig:typique_vs_observee}).
Le fichier typique est ici construit à partir de données couvrant la période de $1985$ à
$2000$. L’irradiation mensuelle du fichier typique est majoritairement inférieure à celle
observée sur les dernières années, en particulier durant la période estivale.
L’irradiation estivale étant importante et les besoins en chauffage inexistants, sa
sous-évaluation n’est pas un facteur limitant pour l’évaluation d’un \abr{SSC}. Cependant
avec l’amélioration de l’isolation des bâtiments, le solaire thermique peut aussi être
valorisé pour la climatisation du bâtiment même si la technologie n’est encore que peu
utilisée. Dans ce cas de figure, il sera alors important de correctement estimer l’irradiation
solaire même durant la période estivale afin d’évaluer correctement le potentiel du \abr{SSC}.

\begin{figure}
    \centering
    \includegraphics[width=0.9\textwidth]{Ressources/Images/Solaire/typique_vs_observee.pdf}
    \caption[Comparaisons mensuelles de l’irradiation solaire entre un fichier typique et des données réelles]
            {Comparaisons mensuelles de l’irradiation horizontale ($I$) (haut) et
            directe à la normale ($I_{dir,\,nor}$) (bas) pour la station météo de Mérignac
            et un fichier typique de Bordeaux.}
    \label{fig:typique_vs_observee}
\end{figure}


\paragraph{} % (fold)
Il existe ainsi deux approches possibles pour évaluer le potentiel d’un \abr{SSC}.
La première consiste à utiliser des données réelles du passé
acquises par la mesure. Ces données peuvent alors être disponibles à plusieurs fréquences
et une distinction est nécessaire entre celles récupérées par des stations météos et
celles récupérées de satellites. Avec les données issues de stations
météo le rayonnement est mesuré à la surface de la terre permettant d’obtenir directement
le rayonnement solaire direct ($G_{dir,\,nor}$), diffus ($G_{dif,\,hor}$), ou global
($G_{hor}$). Néanmoins cette approche nécessite l’installation d’instruments spécifiques
(pyranomètres ou pyrhéliomètres) à une position géographique suffisamment proche du
site étudié afin de fournir des données pertinentes. En France, ces données peuvent aussi
être obtenues à partir des stations météos de \textit{Météo France} ou bien plus récemment auprès
de l’organisme \fnref{http://weather.whiteboxtechnologies.com/}{WhiteBoxTechnologie}.
Les données satellites permettent elles de récupérer les données solaires à n’importe quel
point du globe ($265$ pays) et tiennent compte des spécificités du lieu comme du dénivelé.
Le rayonnement au sol est issu de données satellites (\textit{Meteosat}) et est
calculé à partir d’images couplées à un algorithme de conversion qui dépend du type de
fichier météo construit. Une fois le rayonnement global horizontal obtenu, des
\fnref{http://www.soda-pro.com/help/helioclim/decomposition-models}{algorithmes de décomposition}
sont utilisés afin de déterminer les parts respectives directe et diffuse. La SOlar
radiation DAta (\fnref{http://www.soda-pro.com/}{\abr{SODA}}) est la principale source de
données regroupant les services de quatre pays~: la France, les États-Unis, la Suisse, et
l’Italie. Elle propose des services gratuits et payants permettant d’obtenir le
rayonnement dans des conditions réelles mais aussi dans le cas idéal sans nuages grâce à
un modèle simulant une atmosphère (\textit{Copernicus McClear}) proposé par le
Copernicus Atmosphere Monitoring Service (\fnref{http://www.copernicus.eu/}{\abr{CAMS}}).
Le rayonnement obtenu est vérifié et validé grâce aux données terrestres disponibles.

\paragraph{} % (fold)
La seconde approche permet de limiter la durée de simulation en utilisant des
fichiers \textit{typiques}. Un fichier météo comportant de nombreuses informations
(températures extérieures, humidité relative, vitesse du vent, ou encore ensoleillement),
il est nécessaire de définir un ordre d’importance grâce à des pondérations. De plus
afin d’être cohérent temporellement un fichier typique ne doit être constitué que de blocs de
fichiers réels. La méthode \textit{Sandia} \parencite{Hall1978} est une approche empirique
couramment retenue (\figref{fig:methode_sandia}). Elle consiste à sélectionner sur un
échantillon important de données réelles (\SIrange{10}{30}{ans}) $12$ mois représentatifs
qui sont concaténés pour former le fichier annuel. Ainsi si on considère \SI{30}{ans} de données il existe $30$ mois
potentiels pour chaque mois de l’année. Les mois sont sélectionnés en tenant compte
principalement du rayonnement direct ($G_{dir,\,nor}$) et global ($G_{hor}$) et
secondairement de la vitesse du vent et les températures de l’air et de rosée. Le mois
sélectionné est défini grâce à la distance de \textit{Filkenstein-Schafer} qui correspond
à la différence entre la fonction de répartition au pas horaire d’un mois et celle pour
l’ensemble des mois cumulés. Cette distance est calculée pour chaque paramètre auquel un
coefficient pondérateur est appliqué en fonction de son importance pour l’application
considérée~: le \enquote{driver}. Le mois retenu est alors le mois dont la distance
moyenne pondérée est minimale et ce processus est répété pour chaque mois \parencite{Wilcox2008}.
Des travaux à travers le projet \fnref{http://www.endorse-fp7.eu/}{\textit{ENDORSE}}
ont permis de spécialiser l’approche en fonction de l’application en modifiant le
\enquote{driver} pour tenir compte des spécificités du système considéré. Il est aussi
ajouté la possibilité de considérer des blocs au niveau du mois, de la semaine, ou même du
jour et d’obtenir un pas sub-horaire sur le fichier typique final.

\begin{figure}
    \centering
    \includegraphics[width=0.7\textwidth, clip=true, trim=0mm 2.5mm 0mm 2.5mm]
                    {Ressources/Images/Solaire/tmy_generation.png}
    \caption[Construction d’une année typique (\abr{TMY}) selon la méthode \textit{Sandia}]
            {Sélection des $12$ mois dans un échantillon (\abr{LT}) pour
             construire une année typique (\abr{TMY}) selon la méthode \textit{Sandia}.}
    \label{fig:methode_sandia}
\end{figure}
% paragraph les_fichiers_meteorologiques (end)
% subsubsection les_donnees_meteorologiques (end)

\subsubsection{Déterminer le rayonnement sur une surface inclinée} % (fold)
\label{ssub:determiner_le_rayonnement_sur_une_surface_incline}
L’énergie solaire est disponible sur l’ensemble du globe de manière abondante et il est
possible d’utiliser soit des données réelles, soit des données typiques afin d’évaluer le
potentiel d’un système pour n’importe quel emplacement. Alors que la plupart des systèmes
ont une surface de capteurs inclinée les données sont le plus souvent disponibles ou à
l’horizontal, ou normal à l’incidence. La Terre tournant autour du soleil et sur elle-même, il
est nécessaire pour obtenir le rayonnement sur une surface inclinée de connaître
sa position exacte durant chaque pas de temps simulé. Le temps utilisé couramment
est le temps standard (\abr{UTC}, Temps Universel Coordonné), qui est identique sur
l’ensemble du globe. Le calcul de la position du soleil nécessite d’utiliser le Temps
Universel ou temps solaire (\abr{UT1}). Afin de tenir compte des variations locales
du temps liées à la position géographique du site étudié, l’heure solaire locale (\abr{HSL})
est définie par \eqref{eq:heure_solaire_locale}.

\begin{subequations}\label{eq:heure_solaire_locale}
  \begin{align}
    HSL &= HCL + EQT(J)\\
    \intertext{avec $EQT$ l’équation du temps pour le jour julien $J$}
    EQT(J) &= 10.2 \sin\left(4\pi \frac{J - 80}{373}\right) - 7.74 \sin\left(2\pi \frac{J - 8}{355}\right) \\
    \intertext{et $HCL$ l’heure locale civile}
    HCL &= UTC + \underset{\text{en minutes}}{\underbrace{4(LMS - LON)}}
  \end{align}
  Où \abr{LMS} et \abr{LON} sont respectivement la longitude du méridien de référence
  ($0$ Greenwich) et la longitude (positif vers l’ouest)
\end{subequations}

Sur une surface inclinée le rayonnement total ($G_{tot, inc}$) est défini par
\eqref{eq:rayonnement_inclinee}.
\begin{equation}\label{eq:rayonnement_inclinee}
        G_{inc} = G_{dir, inc} + G_{dif, inc} + G_{r, inc}
\end{equation}

Connaissant la $HSL$ et l’angle d’inclinaison ($\beta_{plan}$) qui est l’angle formé entre
la plan incliné et l’horizontal (Si $\beta > \SI{90}{\degree}$ le soleil fait face au
coté opposé au plan), il alors possible de déterminer chaque composante de $G_{inc}$. D’après les
règles de calcul de l’\textit{\abr{ASHRAE}}, les rayonnements diffus et réfléchi peuvent être
approximés uniquement en connaissant $G_{hor}$, $G_{dif, hor}$ et $\beta_{plan}$
\eqref{eq:rayonnement_surf_incline} avec $\rho_{sol}$ l’albédo du sol.
Cependant la part directe ($G_{dir, inc}$) nécessite de connaître l’angle d’incidence
($\theta$) qui est l’écart angulaire entre le segment reliant le soleil à une surface et
la normale à cette même surface.

\begin{equation}\label{eq:rayonnement_surf_incline}
    \begin{aligned}
        G_{dir, inc} &= G_{dir, nor} \times \cos(\theta) \\
        G_{dif, inc} &= G_{dif, hor} \times \frac{1 + \cos(\beta_{plan})}{2} \\
        G_{r, inc}   &= G_{hor} \times \rho_{sol} \frac{1 - \cos(\beta_{plan})}{2} \\
    \end{aligned}
\end{equation}

Le calcul de l’angle d’incidence ($\theta$) peut être approximé par \eqref{eq:angle_incidence}
\parencite{Cooper1969333,Duffie1980} et la liste des angles solaires
(\figref{fig:angles_solaires}) intervenant est décrite ci-dessous~:
\begin{itemize}
    \item La déclinaison ($\delta$)~: angle entre la droite reliant la terre et le soleil
          (centre) et le plan équatorial variant entre \num{-23.45} et \SI{23.45}{\degree}
          (positive au Nord)
    \item Latitude ($\phi$)~: position Nord-Sud d’un point par rapport à l’équateur
    \item L’angle horaire ($AH$)~: L’écart en minutes séparant du midi solaire (négatif en matinée).
    \item L’angle zénithal ($\theta_{z}$)~: L’angle d’incidence solaire sur une surface horizontale
    \item L’altitude solaire ($\alpha_{sol}$)~: écart angulaire entre une surface horizontale et le segment reliant
          le soleil à cette même surface. C’est l’angle complémentaire de l’angle zénithal.
    \item Azimut solaire ($\gamma_{sol}$)~: Écart angulaire entre la projection du segment reliant
          le soleil et une surface et le Sud (positif dans le sens horaire)
    \item Azimut du plan ($\gamma_{plan}$)~: Déviation par rapport au Sud de la projection
          du plan sur l’horizontale ($\SI{-180}{\degree} \leq \gamma_{plan} \leq \SI{180}{\degree}$).
          Pour une surface orienté Sud~: $\gamma_{plan}=0$.
\end{itemize}

\begin{subequations}\label{eq:angle_incidence}
    \begin{align}
        \cos(\theta) &= \cos(\alpha) \cos(\gamma_{sol} - \gamma_{plan}) \sin{\beta_{plan}} + \sin(\alpha_{sol}) \cos{\beta_{plan}} \\
        \intertext{avec~:}
        \sin(\alpha_{sol}) &= \cos(\phi) \cos(\delta) \cos(AH) + \sin(\phi) \sin(\delta) \\
        \cos\left(\abs{\gamma_{sol}}\right) &= \frac{\sin(\alpha_{sol}) \sin(\phi) - \sin(\delta)}{\cos(\alpha_{sol}) \cos(\phi)} \\
        \intertext{et~:}
        \gamma_{sol} &= \abs{\gamma_{sol}} \times \sign(AH) \\
        \delta &= \SI{23.45}{\degree} \sin\left(360 \times \frac{J + 284)}{365} \right) \\
        AH &= 0.25 \times \underset{\text{en minutes}}{\underbrace{\abs{(HSL - 720)}}}
    \end{align}
\end{subequations}

\begin{figure}
    \centering
    \begin{subfigure}[b]{0.3\textwidth}
        \centering
        \includegraphics[width=\textwidth, clip=true, trim=0mm 140mm 190mm 0mm]{Ressources/Images/Solaire/ear_ray.pdf}
        \caption{}
        \label{fig:ear_ray}
    \end{subfigure}
    \quad
    \begin{subfigure}[b]{0.3\textwidth}
        \centering
        \includegraphics[width=\textwidth, clip=true, trim=0mm 135mm 195mm 0mm]{Ressources/Images/Solaire/zen_sun.pdf}
        \caption{}
        \label{fig:zen_sun}
    \end{subfigure}
    \quad
    \begin{subfigure}[b]{0.3\textwidth}
        \centering
        \includegraphics[width=\textwidth, clip=true, trim=0mm 145mm 190mm 0mm]{Ressources/Images/Solaire/zen_pla.pdf}
        \caption{}
        \label{fig:zen_pla}
    \end{subfigure}
    \caption[Représentation des différents angles solaires]
            {Représentation des différents angles solaires.}
    \label{fig:angles_solaires}
\end{figure}

Ces équations simplifiées couplées à la géométrie du bâtiment permettent aussi de calculer
les masques créés par les obstructions tels qu’une avancée de toiture ou un arbre.
Il est aussi possible par exemple grâce à \fnref{http://pysolar.org/}{Pysolar} de calculer
de manière très \href{http://docs.pysolar.org/en/latest/#validation}{précise} la
position du soleil par rapport à n’importe quel point du globe. Ces outils peuvent
alors être embarqués dans des contrôleurs afin de suivre la course du soleil.
Dans le cadre de ces travaux, il est particulièrement important de pouvoir évaluer
la part de l’énergie solaire qui arrive sur les capteurs afin de pouvoir calculer
le rendement et donc la part transmise au système solaire.
% subsubsection déterminer_le_rayonnement_sur_une_surface_inclinée (end)
% subsection du_soleil_a_la_terre (end)



% ------------------------------------------------------------------------------
\subsection{Valoriser l’énergie solaire} % (fold)
\label{sub:valoriser_l_energie_solaire}
Le système le plus courant dans le domaine du bâtiment pour récupérer l’énergie solaire
est appelé panneau solaire, ou capteur solaire. On distingue trois grandes familles~:
\begin{itemize}
    \item Le capteur plan
    \item Le capteur sous-vide
    \item Le capteur à concentration
\end{itemize}

Afin de pouvoir évaluer le potentiel d’un \abr{SSC}, il est donc nécessaire de
sélectionner un type de capteur adapté à l’application. Cependant chaque capteur a sa
propre géométrie et le calcul de son rendement nécessite la connaissance de nombreux
paramètres. Afin de pouvoir simplifier l’évaluation d’un capteur quelconque, une
méthodologie de test universelle a été mise en place et permet d’exprimer en fonction de
données communes à tous les systèmes solaires, le rendement du capteur indépendamment de
sa géométrie. Ces corrélations sont utilisées dans ces travaux afin de
modéliser le comportement des capteurs solaires.

Ainsi les différents types de capteurs sont dans un premier temps présentés puis
la méthodologie de calcul retenue pour la modélisation des capteurs solaires
thermique est détaillée.



% - - - - - - - - - - - - - - - - - - - - - - - - - - - - - - - - - - - - - - -
\subsubsection{Le capteur plan} % (fold)
\label{ssub:le_capteur_plan}
C’est le système le plus ancien et aujourd’hui encore le plus économique. Parmi les
capteurs plans, une distinction peut être faite entre les capteurs avec et sans vitrage.

\paragraph{Le capteur non-vitré~:} % (fold)
\label{par:le_capteur_non_vitre}
Un capteur solaire sans vitrage est composé d’un tube en plastique noir laissant passer un
fluide caloporteur. Le tube absorbe une partie du rayonnement solaire incident qui est
ensuite transmis au fluide par conduction. L’énergie restante est perdue soit par
convection soit par rayonnement (\figref{fig:capteur_plan}). C’est le système le plus
simple et le plus économique mais les pertes importantes par convection et rayonnement
infrarouge font que ce système n’est pas très performant. N’ayant pas de protection, ni
d’isolation il est en effet très sensible au vent, à la température extérieure, et aux
échanges avec la voute céleste. Pour ces raisons, il est principalement utilisé pour des
opérations annexes telles que le chauffage de piscines.
% paragraph le_capteur_non_vitre (end)

\paragraph{Le capteur vitré~:} % (fold)
\label{par:le_capteur_vitre}
L’invention du capteur vitré a permis de réduire de manière importante les pertes
thermiques par convection et par rayonnement infrarouge. De manière schématique, un
capteur vitré est composé d’une surface transparente, d’un absorbeur, de tubes, et d’un
isolant thermique (\figref{fig:capteur_plan_vitre}). Un vitrage est utilisé pour la
surface transparente afin de laisser passer la majorité du rayonnement solaire (visible et
Proche InfraRouge (\abr{PIR})) de par ces propriétés optiques. L’absorbeur comme son nom
l’indique doit permettre d’absorber au maximum l’énergie solaire et émettre le moins possible dans
l’infrarouge. En effet d’après la loi du déplacement de \textit{Wien}, à température
ambiante les objets émettent spontanément dans l’InfraRouge Moyen (\abr{MIR}) soit pour
une longueur d’onde de \SIrange{3}{50}{\micro\metre}. L’absorbeur est ainsi de
couleur noire ($\leq \SI{95}{\percent}$) afin d’absorber un maximum dans le visible et
subit un traitement chimique (chrome ou céramique à base d’oxydes métalliques) afin
d’obtenir une surface faiblement émissive ($\leq \SI{12}{\percent}$). Le vitrage peut lui
aussi être faiblement émissif afin de laisser passer le rayonnement solaire mais réfléchir
le \abr{MIR}. La combinaison des deux permet ainsi de récupérer la majorité du spectre
solaire au niveau de l’absorbeur tout en limitant les pertes par inter-réflexions. Les
tubes contenant le fluide sont couramment en cuivre car il est simple à travailler et a
une conductivité thermique importante ($\lambda = \SI{390}{\watt\per(\metre\period\kelvin)}$).
Finalement l’isolant en face arrière et sur les côtés permet de réduire les pertes par conduction.
% paragraph le_capteur_vitre (end)

\begin{figure}
    \centering
    \begin{subfigure}[b]{0.4\textwidth}
        \centering
        \includegraphics[width=0.6\textwidth]{Ressources/Images/Solaire/capteur_plan.jpg}
        \caption{}
        \label{fig:capteur_plan}
    \end{subfigure}
    \quad
    \begin{subfigure}[b]{0.5\textwidth}
        \centering
        \includegraphics[width=0.65\textwidth]{Ressources/Images/Solaire/capteur_plan_vitre.jpg}
        \caption{}
        \label{fig:capteur_plan_vitre}
    \end{subfigure}
    \caption[Description des capteurs plans]
             {Exemple de capteur plan non vitré (a), et vue explosée d’un capteur vitré (b).}
    \label{fig:capteurs_plan}
\end{figure}
% subsubsection le_capteur_plan (end)


% - - - - - - - - - - - - - - - - - - - - - - - - - - - - - - - - - - - - - - -
\subsubsection{Le capteur sous-vide} % (fold)
\label{ssub:le_capteur_sous_vide}
À la différence du capteur plan, le capteur sous-vide réduit les déperditions au niveau de
l’absorbeur en remplaçant l’air par du vide supprimant les échanges convectifs. On
distingue quatre types~: le capteur à circulation directe, le capteur à caloduc, le
capteur à effet \enquote{Thermos}, et le capteur à réflecteur intégré. Pour le capteur à
circulation directe, l’absorbeur est une structure plane à ailettes et la circulation de
l’eau est assurée par un système de tubes en \abr{U}. L’absorbeur de chaque tube est
contrairement aux autres modèles, orientable, ce qui permet de les utiliser en façade. Le
capteur à caloduc (\figref{fig:capteur_sous_vide}) utilise le principe de la vaporisation
de l’eau déminéralisée. L’eau en se
vaporisant remonte jusqu’à la tête du caloduc et s’y condense au contact du fluide
caloporteur. Le capteur à effet \enquote{Thermos} est lui composé de deux tubes concentriques,
le tube intérieur étant l’absorbeur. Le vide est réalisé entre les deux tubes et l’échange
de chaleur se fait soit par un système de caloduc, soit directement par circulation du
fluide dans le tube intérieur. Afin d’améliorer la part solaire reçue, un réflecteur peut
être ajouté soit sur la paroi intérieure du tube extérieur soit à l’extérieur
(\abr{CPC}, Schott-Rohglas).

\begin{figure}
    \centering
    \begin{subfigure}[b]{0.35\textwidth}
        \centering
        \includegraphics[width=0.7\textwidth]{Ressources/Images/Solaire/caloduc.jpg}
        \caption{}
        \label{fig:capteur_sous_vide}
    \end{subfigure}
    \quad
    \begin{subfigure}[b]{0.55\textwidth}
        \centering
        \includegraphics[width=0.7\textwidth]{Ressources/Images/Solaire/cylindro_parabolique.jpg}
        \caption{}
        \label{fig:capteur_cylindro_parabolique}
    \end{subfigure}
    \caption[Description d’un capteur sous-vide et exemple d’un capteur cylindro-paraboliques]
             {Fonctionnement d’un capteur sous-vide à caloduc (a). Exemple de
              capteur cylindro-parabolique (b).}
    \label{fig:capteur_vide_parabolique}
\end{figure}
% subsubsection le_capteur_sous_vide (end)


% - - - - - - - - - - - - - - - - - - - - - - - - - - - - - - - - - - - - - - -
\subsubsection{Le capteur à concentration} % (fold)
\label{ssub:le_capteur_a_concentration}
Ces capteurs cherchent à concentrer en un point focal le rayonnement direct grâce à un
système de réflecteur ou de lentille. Grâce à l’énergie concentrée la température
atteinte sur ce point est comprise entre \SIrange{400}{1500}{\celsius} en fonction des
systèmes. Non compatible avec des installations traditionnelles de chauffage ou de production
d’\abr{ECS} les capteurs à concentration sont en majorité utilisés pour des applications
industrielles. Cependant des systèmes de cogénération se développent en couplant
par exemple un moteur à vapeur avec des capteurs à concentration cylindro-paraboliques,
mais les applications restent mineures.
Par contre contrairement aux capteurs précédents, les hautes températures obtenues
permettent de fabriquer de l’électricité soit par conversion directe, soit par détente.
Dans cette famille, une distinction est faite entre, les capteurs cylindro-paraboliques, les capteurs
paraboliques, les centrales à tour, et les fours solaires. Les capteurs cylindro-paraboliques
(\figref{fig:capteur_cylindro_parabolique}) utilisent des réflecteurs
cylindriques orientables sur un axe (pour suivre le soleil) afin de concentrer le
rayonnement direct sur un foyer linéaire où passe un fluide (souvent de l’huile). Le fluide
est ensuite utilisé pour transformer de l’eau en vapeur qui est ensuite envoyée vers une turbine pour fabriquer de l’électricité par détente.
Les capteurs paraboliques emploient un système de suivi du soleil sur deux axes et un
réflecteur en forme de \fnref{https://fr.wikipedia.org/wiki/Paraboloïde}{paraboloïde de
révolution} afin de pouvoir concentrer en un point unique le rayonnement~: le point focal.
L’électricité est produite par conversion directe via un moteur \textit{Stirling}.

Le système de tour solaire est lui aussi basé sur la concentration des rayons solaires sur
un point unique et permet d’obtenir une source chaude allant jusqu’à
\SI{1500}{\celsius}. Un champ de miroirs orientables ayant un système de suivi sur deux axes
(\enquote{héliostats}), réfléchissent toute la journée l’énergie solaire vers un absorbeur au
sommet d’une tour.
L’électricité est fabriquée soit à l’aide d’une turbine et d’un fluide intermédiaire, soit par
production directe. Finalement le dernier dispositif est le four solaire composé d’un
miroir concave concentrant le rayonnement en son foyer. Deux applications peuvent être
notées. La première est la production d’électricité comme par exemple avec le four solaire
français mondialement connu d’\href{http://www.promes.cnrs.fr/index.php?page=historique}{Odeillo} qui permet
d’atteindre une puissance d’\SI{1}{\mega\watt}. Il utilise un ensemble de miroirs
orientables (héliostats) dirigeant les rayons vers un miroir concave géant concentrant les
rayons vers le foyer. La deuxième application est la cuisson avec des appareils plus ou
moins rudimentaires.
% subsubsection le_capteur_a_concentration (end)


% - - - - - - - - - - - - - - - - - - - - - - - - - - - - - - - - - - - - - - -
\subsubsection{Rendement d’un capteur solaire thermique} % (fold)
\label{ssub:rendement_d_un_capteur_solaire_thermique}
Le rendement ($\eta$) d’un capteur solaire \eqref{eq:rendement_sol} est fonction de la
puissance utile rapportée à la surface du capteur ($P_{u}$, [\si{W\per\metre\squared}]) et
du flux solaire incident ($G_{in}$, [\si{W\per\metre\squared}]).

\begin{equation}\label{eq:rendement_sol}
    \eta = \frac{P_{u}}{G_{in}}
\end{equation}

Comme il a été vu à travers les différents capteurs existants, de nombreux facteurs
influencent le calcul de $P_{u}$. Il est en effet nécessaire de considérer les
caractéristiques optiques et thermiques des différents composant du capteur. Si on considère
un capteur plan non-vitré, $P_{u}$ est définie comme la part absorbée moins les pertes
par convection, conduction, et rayonnement infrarouge (\cite{Duffie1980}, voir équations $5.9.2$
et $6.2.1$). Ces relations nécessitent cependant de connaître la température
moyenne de l’absorbeur dont la mesure ou le calcul est ardu. En effet, elle dépend
à la fois du rayonnement incident, du fluide entrant dans le capteur, mais aussi
de la géométrie propre à chaque capteur.
Ainsi afin de simplifier l’évaluation de la performance d’un capteur quelconque, il
est nécessaire de reformuler le problème.
Les corrélations présentées ci-après, aujourd’hui utilisées pour estimer la performance
d’un capteur solaire sont issues des travaux amorcés par \textcite{Hottel1958}.
Dans la première corrélation \eqref{eq:ISO9806}, $\eta_{0}$ est le facteur optique
du capteur, $K$ la conductance thermique totale des pertes, et $T_{ext}$ la température
de l’air extérieur. Cette formulation implique une représentation linéaire de
la performance du capteur et a vite montré ses limites.
\begin{subequations}\label{eq:ISO9806}
    \begin{align}
    P_{u} &= \eta_{0} \times G_{in} - K (T_{moy} - T_{ext}) \\
    \shortintertext{avec}
    T_{moy} &= \frac{T_{fluide}^{entrée} + T_{fluide}^{sortie}}{2}
    \end{align}
\end{subequations}


\paragraph{Approche instantanée~:} % (fold)
\label{par:approche_instantanée}
Dans les années $1990$ avec la norme \abr{ISO}\,$9806$-$1$ une nouvelle corrélation plus
précise est définie. Celle-ci dépend de deux nouveaux coefficients ($c_{1}$ et $c_{2}$)
permettant d’estimer plus finement la performance en fonction de $T_{ext}$
\eqref{eq:rendement_instantanne}. la norme \abr{ISO}\,$9806$-$1$ considère la surface de
l’absorbeur alors que la norme européenne EN\,$12975$-$2:2001$ considère la surface
d’entrée \parencite{EN1297522001}. Ainsi en fonction de la norme considérée la valeur des
coefficients varie et certains organismes de certification fournissent aussi la
valeur de ces coefficients suivant la surface totale du capteur.

\begin{equation}\label{eq:rendement_instantanne}
    P_{u} = \eta_{0} \times G_{in}- c_{1} (T_{moy} - T_{ext}) - c_{2} (T_{moy} - T_{ext})^{2}
\end{equation}

Afin d’obtenir une meilleure approximation, l’approche stationnaire tient compte
de l’angle d’incidence en considérant un modificateur d’angle d’incidence (\abr{IAM}, Incidence Angle Modifier).
Il est défini par $K_{\theta}(\theta)$ comme le rapport entre le rendement optique pour un angle $\theta$
et le rendement optique à incidence normale ($\theta = 0$).
\begin{equation}\label{eq:IAM}
    \begin{aligned}
    K_{\theta}(\theta) = \frac{\eta_{0}(\theta)}{\eta_{0}(\theta = 0)}
    \end{aligned}
\end{equation}
L’approche statique ne fait cependant pas de distinction entre $G_{dir, in}$ et $G_{dif,
in}$. Ainsi l’approche dans ces conditions n’est valide que lorsque $G_{dif, in}$
représente moins de \SI{30}{\percent} du rayonnement hémisphérique total $G_{in}$ \parencite{Osorio2014}.
Finalement la procédure de test ajoute la prise en compte du $C_{p,\,eff}$, la capacité
thermique massique effective définit comme étant la somme pondérée des capacités
massiques des différents composants du capteur. Ces facteurs de pondération indiquent que certains éléments
ne sont impliqués que partiellement dans l'inertie thermique du capteur. $c_{5}$, un nouveau coefficient de
corrélation est alors défini par $\nicefrac{C_{p,\,eff}}{A}$ où $A$ est la surface de
capteur considérée par la norme retenue. En couplant les différentes informations, il est
possible d’obtenir la forme complète de l’équation instantanée
\eqref{eq:instantanee_complete}.

\begin{equation}\label{eq:instantanee_complete}
    \begin{aligned}
        P_{u} = \eta_{0} K_{\theta}(\theta) \times G_{in} - c_{1} (T_{moy} - T_{ext}) - c_{2} (T_{moy}
                - T_{ext})^{2} - c_{5}\frac{dT_{moy}}{dt}
    \end{aligned}
\end{equation}
% paragraph approche_instantanée (end)


\paragraph{Approche quasi-dynamique~:} % (fold)
\label{par:approche_quasi_dynamique}
La procédure de test quasi-dynamique est quant à elle décrite à travers l’ISO\,$9806$-$3$
aussi introduite dans la norme EN\,$12975$-$2:2001$ \parencite{EN1297522001} aujourd’hui
remplacée par la EN\,$12975$-$2:2006$ \parencite{EN1297522006}.
Contrairement à l’approche précédente, trois nouveaux coefficients sont introduits.
Les deux premiers décrivent l’impact du vent sur le rendement optique ($c_{6}$) et sur
les pertes thermiques ($c_{3}$), le dernier, $c_{4}$, permet de tenir compte des pertes
par \abr{MIR}. Comme explicité dans \ref{sub:valoriser_l_energie_solaire}
les capteurs plans vitrés et les capteurs sous-vide sont peu sensibles à l’impact du
vent et du rayonnement \abr{MIR}. Ces coefficients doivent cependant être considérés
pour des capteurs non-vitrés. \textcite{Hunn197733} montrent en effet que les pertes
par rayonnement infrarouge représentent près de \SI{25}{\percent} des pertes annuelles
pour ces capteurs. Que ce soit pour l’approche statique ou dynamique, les coefficients
sont obtenus à partir des données expérimentales et des régressions linéaires multiples.
L’approche quasi-dynamique contrairement à l’approche statique considère aussi séparément
le rayonnement diffus et direct en introduisant deux \abr{IAM}. Il sont notés dans le reste
de ce document respectivement par $K_{\theta,\,dir}$ et $K_{\theta,\,dif}$.
Dans le cas des capteurs plans dont le comportement est
isotropique, il est admis \eqref{eq:iam_dir_plan} où $b_{0}$ est le coefficient
modificateur d’angle d’incidence \parencite{Zambolin20101382}

\begin{equation}\label{eq:iam_dir_plan}
    K_{\theta,\,dir} (\theta) = 1 + b_{0} \times \left(\frac{1}{\cos(\theta)} - 1\right)
\end{equation}

Pour un capteur tubulaire, sa géométrie complexe nécessite de considérer $\theta t$ et
$\theta l$, respectivement la projection transversale et longitudinale de l’angle
d’incidence. L’impact de l’angle d’incidence sur $G_{dir, in}$ est donc défini par
$K_{\theta,\,dir} (\theta t, \theta l)$. D’après \textcite{McIntire1982315} il est possible
d’approximer $K_{\theta,\,dir} (\theta t, \theta l)$ comme étant le produit des deux
projections \eqref{eq:iam_dir_tube}.

\begin{equation}\label{eq:iam_dir_tube}
    K_{\theta,\,dir} (\theta t, \theta l) = K_{\theta t,\,dir} \times K_{\theta l,\,dir}
\end{equation}

Pour $K_{\theta t,\,dir}$ il est admis la même relation que pour le capteur plan \eqref{eq:iam_dir_plan}.
Cependant afin de pouvoir apprécier la complexité de $K_{\theta l,\,dir}$ un polynôme d’ordre
$4$ de la forme \eqref{eq:iam_dir_tube_lon} peut être utilisé \parencite{Zambolin201237}.
Pour note, la norme considère \eqref{eq:iam_dir_plan} dans les deux cas.

\begin{align}\label{eq:iam_dir_tube_lon}
    K_{\theta l,\,dir} (\theta) = 1 &+ b_{0} \times \left(\frac{1}{\cos(\theta)} - 1\right)
                          + b_{1} \times \left(\frac{1}{\cos(\theta)} - 1\right)^{2} \\
                          &+ b_{2} \times \left(\frac{1}{\cos(\theta)} - 1\right)^{3}
                          + b_{3} \times \left(\frac{1}{\cos(\theta)} - 1\right)^{4} \\
\end{align}

Ainsi si on tient compte de l’ensemble des paramètres la forme complète de l’équation
en quasi-dynamique est définie par \eqref{eq:quasi_complete}~: $u$ (\si{\metre\per\second})
étant la vitesse du vent, $\sigma$ la constante de \textit{Stefan–Boltzmann}, et $E_{L}$
(\si{W\per(\metre\squared\period\kelvin^{4})}) la part du rayonnement infrarouge
non comprise dans le spectre solaire (longueur d’onde $> \SI{3}{\micro\metre}$).


\begin{align}\label{eq:quasi_complete}
        P_{u}  &= \eta_{0} K_{\theta,\,dir}(\theta) \times G_{in,\,dir} +
                  \eta_{0} K_{\theta,\,dif}(\theta) \times G_{in,\,dif} \\
                &- c_{6}uG_{in} - c_{3} u(T_{moy} - T_{ext}) - c_{4} (E_{L} - \sigma T^{4}) \\
                &- c_{1} (T_{moy} - T_{ext}) - c_{2} (T_{moy} - T_{ext})^{2} - c_{5}\frac{dT_{moy}}{dt}
\end{align}


\paragraph{Bilan~:} % (fold)
\label{par:bilan_rendement}
Les deux approches permettent aujourd’hui de tester tout type de capteurs solaires
en se basant sur une méthodologie commune facilitant la comparaison entre divers
fabricants. L’ensemble des organismes de certification propose ainsi les données nécessaires
pour appliquer soit la méthode statique, soit la méthode quasi-dynamique.
Les résultats montrent cependant que la méthode quasi-dynamique permet d’obtenir de
une meilleure précision, en particulier en matinée et en soirée
où l’approche statique sur-estime l’énergie récupérée (\figref{fig:compare_static_quasi_dyn})

\begin{figure}
    \centering
    \includegraphics[width=0.9\textwidth]{Ressources/Images/Solaire/static_vs_quasi.png}
    \caption[Comparaison entre la méthode statique et la méthode quasi-dynamique]
            {Comparaisons de résultats expérimentaux et des résultats obtenus
             à partir de la méthode statique (gauche) et de la méthode quasi-dynamique (droite)
             d’après \textcite{Zambolin20101382}.}
    \label{fig:compare_static_quasi_dyn}
\end{figure}

Finalement, la nouvelle norme (EN\,ISO\,$9806$:$2013$ ) actuellement en finition permet d’obtenir
une norme commune sur le plan international, et remplace l’ensemble des précédentes qui font alors doublons~:
EN\,$12975$-$2:2006$ et ISO $9806$-$1,2,3$ \parencite{ISO98062013}.

Maintenant que la méthode de calcul quasi-dynamique utilisée dans ces travaux est
introduite, une présentation de l’évolution globale du marché du solaire thermique
est proposée.
% subsubsection rendement_d_un_capteur_sola ire_thermique (end)


% ------------------------------------------------------------------------------
\subsection{Le solaire thermique en chiffres} % (fold)
\label{sub:le_solaire_thermique_en_chiffres}
Avant de discuter des différents travaux déjà réalisés dans le solaire thermique
et plus particulièrement pour les \abr{SSC}, il est intéressant de faire une description
rapide de l’évolution du secteur à différentes échelles. Cette partie discute aussi
des initiatives internationales actuellement en cours afin d’encourager le développement des
\abr{SSC}.

% - - - - - - - - - - - - - - - - - - - - - - - - - - - - - - - - - - - - - - -
\subsubsection{Les chiffres clés} % (fold)
\label{ssub:les_chiffres_cles}
L’\abr{IEA} actualise tous les ans son rapport sur l’évolution mondiale du solaire thermique
dans le cadre de son programme \enquote{Solar Heating and Cooling} (\abr{SHC}). Dans la dernière
édition, la couverture des données représente \SI{95}{\percent} du
marché mondial \parencite{Weiss2017}. Ainsi fin 2016, près de \SI{456}{\giga\watt_{th}}
sont installés et une évolution de \SI{7.4}{\percent} est enregistrée depuis le début du
siècle. Cette évolution est portée par un fort développement des structures industrielles
($\geq \SI{350}{\kilo\watt_{th}}$ ou $\geq \SI{500}{\metre\squared}$). La plus grande
installation à ce jour étant au Danemark avec près de \SI{110}{\mega\watt_{th}} pour
\SI{156694}{\metre\squared} de capteurs plans représentant \SI{30}{\percent} de la surface
installée courant $2016$. Historiquement, la puissance maximale double tous les ans depuis
2013~:
\begin{description}
    \item[2013~:] Marstal au Danemark avec \SI{\sim 20000}{\metre\squared}
    \item[2014~:] Chili avec \SI{\sim 39000}{\metre\squared}
    \item[2015~:] Vojens au Danemark avec \SI{\sim 70000}{\metre\squared}
    \item[2016~:] Silkeborg  au Danemark avec \SI{\sim 156000}{\metre\squared}
\end{description}
Parmi les $66$ pays analysés, le marché est majoritairement situé en Chine
(\SI{309.5}{\giga\watt_{th}}) et plus modestement en Europe (\SI{49.2}{\giga\watt_{th}})
qui à eux deux représentent \SI{82.3}{\percent} de la puissance thermique installée.
Ramené au nombre d’habitant, il peut être observé que le marché se développe plutôt bien
en Europe avec l’Autriche (\SI{421}{\kilo\watt_{th}\per(1000 hab.)}), la Grèce
(\SI{287}{\kilo\watt_{th}\per(1000 hab.)}), ou l’Allemagne
(\SI{164}{\kilo\watt_{th}\per(1000 hab.)}) contre \SI{226}{\kilo\watt_{th}\per(1000 hab.)}
pour la Chine. En France, le solaire thermique n’est que peu développé avec seulement
\SI{22.03}{\kilo\watt_{th}\per(1000 hab.)} (hors \abr{DOM}) alors que l’ensoleillement
(\SI{1112}{\kilo\watt\hour\per\squared\metre}) est en moyenne plus important que
pour l’Allemagne (\SI{1091}{\kilo\watt\hour\per\squared\metre}).
Le solaire a ainsi permis en $2016$ d’éviter au niveau mondial, l’émission de
\SI{130}{\mega\tonne} de $CO_{2}$ et \SI{40.3}{\mega\tonne} de produits pétroliers et
respectivement \SI{362}{\tonne} et \SI{112}{\mega\tonne} en France.

Malgré une progression de la surface installée chaque année, la tendance décroit de
\SI{14}{\percent} en $2015$ par rapport à $2014$ et la baisse semble continuer en $2016$.
Au niveau économique, le marché stagne en Europe et est décroissant dans le reste du monde
à l’exception de l’Afrique Sub-Saharienne et de l’Asie (Chine non-inclus). L’énergie
solaire thermique assume donc une perte d’intérêt importante depuis le début du siècle
(\figref{fig:tendances_enr}). Depuis $2010$ le solaire thermique progresse de plus en plus
lentement et à partir de $2014$ les autres énergies renouvelables (éolien et
photovoltaïque) affichent une progression nettement plus importante. Ainsi à partir de
$2016$ l’énergie éolienne produit plus d’énergie que le solaire thermique. Finalement le
photovoltaïque qui profite d’une croissance forte et d’une tendance à la hausse depuis
$2014$ semble rapidement rattraper son retard.

\begin{figure}
    \centering
    \includegraphics[width=0.7\textwidth]{Ressources/Images/Environnement/evolution_enr.png}
    \caption[Évolution tendancielle du solaire thermique, éolien, et photovoltaïque]
            {Évolution tendancielle et en puissance installée cumulée de la répartition
             du solaire thermique, de l’éolien, et du photovoltaïque d’après
             \textcite{Weiss2017}.}
    \label{fig:tendances_enr}
\end{figure}

L’association Allemande (\textit{BSW-Solar}, German Solar Industry Association)
a publié une carte des installations industrielles mondiales et met en avant les
principaux facteurs responsables du développement du solaire thermique \parencite{Augsten2017}.
Le  marché semble être en premier influencé par le prix des énergies fossiles, suivi par les considérations
politiques des états imposant une part plus importante d’énergies renouvelables. Arrivées
en second en $2013$, les aides financières sont reléguées à la troisième place avec
cette nouvelle étude en $2016$. Les industriels
mettent aussi en avant le scepticisme des consommateurs vis à vis du solaire thermique
alors que \SI{70}{\percent} des groupes interrogés estiment que le marché est déjà
compétitif. Finalement le marché entre de plus en plus en compétition avec les installations \abr{PV} pour
lesquelles les industriels s’accordent majoritairement sur leur compétitivité.


\paragraph{Applications~:} % (fold)
\label{par:applications}
Au niveau mondial $108$ millions de systèmes sont référencés en $2015$, dont trois-quarts
sont des systèmes à thermosiphon, et sont majoritairement utilisés pour la production
d’\abr{ECS}~: \SI{63}{\percent} dans des maisons individuelles et \SI{28}{\percent} sur des bâtiments plus importants comme
les hôpitaux ou les écoles. La tendance actuelle semble cependant inverser les proportions
avec une forte décroissance en $2015$ de la part de la production d’\abr{ECS} en maison individuelle
et une forte croissance dans les autres bâtiments amenant à une répartition pour les nouvelles installations
de \SI{41}{\percent} en maison individuelle et de \SI{51}{\percent} pour les autres bâtiments .
Les \abr{SSC} restent très minoritaires et représentent seulement \SI{2}{\percent}.
Finalement \SI{6}{\percent} sont utilisés pour le chauffage des piscines et les \SI{1}{\percent}
restants alimentent des réseaux de chaleur, des processus industriels, ou encore des systèmes de
refroidissement.
Au niveau européen le marché est principalement tourné vers les capteurs plans avec \SI{72.3}{\percent}
des parts et contrairement à la tendance mondiale les systèmes comportent pour \SI{61}{\percent} des pompes.
La part des système de production d’\abr{ECS} en maison individuelle est la même qu’au niveau mondial,
cependant les grandes installations ne représentent plus que \SI{12}{\percent}. Alors que
dans la plupart des autres pays du monde, le \abr{SSC} représente une part infime, il
représente en Europe \SI{19}{\percent} des systèmes installés même si la tendance est à
la baisse~: \SI{17}{\percent} en $2014$ \parencite{Mauthner2016} et \SI{16}{\percent} en
$2015$ \parencite{Weiss2017}.
% paragraph applications (end)


\subsubsection{Les travaux en cours} % (fold)
\label{ssub:les_travaux_en_cours}
Afin d’encourager le développement des systèmes solaires, l’\abr{IEA} travaille
actuellement sur le développement économique des systèmes solaires. Ces travaux s’articulent
à travers deux tâches. La première, la tâche $53$ cherche à évaluer le potentiel
de l’énergie solaire (thermique et photovoltaïque) pour le refroidissement qui
avec les \abr{BEPOS} deviendra un secteur clé de la consommation du bâtiment.
La tâche s’articule suivant trois sous-tâches~:
\begin{description}
    \item[A~:] Caractériser le potentiel des systèmes existants et proposer une méthodologie
                commune permettant de les comparer économiquement. Proposition de nouveaux
                systèmes et évaluation de l’interaction avec le réseau.
    \item[B~:] Utilisation de la simulation pour comparer les systèmes et mieux
                comprendre les interactions entre composants et conditions limites.
    \item[C~:] Mise en place d’une évaluation expérimentale pour établir un standard
                de test, vérifier les modèles, collecter, et analyser les données.
\end{description}
La seconde, la tâche $54$ axe ses recherches sur la compétitivité économique des
systèmes solaires. Le but étant d’arriver à une standardisation des
divers composants et de proposer des systèmes \enquote{plug and play} nécessitant
moins de maintenance. Cette évolution implique une réduction du coût de production
et d’installation en plus de simplifier l’offre existante et s’articule ainsi~:
\begin{description}
    \item [A~:] Analyse du marché et définition d’indicateurs pour comparer le coût
                sur les différentes phases~: construction, distribution, installation, et maintenance.
    \item [B~:] Simplification et standardisation des systèmes pour faciliter les différentes
                phases.
    \item [C~:] Recherche sur des composants et matériaux innovants avec en priorité
                leur viabilité économique dans le respect des sous-tâches A et B.
\end{description}
Les deux tâches ont en commun une sous-tâche supplémentaire (D) montrant une volonté
de synergie. Le rôle de cette sous-tâche est en effet d’assurer une coopération
continue entre l’industrie et la recherche, condition nécessaire
pour permettre d’exploiter et valoriser rapidement les résultats de recherche.
Des innovations comme l’invention d’une vanne thermostatique fonctionnant
sans contrôleur électrique (\fnref{http://www.iea-shc.org/article?NewsID=177}{\textit{Conico}})
permettent déjà d’apprécier les retombées.
De plus une \href{http://www.iea-shc.org/article?NewsID=173}{nouvelle tâche} qui doit
commencer début $2018$ devrait permettre d’accompagner le développement de systèmes mixtes
permettant de produire de manière combinée électricité et énergie thermique.
% subsubsection les_travaux_en_cours (end)


\subsubsection{Bilan} % (fold)
\label{ssub:bilan_evolution}
Les applications en solaire thermique sont nombreuses mais subissent
une tendance négative en comparaison avec les autres énergies renouvelables. Le
secteur du solaire thermique est ainsi au niveau mondial en décroissance forte et stagne
sur le continent Européen. Bien que des applications de \abr{SSC} existent au niveau
Européen, la tendance est aussi à la baisse. Ces tendances traduisent une perte de
confiance dans cette technologie déjà observée par les industriels. La même observation
est faite par \textcite{Musall2010} à travers l’analyse de $280$ \textit{Net}\,\abr{ZEB} qui
montrent que moins de \SI{10}{\percent} des maisons individuelles sélectionnent un \abr{SSC}
comme mesure là où près de \SI{70}{\percent} recommande une production d’\abr{ECS}
solaire. Le \abr{SSC} est ainsi aujourd’hui une technologie principalement utilisée dans la rénovation
\parencite{Ellehauge2003}. L’ouverture de nouveaux travaux de recherche
sur l’aspect économique ou sur les applications de refroidissement
permettront à terme d’améliorer la compétitivité du secteur et de mieux valoriser ces systèmes pour les nouvelles
problématiques du bâtiment.

Comme illustré dans les sections précédentes, le solaire thermique doit
afin d’inverser la tendance proposer des solutions performantes et adaptées au \abr{BEPOS}.
Dans cette optique il est nécessaire de pouvoir évaluer la performance potentielle
du couplage entre une \abr{BEPOS} et un \abr{SSC} et de réduire la taille des équipements
nécessaires qui représentent un frein à son développement.
De plus la norme en préparation (\abr{PREN\,$15603$}) amène aussi à reconsidérer la manière
de construire des \abr{BEPOS} et les \abr{SSC} peuvent permettre d’atteindre la couverture
locale en énergie renouvelable nécessaire.
Les \abr{SSC} restent donc une solution du futur mais leur dimensionnement doit être adapté
aux nouveaux bâtiment et non uniquement basé sur l’expérience acquise sur les anciens.
% subsubsection bilan (end)
% subsection le_solaire_thermique_en_chiffres (end)


% ------------------------------------------------------------------------------
\subsection{Les travaux sur le solaire thermique} % (fold)
\label{sub:les_travaux_sur_le_solaire_thermique}
Afin de pouvoir proposer un système adapté, un état de l’art est proposé. Celui-ci
décrit dans un premier temps les travaux sur le solaire thermique de manière générale, puis
plus spécifiquement sur les \abr{SSC} en présentant les divers systèmes existants
et les résultats obtenus. Enfin des exemples d’analyse paramétriques et d’optimisation
sont discutés.

% - - - - - - - - - - - - - - - - - - - - - - - - - - - - - - - - - - - - - - -
\subsubsection{Les études pionnières} % (fold)
\label{ssub:les_etudes_pionnieres}
La recherche sur le chauffage des bâtiment par énergie solaire semble débuter en $1939$ au
\enquote{Massachusetts Institute Technology} (\abr{MIT}). La première maison avec
chauffage solaire, \fnref{https://www.technologyreview.com/s/604079/the-first-us-house-to-go-solar/}
{\textit{Solar House I}} voit le jour. Il faudra attendre $1972$ et $1974$ pour voir apparaître respectivement
en France et en Angleterre les premières maisons solaires utilisant des systèmes passifs
et actifs \parencite{Michaelides1993}. Un modèle numérique développé par \textcite{Hunn197733}
permet en $1977$ d’évaluer pour un pas de temps horaire l’effet des conditions
météorologiques sur la performance d’\abr{SSC}. Étudiant la performance des capteurs, il
conclut qu’une seconde vitre sur les capteurs plans permet d’améliorer les performances
dans un climat caractérisé par une forte couverture nuageuse. Ces travaux sont complétés
par \textcite{Elsayed198989} qui étudie plus en détail la transmission du rayonnement à
l’absorbeur en étudiant par exemple l’impact de l’albédo et du nombre de vitres sur le
capteur. L’étude conclut qu’une variation de \SIrange{10}{15}{\degree} induit une
réduction du rayonnement absorbé de l’ordre de \SIrange{3}{4}{\percent}.

Avant $1979$, afin de couvrir les besoins en \abr{ECS} pour une installation domestique,
un débit de l’ordre de \SI{0.015}{kg\per(\metre\squared\period\second)}
était couramment retenu. \textcite{Koppen1979} montrent qu’un débit réduit
permet d’améliorer la stratification et \textcite{Wuestling1985}
observent le lien entre débit, stratification, et $F_{sol}$ \eqref{eq:f_sol} le taux de couverture
solaire permettant de caractériser la part du solaire par rapport à l’appoint ($Conso_{complément}$).
Il montre alors que lorsque la stratification est importante,
l’utilisation d’un débit réduit (\num{0.002} à \SI{0.007}{kg\per(\metre\squared\period\second)}) permet
d’obtenir une fraction solaire plus importante d’un tiers par rapport à un ballon dont
l’eau est fortement mixée. En pratique un ballon n’est jamais fortement stratifié mais il est possible par son dimensionnement
d’augmenter sa stratification, en favorisant un volume de stockage journalier. \textcite{Loef1967}
s’intéressent eux aux systèmes à circulation naturelle et observent que le différentiel
de température entre l’entrée et la sortie des capteurs reste constant durant la journée
et se situe à \SI{\sim 10}{\celsius}.

\begin{equation}\label{eq:f_sol}
    F_{sol} = \frac{Prod_{sol}}{Conso_{complément} + Prod_{sol}}
\end{equation}

\textcite{Jordan2001197} étudient l’influence du profil de puisage en comparant le
scénario type de la norme \abr{PREN\,$12977$} et un profil statistique tenant compte
des fluctuations annuelles et mensuelles. Une variation de l’ordre
de \SI{3}{\percent} est observée entre les deux scénarios due en majorité à un besoin plus
important durant la période hivernale pour le scénario statistique.
L’importance du maintien de la stratification dans le ballon d’$ECS$ est aussi identifiée
en comparant plusieurs débits de puisage.
% subsubsection les_etudes_pionnieres (end)


% - - - - - - - - - - - - - - - - - - - - - - - - - - - - - - - - - - - - - - -
\subsubsection{La tâche $\bf{26}$} % (fold)
\label{ssub:la_tâche_26}
Une des majeures contributions aux \abr{SSC} pour la maison individuelle a été réalisée au cours
de la tâche\,$26$ qui fait intervenir dix pays dont la France. Elle s’articule en
trois sous-tâches \parencite{Task26C2003}. Au cours
de la tâche, $21$ systèmes ont été étudiés dont neuf de manière détaillée apportant
un nombre conséquent de résultats et recommandations à travers les travaux de nombreux
auteurs indépendants.
Ces travaux sont ainsi utilisés dans ce document comme fil conducteur auxquels des
références plus récentes ou plus anciennes sont greffées. Ce choix est fait afin
de pouvoir apprécier l’étendue des travaux dans le domaine tout en limitant au
maximum les redondances.

Dans la sous-tâche A, une méthodologie est définie pour le développement de modèles
numériques et trois indicateurs sont introduits \eqref{eq:indicateur_performance}. $F_{sav,\,th}$ est le taux d’économie
caractérisant la réduction de la
consommation de l’appoint au regard d’un système de référence. Le second ($F_{sav,\,ext}$) ajoute
la prise en compte des consommations des auxiliaires électriques. Le dernier ($F_{sav, si}$) ajoute en
plus une pénalité qui est proportionnelle à l’écart entre la température de consigne et la température
fournie.
Finalement afin de pouvoir comparer la performance des \abr{SSC} entre eux, un indicateur
indépendant du site et du système étudié est aussi introduit~: le \abr{FSC} (Fractional Solar Consumption) \eqref{eq:FSC}.
Il correspond à la part solaire théorique maximale que le système peut couvrir dans un cas idéal avec $I_{inc} \times A$
l’ensoleillement à la surface des capteurs. Il est donc défini comme étant la somme des productions solaires mensuelles
utiles divisée par la consommation totale de l’appoint \parencite{Letz2009}. Il est alors possible
de comparer entre eux plusieurs \abr{SSC}, le plus performant étant celui qui a le meilleur taux d’économie
($F_{sav,\,therm}$ ou $F_{sav,\,ext}$ ou $F_{sav,\,si}$) pour un même $FSC$. De plus il est montré
que le taux d’économie peut s’exprimer en fonction de $FSC$ sous la forme d’un polynôme de second ordre
permettant d’obtenir une estimation de la performance du système pour différentes conditions climatiques.

\begin{subequations}\label{eq:indicateur_performance}
    \begin{align}
        F_{sav, therm} &= 1 - \frac{Conso_{complément}}{Conso_{ref}}                        \\
        F_{sav, ext}   &= 1 - \frac{Conso_{complément} + Conso_{aux}}{Conso_{ref} + Conso_{aux, ref}}  \\
        F_{sav, si}    &= 1 - \frac{Conso_{complément} + Conso_{aux} + Conso_{pénalité}}{Conso_{ref} + Conso_{aux, ref}}
    \end{align}
\end{subequations}

\begin{equation}\label{eq:FSC}
        FSC = \frac{\sum_{1}^{12} \left( \min(Conso_{ref, mois}, I_{inc} \times A) \right)}{Conso_{ref}}
\end{equation}

Dans la sous-tâche B, un nouveau modèle de capteur solaire détaillé est ajouté à
\textit{TRNSYS}, (type~$132$). Le rendement du capteur est défini à partir des équations
quasi-dynamiques explicitées à travers \ref{ssub:rendement_d_un_capteur_solaire_thermique}.
Finalement dans la sous-tâche C \parencite{Task26C2007}, une analyse détaillée a été
réalisée sur neuf systèmes différents balayant une large variété de paramètres tels que~:
la surface de capteurs, l’orientation, l’inclinaison, le débit de puisage, la position des
capteurs dans le ballon, la position de l’échangeur solaire, la différence de température
minimale pour le chauffage et l’ECS, climat, qualité de l’enveloppe\dots\ Chaque système
est modélisé dans \textit{TRNSYS} et est évalué par rapport au même cas de référence à
travers une analyse paramétrique. La sensibilité du \abr{SSC} pour chaque paramètre est
ensuite déduite des résultats. Les systèmes considérés ont tous une spécificité propre et
la tâche donne ainsi une vision globale des facteurs les plus impactants pour les
différents systèmes existants sur le marché.

\paragraph{Systèmes de la sous-tâche C~:} % (fold)
\label{par:systemes_de_la_sous_tache_c}
Les différents systèmes sont décrits brièvement afin de fournir un aperçu des
technologies et logiques de contrôle existantes (\tabref{tab:diff_ssc}).
Sur les neuf systèmes, deux sont pensés pour une application en maisons ou appartements
collectifs, \emph{\#9b} et \emph{\#19},avec respectivement trois et deux ballons.
\emph{\#9b} comporte deux ballons pour l’$ECS$ dont un immergé
dans le ballon tampon. \emph{\#19} utilise le même réseau secondaire pour le chauffage
et la production \abr{ECS}.

Au niveau des capteurs, la pompe solaire est dans toutes les configurations déclenchée par
le différentiel de température entre la sortie des capteurs et le bas du ballon principal.
Dans le cas de \emph{\#12}, \emph{\#15}, et \emph{\#8}, le débit est modulé afin de
maximiser le rendement des capteurs. Dans \emph{\#12}, le choix a été fait d’utiliser deux
échangeurs, un en partie basse, et un vers le centre (sous l’appoint électrique) afin
d’augmenter la stratification du ballon. Alors que le système \emph{\#2} utilise
un unique échangeur externe à plaques, le système \emph{\#19} fonctionne avec un
échangeur externe à plaque couplé à un échangeur en tube stratifié interne afin de mieux
répartir l’énergie et améliorer la stratification. Les autres systèmes considèrent
uniquement des échangeurs tubulaires internes.
Le système \emph{\#15} est l’unique système tout-en-un regroupant en une unique unité,
un échangeur solaire en tube stratifié, une pompe solaire,
une chaudière gaz, un échangeur à plaques pour l’\abr{ECS}, et un ballon de stockage.
De par son approche compacte, l’installation sur site est facilitée.
Parmi les systèmes, seules les systèmes \emph{\#8}, \emph{\#9b},
et \emph{\#19} assurent la logique de contrôle complète du système. Les autres systèmes
sont dépendants d’un contrôleur externe pour la gestion du chauffage même si le système
\emph{\#12} est capable de désactiver l’appoint électrique si la pompe solaire se met en marche.

Pour la production d’\abr{ECS}, cinq systèmes exploitent une production
d’$ECS$ indirecte, par l’intermédiaire d’un échangeur. \emph{\#19}, \emph{\#15} ayant fait le
choix d’utiliser un échangeur externe à plaque. Pour \emph{\#19}, l’échangeur
assure le transfert d’énergie vers le second ballon chargé une à deux fois par jour afin
de refaire le stock d’\abr{ECS}. \emph{\#12}, \emph{\#11}, et \emph{\#8} optent eux pour
un échangeur interne couvrant la hauteur du ballon de stockage. La géométrie
du serpentin est cependant optimisée uniquement pour récupérer de la chaleur en
partie basse et haute du ballon. Ces options permettent de maintenir une
stratification dans le ballon lors des puisages importants contrairement à une approche
directe. En effet dans les autres systèmes l’\abr{ECS} est puisée directement du ballon.

La couverture du chauffage par le solaire est assurée par un circuit direct pour
\emph{\#3a} et \emph{\#2}. Le premier ne permet pas au solaire d’assurer simultanément
le chauffage et la production d’\abr{ECS} alors que le second ne permet pas de fournir
aux émetteurs de l’énergie provenant en même temps de l’appoint et du solaire.
Les sept autres systèmes s’orientent eux vers un chauffage indirect uniquement. Il est noté
qu’aucun des systèmes étudiés ne permet l’alternance entre chauffage direct et indirect.
% paragraph systemes_de_la_sous_tache_c (end)

\paragraph{Évaluation globale~:} % (fold)
\label{par:evaluation_globale}
Les résultats des différentes approches sont décrits à travers la \figref{fig:compare_perf_ssc}.
Il est observé que pour une surface faible de capteurs solaires ($FSC$ faible), le système \emph{\#15}
obtient des performances plus importantes. Le système \emph{\#3a} arrive en seconde place
mais nécessite une surface de capteur plus importante. Sans surprise le système pensé
pour des installation collectives \emph{\#19} est le moins performant. Le seul système
ne permettant pas de fournir à la fois de l’énergie pour la production d’\abr{ECS}
et pour le chauffage (\emph{\#2}) fait lui aussi partie des systèmes les moins performants.
Finalement, les résultats s’accordent à montrer que la performance du système augmente avec la réduction des
besoins en chauffage et ce indépendamment du système considéré.

\begin{figure}
    \centering
    \includegraphics[width=0.85\textwidth]{Ressources/Images/Solaire/compare_systems.png}
    \caption[Comparaisons de la performance des \abr{SSC} étudiés dans la tâche\,$26$]
            {Comparaisons de la performance des \abr{SSC} étudiés dans la tâche
             \,$26$ d’après \textcite{Task26C2007}.}
    \label{fig:compare_perf_ssc}
\end{figure}
% paragraph evaluation_globale (end)

% Used by previous paragraph
\begin{landscape}
\begin{table}
\centering
\small
\caption[Différences principales entre les différents modèles modélisés dans la tâche\,$26$]
        {Différences principales entre les différents modèles modélisés dans la tâche\,$26$.
         Le CHauffage est désigné par \abr{CH} et l’Eau Chaude Sanitaire par \abr{ECS}.}
\label{tab:diff_ssc}
\begin{tabular}{l C{2cm} C{1.5cm} C{0.6cm} C{1.4cm} C{1.7cm} C{1.7cm} C{1.2cm} C{1.5cm} C{1.2cm} C{2.5cm} C{1cm} r}
    \toprule
            & Type de Chaudière    & Appoint électrique  & \abr{CH} mixte  & \abr{CH} et \abr{ECS} ensemble &
              Type $Prod_{CH}$ & Type $Prod_{ECS}$ & Débit  variable & Contrôleur unique \abr{CH} et \abr{ECS} &
              $Ech_{sol}$ externe & Nombre d’échangeurs & Nombre de ballons & Référence                           \\
    \midrule
    \addlinespace[1.5\defaultaddspace]
    \#2 & Gaz / fioul & Oui & Oui & Non & Directe & Directe & Non & Non & Oui & $2$ & $1$ & \cite{Ellehauge2002}  \\
    \addlinespace[1.5\defaultaddspace]
    \#3a  & Gaz & Non & Non & Oui & Directe & Directe & Non & Non & Non & $2$ & $1$ & \cite{Cheze2002}  \\
    \addlinespace[1.5\defaultaddspace]
    \#4 & Gaz /  fioul  & Non & - & Oui & Indirecte & Directe & Non & Non & Non & $2$ & $1$ & \cite{Shah2002} \\
    \addlinespace[1.5\defaultaddspace]
    \#8 & Gaz / fioul & Non & - & Oui & Indirecte & Indirecte & Oui (capteurs)  & Oui & Non & $2$ & $1$ & \cite{Bony2002} \\
    \addlinespace[1.5\defaultaddspace]
    \#9b  & Gaz / fioul & Oui & - & Oui & Indirecte & Directe (ballon immergé)  & Non & Oui (\abr{CH})  & - & $0$ (circulation directe) & $3$ & \cite{Peter2003}  \\
    \addlinespace[1.5\defaultaddspace]
    \#11  & Bois / gaz / fioul  & Oui & - & Oui & Indirecte & Indirecte & Non & Non & Non & $2$ & $1$ & \cite{Bales2002}  \\
    \addlinespace[1.5\defaultaddspace]
    \#12  & Bois / gaz / fioul  & Oui & - & Oui & Indirecte & Indirecte & Oui (capteurs)  & Non & Non & $3$ ($2$ solaires)  & $1$ & \cite{Bales2002a} \\
    \addlinespace[1.5\defaultaddspace]
    \#15  & Gaz (unité compacte)  & Non & - & Oui & Indirecte & Indirecte (échangeur à plaques)  & Oui (capteurs et \abr{CH})  & Oui  & Non & $2$ + tube stratifiés & $1$ & \cite{Jaehnig2002}  \\
    \addlinespace[1.5\defaultaddspace]
    \#19  & Gaz / fioul & Non & Oui & Non & Indirecte & Indirecte (échangeur à plaques)  & Non & Oui & Oui & $2$ + tube stratifiés & $2$ & \cite{Heimrath2003} \\
    \addlinespace[1.5\defaultaddspace]
    \bottomrule
\end{tabular}
\end{table}
\end{landscape}
% subsubsection la_tâche_26 (end)


% - - - - - - - - - - - - - - - - - - - - - - - - - - - - - - - - - - - - - - -
\subsubsection{Travaux fondamentaux} % (fold)
\label{ssub:travaux_fondamentaux}
\paragraph{Inclinaison et orientation~:} % (fold)
\label{par:inclinaison_et_orientation}
L’orientation optimale des capteurs est au sud, légèrement décalée vers
l’ouest (\SIrange{-5}{-10}{\degree}), la température plus élevée durant l’après midi pouvant
expliquer ces résultats.
L’inclinaison joue aussi un rôle prépondérant et les résultats suggèrent \SI{50}{\degree}
pour une orientation optimale à une latitude de \SI{45}{\degree}. Cette observation est
aussi reportée par \textcite{Shariah2002587} à travers une analyse paramétrique qui montre
que l’inclinaison optimale est fonction de la latitude du site et est définie par
$\phi + $(\SIrange{0}{10}{\degree}) pour les pays de l’union européenne. La plage de performance
maximale est en effet montrée comme dépendante de la surface de capteurs.
Une surface plus importante (relativement aux besoins), autorise une plage plus importante
pour l’inclinaison optimale. À l’inverse l’ensoleillement annuel est maximal pour
une inclinaison $\phi - \SI{8}{\degree}$. Une analyse de l’évolution mensuelle met
en évidence que réduire l’inclinaison favorise le système durant la période estivale
là où le rayonnement solaire est déjà maximal. Il est donc recommandé à la fois pour diminuer
les risques de surchauffe et pour améliorer la performance de favoriser une
inclinaison selon la première relation.
\textcite{Badescu2006129} modélisent grâce au logiciel \textit{TRNSYS}, un \abr{SSC}
couplé à une maison passive (selon le standard \textit{PassivHauss}) de \SI{150}{\metre\squared}
qui par définition impose un système de chauffage utilisant le vecteur air. L’approche montre que des
capteurs verticaux semblent améliorer la performance du système par rapport à des
capteurs implémentés en toiture sur une structure inclinée à \SI{78}{\degree} adaptée à l’optimisation
des apports en hiver.
% paragraph inclinaison_et_orientation (end)

\paragraph{Ballon de stockage~:} % (fold)
\label{par:ballon_de_stockage}
Les différentes études menées durant la tâche $26$ mettent aussi en évidence l’importance
de la position des différents équipements techniques et en particulier au sein du ballon
de stockage. Il est ainsi préférable de disposer en partie basse l’échangeur solaire
afin de maximiser la part solaire récupérée. L’appoint doit lui être en partie haute afin de réserver
le chauffage de la majeure partie du volume au solaire. Il est cependant nécessaire
de ne pas le positionner trop haut afin de garantir la température de puisage~: le volume
réservé est donc fonction de la réactivité de l’appoint. Les systèmes \emph{\#11} et
\emph{\#12} recommandent ainsi pour une chaudière bois de placer le retour en partie
basse du ballon alors qu’il est placé en partie haute sur des chaudières gaz.
Au regard de ces résultats, le volume du ballon maintenu par l’appoint doit donc
être le plus faible possible. De même la température de maintien doit être minimale.
Un bon réglage permet donc de maximiser la part solaire sans dégrader le confort des occupants.
Dans le cas d’un système indirect, il est préférable de réserver la partie centrale
du ballon
pour fournir de l’énergie au circuit secondaire de chauffage. Il est aussi noté que
la proximité de l’échangeur pour le chauffage risque de parasiter le volume supérieur
qui doit être maintenu à une température plus importante et donc affecter les performances.
La logique de contrôle étant dépendante des informations retournées par les sondes, il est
recommandé de placer la sonde de température pour l’échangeur solaire à un tiers
en dessous de sa position haute afin de maximiser la part d’énergie récupérée.
% paragraph ballon_de_stockage (end)

\paragraph{Relation entre capteurs et ballon~:} % (fold)
\label{par:relation_entre_capteurs_et_ballon}
Il est identifié que l’impact des capteurs est moins important à mesure
que la surface augmente \parencite{Task26C2007}. En effet, la température du stockage augmentant les pompes
tournent plus longtemps sans pour autant apporter plus d’énergie au système. En dehors
de la période hivernale, remplacer la chaudière dont le rendement décroit par un appoint
électrique est montré comme pertinent économiquement et énergétiquement.
Une forte interaction entre surface de capteur et volume de ballon est aussi identifiée
mais aucune corrélation n’est mise en avant au regard des résultats très disparates
obtenus en fonction du système considéré. Il est cependant conseillé de ne pas utiliser
un volume de stockage plus grand que \SI{100}{\litre\per\metre\squared (capteurs)}.
\textcite{Lund200559} montre à l’aide d’un modèle analytique journalier que le volume
du ballon de stockage doit permettre de stocker de l’énergie pour quelques jours et que
chercher à l’augmenter au delà n’apporte pas de bénéfices.
Ces résultats sont intégrés et complétés avec l’ajout de six nouveaux systèmes à travers
le projet européen \href{http://www.combisol.eu/}{\textit{CombiSol}} ($2007$-$2010$) qui reprend
les travaux existants (notamment ceux de la tâche $26$) dans une approche plus pragmatique où
chaque composant est analysé pour lister les bonnes pratiques et recommendations \parencite{Thur2011}.

Une étude plus récente \parencite{Tsalikis2015743} identifie aussi cette interaction sur
un \abr{\textit{Net} ZEB}. Le système est implémenté en Grèce en suivant la méthodologie
\textit{F-Chart} et considère un plancher chauffant et une chaudière fuel ou gaz selon
l’accessibilité au réseau du site étudié. L’étude montre que le meilleur temps de retour
sur investissement obtenu varie entre \SIrange{4.87}{6.45}{ans} en fonction du climat
considéré. Il est aussi montré que pour toutes les configurations étudiées
(\SIrange{12}{24}{\metre\squared} (capteurs) et \SIrange{750}{2000}{\litre} (ballon))
la valeur nette actuelle est toujours positive faisant du système solaire un
investissement toujours favorable pour le \abr{\textit{Net} ZEB} considéré. De plus un
investissement plus important permet d’améliorer la valeur nette actuelle. La prise en
compte de la production des panneaux photovoltaïques et du système solaire thermique permet
de couvrir de \SIrange{76}{97}{\percent} de la consommation totale en énergie primaire.
Dans le cas d’une maison efficiente \parencite{Martinopoulos2014130}, la même méthodologie
est suivie. Les résultats indiquent un temps de retour sur investissement similaire pour une
chaudière au fioul et de \SIrange{8.5}{10}{ans} lorsque une chaudière au gaz est installée.
Dans les deux cas le système est capable de couvrir entre \SI{45}{\percent} et
\SI{95}{\percent} des besoins totaux variant de \SIrange{250}{3400}{kWh}.
% paragraph relation_entre_capteurs_et_ballon (end)

\paragraph{Stockage saisonnier~:} % (fold)
\label{par:stockage_saisonnier}
Des systèmes moins conventionnels sont aussi traités dans la littérature.
\textcite{Hartl2012623} décrivent ainsi une étude numérique et expérimentale sur une
chaudière biomasse couplée à des capteurs solaires. Les résultats acquis hors période
estivale (novembre à mai) sont encourageants avec un taux de couverture de
\SI{82}{\percent}. \textcite{Ucar20082532} évaluent le potentiel d’un stockage
solaire saisonnier à travers quatre configurations couplées à une \abr{PAC} pour couvrir les
besoins en chauffage de respectivement $25$, $250$, $1000$ maisons~: ballon non-isolé,
ballon isolé, ballon non-isolé enterré, ballon isolé enterré. En se basant sur une
valeur mensuelle pour l’irradiation et de la température extérieure, il est observé que
le stockage solaire enterré permet d’atteindre une meilleure couverture et que les
économies financières et énergétiques sont réduites à mesure que le nombre de bâtiments
alimentés par le système augmente. Des travaux similaires appliqués à une maison
individuelle ont aussi été réalisés \parencite{Yumrutas2012983} complétant les travaux
précédents en indiquant une forte dépendance entre la température moyenne du ballon enterré
et le type de sol, exception faite de la période hivernale.
% paragraph stockage_saisonnier (end)
% subsubsection travaux_fondamentaux (end)


% - - - - - - - - - - - - - - - - - - - - - - - - - - - - - - - - - - - - - - -
\subsubsection{Optimisation de \abr{SSC}} % (fold)
\label{ssub:optimisation_de_ssc}
Des travaux récents cherchent à évaluer la performance des systèmes solaires à travers
un processus d’optimisation. L’optimisation est le processus permettant de trouver les
solutions maximisant ou minimisant un ou plusieurs objectifs et sera détaillée dans la
suite de ce document. Contrairement aux approches paramétriques qui utilisent des variations
successives sur un paramètre, l’optimisation explore l’espace des solutions de manière automatisée
et pouvant faire varier plusieurs paramètres simultanément.

\textcite{Fraisse2009232} présentent une étude comparative des différents objectifs retenus lors des études antérieures
d’optimisation de système solaires pour la production d’\abr{ECS}. Le système retenu est
évalué successivement suivant sa performance d’un point de vue énergétique (exergie,
$F_{sav, ext}$, et $F_{sav, si}$), économique (coût du cycle de vie, \abr{CCV}), et
écologique (émissions de $CO_{2}$ puis en combinant plusieurs objectifs en un objectif
unique. L’exergie permet de quantifier la qualité d’une énergie, celle ci diminuant au cours des
transformations successives. Le travail maximum récupérable est ainsi égal à l’opposé de
la variation d’exergie au cours d’une transformation. L’auteur de l’article
cherche donc à maximiser le rendement exergétique qui est le rapport entre l’exergie en
sortie du système solaire (exergie fournie à l’\abr{ECS} pour l’occupant) et l’exergie
en entrée du système (auxiliaires électriques et irradiation solaire).
Le système est décrit dans \textit{TrnSys} et optimisé grâce à
\textit{GenOpt} en faisant varier la surface de capteurs, le volume du ballon sanitaire,
et le débit du fluide au travers des capteurs. Pour les différents objectifs la surface de
capteur est toujours proche de la borne maximale exception faite lorsque l’objectif est de
minimiser l’investissement pour chaque
\si{kWh} économisé. Au regard des résultats, l’exergie ou le taux de couverture solaire
semblent favoriser une part importante d’énergie solaire mais ne permettent pas de refléter
la performance réelle du système. Il est en effet mis en avant que favoriser la production
solaire au lieu de favoriser la réduction de la part de l’appoint tend à sur-dimensionner
le système afin d’augmenter la part solaire en période estivale. Finalement, il est mis en
avant trois approches permettant d’obtenir une  bonne performance tant sur la réduction du
coût, de la consommation, et des émissions en $CO_{2}$~:
\begin{itemize}
    \item Minimiser le coût sur le cycle de vie du système (\abr{CCV})
    \item Minimiser la consommation de l’appoint et des auxiliaires ($F_{sav, ext}$)
    \item Minimiser la combinaison du coût et des consommations
    \item Minimiser \eqref{eq:qmin_qsol}
\end{itemize}

\begin{equation}\label{eq:qmin_qsol}
    \frac{Conso_{complément} + Conso_{aux} + Conso_{pénalité}}{Prod_{sol}}
\end{equation}

Bien que le système considéré ne permet pas de couvrir les besoins en chauffage du
bâtiment, les remarques restent transposables aux \abr{SSC} qui doivent eux aussi
couvrir les besoins en \abr{ECS}.
\textcite{Deng2013212} évaluent le potentiel d’un couplage entre une \abr{PAC} $CO_{2}$ et
d’un \abr{SSC} à travers une étude paramétrique puis une optimisation. L’optimisation est
réalisée sur le \abr{CCV} et il est montré que coupler la \abr{PAC} avec le
solaire permet d’améliorer son \abr{COP}. Le temps de retour du système optimal est estimé
à un peu moins de \SI{10}{ans} avec un taux de couverture de \SI{69}{\percent}. Une autre
approche utilisant un appoint électrique est investiguée sur trois indicateurs
\parencite{Hin2012,Hin2014102}~:
le \abr{CCV}, le cumul de l’énergie grise et de l’énergie consommée sur la durée de vie
(\abr{LCE}), et de l’exergie \abr{LCX}. Les résultats montrent que le choix de l’objectif
influence fortement les résultats autant sur la performance du système que sur son coût.
Optimiser en fonction du \abr{CCV} le système optimal obtenu considère un unique capteur
solaire offrant un taux de couverture de seulement \SI{21}{\percent} alors que si on
considère l’objectif \abr{LCE}, le taux de couverture obtenu est de \SI{74}{\percent}.
Cependant la première approche permet d’obtenir un \abr{CCV} $\nicefrac{1}{3}$ inférieur à
la solution optimisée sur le \abr{LCE}. \textcite{Bornatico201231} utilisent eux le
logiciel \textit{PolySun} pour évaluer un \abr{SCC} à travers $F_{sol}$, la
consommation totale, et le coût d’investissement groupés en un unique objectif. Le
système est optimisé avec deux méta-heuristiques, un essaim particulaire et un algorithme
génétique. La solution optimale obtenue dépend peu de l’algorithme utilisé.
\textcite{Asaee2014510} évaluent la performance d’un \abr{SSC} (système \emph{\#14} de la tâche
$26$) couvrant chauffage, \abr{ECS}, et climatisation. L’étude considère un modèle
mono-zone et trois ballons dont un pour stocker l’énergie solaire lorsque les besoins sont déjà
couverts. La logique de contrôle retenue considère deux modes de fonctionnement, un mode
hiver et un mode été. Durant l’été la priorité est toujours donnée à la production
d’\abr{ECS} alors que durant l’hiver la priorité est donnée à la production d’\abr{ECS} la
nuit et au chauffage le jour. Le système montre de très bonnes performances pour les
quatre climats étudiés avec un $F_{sol}$ supérieur à \SI{70}{\percent}. Finalement
plus récemment, une comparaison entre une optimisation multi-objectifs et une optimisation
mono-objectif d’un \abr{SSC} couplé à un \abr{\textit{Net} ZEB} est discutée
\parencite{Rey2016622}. Comme pour les autres optimisations, seuls les équipements
composant le \abr{SSC} sont retenus comme critères de décision. L’optimisation
mono-objectif est réalisée sur la somme pondérée des différents objectifs (\abr{LCE}
et \abr{CCV}). L’étude montre que l’obtention de solution résultant d’un compromis entre
les divers objectifs est un travail fastidieux et coûteux en temps lorsque l’approche
mono-objectif est utilisée.
% subsubsection optimisation_de_ssc (end)


% - - - - - - - - - - - - - - - - - - - - - - - - - - - - - - - - - - - - - - -
\subsubsection{Bilan} % (fold)
\label{ssub:bilan_travaux}
Les \abr{SSC} existent ainsi sous de nombreuses variations utilisant soit une chaudière,
soit une \abr{PAC}, soit simplement un appoint électrique. La performance de ces systèmes
est très variable et de nombreux facteurs influents ont été identifiés, et leur
sensibilité analysée. Il est cependant aussi observé que la majorité des études ne
considèrent que les différents équipements principaux du \abr{SSC} (capteurs, volume des
ballons). Dans ces approches, la qualité de l’enveloppe et la logique de contrôle sont
définies en amont et fixée même lors des études d’optimisation. Avec une variation de \SIrange{5}{100}{ans}
\parencite{Tsalikis2015743,Hin2014102}, une forte disparité des résultats obtenus pour le
calcul du temps de retour sur investissement est identifiée. Les différents travaux
assument une approche plus ou moins simplifiée qui ne prend pas en compte toute
la complexité inhérente à l’évaluation du coût d’une installation solaire. Il existe en
effet un nombre important d’incertitudes, notamment le coût de la main
d’œuvre, l’évolution des prix des énergies, le taux d’inflation, ou encore les coûts de la
maintenance. De plus en fonction des pays considérés, les aides de l’état ou des régions,
les taux d’intérêts, le prix des équipements et des énergies varient. L’évaluation de la
rentabilité d’une installation solaire devrait aussi tenir compte de la valeur ajoutée au
bâtiment (la valeur verte), des spécificités de la région, du coût de démantèlement, ainsi
que du coût lié à l’environnement. Elle doit ainsi faire l’objet
d’une attention particulière afin de tenir compte de
l’ensemble des facteurs économiques et politiques et nécessite ainsi l’accès aux données
d’acteurs du marché à partir desquelles des hypothèses peuvent être avancées. Sur
l’évaluation de la performance énergétique de nombreux indicateurs sont identifiés
(\eqref{eq:f_sol}, \eqref{eq:indicateur_performance}, exergie). Il est aussi noté que
suivant l’indicateur considéré, des précautions doivent être prises afin d’éviter une
surproduction en période estivale et/ou une sur-estimation de la performance du \abr{SSC}.
Finalement, au cours de la tâche $26$ une corrélation a
été trouvée permettant de comparer entre eux les différents \abr{SSC} indépendamment des
conditions climatiques ou des besoins à travers l’indicateur $FSC$ \eqref{eq:FSC}.

Des verrous et conclusions propres au sujet de ces travaux ont ainsi pu être identifiés~:
\begin{itemize}
  \item Les énergies renouvelables et donc l’énergie solaire se placent comme une
        alternative pertinente pour lutter contre le réchauffement climatique.
  \item La norme \abr{PREN\,$15603$} amène à reconsidérer les moyens mis à mettre
        en œuvre pour la conception de \abr{BEPOS}.
  \item Compenser les consommations par une production locale renouvelable n’est plus un critère
        suffisant, cette production doit permettre de couvrir directement une part
        importante des consommations du bâtiment.
  \item Le développement du solaire thermique stagne en Europe et les \abr{SSC}
        sont très peu utilisés.
  \item Il est nécessaire de repenser la conception des \abr{SSC} pour les revaloriser
        dans le cadre des \abr{MEPOS}.
  \item Les \abr{SSC} doivent être évalués suivant un bilan exportation / importation
        afin d’optimiser non pas la production solaire mais la production solaire
        en adéquation avec la demande.
  \item La conception d’une \abr{MEPOS} solaire doit être le compromis entre qualité
        de l’enveloppe et performance du \abr{SSC}.
  \item Afin d’évaluer le potentiel solaire, le modèle doit être détaillé pour
        permettre l’implémentation d’une logique de contrôle adaptée et évaluer
        son influence.
  \item La logique de contrôle doit faire partie du processus de décision
  \item La taille des équipements du \abr{SSC} doit être réduite afin d’éviter des
        surcoûts et un encombrement importants.
\end{itemize}
% subsubsection bilan (end)
% subsection les_travaux_sur_le_solaire_thermique (end)
% section les_systemes_solaires (end)




% ..............................................................................
% ..............................................................................
\section{Conception de maisons solaires à énergie positive} % (fold)
\label{sec:conception_de_maisons_solaires_a_energie_positive}
Cette dernière section décrit l’approche retenue dans ces travaux afin de répondre à la
problématique et aux contraintes listées ci-avant. La méthodologie est développée pour
répondre à la problématique de conception dans le bâtiment durant
laquelle la faisabilité technique du projet est vérifiée et le planning des travaux
détaillé (\ref{cha:construction_batiment}).

% ------------------------------------------------------------------------------
\subsection{Critique de l’approche actuelle dans le bâtiment} % (fold)
\label{sub:critique_de_l_approche_actuelle_dans_le_batiment}
La construction d’un bâtiment est à l’origine guidée par le bon sens et l’expérience
acquise dans l’optique de satisfaire aux besoins immédiats du client.
Les ressources énergétiques d’origine fossile étant limitées, des
réglementations sont mises en place au $20^{ème}$ siècle afin de sécuriser
l’approvisionnement. Ces réglementations sont ensuite renforcées afin de tenir
compte des changements climatiques induits par la sur-utilisation des énergies
fossiles en visant à toujours plus d’exemplarité. Le secteur du bâtiment reste
cependant encore aujourd’hui le plus gros consommateur d’énergie.

Parallèlement les progrès de l’informatique ont permis le développement d’outils
techniques et économiques, marquant une évolution dans le processus de conception.
Au début réservée à une élite, sa démocratisation a permis de l’exploiter dans le domaine de
l’énergétique du bâtiment et plus particulièrement dans les bureaux d’études. Cependant
les habitudes sont fortement ancrées et l’outil numérique est aujourd’hui principalement
employé à travers une approche par essai/erreur. À partir d’un cas de référence
généralement défini par l’expérience, un ensemble de variations sont introduites à l’aide
d’itérations successives. Ce processus est coûteux en temps humain et ne permet pas de
valoriser la puissance de calcul disponible. Elle offre pourtant la possibilité d’explorer
un nombre important de solutions et de vérifier des suppositions plus rapidement. De plus,
les réalités économiques et la forte concurrence lors des appels d’offres amènent les
bureaux d’études à répondre de plus en plus rapidement aux marchés dont la complexité et
la combinatoire augmentent parallèlement aux réglementations.

Ainsi, afin de pouvoir répondre au nouvelles contraintes du marché, le temps humain doit
être valorisé pour des tâches cognitives, et le temps machine assujetti aux tâches
répétitives. En effet la miniaturisation, la finesse de gravure, et l’augmentation du
nombre de transistors par processeur a permis une croissance exponentielle de la puissance
de calcul. L’arrivée des processeurs multi-cœur a encore repoussé les limites de calcul,
si bien que le développement d’une méthodologie basée sur le temps machine est tout à fait
pertinente. Dans cette optique il est nécessaire d’automatiser le processus de décision en
phase de conception grâce à l’intelligence artificielle. Cette approche nécessite cependant
le développement d’une méthodologie adaptée.
% subsection critique_de_l_approche_actuelle_dans_le_batiment (end)


% - - - - - - - - - - - - - - - - - - - - - - - - - - - - - - - - - - - - - - -
\subsection{Approche proposée} % (fold)
\label{sub:approche_proposee}
La construction d’un bâtiment est par définition multi-objectifs, elle fait intervenir un
nombre conséquent de paramètres que ce soit sur l’enveloppe, les systèmes, la logique de
contrôle, le climat, ou encore le comportement des occupants. Elle doit de plus à minima
satisfaire un cadre réglementaire au niveau acoustique, thermique, et mécanique. Par extension
la construction d’une \abr{MEPOS} solaire est donc aussi multi-objectifs.

\paragraph{} % (fold)
Au cours de la première section, il a été mis en évidence la nécessité de s’émanciper des
énergies fossiles, dont la sur-utilisation est responsable des fortes émissions de
$CO_{2}$. Dans cette optique, un cadre réglementaire a évolué et des approches
associatives ont aidé à son développement notamment avec la mise en place de labels
certifiant une qualité supérieure favorisant le recours aux énergies alternatives. L’état est
aussi un acteur clé à travers la mise en place d’une coopération internationale et de
mesures incitatrices~: la politique de rachat de l’énergie électrique produite localement
ou encore les aides financières obtenables lors de travaux de rénovation ou d’exemplarité
énergétiques. Les actions combinées de tous ces intervenants permettent
aujourd’hui d’être à l’aube d’un cadre réglementaire européen (\abr{PREN\,$15603$}) sur la
\abr{MEPOS} dont chaque pays membre devra s’inspirer pour la définition de leur nouvelle
réglementation thermique nationale. Bien que la norme ne soit pour l’heure qu’un
brouillon, les axes principaux sont d’ores et déjà définis. La norme s’articule autour de
quatre objectifs péremptoires~: assurer une performance minimale sur l’enveloppe, assurer
l’efficacité des systèmes installés, assurer un bilan énergétique global positif, et
assurer une part minimale d’énergie renouvelable pour la production locale.

Ces travaux cherchent ainsi à travers une approche innovante à valoriser l’énergie
solaire thermique aussi bien sur le chauffage que sur l’\abr{ECS} afin d’assurer
la construction de maisons à énergie positive ayant une empreinte énergétique sur
sa durée de vie la plus faible possible. Comme le souligne l’association \textit{négaWatt}
\enquote{L’énergie qui pollue le moins est celle que l’on ne consomme pas.}.
De plus comme l’indique la norme \abr{PREN\,$15603$} à travers l’indicateur $RER$,
la consommation locale en énergie primaire devra être fortement couverte
par des énergies renouvelables et l’évaluation de ce critère est réalisée par vecteur énergétique.
L’approche choisie à travers ces travaux est donc la minimisation de la consommation
d’énergie non renouvelable en couvrant au maximum la demande énergétique par le solaire
thermique. La production d’énergie photovoltaïque
est considérée comme complémentaire et doit permettre de couvrir les besoins électriques
de plus en plus importants dans les maisons individuelles.
Il est évalué à la fois les besoins de chauffage et d’\abr{ECS}, mais aussi les charges internes
(éclairage et électro-domestiques), qui représentent une part importante de la consommation des \abr{MEPOS}.
Les autres objectifs de la norme \abr{PREN\,$15603$} sont aussi pris en compte
même si plus de liberté est prise afin d’explorer l’ensemble des solutions envisageables.

Ainsi la méthodologie retenue cherche à caractériser le bâtiment et ses systèmes conjointement
afin d’offrir une cohésion de l’ensemble. En effet la norme restreint fortement l’exploratoire
en imposant d’une part une enveloppe performante et d’autre part une limitation de la
consommation sur site. Il apparaît pour l’auteur que ces contraintes relèvent
d’une volonté de sobriété énergétique afin de garantir une empreinte énergétique
la plus faible possible. Cependant dans le cas d’une maison solaire il est intéressant
d’explorer les combinaisons complémentaires possibles qui pourraient offrir des alternatives
cohérentes, sans aller à l’encontre de la sobriété.
Finalement ces travaux intègrent l’énergie primaire comme indicateur de référence
pour la définition du bilan positif comme décrit par la $4^{ème}$ directive de la
prochaine norme européenne. La méthodologie retenue impose ainsi un bilan positif
sur les quatre usages considérés~: le chauffage, la production d’\abr{ECS}, la
consommation de l’électro-ménager, et la consommation de l’éclairage.

\paragraph{} % (fold)
Parallèlement, il a été mis en évidence que le solaire thermique est une technologie ancienne
permettant de valoriser une énergie quasi-infinie. Les bâtiments sont ainsi aujourd’hui
pensés pour maximiser la part solaire passive durant la période de chauffage afin de
limiter au maximum sa consommation. Le solaire actif est aussi fortement utilisé dans le
neuf comme dans la rénovation afin de limiter les consommations nécessaires à la
production d’\abr{ECS}. Cependant les \abr{SSC}, développés pour couvrir à la fois les
besoins en \abr{ECS} et le chauffage restent principalement une solution pour la
rénovation. De nombreux systèmes existent sur le marché et une majorité ont été étudiés à
travers la tâche $26$ et le projet \textit{CombiSol}, mais peu de travaux cherchent à
déterminer la performance pouvant être obtenue pour des \abr{MEPOS}. De plus parmi les
travaux existants, une partie évalue la performance des \abr{SSC} à travers des outils
simplifiés tels que \textit{F-chart} qui ne permettent pas de tenir compte de la dynamique
du système, ni des spécificités du \abr{SSC} retenu. D’autres approches utilisent des
outils permettant une approche détaillée comme \textit{TrnSys}. Cependant
même dans ces approches, l’algorithme de contrôle reste secondaire et les critères de
décision sont liés aux équipements solaires uniquement (surface de capteur, volume du
ballon\dots). Finalement, la performance est évaluée globalement, ne faisant pas de
distinction entre la part solaire apportée au chauffage et la part solaire apportée à la
production d’\abr{ECS}.

La performance d’un \abr{SSC} innovant est ainsi caractérisée à la
fois sur le chauffage, et sur la production d’\abr{ECS}, à travers une approche
multi-objectifs. De plus le système solaire, l’enveloppe du bâtiment, et la logique de contrôle
seront évalués simultanément afin d’explorer les différentes combinaisons existantes
permettant d’atteindre un bilan positif.
Dans cette optique, une modélisation détaillée du \abr{SSC} et de son algorithme de
contrôle est nécessaire. Le parti pris est le suivant~: les contraintes techniques sont
propres à chaque site, chaque industriel, mais aussi à chaque client dont les attentes
varient fortement. Partant de ce constat, il apparaît plus intéressant de proposer
un ensemble de solutions techniquement différentes mais répondant toutes aux contraintes
de la \abr{MEPOS} à travers une approche multi-objectifs. En effet une approche mono-objectif
contraint à l’obtention d’une solution unique. Les limitations propres aux approches
mono-objectif et multi-objectifs sont discutées plus amplement dans le chapitre III de ce
manuscrit.
% subsection approche_proposee (end)


% - - - - - - - - - - - - - - - - - - - - - - - - - - - - - - - - - - - - - - -
\subsection{Bilan} % (fold)
\label{sub:bilan_methode}
Ces travaux s’inscrivent donc en phase de conception en proposant une méthodologie
exploratoire permettant d’identifier des combinaisons pertinentes entre les équipements du
\abr{SSC}, la logique de contrôle, et l’enveloppe du bâti. La méthodologie s’inspire des
contraintes décrites à travers la norme \abr{PREN\,$15603$} afin d’aider au développement
des \abr{BEPOS} solaires adaptés aux contraintes du secteur du bâtiment et plus
particulièrement de la maison individuelle. S’appuyant principalement sur
l’intelligence artificielle et l’automatisation, elle permet d’optimiser le temps machine
et de revaloriser le temps humain disponible pour chaque marché.

Dans un premier temps, le choix des outils est introduit à travers une étude comparative.
Suit une description du bâtiment de référence et des développements réalisés sur le
\abr{SSC} à travers une présentation des équipements du système et d’une description détaillée
de la logique de contrôle implémentée. Le chapitre II permet ainsi de préciser
les scénarios d’usages tels que le profil de puisage retenu mais aussi d’évaluer
a priori la performance du \abr{SSC} pour différentes configurations et différents climats
à travers une étude paramétrique.

Le troisième chapitre décrit la méthodologie d’aide à la décision retenue. Il y est
présenté dans un premier temps les méthodes d’aide à la décision existantes mais
aussi les outils complémentaires comme l’analyse de sensibilité ou les modèles
de substitution. Dans un second temps l’optimisation multi-objectifs est présentée
et une analyse circonstanciée des méthodes existantes est réalisée afin de justifier
le choix de la méthodologie retenue.

Finalement le quatrième chapitre explicite la méthodologie d’optimisation retenue à
travers deux cas d’étude. Le processus complet d’aide à la décision est ainsi illustré
en parallèle pour des conditions climatiques distinctes, Bordeaux et Strasbourg. L’ensemble
des solutions obtenues par le processus d’optimisation multi-objectifs est alors discuté
et permet de mettre en avant de nombreuses remarques pertinentes pour aider au développement
des \abr{SSC}.
% subsection bilan (end)
% section conception_de_maisons_solaires_a_energie_positive (end)
