%!TEX root = ../main.tex
% Chapitres\Chap1-ContexteTravaux.tex



% ..............................................................................
% ..............................................................................
\section{Le solaire thermique appliqué au bâtiment} % (fold)
\label{sec:le_solaire_thermique_applique_au_batiment}

% ------------------------------------------------------------------------------
\subsection{Contexte énergétique} % (fold)
\label{sub:contexte_energetique}
\itodo{État de l’art sur performance des bâtiments, danger réchauffement, augmentation
       de la population}
\itodo{Parler de ce que fait NegaWatt notament avec le projet <Europe,Territoires>.\\
       \url{http://www.negawatt.org/telechargement/Docs/160615_Rapport-final_Europe-territoire_Phase1.pdf}\\
       \url{http://www.negawatt.org/telechargement/Docs/160324_Synthese_Etude_Europe-territoires_Phase1.pdf}}
% subsection contexte_énergétique (end)

% ------------------------------------------------------------------------------
\subsection{Le concept MEPOS} % (fold)
\label{sub:le_concept_mepos}
\itodo{Description des approches existantes: Passivhaus, NZEB, MEPOS}
\itodo{Décrire l’initiative de COMEPOS pour le label MEPOS}
\itodo{Décrire pourquoi ce choix (car il a apprit des anciens labels)}
% subsection le_concept_mepos (end)

% ------------------------------------------------------------------------------
\subsection{Le solaire thermique pour une production énergétique respectueuse} % (fold)
\label{sub:le_solaire_thermique_pour_une_production_energetique_respectueuse}
\itodo{État de l’art sur les applications du solaire thermique.}
\itodo{Montrer que le solaire thermique n’a pas la côte dans la construction à énergie positive.
       Problème de coût, de confiance, ou de performance ?}
\itodo{Montrer que l’innovation passe par le neuf avant d’être intégré à la rénovation.}
% subsection le_solaire_thermique_pour_une_production_énergétique_respectueuse (end)
% section le_solaire_thermique_appliqué_au_bâtiment (end)



% ..............................................................................
% ..............................................................................
\section{Une approche par optimisation} % (fold)
\label{sec:une_approche_par_optimisation}

% ------------------------------------------------------------------------------
\subsection{Approches explorées} % (fold)
\label{sub:approches_explorees}
\itodo{Décrire les différentes approches déjà explorées}
\itodo{Augmentation surface capteur, sur-isolation, mon approche couplée pour un
       compromis entre coût/surface capteur/isolation}
% subsection approches_explorées (end)

% ------------------------------------------------------------------------------
\subsection{L’optimisation d’une maison solaire: un problème multi-critère} % (fold)
\label{sub:l_optimisation_d_une_maison_solaire_un_probleme_multi_critere}
\itodo{Décrire brièvement les methodes existantes}
\itodo{Décrire les outils nécessaires (sensibilité, opimisation, aide à la décision)}
% subsection l_optimisation_d_une_maison_solaire_un_problème_multi_critère (end)
% section une_approche_par_optimisation (end)


% ..............................................................................
% ..............................................................................
\section{Le choix d’un modèle de système solaire couplé au bâtiment} % (fold)
\label{sec:le_choix_d_un_modele_de_systeme_solaire_couple_au_batiment}
% ------------------------------------------------------------------------------
\subsection{Les modèles existants} % (fold)
\label{sub:les_modeles_existants}
\itodo{Décrire le choix de l’approche par modélisation}
Il existe plusieurs moyens permettant d’évaluer un système énergétique. Le premier
consiste à reproduire expérimentalement le système et son environnement. Cependant
ce processus est couteux , spécialement lorsque on essayer d’évaluer un système à l’échelle
du bâtiment. On est de plus contraint par les conditions extérieures que l’on ne contrôle
pas. Ainsi ces deux raisons font qu’il est compliqué et couteux de chercher à dimensionner
expérimentalement un système. Pour réduire le coût de la recherche et explorer plus de
variation il est nécessaire de pouvoir contrôler les conditions limites et de pouvoir
itérer rapidement entre différents compositions/régulations. La modélisation entre alors
en jeu proposant un contrôle complet des systèmes, de leur régulation, et des conditions
limites. Le système n’étant pas physique il est alors possible de faire varier n’importe
quel paramètre simplement afin d’évaluer son importance, son impact, ...
La modélisation est donc l’outil de choix pour réaliser une étude de faisabilité, un
dimensionnement, une optimisation.

\itodo{État de l’art des modèles numériques}
\itodo{Détail des modèles existant, contrôle, couplages}
\itodo{Montrer que ces modèles sont très génériques et souvent très simplifiés.
       De plus l’algorithme de contrôle est non évalué/optimisé.}
% subsection les_modèles_existants (end)

% ------------------------------------------------------------------------------
\subsection{Un modèle solaire couplé au bâtiment} % (fold)
\label{sub:un_modele_solaire_couple_au_bâtiment}
\itodo{Décrire le choix de l’approche par modélisation}


\itodo{Décrire les outils utilisés: Modelica et les bibliothèques, Dymola et les solveurs}
Il existe de nombreux langages, logiciels pour réaliser des simulations plus ou moins complexes.
La première choses à définir est donc le niveau de précision que l’on souhaite pour son modèle ou
pour les différentes parties du modèle. Notre cas d’étude se place à l’échelle du bâtiment mais
l’on souhaite aussi conserver un contrôle important sur la gestion des équipements.
Ensuite il est nécessaire de déterminer le niveau d’accessibilité que l’on souhaite avoir
sur chaque composant du système. Le modèle \textbf{boîte noire} ne pourrait pas correspondre à notre
demande. Celui-ci ne nous offre pas le liberté de comprendre comment évolue chaque
composants du système et nous empêche d’explorer/modifier le code.
Pour la même raison un modèle \textbf{boîte grise} nous limite dans l’accès à certaines
partie du code et donc à la compréhension interne du fonctionnement du système.
Il est alors nécessaire d’utiliser un modèle \textbf{boîte blanche} garantissant
un contrôle total sur chaque partie du système et permettant d’évaluer le comportement
au niveau global mais aussi composant par composant.
Nous avons ainsi opté pour le langage Modelica et la plateforme de développement
Dymola (Dynamics Modeling Laboratory).


\itodo{Modelica description}
\mtodo{Ajouter référence}{Modelica} est un langage de programmation libre et ouvert développé pour répondre aux
contraintes de la modélisation multi-physique. Il a été pensé pour être intuitif
et offre une approche équationnelle et orienté objet au développeur.
L’approche objet est très intuitive et permet d’encapsuler un ensemble de données
et d’offrir des interface pour accéder à ces données. On peut alors composer de nouveaux
objets grâce à des références vers d’autre objets (composition) ou en héritant
du comportement d’un objet pour lui ajouter une spécialisation (héritage).
Enfin le langage est acausal permettant d’itérer entre différentes formulation
d’un problème facilement. Un système acausal récupère l’ensemble des variables qu’il
connait et défini les inconnus à partir de celles-ci. L’ordre d’écriture des équations
n’est donc plus importante et modifier un système n’oblige pas à re-écrire complètement
les équations pour isoler la variable que l’on cherche à déterminer. Prenons l’exemple
de la formule $U = R \times I$. Un problème causal requiert de connaître $R$ et $I$
pour trouver $U$.L’équation doit être re-écrite si on cherche à trouver $R$ ou $I$.
Si on utilise une approche acausal alors cette équation a une solution si on a deux des
trois inconnus ($U et R$ ou $R et I$, ...) sans avoir à modifier l’équation. On peut ainsi utiliser
la même formulation pour peut importe les variables connues.
On peut ainsi voir que le développement sous Modelica permet de rapidement proto-typer
des systèmes complexes.


\itodo{Dymola description}
Dymola est une suite de logiciels développée par \mtodo{Ajouter référence}{Dassault System}
permettant d’ajouter de nombreuses fonctionnalités.
La première étant l’interface graphique permettant de connecter différentes portion
de code de manière plus intuitive. Il ajoute aussi un débogueur puissant permettant
de trouver rapidement la portion de code qui pose problème, un outil pour faire du
refactoring. Enfin il offre un outil puissant pour compiler, initialiser et intégrer
le modèle avec un large choix d’intégrateurs avec le programme \mtodo{Ajouter référence}{Dymosim}.
Il propose aussi du support pour la parallélisation, les FMU, et, un outil
puissant pour faire du traitement des données durant et après les simulations. Enfin
Dymola propose grâce à des scripts d’accéder à l’ensemble des fonctionnalités
comme lancer une simulation, modifier un modèle, exporter un modèle, ...
Dymola est ainsi un outil puissant pour accélérer le développement de modèle Modelica.


\itodo{Couplage Dymola + Modelica description}
Ces deux outils offrent alors de nombreux avantages. On contrôle chaque partie
du système, le code est réutilisable, on peut choisir le détail de chaque partie
du modèle, et on peut le coupler avec d’autres logiciels si besoin est.

Le langage Modelica étant largement utilisé, de nombreuses bibliothèques open source
on été développées dont la liste peut être trouvée sur le site officiel de l’association Modelica
(\url{https://www.modelica.org/libraries}). Dans notre étude nous avons utilisé
la bibliothèque \mtodo{Ajouter référence}{Buildings} qui est développé par le
Laboratoire National Lawrence Berkeley (LBNL). C’est une bibliothèque libre et ouverte
orienté pour le secteur du bâtiment offrant de nombreux modèles de base.


\itodo{Pourquoi une modélisation détaillée d’un système existant}
Le but de cette étude est d’évaluer le potentiel de couverture d’un système solaire,
il apparaît donc important de pouvoir évaluer dans le détail son comportement au niveau
de chaque élément. On va chercher à comprendre comment évolue la température au sein
de la maison mais aussi des ballons, des capteurs, ...
Il est aussi important de pouvoir modifier facilement les différents éléments composant
le système comme les temporisations, les consignes, la taille des différents équipements, ...
L’étude ne s’intéresse en effet pas seulement à la performance finale du système mais
aux raisons et limitations qui ont pour conséquence ce résultat.


\itodo{Récapituler les besoins de l’étude}
Si on résume on a donc besoin de contrôler chaque composant pour évaluer si il se
comporte comme on l’entends. Il est aussi nécessaire de pouvoir facilement de manière
intuitive les sous-modèles sans affecter le modèle principal. On veut de plus pouvoir
évaluer le système au niveau du bâtiment et il est donc nécessaire de réaliser des
simulation sur une échelle de temps importante (de l’ordre de l’année).


\itodo{Conclure sur le choix de Modelica et Dymola}
Le choix du couple Modelica + Dymola est donc le résultat d’une recherche d’un outil
répondant à nos contraintes. On veut avoir un contrôle complet de chaque composant, leur
physique comme leur régulation et évaluer au niveau bâtiment la performance de celui-ci.
Ces deux outils nous permet d’évaluer/modifier/comprendre efficacement le système
sur différentes échelles tout en encourageant le processus itératif de cette étude.



\itodo{Montrer que aujourd’hui peu de travail a été fait sur l’optimisation couplée
       et que par conséquent ce sujet est innovant dans son approche du problème}
\itodo{Introduire la partie suivante}
% subsection un_modèle_solaire_couplé_au_bâtiment (end)
% section le_choix_d_un_modèle_de_système_solaire_couplé_au_bâtiment (end)
