%!TEX root = ../main.tex
% Chapitres/Chap2-ApprocheModelisation.tex

% ..............................................................................
% ..............................................................................
\section{Modélisation détaillée d’un système solaire innovant} % (fold)
\label{sec:modelisation_detaillee_d_un_systeme_solaire_innovant}
% ------------------------------------------------------------------------------
\subsection{Quelles contraintes} % (fold)
\label{sub:quelles_contraintes}
Il existe plusieurs méthodes permettant d’évaluer un système énergétique. Le premier
consiste à reproduire expérimentalement le système et son environnement. Ce processus est
coûteux, particulièrement si le système est évalué à l’échelle du bâtiment. De plus les
conditions extérieures ne peuvent être contrôlées et les modifications du système sont complexes à mettre en place.
Le second, la modélisation, permet de contrôler l’ensemble des éléments~: équipements,
algorithme de contrôle, et les conditions limites (météo).
Il est alors possible d’évaluer le système de
manière totalement libre rendant cette approche particulièrement adaptée pour une
étude de faisabilité, un dimensionnement, ou encore l’optimisation d’un système. Pour ces raisons,
une approche par modélisation est retenue dans ces travaux.

Il existe cependant de nombreux langages ou logiciels permettant de réaliser des
simulations plus ou moins complexes. Il est ainsi nécessaire, dans un premier temps, de
définir le niveau de précision nécessaire pour les différentes parties du modèle. Il est
aussi important de déterminer le niveau de contrôle que l’on souhaite avoir sur chaque
composant. En effet il existe principalement deux familles d’outils~: les modèles opaques
(boîtes noires) qui masquent à l’utilisateur le fonctionnement internes, et les modèles
ouverts (boîtes blanches) dont la composition est connue et souvent modifiable. Notre cas
d’étude se place à l’échelle du bâtiment mais il est souhaitable de conserver un contrôle
important sur la gestion des équipements. Ce contrôle est nécessaire pour permettre
l’analyse des interactions entre des variations au niveau du composant et le
fonctionnement global du système. De plus, le processus de construction du modèle est dans un
premier temps fortement itératif afin de permettre la définition d’un système solaire
performant dont l’algorithme de contrôle est détaillé ci-après. L’outil sélectionné doit par conséquent
permettre de modifier, ajouter, et supprimer rapidement et simplement des composants. Finalement le
problème à modéliser est un problème multi-physique et l’outil doit être adapté à la
modélisation d’un système hydraulique (solaire), d’un système aéraulique (ventilation),
d’un algorithme de contrôle (régulation), le tout couplé à un modèle thermique (bâtiment).
% subsection quelles_contraintes (end)



% ------------------------------------------------------------------------------
\subsection{Quelles solutions} % (fold)
\label{sub:quelles_solutions}
Le langage \textit{Modelica} (3.2.2) et le logiciel \textit{Dymola} (Dynamics MOdeling
LAboratory) (2017) permettent de répondre à l’ensemble des contraintes décrites dans la
section précédente. \textit{Modelica} est un langage de programmation libre et ouvert
développé pour répondre aux contraintes de la modélisation multi-physique. Il a été pensé
pour être intuitif en offrant une approche équationnelle et orientée objet au développeur
\parencite{Wetter2016290}.
L’approche objet permet d’encapsuler un ensemble de données dont l’accès est restreint par
une interface publique nommer $API$ (Application Programming Interface). Un modèle est
ainsi une combinaison d’un ensemble de sous- modèles (composition) ou certains sous-modèles
peuvent être améliorés afin de leur ajouter une spécialisation (héritage). Cette
approche permet alors une forte réutilisation des modèles déjà existant tout en
simplifiant l’ajout de nouvelles fonctionnalités ou le couplage des deux. Finalement, le
langage est acausal (\defref{def:acausal}), permettant au modeleur de rapidement
créer des prototypes de systèmes complexes sans devoir à chaque modification réécrire
l’ensemble des systèmes d’équations. Il est cependant toujours possible de définir
explicitement quelles sont les inconnus d’un modèle en héritant de la classe \textit{Bloc}.
En effet en plus de proposer une bibliothèque avec un nombre conséquent de briques
permettant de construire des modèles complexes, le langage propose aussi des classes de
référence permettant à l’utilisateur de spécialiser chaque élément de son modèle. Cet
outil assure alors un processus itératif efficace tout en offrant une grande liberté sur
le détail de chaque composant du modèle final.


\begin{Def}[Acausal]\label{def:acausal}
Un modèle est dit acausal lorsque le sens de la définition d’une équation n’est pas
imposé (\figref{fig:acausal_vs_causal}). Le solveur va ainsi déterminer par lui-même les
paramètres dont il connaît la valeur et réécrire les systèmes d’équations afin d’exprimer
les inconnues à partir des valeurs connues. À l’inverse un système causal nécessite
que l’inconnu soit isolé a priori par le programmeur.
\end{Def}

\begin{figure}
    \centering
    \includegraphics{Ressources/Images/Modelisation/composant_vs_bloc.png}
    \caption{Différences entre modélisation acausale (par composant) et causal (par bloc)
             (\href{http://www.wolfram.com/system-modeler/}{Wolfram}).}
    \label{fig:acausal_vs_causal}
\end{figure}

\subsubsection{Dymola} % (fold)
\label{ssub:dymola}
\textit{Dymola} est une suite de logiciels développée par \textit{Dassault Systèmes}
permettant de simplifier le développement de modèles numériques grâce au langage
\textit{Modelica}. Il offre une interface graphique permettant de connecter les modèles
entre eux de manière intuitive (\figref{fig:exemple_modelica}) et un outil de
post-traitement afin d’évaluer rapidement les résultats. Il propose aussi de nombreuses fonctions
simplifiant le développement afin de se focaliser sur le problème physique à modéliser~:
\begin{itemize}
    \item \textit{Dymosim}~: un large choix de solveurs pour intégrer les modèles
    \item Un format texte permettant la coopération, le \enquote{versionnage}, et la construction dynamique.
    \item Un langage de script interne permettant l’automatisation.
    \item Le \enquote{refactoring} permettant de renommer de manière dynamique une variable.
    \item Le \enquote{drag and drop} permettant de composer et connecter rapidement des modèles.
    \item Le support de la parallélisation.
    \item La création de $FMU$ (Functional Mock-up Unit).
    \item La génération de modèles autonomes.
    \item \dots
\end{itemize}

\begin{figure}
    \centering
    \includegraphics{Ressources/Images/Modelisation/exemple.PNG}
    \caption{Modélisation d’un échange thermique entre deux masses en Modelica.}
    \label{fig:exemple_modelica}
\end{figure}
% subsubsection dymola (end)


\subsubsection{Buildings} % (fold)
\label{ssub:buildings}
Le couplage de \textit{Modelica} et de \textit{Dymola} permet le développement itératif
nécessaire à notre problème tout en offrant un contrôle total sur chaque partie du modèle.
Le langage \textit{Modelica}, en pleine croissance dans le secteur du bâtiment est cependant
encore peu utilisé en particulier lorsque ces travaux ont commencé. Depuis, de nombreuses
bibliothèques open source ont été développées dont la liste peut être trouvée sur le site
officiel de l’association \href{https://www.modelica.org/libraries}{\textit{Modelica}}. Ces
travaux s’appuient sur la bibliothèque Buildings \parencite{Wetter2014253} développée par le laboratoire
national Lawrence Berkeley ($LBNL$, Lawrence Berkeley National Laboratory) dont le
développement était le plus avancé et le plus actif. C’est une bibliothèque libre et
ouverte, développée pour la modélisation des systèmes du bâtiment (hydraulique,
aéraulique, électrique, thermique\dots) et son développement est toujours en croissance. Enfin
afin d’automatiser la simulation de nombreux modèles la bibliothèque
\href{http://simulationresearch.lbl.gov/modelica/buildingspy/}{\textit{Buildingspy}}
est utilisée. Développée par le $LBNL$, elle permet d’interfacer le langage \textit{Python}
avec le langage de script de \textit{Dymola}.
% subsubsection buildings (end)
% subsection quelles_solutions (end)
% section modelisation_detaillee_d_un_systeme_solaire_innovant (end)




% ..............................................................................
% ..............................................................................
\section{Description du cas d’étude~: le bâtiment} % (fold)
\label{sec:description_du_cas_d_etude_le_batiment}
% ------------------------------------------------------------------------------
\subsection{Modélisation mono-zone} % (fold)
\label{sub:modelisation_monozone}
Le bâtiment, une maison de plain-pied (\figref{fig:plan_maison}) avec une surface habitable
de \SI{98.4}{\meter\squared}. Elle comporte trois chambres, une cuisine/salon, et un local
technique où se trouvent les équipements du système solaire combiné ($SSC$). En plus de
l’espace chauffée, les combles et le vide sanitaire ont été modélisés comme des conditions
limites (\figref{fig:modelisation_maison}). Enfin une fenêtre de toit est ajoutée dans
le salon afin d’améliorer le confort lumineux (\tabref{tab:compo_velux}).

Dans le cadre de ces travaux un couplage fort entre bâtiment, systèmes, et algorithme de
contrôle est envisagé. Dans cette optique, un modèle mono-zone validé à travers une suite
de test issue de l’\textit{ASHRAE} est retenu pour modéliser le bâtiment
\parencite{Wetter2011,Nouidui2012}. Ce choix est discuté ci-après en particulier en
comparant les besoins obtenus entre cette approche et une approche multi-zonale réalisée
à l’aide d’\textit{Energy Plus}. Ainsi la température dans l’ensemble de la zone chauffée
est considérée comme uniforme.

\begin{figure}
    \centering
    \includegraphics{Ressources/Images/Modelisation/Batiment/plan.png}
    \caption{Plan du bâtiment utilisé à travers ces travaux.}
    \label{fig:plan_maison}
\end{figure}

\begin{figure}
    \centering
    \includegraphics{Ressources/Images/Modelisation/maison.png}
    \caption{Représentation du bâtiment sous Modelica.}
    \label{fig:modelisation_maison}
\end{figure}

Les caractéristiques thermiques des parois opaques sont décrites en fonction de
leur orientation (\tabref{tab:perf_parois_opaques}) et une description détaillée et exhaustive est
disponible en annexe (\tabref{tab:compo_parois}). La composition retenue permet ainsi d’obtenir une
enveloppe très isolante.

\begin{table}
\centering
\caption{Description de la performance des parois opaques.}
\label{tab:perf_parois_opaques}
\begin{tabular}{l *{3}{c} r}
    \toprule
                       & Murs           & Plancher     & Plafond & Unité     \\
    \midrule
    $U$                & \num{0.174}    & \num{0.110}  & \num{0.123}  & \si{\watt\per(\meter\squared\period\kelvin)}\\
    $Surface$          & \num{91.17}    & \num{98.40}  & \num{97.06}  & \si{\meter\squared}\\
    $U \times Surface$ &  \num{15.864}  & \num{10.824} & \num{11.938} & \si{\watt\period\kelvin}\\
    \bottomrule
\end{tabular}
\end{table}



% - - - - - - - - - - - - - - - - - - - - - - - - - - - - - - - - - - - - - - -
\subsubsection{Déperditions à travers les fenêtres} % (fold)
\label{ssub:deperditions_a_travers_les_fenetres}
Les vitrages utilisés sont des vitrages doubles avec une lame d’air ou d’argon et une
couche faiblement émissive est ajoutée pour les vitrages des parois verticales.
Il est considéré pour l’ensemble des vitrages, une composition
similaire (Planilux SGG \SI{4}{mm} - Argon - Planitherm XN \SI{4}{mm}), exception
faite de la fenêtre de toit.

Une attention particulière a été apportée à la description des vitrages afin d’être
cohérente avec les fiches techniques et satisfaire les normes de calculs implicites de la
bibliothèque \textit{Buildings}. En effet, le modèle de vitrage de la bibliothèque
\textit{Buildings} sépare le calcul de la part transmise en conductif et en radiatif~: les
caractéristiques des vitrages doivent alors être renseignées de manière détaillées.
Cependant, les fabricants fournissent seulement les caractéristiques globales~: le
coefficient de transmission thermique $U_{g}$ et le facteur solaire $g$. Le logiciel
\textit{Windows 7.4} a donc été utilisé afin d’obtenir la composition détaillée des vitrages
retenus.

Dans un premier temps, le vitrage est construit à partir de la composition décrite dans sa
fiche technique puis les coefficients caractéristiques sont calculés. Afin de pouvoir
comparer avec les données fabricants, les conditions limites issues des normes européennes
sont implémentées dans le logiciel (\tabref{tab:detail_calcul_fenetre}).
En effet, le choix des conditions limites impactent fortement le résultat
et il est observé une variation moyenne de \SI{20}{\percent} entre les normes européennes
et états-uniennes \parencite{RDH2014}.

\begin{table}
\centering
\caption{Détail du calcul du $U_{g}$ et du $g$ selon \textcite{NFEN673} et \textcite{NFEN410}.}
\label{tab:detail_calcul_fenetre}
\begin{tabular}{l *5{c}}
    \toprule
    & $T_{int}$ & $T_{ext}$            & $h_{c}^{int}$ & $h_{c}^{ext}$                                    & Ensoleillement \\
    \addlinespace[\defaultaddspace]
    & \multicolumn{2}{c}{[\si{\kelvin}]} & \multicolumn{2}{c}{[\si{\watt\per(\meter\squared\period\kelvin)}]} & [\si[per-mode=symbol]{W\per\metre\squared}] \\
    \midrule
    Calcul du $U_{g}$       & \num{20}         & \num{0}       & \num{3.6}   & \num{25}    & -    \\
    Calcul du $SHGC$        & \num{30}         & \num{25}       & \num{3.6}   & \num{25}    & \num{500} \\
    \bottomrule
\end{tabular}
\end{table}

\itodo{Aurélie~: informations relatives aux ponts thermiques}
Finalement, la bibliothèque \textit{Buildings} ne permettant pas l’ajout de ponts
thermiques, le choix a été fait de les intégrer dans le coefficient thermique
caractéristique du cadre, $U_{f}$. Une description détaillée et exhaustive de la
composition des fenêtres est disponible en annexe (\tabref{tab:compo_vitrage},
\tabref{tab:compo_velux}, et \tabref{tab:compo_gaz}).
% subsubsection deperditions_a_travers_les_fenetres (end)


% - - - - - - - - - - - - - - - - - - - - - - - - - - - - - - - - - - - - - - -
\subsubsection{Description des infiltrations} % (fold)
\label{ssub:description_des_infiltrations}
Les infiltrations ont été définies en utilisant une perméabilité de
\SI{0.4}{m^{3}\per(\hour\period\meter\squared)}. Le calcul tient compte de la
somme des surfaces dites froides correspondant à la somme des surfaces donnant
vers l’extérieur à l’exception du plancher bas \eqref{eq:infiltrations}.
\begin{equation}
    \begin{aligned}
    Infiltrations &= \num{0.4} \times (Parois_{verticales} + Parois_{velux} + Plafond)\\
    &              \backsimeq \SI{85}{m^{3}/h}
    \label{eq:infiltrations}
    \end{aligned}
\end{equation}

% subsubsection description_des_infiltrations (end)


\subsubsection{Limitations du modèle} % (fold)
\label{ssub:limitations_du_modele}
Certains choix ont été fait du aux limitations de modélisation~:
\begin{itemize}
    \item Les coefficients de convection intérieurs utilisés sont les mêmes pour toutes
          les parois (\SI{7.7}{\watt\per(\meter\squared\period\kelvin)}) et le coefficient extérieur
          est fixé à \SI{25}{\watt\per(\meter\squared\period\kelvin)} (\textcite{NFENISO6946}).
    \item Les ponts thermiques ont été définies comme une surface de déperdition
          équivalente à \SI{13.2}{\meter\squared} (\mtodo{Aurélie~: Récup détail}).
    \item La fenêtre de toit est placée horizontalement et non à \SI{33}{\percent}
          (inclinaison réelle) car la bibliothèque \textit{Buildings} ne supporte pas les
          surfaces vitrés inclinées. Le $g$ utilisé est cependant celui défini
          pour \SI{33}{\percent} dans sa fiche technique.
\end{itemize}
% subsubsection limitations_du_modele (end)


% - - - - - - - - - - - - - - - - - - - - - - - - - - - - - - - - - - - - - - -
\subsubsection{Comparaison avec un modèle multi-zone} % (fold)
\label{ssub:comparaison_avec_un_modele_multi_zone}
Une comparaison a été réalisée entre un modèle multi-zone développé sous \textit{Energy Plus}
par le \textit{CEA} et le modèle mono-zone retenue dans ces travaux. Le fichier météo
utilisé est celui de Bordeaux
(\href{https://www.energyplus.net/weather-download/europe_wmo_region_6/FRA//FRA_Bordeaux.075100_IWEC/all}{IWEC - WMO 075100}).
La consigne de chauffage est de \SI{19}{\celsius} en occupation et de \SI{16}{\celsius} en
inoccupation. L’étude ne portant pas sur l’évaluation du confort estival, seule les
besoins ont été comparés. Les résultats pour l’approche mono-zone et de l’approche multi-
zone sont similaires~: respectivement \SI{1114}{\kilo\watt\hour} et
\SI{1190}{\kilo\watt\hour}.

\iunsure{Ajouter comparatif des puissances entre Energy Plus et Buildings}.
% subsubsection comparaison_avec_un_modele_multi_zone (end)
% subsection modelisation_monozone (end)


% % ------------------------------------------------------------------------------
\subsection{Scénarios de référence} % (fold)
\label{sub:scenarios_de_reference}
% - - - - - - - - - - - - - - - - - - - - - - - - - - - - - - - - - - - - - - -
\subsubsection{Occupation} % (fold)
\label{ssub:profil_d_occupation}
Un profil unique pour l’ensemble de la zone et pour chaque occupant est retenu
(\figref{fig:scenario_reference}). Il est considéré quatre personnes, deux enfants, et deux parents. Dans ce profil,
les occupants sont considérés à la maison durant le week-end et absent durant les jours
ouvrés (travail) exception faite du mercredi après-midi.
La puissance par habitant est de \SI{97.5}{\watt} (\SI{70}{\percent} convective), résultat
de la moyenne pondérée de la puissance dégagée par une personne en fonction de la surface
de chaque pièce dans le modèle multi-zone (\tabref{tab:puissance_occupants}).

\begin{table}
\centering
\itodo{Aurélie~: Récupérer plus d’informations sur le profil d’occupation}
\caption{Récapitulatif des puissances dissipées en fonction des pièces.}
\label{tab:puissance_occupants}
\begin{tabular}{*8{c}}
    \toprule
    Chambre 1 & Chambre 2  & Chambre 3 & Séjour     & Cuisine    & Sanitaire   & SdB         & Cellier     \\
    \midrule
    \num{78}  & \num{78}   & \num{117} & \num{97.5} & \num{97.5} & \num{114.3} & \num{114.3} & \num{114.3} \\
    \bottomrule
\end{tabular}
\end{table}
% subsubsection profil_d_occupation (end)


% - - - - - - - - - - - - - - - - - - - - - - - - - - - - - - - - - - - - - - -
\subsubsection{Charges internes} % (fold)
\label{ssub:charges_internes}
En plus des occupants, les charges internes dues aux équipements sont prises en compte. Il
est entendu comme charges internes, les consommations des équipements électriques
(électroménager, ordinateurs\dots) et la consommation de l’éclairage. Aucunes informations
n’étant disponible afin d’estimer les consommations réelles de la maison, les
consommations réglementaires, \textit{RT\,2012} \parencite{CSTB2011} ont été retenues.
Dans les deux cas il est considéré une consommation type durant l’occupation
et réduite en inoccupation.

\paragraph{Équipements électriques~:} % (fold)
\label{par:equipements_electriques}
La consommation est fixée à \SI{5.7}{\watt\per m^{2}} (\SI{80}{\percent}
convective) et durant l’occupation à \SI{1.14}{\watt\per m^{2}}, soit une
réduction de \SI{80}{\percent} (\figref{fig:scenario_reference}).
% paragraph equipements_electriques (end)

\paragraph{Éclairage~:} % (fold)
\label{par:eclairage}
La consommation est fixée à \SI{1.4}{\watt\per m^{2}} (\SI{42}{\percent} convective) de
\SI{7}{\hour} à \SI{9}{\hour} et de \SI{19}{\hour} à \SI{22}{\hour}
(\figref{fig:scenario_reference}). Durant le week-end, la consommation est réduit de
\SI{50}{\percent}, soit \SI{0.7}{\watt\per m^{2}} de \SI{10}{\hour} à \SI{19}{\hour}.
Finalement, en inoccupation la consommation est considérée comme nulle.
% paragraph eclairage (end)
% subsubsection charges_internes (end)


% - - - - - - - - - - - - - - - - - - - - - - - - - - - - - - - - - - - - - - -
\subsubsection{Consigne de chauffage} % (fold)
\label{ssub:consigne_de_chauffage}
Le profil de température de référence (\figref{fig:scenario_reference}) est issue de la
réglementation thermique (RT\,2012) qui prévoit le maintien d’une température de
\SI{19}{\celsius} en occupation, un réduit de \SI{16}{\celsius} en inoccupation. De plus,
elle autorise un abaissement de la consigne durant la période nocturne~: \SI{18}{\celsius} est
retenue.
% subsubsection consigne_de_chauffage (end)

% - - - - - - - - - - - - - - - - - - - - - - - - - - - - - - - - - - - - - - -
\subsubsection{Ventilation} % (fold)
\label{ssub:ventilation_ref}
Finalement le profil de ventilation de référence est considéré à \SI[per-mode=symbol]{90}{\meter\cubed\per\hour}
comme le prévoit l’arrêté du \href{https://www.legifrance.gouv.fr/affichTexte.do?cidTexte=JORFTEXT000000862344}{24 Mars
1982 et du 28 Octobre 1983} pour une maison comportant quatre pièces principales. Il est aussi
considéré une réduction du débit d’air neuf minimal à \SI[per-mode=symbol]{20}{\meter\cubed\per\hour}
en inoccupation.
% subsubsection ventilation (end)
% subsection scenarios_de_reference (end)

\begin{figure}
    \centering
    \includegraphics{Ressources/Images/Modelisation/Scenario/charges_internes.pdf}
    \caption{Profil de référence pour les charges internes (occupants, équipements et éclairage)
             et la consigne de chauffage.}
    \label{fig:scenario_reference}
\end{figure}

% section description_du_cas_d_etude_le_batiment (end)



% ..............................................................................
% ..............................................................................
\section{Description du cas d’étude : le système solaire} % (fold)
\label{sec:description_du_cas_d_etude_le_systeme_solaire}
% ------------------------------------------------------------------------------
\iunsure{Parler du système solaire par vecteur eau}
\iunsure{Mettre des capture du modèle \textit{Modelica} (système et algorithme)}
Dans un premier temps un modèle existant basé sur le vecteur eau a été modélisé.
Le comportement du modèle a ainsi pu être validé par l’entreprise (SolisArt) à
l’origine du système.

\iunsure{Mentionner IGC}
\iunsure{Mentionner COMEPOS ou seulement dans chapitre 1}
Le système décrit dans ce chapitre est un $SSC$ utilisant l’air comme vecteur de chaleur
afin d’éviter l’installation d’un système de chauffage conventionnel (radiateurs, plancher
chauffant\dots). Le système permet donc, par l’intermédiaire du réseau de ventilation, de
couvrir les besoins en $ECS$ et en chauffage ($CH$) pour l’ensemble du bâtiment. De plus
le choix de l’air comme vecteur de chaleur est motivé par la volonté d’obtenir un système
réactif par l’entreprise partenaire (\textit{IGC}). Le système modélisé est composé de trois
parties principales~: la partie hydraulique, la partie aéraulique, et l’algorithme de
contrôle qui orchestre le fonctionnement couplé entre le système solaire et le bâtiment
(\figref{fig:air_complet_mono}).

Dans un premier temps le fonctionnement de la partie hydraulique sera détaillé. Dans un
second temps la partie aéraulique sera présentée. Finalement la partie contrôle sera
décrite et le fonctionnement combiné du système et de la maison explicité.

\begin{figure}
    \centering
    \includegraphics{Ressources/Images/Modelisation/Principe/air_complet_mono.pdf}
    \caption{Description schématique du système solaire couplé à la ventilation.}
    \label{fig:air_complet_mono}
\end{figure}




% ------------------------------------------------------------------------------
\subsection{Description du système solaire} % (fold)
\label{sub:description_du_systeme_solaire}
% - - - - - - - - - - - - - - - - - - - - - - - - - - - - - - - - - - - - - - -
\subsubsection{Partie hydraulique} % (fold)
\label{ssub:partie_hyraulique}
Le système hydraulique (\figref{fig:air_complet_mono}) est composé de deux ballons dont les
caractéristiques sont disponibles dans le \tabref{tab:tanks_specs}. Le premier, le ballon
sanitaire, permet de lisser la demande en énergie nécessaire pour couvrir les besoins en
$ECS$. Le système est alors dit semi-accumulée car la réserve d’eau chaude est inférieure
à la quantité totale puisée durant la journée. Le ballon sanitaire est connecté en
continue avec le réseau d’eau froide public. Il est donc nécessaire, soit de maintenir
à une température minimale de \SI{55}{\celsius} l’eau du ballon, soit de réaliser une surchauffe journalière
(\href{https://www.legifrance.gouv.fr/affichTexte.do?cidTexte=JORFTEXT000000423756}{Arrêté
du 30 novembre 2005}). La première option a été retenue afin de garantir le respect de la
réglementation dans les limites techniques imposées par la modélisation. Le second, le
ballon de stockage, est utilisé pour stocker l’énergie accumulée durant la journée afin de
la valoriser en période nocturne. Dans les deux cas, le volume du ballon est discrétisé en
\num{20} sous-volumes afin de tenir compte de la stratification. Cette stratification est
particulièrement importante pour le ballon sanitaire où l’eau froide en partie basse, et
l’eau à \SI{55}{\celsius} en partie haute, imposent un différentiel de température
important. Enfin, les deux ballons étant dans la partie chauffée de la maison, leurs
déperditions sont considérées comme des charges internes au bâtiment.

Afin de garantir le maintien du confort thermique des occupants, un appoint électrique en
partie haute du ballon sanitaire est aussi ajouté. Ce dernier n’est activé que lorsque
l’énergie solaire disponible n’est pas suffisante. L’énergie solaire est elle transmise à
l’eau par l’intermédiaire de panneaux solaires (\tabref{tab:idmk_specs}) puis vers le
système aéraulique à l’aide d’un échangeur de chaleur. Finalement, le système comporte trois
pompes~: $S6$, $S5$, et $S2$. Un débit nominal de \SI{40}{\litre\per(\hour\period\meter\squared)}
(capteur) pour les pompes est considéré comme première approximation
\parencite{Peuser2005}. Les pompes utilisées sont des pompes à vitesse variable afin de
limiter les arrêts intempestifs \parencite{Kicsiny20123489}.

\begin{table}
\centering
\caption{Caractéristiques techniques des ballons de référence (tampon et sanitaire).}
\label{tab:tanks_specs}
\begin{tabular}{l*{2}{c}r}
    \toprule
    Paramètre & Ballon tampon & Ballon sanitaire & Unité\\
    \midrule
    Volume                                       & \num{300}   & \num{300}    & \si{\litre}              \\
    Hauteur                                      & \num{1.05}  & \num{1.25}   & \si{\metre}              \\
    Épaisseur isolation                          & \num{100}   & \num{55}     & \si{\milli\metre}             \\
    $\lambda$ isolant                            & \num{0.04}  & \num{0.04}   & \si{W/m^{2}\period K}      \\
    Échangeur haut                               & \num{0.85}  & \num{0.64}   & \si{\metre}              \\
    Échangeur bas                                & \num{0.15}  & \num{0.13}   & \si{\metre}              \\
    Diamètre échangeur (extérieur)               & \num{34.6}  & \num{33.7}   & \si{\milli\metre}             \\
    Chaleur spécifique de échangeur (acier noir) & \num{490}   & \num{490}    & \si{J/kg\period K}         \\
    Puissance nominale                           & \num{103}   & \num{53}     & \si{\kilo\watt}             \\
    Température nominale (ballon)                & \num{10}    & \num{45}     & \si{\celsius} \\
    Température nominale (échangeur)             & \num{45}    & \num{10}     & \si{\celsius} \\
    Débit nominal                                & \num{0.36}  & \num{0.366}  & \si{kg\per\second}           \\
    \bottomrule
\end{tabular}
\end{table}

\begin{table}
\centering
\caption{Caractéristiques du collecteur de référence (modèle IDMK\,25 de chez Sonnenkraft).}
\label{tab:idmk_specs}
\begin{tabular}{lcr}
    \toprule
    Paramètre                                   & Valeur         & Unité                 \\
    \midrule
    Surface nette                               & \num{2.32}           & \si{m^{2}}            \\
    Poids à vide                                & \num{54}             & \si{kg}               \\
    Contenance                                  & \num{1.35}           & \si{l}                \\
    Rendement optique ($\eta_{0}$)              & \num{78}             & \si{\%}               \\
    Pente                                       & \num{-5.103}         & -                     \\
    Coefficient de pertes linéiques ($a_{1}$)   & \num{3.796}          & \si{W/(m^{2}\period K)}      \\
    Coefficient de pertes surfaciques ($a_{2}$) & \num{0.013}          & \si{W/(m^{2}\period K^{2})}  \\
    Modulation du diffus ($IMDiff$)             & \num{100}            & \si{\%}               \\
    \bottomrule
\end{tabular}
\end{table}

\paragraph{Équilibrage du réseau~:} % (fold)
\label{par:equilibrage_du_reseau}
Le système décrit est complexe et plusieurs circulateurs en série peuvent être actifs
simultanément. Ainsi, si on considère deux circulateurs en série ayant toutes les deux pour
consigne un débit de \SI[per- mode=symbol]{2}{\meter\cubed\per\hour}, le débit résultant
sera de \SI[per-mode=symbol]{4}{\meter\cubed\per\hour} car le débit à chaque pompe est
\textbf{imposé}. Afin de corriger ce comportement il est nécessaire d’imposer soit une
vitesse de rotation, soit un différentiel de pression.

Un circulateur à vitesse variable (\textit{Wilo-Yonos ECO 25/1-5 BMS} adapté aux
températures élevées d’un réseau solaire est retenu (voir \figref{fig:caracs_pompes}).
Ces caractéristiques sont implémentées afin de permettre à partir d’une vitesse de
rotation imposée, de trouver le débit correspondant. De cette manière, les pertes de
charges du réseau (équilibré pour le débit nominal) sont prises en compte~: le débit est
correctement modulé si plusieurs pompes en série sont actives.
% paragraph equilibrage_du_reseau (end)
% subsubsection partie_hyraulique (end)


% - - - - - - - - - - - - - - - - - - - - - - - - - - - - - - - - - - - - - - -
\subsubsection{Partie aéraulique} % (fold)
\label{ssub:partie_aeraulique}
La partie aéraulique du système est responsable du renouvellement d’air et du chauffage
par l’intermédiaire du solaire thermique ou bien de la batterie électrique en
position terminale. Un caisson de mélange permet de récupérer une partie de l’énergie
sur l’air extrait (\figref{fig:air_complet_mono}). L’air soufflé est un mélange entre
l’air neuf et l’air repris dans le respect de la réglementation en vigueur (\ref{ssub:ventilation}).
Le système de ventilation peut ainsi être apparenté à une ventilation mécanique par insufflation
($VMI$).

% \begin{figure}
%     \centering
%         \includegraphics{Ressources/Images/Modelisation/air_aeraulique_mono.pdf}
%     \caption{Description schématique de la partie aéraulique du système solaire.
%              \label{fig:schema_aeraulique}}
% \end{figure}
% subsubsection partie_aeraulique (end)

\subsubsection{Vérifications} % (fold)
\label{ssub:verifications}
Dans l’optique de l’évaluation de la performance  du système solaire couplé à un bâtiment,
une attention particulière doit être apporté au modèle de panneaux solaires. Il est
nécessaire d’utiliser un modèle étant à la fois représentatif du comportement réel et
compatible avec les informations techniques disponibles sur les fiches techniques des
différents fabricants~: ce dernier point exclue une modélisation détaillée. L’approche
retenue utilise le modèle disponible dans la bibliothèque \textit{Buildings} basant son
système d’équations sur les relations empiriques décrites dans le volet 2 de la norme
\textcite{EN129752}. Dans cette approche, deux principaux coefficients permettent de
caractériser les pertes thermiques des capteurs : les coefficients de pertes linéiques et
surfaciques notés respectivement $a_{1}$ et $a_{2}$. Deux autres coefficients, le rendement
solaire sans pertes ($\eta_{0}$) et le coefficient de modulation de la part solaire issue du diffus
($IMDiff$) permettent de caractériser la part solaire absorbée.

Afin de vérifier la pertinence du modèle des données expérimentales ont été utilisées.
Pour ce faire la température, le débit de l’eau en entrée des collecteurs, et les
irradiations directe et indirecte ont été utilisées comme conditions limites pour le
modèle~: la production des capteurs a ainsi pu être comparée. Les résultats montrent une
production similaire sur les trois mois couverts par les données expérimentales
(\figref{fig:compare_capteurs}, gauche). Enfin le modèle utilisé pour convertir
l’irradiation disponible à l’horizontal (diffus) et à la normale (direct) sur une surface
inclinée (capteur) a aussi été comparé avec les résultats du modèle implémenté dans
\textit{TrnSys}. Les deux modèles estiment de manière similaire l’irradiation sur une
surface inclinée (\figref{fig:compare_capteurs}, droite).

\begin{figure}
    \centering
    \includegraphics{Ressources/Images/Modelisation/compare_capteurs.pdf}
    \caption[Comparaisons de l’irradiation entre \textit{TrnSys}, \textit{Modelica}]
             {Comparaisons par mois~: (gauche) l’irradiation sur les capteurs entre un modèle
             \textit{TrnSys} et \textit{Modelica}, (droite) la production des capteurs entre des résultats
             expérimentaux et le modèle \textit{Modelica}.}
    \label{fig:compare_capteurs}
\end{figure}
% subsubsection verifications (end)
% subsection description_du_systeme_solaire (end)


% ------------------------------------------------------------------------------
\subsection{Algorithme de contrôle} % (fold)
\label{sub:algorithme_de_controle}
% - - - - - - - - - - - - - - - - - - - - - - - - - - - - - - - - - - - - - - -
\subsubsection{Fonctionnement global} % (fold)
\label{ssub:fonctionnement_global}
La partie hydraulique est contrôlée par un ensemble hiérarchisé de contrôleurs. Le niveau
le plus élevé de contrôle a pour rôle l’activation ou la désactivation des différents
éléments comme les pompes ou les vannes. Au plus bas niveau chaque pompe et chaque
équipement électrique utilise un contrôleur Proportionnel, Intégral, Dérivé ($PID$). Cette
approche permet d’adapter dynamiquement le comportement du système tout en conservant un
stricte séparation entre la logique de chaque composant. Le système solaire fonctionne
ainsi suivant trois principaux modes ordonnées afin de valoriser l’énergie solaire captée.
(\figref{fig:schema_modes}).
\begin{figure}
    \centering
    \includegraphics{Ressources/Images/Modelisation/Principe/air_modes.pdf}
    \caption[Description schématique des différents modes de fonctionnements du $SSC$]
    {Description schématique des différents modes de fonctionnements du $SSC$. Système
    global (a), fonctionnement $Indirect$ (b), fonctionnement $Direct$ avec besoin de
    chauffage (c), et, fonctionnement $Direct$ sans besoin de chauffage (d).}
    \label{fig:schema_modes}
\end{figure}

La priorité revient respectivement au maintien du ballon sanitaire, au respect de la température de
consigne, puis à l’élévation de la température dans le ballon de stockage. Le respect de
ces priorité n’est cependant pas exclusif. En effet, lorsque l’énergie solaire disponible est
suffisante, le chauffage, l’$ECS$, et la charge du ballon tampon peuvent être actifs de
manière simultanée.
Durant la période diurne, l’énergie solaire est récupérée directement au niveau des
capteurs solaires (mode $Direct$) et la vanne trois voies ($V3V$) est ouverte.
L’activation des pompes $S5$, $S2$, et $S6$ permet alors respectivement de charger le
ballon tampon, couvrir les besoins en $ECS$, et les besoins en chauffage. Le système
ajuste la vitesse des pompes afin de maintenir une différence de température minimale de
\SI{10}{\celsius} entre la sortie des échangeurs et la sortie des capteurs solaires
($DeltaT_{sol}$). Contrairement aux approches plus classiques, la différence de
température entre l’entrée et la sortie du collecteur n’est pas utilisée comme un élément
de régulation. En effet, cette approche ne permet pas de tenir compte de manière dynamique
des fluctuations entre besoins, pertes, et énergie disponible \parencite{Mosallat2013686}.
En considérant la différence de température entre la source (sortie des capteurs) et la
cible (sortie des échangeurs) on tient compte des pertes en lignes mais aussi des
fluctuations de la part d’énergie fournie aux ballons ou à l’air. Ainsi dans le cas où
l’énergie solaire est importante, il est possible de la distribuer entre les différentes
cibles. Dans le cas où l’énergie solaire est limitée la différence minimale de température
assure de valoriser cette énergie en contrôlant dynamiquement le nombre de pompes activés
grâce à un $PID$ associé à chaque pompe.

En dehors des heures d’ensoleillement, le système utilise l’énergie solaire stockée dans
le ballon tampon pour couvrir les besoins de chauffage (mode $Indirect$). Ainsi le sens de
circulation du fluide dans l’échangeur du ballon tampon est déterminé par la position de
la $V3V$. Dans le mode $Direct$, la $V3V$ est ouverte vers les capteurs permettant à la
pompe $S6$ de s’activer~: l’eau circule du haut vers le bas du ballon. Dans le mode
$Indirect$ la $V3V$ bascule vers le ballon tampon (fermée coté capteurs)~: l’eau circule
du bas vers le haut (inverse) et la pompe $S6$ ne peut donc pas être active.
Dans le cas où l’énergie du ballon sanitaire n’est pas suffisante (température inférieure
à \SI{55}{\celsius}), l’appoint électrique est activé afin de garantir la température
minimale réglementaire. De même, pour le chauffage, une batterie électrique en partie
terminale du réseau aéraulique s’active si la consigne de soufflage n’est pas respectée.
Il est important de noter que les besoins en énergie peuvent être couverts simultanément
par le solaire et par l’appoint électrique si l’énergie solaire disponible n’est pas
suffisante.

Afin d’éviter les instabilités, des hystérésis sont utilisés. Sur le ballon sanitaire par
exemple, la température à atteindre est de \SI{55}{\celsius} mais une variation de
\SI{5}{\celsius} est autorisée. Lorsque la température descend en dessous de
\SI{55}{\celsius} la demande en énergie est active. Lorsque le ballon atteint une
température supérieure à \SI{60}{\celsius} la demande est de nouveau inactive.
L’intervalle de température (\num{55} - \num{60}) représente alors une zone morte assurant
une meilleure stabilité au système. De plus, l’activation d’une pompe nécessite que la
température de la source (sortie des capteurs ou du ballon de stockage) est atteint un
seuil. Ce seuil minimal permet d’éviter l’activation trop rapide d’une pompe. Par exemple,
si la température dans le ballon tampon (stockage) est inférieure à \SI{45}{\celsius}
alors le chauffage solaire $Indirect$ ne peut pas être activé. Aussi, bien que le système
puissent remplir plusieurs fonctions concurrentes, certaines contraintes permettent de
garantir l’ordre des priorités. L’activation de la charge du ballon de stockage est ainsi
seulement autorisée si la $V3V$ est ouverte vers les capteurs solaires et si le ballon
sanitaire en partie basse (position de l’échangeur solaire) a atteint au minimum
\SI{30}{\celsius} ($T3$). Cette règle permet de garantir que la couverture des besoins en
$ECS$ est prioritaire.

\paragraph{Architecture de l’algorithme~:} % (fold)
\label{par:architecture_de_l_algorithme}
Chaque équipement ou entité est dans ces travaux représenté comme un automate fini ($FSM$,
Finite State Machine) (\figref{fig:automate_fini})~: la position de la V3V, l’état des
pompes, et le mode de chauffage actif. Un $FSM$ décrit un algorithme comme étant un
ensemble d’états ayant un jeu de conditions et/ou temporisations permettent de passer d’un
état à l’autre~: les transitions. L’unicité garantie par cette formulation permet de
renforcer la stabilité de l’algorithme. Elle impose cependant que chaque état soit capable
de réaliser la transition vers le bon état et ce pour toutes les conditions existantes.
Afin de simplifier sa création, les Automates Finis Hiérarchisés ($HFSM$, Hierarchical
Finite State Machine) ont été développés. Cette formulation admet que chaque état peut
être lui-même un $FSM$ dont les sous-états ne sont connus que de l’état maître. Cette
nouvelle formulation permet ainsi de réduire le nombre de conditions de transition et
l’encapsulation des sous-états simplifie la compréhension et la construction de
l’algorithme.

\begin{figure}
    \centering
    \includegraphics{Ressources/Images/Modelisation/Regulation/exemple_fsm.pdf}
    \caption{Illustration du fonctionnement d’un automate finie appliqué au contrôle
             d’un personnage fictif grâce à deux boutons~: A et B.}
    \label{fig:automate_fini}
\end{figure}

Le $FSM$ gouvernant le chauffage admet trois états primaires~: $Chauffage_{off}$,
$Chauffage_{on}$, et $Surchauffe_{on}$ (\figref{fig:automate_chauffage}). L’état
$Chauffage_{on}$ admet deux $FSM$ fonctionnant en parallèle~: celui du chauffage
solaire, et celui du chauffage électrique. Le chauffage solaire admet ainsi trois sous-états~:
$Solaire_{off}$, $Solaire_{direct}$, ou $Solaire_{indirect}$. Le premier est actif lorsque
il n’y a pas d’énergie solaire disponible ou pas assez pour couvrir les besoins en ECS et
en chauffage. Le second ($Solaire_{direct}$) est actif lorsque il y a une demande en
chauffage et que le système est en mode $Direct$. Finalement le dernier état est atteint
lorsque il existe toujours une demande en chauffage et que le système est en mode
$Indirect$.
Il peut être noté l’ajout d’un hystérésis de \SI{5}{\celsius} mis en place sur la
transition entre $Solaire_{off}$ et les deux autres états. Dans les deux cas, une
temporisation permet d’assurer que l’énergie est vraiment disponible et que ce n’est pas
le résultat d’une variation parasite temporaire (activation d’une pompe, fermeture d’une
vanne\dots). Le chauffage électrique lui admet deux états~: $Appoint_{off}$ et
$Appoint_{on}$. Seul une temporisation est utilisée afin de favoriser le solaire thermique.
L’appoint électrique reste ainsi actif tant que le besoin de chauffage n’est pas couvert.
Lorsque le besoin de chauffage est couvert, le chauffage passe dans l’état
$Surchauffe_{on}$. Comme décrit ci-avant, cet état autorise une élévation de la température
dans le bâtiment. Finalement, dans le cas où la surchauffe n’est pas autorisée ou que de nouveaux
besoins en chauffage apparaissent l’état du chauffage est réinitialisé à $Chauffage_{off}$
et la boucle de contrôle recommence.
L’exemple suivant illustre le fonctionnement de l’algorithme.
\blockquote{L’énergie disponible au niveau des capteurs est suffisante pour couvrir les besoins en chauffage de
la maison. Le système solaire est alors en mode $Direct$ et le chauffage dans l’état
$Chauffage_{on}$ avec le chauffage solaire dans le sous-état $Solaire_{direct}$ et
l’appoint électrique dans le sous-état $Appoint_{off}$.}
Un des habitants décide de prendre sa douche, abaissant la température de l’eau du ballon
sanitaire en dessous du seuil minimal (\SI{55}{\celsius}). Dans cette configuration la
pompe $S5$ va s’enclencher pour fournir de l’énergie au ballon sanitaire. Si l’énergie
disponible au niveau des capteurs est insuffisante pour le chauffage et le maintien du
ballon sanitaire alors la pompe $S2$ (chauffage) va s’arrêter car l’ECS est prioritaire.
Le chauffage solaire passe alors dans l’état $Solaire_{off}$ et l’appoint électrique est
en attente d’activation pour permettre de couvrir le chauffage (temporisation de
\SI{3}{\minute}).

\begin{figure}
    \centering
    \includegraphics{Ressources/Images/Modelisation/Regulation/chauffage_fsm.pdf}
    \caption{Description détaillée de l’$FSM$ contrôlant la transition entre
             les différents modes de chauffage.}
    \label{fig:automate_chauffage}
\end{figure}
% paragraph architecture_de_l_algorithme_ (end)


\paragraph{} % (fold)
\label{par:conclusion_algo}
L’algorithme de contrôle de la partie hydraulique fonctionne ainsi grâce à un algorithme
maître en cascade utilisant une approche par prospection (feed-forward) et délégant le
contrôle des équipements à une combinaison de $PID$ fonctionnant par rétroaction (feed-back).
C’est approche permet au système de prévenir au niveau global les changements et de
s’adapter aux besoins.
% paragraph conclusion_algo (end)
% subsubsection fonctionnement_global (end)


% - - - - - - - - - - - - - - - - - - - - - - - - - - - - - - - - - - - - - - -
\subsubsection{Contrôle du chauffage solaire (partie aéraulique)} % (fold)
\label{ssub:controle_du_chauffage_solaire}
Afin de favoriser le chauffage par l’énergie solaire, l’échangeur de chaleur entre l’eau
et l’air est placé en amont du terminal électrique. Grâce à une loi d’air, loi
proportionnelle par rapport à la température extérieure (similaire à une loi d’eau), la
température de soufflage nécessaire est déterminée~: $T_{air}^{cons}$ (\tabref{eq:temp_soufflage}).
La température d’air en sortie de l’échangeur est directement liée à la température et au
débit de l’eau en entrée de l’échangeur. Pour cette raison, la différence de température
entre l’air en sortie de l’échangeur et la consigne ($T_{air}^{cons}$), est utilisé comme
variable d’apprentissage par le $PID$ contrôlant la pompe $S2$.

\begin{equation}\label{eq:temp_soufflage}
    T_{air}^{cons} = T_{air}^{ext} + PI_{souff} \times (T_{air}^{max} - T_{air}^{ext})
\end{equation}
avec $PI_{souff}$, $T_{air}^{max}$, et $T_{air}^{ext}$ respectivement l’impact du
contrôleur $PI$ associé, la température maximale de l’air soufflé, et la température
extérieure de l’air. Cette formulation permet d’éviter de souffler un air à une
température élevée en continue. Cependant dans le cas où les besoins sont importants, le
débit hygiénique ($Q_{v}^{min} = \SI[per-mode=symbol]{90}{\meter\cubed\per\hour}$) est insuffisant
et le débit doit être augmenté.

Une temporisation de \SI{10}{min} est alors mis en place lorsque $T_{air}^{cons} = T_{air}^{max}$.
À la fin de la temporisation si la température de la maison est
toujours inférieure à $T_{ins}$, alors le débit soufflé est modulé jusqu’à un débit maximal de
$Q_{v}^{max} = \SI[per-mode=symbol]{900}{\meter\cubed\per\hour}$. Ces
temporisations permettent respectivement de favoriser en priorité le solaire et de
garantir le maintien du confort thermique (\figref{fig:chauffage_aeraulique}).

\begin{figure}
    \centering
    \includegraphics{Ressources/Images/Modelisation/Regulation/chauffage_aeraulique.pdf}
    \caption{Fonctionnement de la régulation du chauffage par vecteur air.}
    \label{fig:chauffage_aeraulique}
\end{figure}

Afin d’améliorer la couverture solaire sur le chauffage, l’algorithme autorise la
surchauffe de la maison durant les périodes d’inoccupation diurnes
(\figref{fig:control_air}) mais seul l’état $Solaire_{direct}$ est autorisé. Dans ce
cas, le débit minimum sanitaire est maintenue et seul $T_{air}^{cons}$ varie afin que la
température intérieure atteigne la température limite de surchauffe~: $T^{cons}_{sol}$.
L’augmentation de la température intérieure permet de profiter de l’inertie du bâtiment, et
ainsi retarder ou éviter la relance du chauffage~: l’utilisation potentielle de
l’appoint électrique sur le chauffage est réduit.
\begin{figure}
    \centering
    \includegraphics{Ressources/Images/Modelisation/Regulation/control_air_curve.pdf}
    \caption{Principe de la surchauffe diurne durant une journée hivernale.}
    \label{fig:control_air}
\end{figure}
% subsubsection controle_du_chauffage_solaire (end)
% subsection algorithme_de_controle (end)
% section description_du_cas_d_etude_le_systeme_solaire (end)








% ..............................................................................
% ..............................................................................
\section{Étude paramétrique~:} % (fold)
\label{sec:etude_parametrique}
% ------------------------------------------------------------------------------
L’étude paramétrique détaillée dans ce chapitre a été réalisée à partir de la
version définitive de l’algorithme et du bâtiment. En effet, le projet étant
porté à travers le projet $COMEPOS$ (\munsure{Plus interne IGC mais possible de le dire ?})
il est le fruit d’un travail itératif. Le bâtiment comme le $SSC$ ont évolués au fur et à mesure
de l’avancement du projet.

Dans un premier temps, les différents scénarios puis les variations techniques
sont présentées. Les résultats sont ensuite détaillés et les limites
de l’approche discutées.


% ------------------------------------------------------------------------------
\subsection{Climats étudiés} % (fold)
\label{sub:climats_etudies}
Afin d’être représentatif des zones climatiques de la France, l’étude a été réalisée
pour 5 villes~: Bordeaux, Nantes, Strasbourg, Limoges, et Marseille (\figref{fig:carte_france},
\tabref{tab:description_climat}).
\begin{figure}
    \centering
    \includegraphics{Ressources/Images/Modelisation/Batiment/France_map.pdf}
    \caption{Cartographie des villes sélectionnées pour l’étude paramétrique.}
    \label{fig:carte_france}
\end{figure}
La ville de Marseille correspond à un climat très ensoleillé avec une faible demande en
chauffage. À l’opposée le climat de Strasbourg est rude avec
une forte demande en chauffage. Le climat de Bordeaux est lui modéré~: l’ensoleillement est bon
et la demande en chauffage faible. Limoges et Nantes se placent entre Bordeaux et
Strasbourg avec un bon ensoleillement mais une demande en énergie pour le chauffage plus
importante. Pour ces différents sites, la température du sol est
considéré comme constante et est fixée à \SI{10}{\celsius}. Bien que impactant les besoins
d’un bâtiment, sa détermination de manière précise est un processus complexe spécialement
lorsque de nombreuses variations de l’enveloppe comme des charges internes sont envisagées.

Dans un premier temps, l’impact du climat est investigué puis de nombreuses variations au
niveau des équipements, des scénarios, et de la régulation sont évaluées pour deux
villes~: Bordeaux et Strasbourg.

\begin{table}
\centering
\caption{Description des différents climats retenues}
\label{tab:description_climat}
\begin{tabular}{ l c c  c  c  c  c }
  \toprule
                                          &    & \textbf{Bordeaux} & \textbf{Nantes} & \textbf{Strasbourg} & \textbf{Limoges} & \textbf{Marseille} \\
  \midrule
  \addlinespace[\defaultaddspace]
  \multirow{3}{*}{Irradiation solaire} & $IGH$   & \num{1264}              & \num{1184}               & \num{1091}                & \num{1257}              & \num{1545}              \\
                                       & $IDN$   & \num{929}               & \num{885}               & \num{721}                 & \num{1209}              & \num{1503}              \\
                                       & $IDH$   & \num{712}               & \num{665}               & \num{650}                 & \num{602}              & \num{615}               \\
  \addlinespace[\defaultaddspace]
  \multirow{2}{*}{Température eau froide} & Min     & \num{8.9}               & \num{8.3}               & \num{5.3}                 & \num{7}                 & \num{12}                \\
                                          & Max     & \num{16}                & \num{15}               & \num{14}                  & \num{14}                & \num{19}                \\
  \addlinespace[\defaultaddspace]
  DJU (\SI{19}{\celsius})                 & -  & \num{2408}              & \num{2660}               & \num{3360}                & \num{2972}              & \num{2049}              \\
  \bottomrule
\end{tabular}
\end{table}
% subsection climats_etudies (end)


% ------------------------------------------------------------------------------
\subsection{Scénarios étudiés} % (fold)
\label{sub:scenarios_etudies}
% - - - - - - - - - - - - - - - - - - - - - - - - - - - - - - - - - - - - - - -
\subsubsection{Puisage en eau chaude sanitaire} % (fold)
\label{ssub:puisage_en_eau_chaude_sanitaire}
Afin de pouvoir estimer la consommation nécessaire pour la production d’$ECS$, un profil
de puisage type est nécessaire. Pour chaque climat, la température de l’eau du réseau suit
une évolution mensuelle extrapolée durant la simulation
(\tabref{tab:temp_eau}). Il existe donc une forte disparité entre les différents climats
avec des extremums représentés respectivement par Strasbourg et Marseille.
\begin{itemize}
    \item Les minimums varient entre \SI{5.3}{\celsius} et \SI{12}{\celsius}.
    \item Les maximums varient entre \SI{14}{\celsius} et \SI{19}{\celsius}.
\end{itemize}

\begin{table}
\centering
\caption{Température de l'eau du réseau (\si{\kelvin}) au cours de l'année en fonction de la
         position géographique.}
\label{tab:temp_eau}
\begin{tabular}{l*{12}{c}}
    \toprule
               & Janv. & Fevr. & Mars & Avr. & Mai & Juin & Juil. & Août & Sept. & Oct. & Nov. & Dec. \\
    \midrule
    Strasbourg & \num{5.3}   & \num{5.8}   & \num{7.7}  & \num{9.5}  & \num{11}  & \num{13}   & \num{14}    & \num{14}   & \num{12}    & \num{9.8}  & \num{7.5}  & \num{5.8}  \\
    Limoges    & \num{7}     & \num{7.4}   & \num{9}    & \num{10}   & \num{12}  & \num{14}   & \num{14}    & \num{14}   & \num{13}    & \num{11}   & \num{8.8}  & \num{7.3}  \\
    Toulouse   & \num{8.6}   & \num{9.2}   & \num{11}   & \num{12}   & \num{14}  & \num{16}   & \num{17}    & \num{17}   & \num{16}    & \num{13}   & \num{11}   & \num{9}    \\
    Bordeaux   & \num{8.9}   & \num{9.3}   & \num{11}   & \num{12}   & \num{14}  & \num{15}   & \num{16}    & \num{16}   & \num{15}    & \num{13}   & \num{11}   & \num{9.2}  \\
    Nantes     & \num{8.3}   & \num{8.5}   & \num{9.9}  & \num{11}   & \num{13}  & \num{14}   & \num{15}    & \num{15}   & \num{14}    & \num{12}   & \num{9.8}  & \num{8.6}  \\
    Marseille  & \num{12}    & \num{12}    & \num{13}   & \num{14}   & \num{16}  & \num{18}   & \num{19}    & \num{19}   & \num{18}    & \num{16}   & \num{14}   & \num{12}   \\
    \bottomrule
\end{tabular}
\end{table}


\paragraph{Profils retenus~:} % (fold)
\label{par:profils_retenus}
Cinq scénarios ont été ont été considérés dans ces travaux (\figref{fig:profil_puisage}).
Pour l’ensemble des profils, un volume de puisage typique moyen de
\SI{33}{\litre\per(jour\period pers)}\,(\SI{60}{\celsius}), soit un
volume total de \SI{220}{\litre/jour}\,(\SI{40}{\celsius}) pour les quatre occupants est
retenu. Le profil de référence utilisé est le scénario décrit dans la norme
\textcite{EN129771}, \textbf{EN\,12977}. Trois variantes de celui-ci ont aussi été évaluées favorisant
respectivement un puisage en matinée (\textbf{Matin}), en soirée (\textbf{Soir}) et un
puisage similaire pour les trois pics journaliers (\textbf{Réparti}). Finalement un scénario
issue de données statistiques \parencite{ADEME2016} a aussi été implémenté
(\textbf{Réaliste}). Ce dernier décrit de manière plus représentative le comportement
d’une famille dans une habitation individuelle ou un appartement, et les besoins
journaliers sont pondérées en fonction du jour de la semaine (\tabref{tab:coef_semaine}).
\begin{figure}
    \centering
    \includegraphics{Ressources/Images/Modelisation/Scenario/puisage.pdf}
    \caption{Description des profils de puisage envisagés.}
    \label{fig:profil_puisage}
\end{figure}
% paragraph profils_retenus (end)

\paragraph{Pondération de la demande~:} % (fold)
\label{par:ponderation_de_la_demande}
Afin d’évaluer l’impact de la variation mensuelle des besoins en $ECS$, l’ajout d’un
coefficient mensuel est investigué. Les coefficients utilisés sont issues des travaux de
l’\textit{ADEME} \parencite{ADEME2016}. Le demande totale reste inchangée~; seule sa
répartition est altérée. Les résultats de l’\textit{ADEME} montrent ainsi que le puisage est plus
important durant la période hivernale (\tabref{tab:coef_mois}).

\begin{table}
\centering
\caption{Détail des coefficients de pondération journaliers pour le profil de
         puisage Réaliste.}
\label{tab:coef_semaine}
\begin{tabular}{l*{7}{c}}
    \toprule
                & Lundi & Mardi & Mercredi & Jeudi & Vendredi & Samedi & Dimanche \\
    \midrule
    Coefficient & \num{0.97}  & \num{0.95}  & \num{1.00}     & \num{0.97}  & \num{0.96}     & \num{1.02}   & \num{1.13}     \\
    \bottomrule
\end{tabular}
\end{table}

\begin{table}
\centering
\caption{Détail des coefficients de pondération mensuels pour le profil de
         puisage Réaliste.}
\label{tab:coef_mois}
\begin{tabular}{l*{12}{c}}
    \toprule
                & Janv. & Fevr. & Mars & Avr. & Mai & Juin & Juil. & Août & Sept. & Oct. & Nov. & Dec. \\
    \midrule
    Coefficient & \num{1.11}   & \num{1.2}   & \num{1.11}  & \num{1.06}  & \num{1.03}  & \num{0.93}   & \num{0.84}    & \num{0.72}   & \num{0.92}    & \num{1.03}  & \num{1.04}  & \num{1.01}  \\
    \bottomrule
\end{tabular}
\end{table}
% paragraph ponderation_de_la_demande (end)

\paragraph{Variations paramétriques~:} % (fold)
\label{par:variations_parametriques}
L’évaluation de l’impact de la variation des profils de puisage comme de la quantité des
besoins sur la performance du $SSC$ est donc retenue. Les consommations variant fortement
d’une habitation à une autre, des consommations de \num{27}, \num{33}, et
\SI{40}{\litre\per(jour \period  pers)}\,(\SI{60}{\celsius}) sont aussi évaluées.
% paragraph variations_parametriques (end)
% subsubsection puisage_en_eau_chaude_sanitaire (end)


% - - - - - - - - - - - - - - - - - - - - - - - - - - - - - - - - - - - - - - -
\subsubsection{Ventilation} % (fold)
\label{ssub:ventilation}
Deux scénarios de ventilation ont été retenus pour l’étude paramétrique~:
\begin{itemize}
    \item Référence~: $90-20$\,\si[per-mode=symbol]{\meter\cubed\per\hour} (\ref{ssub:ventilation_ref})
    \item Constant~: $90-90$\,\si[per-mode=symbol]{\meter\cubed\per\hour}
\end{itemize}
L’analyse de ces deux scénarios permettra d’identifier l’impact d’un débit réduit sur
les consommations nécessaires afin de maintenir la température de consigne dans la
maison.
% subsubsection ventilation (end)


% - - - - - - - - - - - - - - - - - - - - - - - - - - - - - - - - - - - - - - -
\subsubsection{Température de consigne} % (fold)
\label{ssub:temperature_de_consigne}
\paragraph{Consigne de chauffage~:} % (fold)
\label{par:consigne_de_chauffage}
Afin de couvrir une large palette de scénarios, la consigne en inoccupation, en occupation
diurne, et en inoccupation nocturne sont modifiées indépendamment. La consigne de
référence étant, pour rappel, définie comme étant $19$-$18$-$16$~: \SI{19}{\celsius} durant les
occupations diurnes, \SI{18}{\celsius} durant les occupations nocturnes et
\SI{16}{\celsius} en inoccupation. À partir des règles et contraintes définies dans
\tabref{tab:consigne_chauffage} il est alors possible de construire un lot de scénarios
représentatifs. L’étude cherchera en particulier à évaluer l’impact d’une augmentation de
la consigne de chauffage, ainsi que du réduit en inoccupation et en période
nocturne. Les scénarios construits sont les suivants~: $19$-$18$-$16$, $19$-$19$-$16$, $19$-$19$-$19$,
$20$-$18$-$16$, $20$-$20$-$16$.

\begin{table}
\centering
\caption{Variations envisagées pour la consigne de chauffage en fonction de la période.}
\label{tab:consigne_chauffage}
\begin{tabular}{l c c c c}
    \toprule
                           & \textbf{\num{16}}                     & \textbf{\num{18}}                     & \textbf{\num{19}}                     & \textbf{\num{20}}              \\
    \midrule
    Occupation (diurne)    &                             &                             & \cellcolor{SolarizedBrBlue} & \cellcolor{SolarizedBrBlue} \\
    Occupation (nocturne)  &                             & \cellcolor{SolarizedBrBlue} & \cellcolor{SolarizedBrBlue} & \cellcolor{SolarizedBrBlue} \\
    Inoccupation           & \cellcolor{SolarizedBrBlue} &                             & \cellcolor{SolarizedBrBlue} &                     \\
    \bottomrule
\end{tabular}
\end{table}
% paragraph consigne_de_chauffage (end)


\paragraph{Consigne de chauffage solaire~:} % (fold)
\label{par:consigne_de_chauffage_solaire}
L’impact de la température de consigne solaire est aussi discuté. Deux configurations sont
simulées. La première admet une température de consigne de \SI{22}{\celsius} et la seconde
ne considère pas d’élévation de la température par le solaire. Pour rappel l’élévation de
la température n’est réalisé que par l’énergie solaire provenant des capteurs, et donc
uniquement si le système est dans l’état $Solaire_{direct}$. Finalement les interactions
entre consigne et consigne solaire de chauffage sont aussi discutées
% paragraph consigne_de_chauffage_solaire (end)
% subsubsection temperature_de_consigne (end)
% subsection scenarios_etudies (end)


% ------------------------------------------------------------------------------
\subsection{Variations techniques étudiées} % (fold)
\label{sub:variations_techniques_etudiees}
% - - - - - - - - - - - - - - - - - - - - - - - - - - - - - - - - - - - - - - -
\subsubsection{Fluide caloporteur} % (fold)
\label{ssub:fluide_caloporteur}
Afin de pouvoir implémenter le modèle pour tous les climats, il est nécessaire de
tenir compte des risques de gel. Le fluide caloporteur utilisé à travers les capteurs
solaires est donc un mélange~: eau (\SI{70}{\percent}) et éthylène-glycol (\SI{30}{\percent}).

Seul les capteurs solaires sont à l’extérieur, il n’est alors pas nécessaire de protéger
l’ensemble du système contre le gel. En effet, l’eau du ballon sanitaire provient du
réseau d’eau de ville, et le ballon de stockage est dans une zone chauffée. Ainsi dans les
deux cas une eau sans glycol est considérée. Ainsi seules les canalisations reliant les
échangeurs contiennent du glycol et la modélisation admet deux fluides différents
(\tabref{tab:fluide_carac}). Enfin, la variation de la capacité massique étant négligeable sur
les plages de variation considérées (\num{0} à \SI{120}{\celsius}), le modèle assume une
capacité massique constante permettant de simplifier les systèmes d’équations.

\begin{table}
\centering
\caption{Caractéristiques des deux fluides caloporteurs utilisés pour la modélisation.}
\label{tab:fluide_carac}
\begin{tabular}{l *{2}{c} r}
    \toprule
                       & Eau                 & Eau + glycol          & Unité                             \\
    \midrule
    Chaleur massique   & \num{4180}          & \num{3608}            & \si{\joule\per(kg\period\kelvin)} \\
    Masse volumique    & \num{1000}          & \num{1034}            & \si{kg\per\meter\cubed}           \\
    Plage de variation & \num{0} à \num{100} & \num{-20} à \num{110} & \si{\celsius}                     \\
    \bottomrule
\end{tabular}
\end{table}
% subsubsection fluide_caloporteur (end)

% - - - - - - - - - - - - - - - - - - - - - - - - - - - - - - - - - - - - - - -
\subsubsection{Capteurs solaires} % (fold)
\label{ssub:capteurs_solaires}
Afin d’évaluer l’impact du choix du capteur, \num{3} capteurs différents ont été retenus
(\tabref{tab:capteurs_specs}). Le capteur de référence (IDMK\,25), un capteur sous-vide
(12\,CPC58) et un autre capteur plan très performant (308C\,HP). Les caractéristiques du
premier (IDMK\,25) sont issues de sa fiche technique
(\ref{ssub:capteurs_solaires}). Les deux autres types sont issue d’une source
commune~: \href{www.solar-rating.org}{ICC SRCC}. Cette source a été sélectionnée car elle
regroupe les résultats d’essais réalisés dans des conditions similaires par un organisme indépendant.
De plus des informations supplémentaires non accessibles sur les fiches techniques mais nécessaires
pour le modèle numérique sont explicitement définis. Les résultats de ces essais sont disponibles en
\figref{fig:caracs_radco} et \figref{fig:caracs_skypro}.
Enfin les modèles retenues ont une surface nette similaire facilitant la comparaison entre
eux.

\begin{table}
\centering
\caption{Caractéristiques des panneaux solaires.
\label{tab:capteurs_specs}}
\begin{tabular}{l c c c r}
    \toprule
                                 & IDMK\,25 (référence)  & 308C\,HP              & 12\,CPC58        & Unité                       \\
    \midrule
    Fabricant                    & Sonnenkraft          & Radco                & Sky Pro         & -                           \\
    Type                         & Plan vitrée          & Plan vitrée          & Tubulaire       & -                           \\
    Surface nette                & \num{2.32}           & \num{2.193}          & \num{2.28}      & \si{m^{2}}                  \\
    Poids à vide                 & \num{54}             & \num{36}             & \num{53}        & \si{kg}                     \\
    Contenance                   & \num{1.35}           & \num{3.5}            & \num{1.83}      & \si{\litre}                 \\
    $\eta_{0}$                   & \num{78}             & \num{83.4}           & \num{63}        & \si{\%}                     \\
    Pente                        & \num{-5.103}         & \num{-4.777}         & \num{-0.975}    & -                           \\
    $a_{1}$                      & \num{3.796}          & \num{1.4539}         & \num{0.9249}    & \si{W/(m^{2}\period K)}     \\
    $a_{2}$                      & \num{0,013}          & \num{0.0589}         & \num{0.00069}   & \si{W/(m^{2}\period K^{2})} \\
    $IMDiff$                     & \num{100}            & \num{96}             & \num{102}       & \si{\%}                     \\
    \bottomrule
\end{tabular}
\end{table}

Au cours de l’étude paramétrique, l’orientation, la surface, et l’inclinaison
des capteurs est évaluée afin de comparer les résultats avec les observations déjà
relevées au cours des études existantes \parencite{Task262003,Shariah2002587}.
Les variations suivantes sont retenues~:
\begin{itemize}
  \item Inclinaison~: \SI{18.9}{\degree} (\SI{33}{\percent}), \SI{33}{\degree}, \SI{45}{\degree}, \SI{60}{\degree}
  \item Orientation~: Est, Ouest, Sud
  \item Nombre de capteurs~: \num{2}, \num{4}, \num{6}, \num{8}
\end{itemize}
Finalement, la maison de référence est orientée Sud, comporte 4 capteurs
de type IDMK\,25, et une toiture ayant une pente de \SI{33}{\percent}.
% subsubsection capteurs_solaires (end)


% - - - - - - - - - - - - - - - - - - - - - - - - - - - - - - - - - - - - - - -
\subsubsection{Ballons} % (fold)
\label{ssub:ballons}
L’analyse paramétrique tient aussi compte de variations au niveau des ballons comme
son volume. Celui-ci est discrétisé en \num{20} segments numérotés de haut en bas
permettant de prendre en compte le phénomène de stratification.
Afin d’être cohérent, les caractéristiques des ballons sont proportionnelles à sa taille~:
si le volume du ballon augmente, sa hauteur augmente aussi.
De même, la taille de l’échangeur est adaptée à la nouvelle hauteur du ballon
afin que sa position relative reste identique. De cette manière, le ballon de
stockage conserve un échangeur couvrant la quasi totalité du volume, et
l’échangeur solaire sur le ballon sanitaire reste adapté à la nouvelle taille du
ballon. Les variations retenues pour chaque ballon sont identiques~:
\SI{150}{l}, \SI{300}{l}, et \SI{450}{l}.

La position de l’échangeur solaire est aussi étudiée. Ces travaux ne considère que
la position de l’échangeur du ballon sanitaire ($Ech_{sol}^{pos}$) car l’échangeur du
ballon de stockage couvre presque l’intégralité de sa hauteur. De plus, afin d’être
cohérent avec les variations réalisées sur le volume des ballons, la variation de
$Ech_{sol}^{pos}$ est réalisée relativement à sa position d’origine, laissant intact sa
longueur. Il est alors possible d’évaluer l’impact du volume du ballon et de la $Ech_{sol}^{pos}$
de manière combinés ou bien indépendamment. Les coefficients
retenues sont issus des contraintes imposées par les spécificités techniques des ballons~:
\num{0.8}, \num{1}, et \num{1.3}. Avec un coefficient de \num{0.8} l’échangeur sera ainsi
positionné plus bas et avec un coefficient de \num{1.3}, il sera à l’inverse plus haut. La
position de référence pour l’échangeur(\num{1}) correspond au $12^{ème}$ segment.
% subsubsection ballons (end)


% ------------------------------------------------------------------------------
\subsubsection{Algorithme} % (fold)
\label{ssub:variations_algorithmiques}
Finalement il est aussi étudié à travers cette étude des variations au niveau de
l’algorithme de contrôle (\tabref{tab:variations_algo}). $DeltaT_{sol}$, la différence de
température entre la source (sortie des capteurs) et la cible (sortie des échangeurs) est
modulée autour du cas de référence (\SI{10}{\celsius}). Augmenter le $DeltaT_{sol}$
revient à attendre plus longtemps avant d’autoriser l’utilisation de l’énergie solaire et
impose une température en sortie des capteurs plus élevée afin de valoriser les apports
solaires. À l’inverse diminuer ce paramètre amène le système a être plus réactif mais le
risque d’instabilité est cependant plus important. Il est aussi évalué l’impact des
temporisations au niveau de la régulation du chauffage par vecteur air~:
\begin{itemize}
  \item Activation de la batterie électrique terminale
  \item Modulation du débit de soufflage
  \item Modulation de la température de soufflage
\end{itemize}
La première temporisation, \SI{3}{min} pour le cas de référence, sera notée
$Tempo_{batterie}$. Elle permet d’éviter d’activer la batterie terminale électrique
directement à chaque demande de chauffage. Elle permet donc de retarder son activation
quand de l’énergie solaire est disponible. La seconde temporisation correspond à la temporisation
entre la modulation de la température, et la modulation du débit pour le chauffage par
l’air. Le cas de référence admet une temporisation de \SI{10}{min} notée $Tempo_{souff}$.

\begin{table}
\centering
\caption{Variations algorithmiques étudiées lors de l’étude paramétrique.
\label{tab:variations_algo}}
\begin{tabular}{l c c r}
    \toprule
    Paramètre          & Référence & Variations          & Unité         \\
    \midrule
    $DeltaT_{sol}$     & \num{10}  & \num{5}, \num{15}   & \si{\celsius} \\
    $Tempo_{batterie}$ & \num{3}   & \num{2}, \num{4}    & \si{min}      \\
    $Tempo_{souff}$    & \num{10}  & \num{5}, \num{15}   & \si{min}      \\
    \bottomrule
\end{tabular}
\end{table}
% subsubsection algorithme (end)
% subsection variations_techniques_etudiees (end)



% ------------------------------------------------------------------------------
\subsection{Analyse des résultats} % (fold)
\label{sub:analyse_des_resultats}
% - - - - - - - - - - - - - - - - - - - - - - - - - - - - - - - - - - - - - - -
\subsubsection{Méthodologie} % (fold)
\label{ssub:methodologie}
L’analyse a été réalisée sur la base d’un cas de référence dont les scénarios et les
caractéristiques ont été présentés dans les parties précédentes. Sans mentions explicites les
simulations utilisent ces paramètres. L’ensemble des variations possibles ainsi que
le cas de référence sont décrits à travers un tableau récapitulatif (\tabref{tab:ref_description})
afin d’offrir une vue globale des variations étudiées.
Enfin afin d’évaluer le système solaire différent indicateurs sont introduit
\eqref{eq:indicators}. La consommation électrique tient compte de la consommation des deux
appoints (chauffage et $ECS$) mais aussi de la consommation des pompes du $SSC$. Elle est
notée $Conso_{app}$ dans le reste de l’étude. Enfin $F_{sol}^{ECS}$ et $F_{sol}^{CH}$
sont respectivement le taux de couverture sur la production de l’eau chaude sanitaire
et sur le chauffage.

\begin{table}
\centering
\caption{Description de la solution de référence et de variations étudiées.}
\label{tab:ref_description}
  \begin{tabular}{l c C{4cm} r r}
    \toprule
    \addlinespace
                                           & Référence & Variations                             & Unité         & Détails                                               \\
    \multicolumn{5}{l}{\textbf{Scénarios}}                                                                                                                                 \\
    \midrule
    Ventilation                            & $90-20$   & $$90-90$$                                  & \si{m^{3}/h}  & \ref{ssub:ventilation}                                \\
    Chauffage                              & $19$-$18$-$16$  & $19$-$19$-16, $19$-$19$-$19$, $20$-$18$-$16$, $20$-$20$-$16$ & \si{\celsius} & \multirow{2}{*}{\ref{ssub:temperature_de_consigne}}   \\
    Chauffage solaire                      & Avec (22) & Sans                                   & \si{\celsius} &                                                       \\
    $Puisage_{scenario}$                   & EN\,12977 & Soir, Matin, Réparti, Réaliste         & \si{l/h}      & \multirow{3}{*}{\ref{ssub:puisage_en_eau_chaude_sanitaire}}            \\
    $Puisage_{vol}$ (\SI{60}{\celsius})    & \num{33}  & \num{27}, \num{40}                     & \si{\litre/(jour\period pers)}      &                                 \\
    $Puisage_{mod}$                        & Non       & Mensuelle                              & -             &                                                       \\
    \\
    \addlinespace[\defaultaddspace]
    \multicolumn{5}{l}{\textbf{Équipements}}                                                                                                                              \\
    \midrule
    Modèle des capteurs                    & IDMK\,25  & 308C\,HP, 12\,CPC58                    & -             & \multirow{4}{*}{\ref{ssub:capteurs_solaires}}         \\
    Nombre de capteurs                     & \num{4}   & \num{2}, \num{6}, \num{8}              & -             &                                                       \\
    Inclinaison des capteurs               & \num{33}  & \num{18.9}, \num{45}, \num{60}         & \si{\degree}  &                                                       \\
    Orientation des capteurs               & Sud       & Est, Ouest                             & -             &                                                       \\
    Volume ballon de stockage              & \num{300} & \num{150}, \num{450}                   & \si{\litre}   & \multirow{3}{*}{\ref{ssub:ballons}}                   \\
    Volume ballon $ECS$                    & \num{300} & \num{150}, \num{450}                   & \si{\litre}   &                                                       \\
    $Ech_{sol}^{pos}$                      & \num{1}   & \num{0.8}, \num{1.3}                   & \si{m}        &                                                       \\
    \\
    \addlinespace
    \multicolumn{5}{l}{\textbf{Algorithme}}                                                                                                                                 \\
    \midrule
    $Tempo_{batterie}$                     & \num{3}   & \num{2}, \num{4}                       & \si{\min}     & \multirow{3}{*}{\ref{ssub:variations_algorithmiques}} \\
    $Tempo_{souff}$                        & \num{10}  & \num{5}, \num{15}                      & \si{\min}     &                                                       \\
    $DeltaT_{sol}$                         & \num{10}  & \num{5}, \num{15}                      & \si{\celsius} &                                                       \\
    \addlinespace[\defaultaddspace]
    \bottomrule
  \end{tabular}
\end{table}

\begin{equation}
\label{eq:indicators}
\setlength{\jot}{10pt}  % Change equation vertical space only for this environment
\begin{aligned}
    Conso_{app}   &= Conso_{app}^{ECS} + Conso_{app}^{CH} + Conso_{pompes} \\
    F_{sol}       &= \frac{Prod_{sol}}{Conso_{app}}     \\
    F_{sol}^{ECS} &= \frac{Prod_{sol}^{ECS}}{Conso_{app}^{CH}} \\
    F_{sol}^{CH}  &= \frac{Prod_{sol}^{CH}}{Conso_{app}^{ECS}}
\end{aligned}
\end{equation}
% subsubsection methodologie (end)



% - - - - - - - - - - - - - - - - - - - - - - - - - - - - - - - - - - - - - - -
\subsubsection{Impact des conditions météorologiques} % (fold)
\label{ssub:impact_des_conditions_meteorologiques}
Les résultats obtenues à enveloppe identique pour les différents climats sont disponibles
à travers le \tabref{tab:performance_annuelles}. Le système solaire permet d’obtenir une
bonne performance pour tous les climats autant sur $F_{sol}^{ECS}$ que sur $F_{sol}^{CH}$.
L’énergie solaire transmise à l’eau ($Solaire_{abs}$) comme les pertes en lignes à travers
les canalisations ($Pertes_{réseau}$) sont similaires pour les différents climats.
Cependant, il est observé un impact important de la température de l’eau froide du réseau.
En effet les simulations admettent le même scénario de puisage mais la $Conso_{app}$
varient de \SI{400}{\kilo\watt\hour} entre Marseille et Strasbourg. De plus, malgré des
conditions plus favorable à Nantes, la $Conso_{app}$ à Limoges est inférieure grâce à un
ensoleillement plus favorable (\tabref{tab:temp_eau}, \tabref{tab:description_climat}).
La même chose peut être observé au niveau du chauffage. La consommation totale à Limoges
est supérieure de \SI{200}{kWh} par rapport à Nantes mais la $Conso_{app}$ est elle
seulement supérieure de \SI{100}{kWh}. Enfin, la consommation des pompes ne semble pas
être un facteur impactant sauf pour Marseille où la $Conso_{app}$ cumulée du chauffage et
de la production d’$ECS$ est presque nulle. À l’opposé sur Strasbourg, quatre panneaux
solaires semblent insuffisants pour couvrir les besoins importants.

\begin{table}
\small
\centering
\caption{Performances annuelles du système solaire pour différents climats.}
\label{tab:performance_annuelles}
\begin{tabular}{l c c c c c c c c c c c c}
    \toprule
               &   \multicolumn{9}{c}{Consommation [\si{\kilo\watt\hour}]} & & \multicolumn{2}{c}{\multirow{2}{*}{\%}} \\
    \cmidrule(r){2-10}
               & \multicolumn{2}{c}{\textbf{Totale}} &  \multicolumn{3}{c}{\textbf{Électrique}}  & \multicolumn{4}{c}{\textbf{Solaire}} & \\
    \cmidrule(r){2-3}
    \cmidrule(r){4-6}
    \cmidrule(r){7-10}
    \cmidrule(r){12-13}
               & $ECS$    & $CH$      &  $ECS$        & $CH$ & Pompes    & $Solaire_{abs}$  & $Pertes_{réseau}$ & $ECS$  & $CH$ & & $F_{sol}^{ECS}$  & $F_{sol}^{CH}$ \\
    \midrule
    Bordeaux   & \num{2983}     & \num{1024}      &  \num{101}          & \num{91}          &  \num{7}                 & \num{3387}                  & \num{219}       & \num{2444}   &  \num{949}    &   & \num{95}         & \num{91}  \\
    Strasbourg & \num{3180}     & \num{1986}      &  \num{479}          & \num{1203}        &  \num{8}                 & \num{3164}                  & \num{199}       & \num{2332}   &  \num{845}    &   & \num{83}         & \num{42}  \\
    Marseille  & \num{2784}     & \num{975}       &  \num{2}            & \num{1}           &  \num{7}                 & \num{3280}                  & \num{218}       & \num{2300}   &  \num{974}    &   & \num{100}        & \num{100} \\
    Nantes     & \num{3044}     & \num{1114}      &  \num{233}          & \num{248}         &  \num{7}                 & \num{3295}                  & \num{209}       & \num{2399}   &  \num{902}    &   & \num{91}         & \num{78}  \\
    Limoges    & \num{3120}     & \num{1311}      &  \num{217}          & \num{359}         &  \num{9}                 & \num{3474}                  & \num{209}       & \num{2502}   &  \num{983}    &   & \num{92}         & \num{73}  \\
    \bottomrule
\end{tabular}
\end{table}

L’évolution des températures des différents ballons durant la période de chauffage pour
les différents climats donnent plus d’informations (\figref{fig:temp_ballon_mensuel}). Du
fait des priorités de régulation le ballon de stockage ne peut être chargé si la partie
basse du ballon sanitaire ($T3$) n’est pas à une température suffisante. Ainsi l’évolution
des température des ballons permet de renseigner sur l’activité du système, et plus
particulièrement sur l’importance de l’énergie solaire récupérée par rapport à la demande.
Ainsi sur Limoges, le système utilise principalement l’énergie solaire durant tous les
mois à l’exception de décembre où $T5$ stagne. Durant janvier, l’énergie solaire est
suffisante pour charger le ballon de stockage et réduire la $Conso_{app}$ sur
l’$ECS$. Cependant, $T4$ stagne autour de \SI{55}{\celsius} car l’énergie solaire fournie
ne permet pas de charger le ballon sanitaire suffisamment. Sur Strasbourg, le système ne
collecte pas assez d’énergie de novembre à février afin d’être autonome mais l’énergie
accumulée permet cependant de préchauffer l’eau du réseau et donc réduire la
$Conso_{app}$. Finalement sur Bordeaux le système permet de couvrir la majorité des besoins
même si durant décembre et janvier l’appoint électrique est nécessaire ponctuellement.

\begin{figure}
    \centering
    \includegraphics{Ressources/Images/Parametrique/temp_ballons.pdf}
    \caption[Évolution des température internes des ballons du $SSC$]
            {Évolution des température internes des ballons du $SSC$ en fonction du mois
             de l’année avec indication de la moyenne (triangle vert) et
             de la médiane (rectangle horizontal bleu).}
    \label{fig:temp_ballon_mensuel}
\end{figure}

\iunsure{Ajouter couverture par mois en fonction des climats}

\paragraph{Période de simulation~:} % (fold)
\label{par:periode_de_simulation}
L’optique de cette étude paramétrique étant d’évaluer l’impact de différents éléments du
$SSC$ (technique et algorithmique) il est n’est pas nécessaire de simuler les périodes ou
le système est autonome. Chaque simulation annuelle nécessite en effet entre
\SI{2}{\hour} et \SI{3}{\hour} et réduire la période de simulation permet de réduire la durée
d’intégration. Des simulations comprenant un jeu de paramètres défavorables ont aussi été
réalisées afin d’évaluer la période de simulation minimale nécessaire. Pour ces
simulations, un scénario de chauffage $19$-$19$-$19$, un volume de \SI{450}{\litre} pour les
deux ballons, ainsi que \num{2} capteurs avec une orientation Ouest et une inclinaison de
\SI{18.9}{\degree} sont retenus. Les résultats montrent que durant la
période estivale, le système solaire est autonome avec une $Conso_{app}$ de seulement
\SI[per-mode=symbol]{0.01}{\kilo\watt\hour\per\metre\squared} et
\SI[per-mode=symbol]{0.08}{\kilo\watt\hour\per\metre\squared}
pour respectivement Bordeaux et Strasbourg. Dans le reste de l’étude paramétrique
les résultats sont donc discutés \emph{sur une période s’étendant du $1^{er}$
octobre au $30$ avril} permettant de réduire à \SI{\approx 1}{\hour} par simulation. Les
résultats sont fournis pour la période concernée et doivent donc être vu comme la
\textbf{performance minimale} du $SSC$. En effet, le système est autonome durant le reste
de l’année car les besoins en chauffage et en $ECS$ sont complètement couverts par la
production solaire~: le $F_{sol}^{ECS}$ et le $F_{sol}^{CH}$ annuelles sont donc
supérieures car ils tiennent compte de la période d’autonomie complète du $SSC$.

% \begin{figure}
%     \centering
%         \ftodo{Ajouter évolution de la puissance et de l’énergie pour Strasbourg et Bordeaux}
%         % \includegraphics{Ressources/Images/Scenario/puisage.pdf}
%     \caption{Évolution des appels de puissance des appoints électriques et du cumul d’énergie pour
%              une année complète \label{fig:puissance_annuelle_defavorable}}
% \end{figure}
% paragraph période_de_simulation (end)
% subsubsection impact_des_conditions_meteorologiques (end)



% - - - - - - - - - - - - - - - - - - - - - - - - - - - - - - - - - - - - - - -
\subsubsection{Impact des scénarios internes} % (fold)
\label{ssub:impact_des_scenarios_internes}
Dans cette section l’impact des charges internes et de la consigne de température est
investigué. Afin d’éviter la redondance, les différentes variations sont seulement
réalisées pour le climat de Bordeaux et de Strasbourg. L’analyse des résultats montrent
que le comportement du système est similaire pour les deux climats~: l’analyse est donc
illustrée uniquement pour le climat de Strasbourg où les modifications ont un impact plus
net.

Afin de pouvoir comparer de manière efficace les variations paramétriques une
représentation graphique est introduite (\figref{fig:impact_temp_consigne}). Elle permet
de visualiser les paramètres caractéristiques du système combiné~:
\begin{itemize}
    \item $F_{sol}^{ECS}$~: la longueur des barres horizontales
    \item $F_{sol}^{CH}$~: la largeur des barres dont l’étendue de variation quantitative
          est décrite en bas à droite.
\end{itemize}.
Il est aussi possible d’évaluer la production solaire valorisée
($Prod_{sol}^{valorisée}$), définit comme la part solaire transmise à l’eau après
soustraction des pertes au niveau des canalisations~; et la part d’énergie consommée
par l’appoint ($Conso_{app}$). Ces deux indications sont fournies à l’extrémité de chaque
barre horizontale et une légende est disponible en dessous.
Enfin la couleur et l’ordre des barres sont définis à partir des valeurs normalisées
du rendement solaire des capteurs ($\eta_{sol}$) dont l’échelle est disponible en partie droite. Le
$\eta_{sol}$ traduit le rapport entre la $Prod_{sol}^{valorisée}$ et l’énergie
incidente sur les capteurs~: il indique donc la part utile de l’énergie récupérée par rapport à
l’énergie solaire incidente.

L’analyse de la \figref{fig:impact_temp_consigne} permet d’obtenir de nombreuses
informations. Augmenter la température de consigne durant le période
d’occupation ($20$-$20$-$16$) augmente de \SI{8}{\percent} de la $Conso_{app}$
sur le chauffage, mais n’impacte pas le $F_{sol}^{ECS}$. Il est aussi important de noter que le
réduit de nuit permet de réduire sensiblement la $Conso_{app}$ mais le $F_{sol}^{ECS}$ et le
$F_{sol}^{CH}$ restent inchangés. Le $SSC$ fonctionne ainsi aussi bien avec que sans réduit de
nuit et la $Conso_{app}$ est seulement proportionnelle à l’augmentation de la demande en
énergie pour le chauffage. Cependant la performance du $SSC$ est impactée si on considère
une consigne de chauffage constante ($19$-$19$-$19$).

L’impact de la ventilation est plus subtil. Le scénario de ventilation $90-90$
augmente le $F_{sol}^{CH}$ (\SI{\approx +3}{\percent}) mais réduit le $F_{sol}^{ECS}$
(\SI{\approx -1}{\percent}) en comparaison avec le scénario de référence qui considère un
réduit en inoccupation ($90-20$). Un débit minimal réduit en inoccupation ($90-20$) augmente
alors la $Conso_{app}$ pour le chauffage (\SI{\approx +15}{kWh}). Cependant la
$Conso_{app}$ diminue (\SI{\approx -8}{\kilo\watt\hour}) car la part
électrique couvrant la production d’$ECS$ diminue. En effet, lorsque le débit de ventilation
est plus élevé le $SSC$ est capable de fournir plus d’énergie pour le chauffage au
détriment de l’$ECS$ expliquant cet équilibre. Il est aussi intéressant de noter le
rendement des capteurs augmentent si on considère un débit de ventilation constant
($90-90$). Au regard des différentes remarques il semble donc que
favoriser en priorité la production d’$ECS$ permet de mieux valoriser l’énergie solaire.

Enfin il apparaît sans surprise que la diminution des apports internes, réduction par deux
des charges internes électriques (équipements et éclairage), impacte fortement l’ensemble
des indicateurs. En effet les charges internes représente la majorité des consommations
pour des bâtiments dont l’enveloppe est très performante (\figref{fig:besoins_charges_internes}).
Il est donc important de tenir compte des scénarios d’éclairage, de l’utilisation des
équipements et des occupants pour analyser la performance d’un $SSC$ pour des maisons
dites à énergie positive ($NZEB$, Near Zero Energy House).

\begin{figure}
    \centering
    \includegraphics{Ressources/Images/Parametrique/parametric_consigne.pdf}
    \caption{Impact de la température de consigne sur la performance
             du $SSC$ à Strasbourg durant la période de chauffage (01/10 - 30/04)}
    \label{fig:impact_temp_consigne}
\end{figure}

\begin{figure}
    \centering
    \includegraphics{Ressources/Images/Parametrique/besoins_charges_internes.pdf}
    \caption{Comparaison annuelle des charges internes et des besoins du bâtiment
             pour la simulation de référence~: (haut) Bordeaux, (bas) Strasbourg.}
    \label{fig:besoins_charges_internes}
\end{figure}
% subsubsection impact_des_scenarios_internes (end)


% - - - - - - - - - - - - - - - - - - - - - - - - - - - - - - - - - - - - - - -
\subsubsection{Impact des ballons} % (fold)
\label{ssub:impact_des_ballons}
Outre la surface de capteur, le choix du volume des ballons, stockage comme sanitaire,
reste un élément déterminant lors du dimensionnement d’un $SSC$. Cette section discute
l’impact du volume mais aussi de la position de l’échangeur solaire. Pour rappel, la
position de l’échangeur est définie de manière relative afin d’être indépendante des
variations réalisées sur le volume du ballon.

Le climat de Strasbourg est retenue afin d’illustrer les résultats et ceux sur Bordeaux
sont ensuite discutés. Au regard des résultat, augmenter le volume du ballon sanitaire
(\num{300} à \SI{450}{\litre}) améliore les indicateurs principaux (le $F_{sol}^{ECS}$, le
$F_{sol}^{CH}$, et la $Conso_{app}$). Cependant la part électrique pour le chauffage
augmente. Au niveau du ballon de stockage, un volume plus important améliore le $F_{sol}^{CH}$
et réduit la $Conso_{app}$ ainsi que le $F_{sol}^{ECS}$. Dans les deux cas, le $\eta_{sol}$
augmente traduisant une meilleure valorisation de l’énergie incidente disponible.
De plus les températures des ballons ($T3$, $T4$, $T5$) fluctuent plus doucement lorsque le
volume est important. Réduire le volume des deux ballons (\num{300} à \SI{150}{\litre})
impacte négativement les performances du $SSC$ mais l’inverse améliore l’ensemble des
indicateurs. Aussi, le volume du ballon sanitaire importe plus que celui du ballon de
stockage et un volume de \SI{450}{\litre} semble préférable. Pour le ballon de stockage,
un volume compris entre \num{300} de \SI{450}{\litre} est suffisant. La même analyse sur
le climat de Bordeaux traduit un comportement similaire. Cependant, la $Conso_{app}$
varie de manière plus importante.

Il est aussi mis en évidence un élément important lors de l’évaluation d’$SSC$. La
couverture solaire ($F_{sol}$), ne permet pas de clairement évaluer la performance d’un
système solaire. En effet au regard des résultats, faire varier le volume du ballon de
stockage ou sanitaire n’a pas d’effet (\figref{fig:importance_chauffage_ecs}). Il est
nécessaire, comme cette analyse le fait, de considérer séparément le $F_{sol}^{ECS}$ et
$F_{sol}^{CH}$ afin de comprendre que le système favorise dans un cas le chauffage et dans
l’autre la production en $ECS$. Le système est donc capable de s’adapter à
différentes conditions de fonctionnement.

\begin{figure}
    \centering
    \includegraphics{Ressources/Images/Parametrique/importance_chauffage_ecs.pdf}
    \caption{Évolution mensuelle de la répartition des consommations et du taux
             de couverture solaire ($F_{sol}$) à Bordeaux (01/10 - 30/04)}
    \label{fig:importance_chauffage_ecs}
\end{figure}

L’analyse de l’impact de la position de l’échangeur est illustré à travers la
\figref{fig:impact_pos_ech}. Il est observé que la position de l’échangeur impacte
fortement les performances du $SSC$ ($F_{sol}^{CH}$ et $F_{sol}^{ECS}$), en particulier
lorsque le volume du ballon est faible. Pour les deux climats, une position basse
(\num{0.8}) est à favoriser afin d’améliorer le rendement des capteurs et réduire la part
couverte par l’appoint. Il est cependant noté qu’une position plus élevée (\num{1.3})
permet de réduire sensiblement la $Conso_{app}$ sur le chauffage.
Enfin contrairement au volume du ballon dont l’impact est plus significatif sur Bordeaux, le
climat de Strasbourg est plus sensible à la $Ech_{sol}^{pos}$. En effet, la position de l’échangeur
est un facteur influençant directement la $Prod_{sol}^{valorisée}$. Il est donc clair que le gain potentiel
est plus notable lorsque l’ensoleillement est moins important comme à Strasbourg.
Finalement l’impact du matériau utilisé pour l’échangeur a aussi été évalué et sa
composition n’impacte pas les performances du $SSC$.

En résumé, la position de l’échangeur et le volume des ballons sont des facteurs
impactant la performance du $SSC$. Il semble aussi plus important de favoriser
le volume du ballon sanitaire et de positionner en partie basse l’échangeur afin
d’améliorer la $Prod_{sol}^{valorisée}$.

\begin{figure}
    \centering
    \includegraphics{Ressources/Images/Parametrique/parametric_echangeur.pdf}
    \caption{Impact de la position de l’échangeur solaire sur la performance
             du $SSC$ à Strasbourg durant la période de chauffage (01/10 - 30/04).}
    \label{fig:impact_pos_ech}
\end{figure}
% subsubsection impact_des_ballons (end)


% - - - - - - - - - - - - - - - - - - - - - - - - - - - - - - - - - - - - - - -
\subsubsection{Impact du profil de puisage} % (fold)
\label{ssub:impact_du_profil_de_puisage}
Dans cette avant dernière partie, l’impact des différents profils de puisage comme du
volume de puisage est discuté. Puis, la prise en compte de coefficients modulateurs,
mensuels et hebdomadaires est aussi détaillé. Pour rappel, les différents profils ne
modifient pas la quantité puisée mais uniquement sa répartition.

Les résultats obtenues pour Strasbourg montrent que le scénario de puisage n’impacte pas
les différents indicateurs même si le scénario \textbf{Réaliste} améliore sensiblement la
$Prod_{sol}^{valorisée}$ (\figref{fig:impact_profil_puisage}). Les résultats des
simulations faisant varier le volume puisé montre que le $SSC$ est robuste. Il est en
effet capable de fournir plus d’énergie solaire afin de couvrir l’augmentation des besoins
en $ECS$. Si on compare les deux consommations extrêmes (\num{27} et
\SI{40}{\litre/(jour\period pers)}\,(\SI{60}{\celsius})), la $Prod_{sol}^{valorisée}$
augmente de \SI{248}{\kilo\watt\hour} et la consommation de l’appoint de \SI{390}{\kilo\watt\hour}.
Ainsi, le système s’adapte sans problème à différent profils de puisage et le
$F_{sol}^{ECS}$ et le $F_{sol}^{CH}$ restent élevés même lors que les besoins en $ECS$ augmentent.

L’utilisation des coefficients mensuels n’affectent ni le $F_{sol}^{CH}$, ni la $Conso_{app}$
électrique sur le chauffage. Cependant le $F_{sol}^{ECS}$ est fortement impacté, montrant une
nouvelle fois l’importance d’évaluer de manière séparée le chauffage et la production
d’$ECS$. Sur Bordeaux où l’ensoleillement est plus propice, la variation est moindre. Ces
résultats montrent ainsi que les habitudes des occupants (prendre une douche plus longue
ou plus chaude en hiver\dots), doivent être pris en compte afin d’évaluer correctement la
performance du $SSC$. Ils montrent aussi que le choix du profil journalier n’est pas impactant lors
de l’évaluation du $SSC$.

\begin{figure}
    \centering
    \includegraphics{Ressources/Images/Parametrique/parametric_puisage.pdf}
    \caption{Impact du profil de puisage sur la performance
             du $SSC$ à Strasbourg durant la période de chauffage (01/10 - 30/04).}
    \label{fig:impact_profil_puisage}
\end{figure}
% subsubsection impact_du_profil_de_puisage (end)


% - - - - - - - - - - - - - - - - - - - - - - - - - - - - - - - - - - - - - - -
\subsubsection{Autres variations} % (fold)
\label{ssub:autres_variations}
Cette dernière partie discute des variations restantes qui sont plus communes lors
de l’évaluation d’un $SSC$~: le nombre ou la surface des panneaux, leur
orientation et inclinaison, et leur performance au regard des caractéristiques techniques.
Enfin, les variations algorithmiques sont aussi discutées.

L’inclinaison des capteurs jouent un rôle important dans la caractérisation de la
performance d’un système $SSC$ et affectent l’ensemble des indicateurs. La $Conso_{app}$
est ainsi réduite lorsque l’inclinaison des capteurs augmente. Le choix d’une inclinaison
importante est en effet fortement bénéfique. Durant la période hivernale, une inclinaison
importante (\num{45} ou \SI{60}{\degree}) permet d’augmenter l’énergie récupérée par les
capteurs et réduit les risques de surchauffes en périodes estivales. Sans surprise, le
nombre de capteurs impactent fortement la consommation de l’appoint. À Strasbourg, une
variation de surface de capteurs de \SI{14}{\meter\squared} (\num{2} à \num{8} capteurs)
fait varier le $F_{sol}^{CH}$ de \num{11} à \SI{52}{\percent} et la $Conso_{app}$ varie de
\num{2232} à \SI{1133}{\kilo\watt\hour}. L’orientation joue aussi un rôle important sur la
performance du $SSC$. À surface équivalente, des capteurs orientés au sud produisent deux
fois plus que des capteurs à l’est. Les résultats montrent aussi que le capteur plan vitrée du
fabricant Radco (308C\,HP) permet d’obtenir les meilleures performances mais l’écart est sensible. Le
capteur sous-vide (12\,CPC58) apparaît comme le moins performant même pour le climat Strasbourgeois.

Au niveau des variations algorithmiques, les temporisations n’impactent que sensiblement
le $SSC$. Augmenter la $Tempo_{souff}$ (\SI{15}{min}) améliore le $F_{sol}^{ECS}$ mais dégrade
le $F_{sol}^{CH}$. La $Tempo_{batterie}$ elle, n’a pas d’impact notable. Le $DeltaT_{sol}$
impacte le système solaire de manière modéré~: l’augmenter dégrade à la fois le rendement
solaire, le $F_{sol}^{ECS}$, et le $F_{sol}^{CH}$. À l’opposé le réduire permet de valoriser une plus
grande quantité d’énergie solaire. Enfin, autoriser une sur-chauffage par le solaire
durant la journée (consigne solaire à \SI{22}{\celsius}) améliore le $F_{sol}^{CH}$ mais dégrade
de manière plus importante le $F_{sol}^{ECS}$. Ainsi, la $Conso_{app}$ augmente. Ce résultat est
cohérent avec les observations faites lors de l’évaluation des variations sur la
ventilation (\ref{ssub:impact_des_scenarios_internes}) et montrent que l’énergie solaire
est mieux valorisée si elle est attribuée en priorité pour la production d’$ECS$.

En résumé, les capteurs solaires thermiques doivent être placés en priorité sur un pan sud
avec une inclinaison supérieure de \SI{15}{\degree} par rapport à la latitude du lieu afin
d’améliorer les performances du $SSC$ sur l’ensemble des indicateurs. Il apparaît aussi
que le $DeltaT_{sol}$ impacte le système est doit donc être considéré. À l’opposé la surchauffe
diurne impacte négativement le bilan global du système. Cependant, il permet de favoriser le
chauffage ou apporter un confort thermique hivernale plus important sans surcoût.
% subsubsection autres_variations (end)
% subsection analyse_des_resultats (end)
% section etude_parametrique (end)





% ..............................................................................
% ..............................................................................
\section{Vers une méthodologie d’aide à la décision} % (fold)
\label{sec:vers_une_methodologie_d_aide_a_la_decision}
Au cours de ce chapitre, la démarche entreprise a été explicitée. Dans l’optique de mieux
comprendre les interactions entre le bâtiment et le $SSC$, un modèle mixte a été réalisé à
l’aide du langage \textit{Modelica}. Pour le bâtiment, un modèle mono-zone a été retenu
suite à une comparaison avec une approche plus détaillée. Pour le $SSC$, une approche
détaillée a été retenue afin de tenir compte de la complexité de l’algorithme de contrôle.
En effet un algorithme innovant et détaillé a été utilisé afin d’orchestrer le
fonctionnement couplé du bâtiment et du $SSC$. Finalement, une étude paramétrique a permis
de mieux comprendre le comportement du $SSC$.

De nombreux paramètres dont la littérature avait identifié l’impact ont été identifiés dans
ces travaux et les conclusions sont similaires. Cependant d’autres paramètres et d’autres
conclusions ont aussi été obtenues. Ce $SSC$ est robuste, faire varier la température de
consigne, supprimer le réduit de nuit, ou encore modifier la répartition des puisages
journaliers impacte de manière très sensible sa performance. Aussi, il a été mis en exergue que
le $F_{sol}^{ECS}$ et le $F_{sol}^{CH}$ ne sont pas influencés de manière similaire en
fonction des variations techniques. Le système est en effet capable de s’adapter~: dans
certains cas le $F_{sol}^{ECS}$ est favorisé, dans d’autre c’est le $F_{sol}^{CH}$. Ce
comportement est aisément observable lorsque le volume des ballons est modifié et que
$F_{sol}$ demeure intact mais où $F_{sol}^{ECS}$ et $F_{sol}^{CH}$ subissent de fortes
variations. Il est donc indispensable de considérer de manière indépendante la performance
du système solaire sur le chauffage ($F_{sol}^{CH}$) et sur la production d’$ECS$
($F_{sol}^{ECS}$). Il a aussi été mis en évidence que l’énergie solaire est mieux
valorisée lorsque elle est en priorité utilisée afin de monter en température le ballon
sanitaire. Ceci a en effet été noté à plusieurs reprises~: lors de la charge diurne du
bâtiment afin de réduire la $Conso_{app}$, ou bien lors de la variation du débit de
ventilation. Cette étude a donc permise de confronter le modèle aux variations techniques
communes dans l’étude des $SSC$ mais aussi d’apprendre plus à travers d’autres variations
au niveau du bâtiment, du $SSC$ ou de son algorithme. Les résultats tendent à
montrer que le $SSC$ permet d’obtenir une couverture importante sur le chauffage comme la
production $ECS$ lorsque un algorithme détaillé est utilisé. Il est donc propice à
l’utilisation pour des bâtiments à faible consommation et pas uniquement pour couvrir les
besoins en $ECS$ comme les constructions performantes actuelles le suggère.

Fort de ces résultats, trois indicateurs principaux ont pu être identifié. La performance
d’un $SSC$ nécessite l’optimisation de la $Prod_{sol}^{valorisée}$ ainsi que de la
$Conso_{app}$ pour le chauffage et la production d’$ECS$. Uniquement considérer la
$Conso_{app}$ ne permet en effet pas de tenir compte des influences bidirectionnelles
entre le $SSC$ et le bâtiment illustrées à travers l’étude paramétrique. Dans l’optique
d’une aide à la décision, il apparaît donc important de considérer à la fois la
$Conso_{app}$, le $F_{sol}^{ECS}$, et le $F_{sol}^{CH}$. L’approche est donc multi-objectif en plus
d’être multi-critère et une méthode spécialisée est nécessaire. Dans le chapitre suivant,
le choix de la méthode retenue est discutée, et sa mise en place illustrée à travers les
résultats dans le dernier chapitre.
% subsection bilan (end)
% section vers_une_methodologie_d_aide_a_la_decision (end)
