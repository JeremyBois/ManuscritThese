% Chapitres\Chap2-ApprocheModelisation.tex


% ..............................................................................
% ..............................................................................
\section{Modélisation des systèmes} % (fold)
\label{sec:modelisation_des_systemes}

% ------------------------------------------------------------------------------
\subsection{Pourquoi et comment ?} % (fold)
\label{sub:pourquoi_et_comment}
\itodo{Décrire le choix de l’approche par modélisation}
\itodo{Décrire les outils utilisés: Modelica et les bibliothèques, Dymola et les solveurs}
\itodo{Pourquoi une modélisation détaillée d’un système existant}
% subsection pourquoi_et_comment (end)


% ------------------------------------------------------------------------------
\subsection{Un système solaire couplée au vecteur air} % (fold)
\label{sub:description_du_système_vecteur_air}
\itodo{Choix de modélisation pour les systèmes}
\itodo{Description de l’algorithme de fonctionnement}
\itodo{Description des modèles validés}
\itodo{Description schématique du système}
% subsection description_du_système_vecteur_air (end)


% ------------------------------------------------------------------------------
\subsection{Un système solaire couplée au vecteur eau} % (fold)
\label{sub:description_du_système_vecteur_eau}
\itodo{Déscription des systèmes}
% subsection description_du_système_vecteur_eau (end)
% section modelisation_des_systemes (end)




% ..............................................................................
% ..............................................................................
\section{Modélisation du bâtiment} % (fold)
\label{sec:modelisation_du_batiment}


% ------------------------------------------------------------------------------
\subsection{Description de la maison} % (fold)
\label{sub:description_de_la_maison}

% - - - - - - - - - - - - - - - - - - - - - - - - - - - - - - - - - - - - - - -
\subsubsection{Description du site} % (fold)
\label{ssub:modelisation_du_site}
\itodo{Description du site étudié: climat, données metéos, ...}

Afin d’évaluer et d’optimiser la performance d’un système solaire, il est nécessaire
que les conditions extérieures ne soit pas fortement favorable.
Les climats méditerranéens ne sont donc pas des cas d’étude intéressant du fait
de la forte couverture solaire.
À l’opposé le climat de Limoges est assez rude et l’ensoleillement durant la période
hivernale est faible. C’est donc un cas d’étude intéressant.
\itodo{Ajouter une carte pour localiser la ville}
\itodo{Ajouter une description plus complète avec DJU, extrèmes, ...}
\itodo{Refaire complètement ce paragraphe car il fait pitié actuellement}
% subsubsection description_du_site (end)


% - - - - - - - - - - - - - - - - - - - - - - - - - - - - - - - - - - - - - - -
\subsubsection{Description de l’enveloppe} % (fold)
\label{ssub:description_de_l_enveloppe}
\itodo{Description complète de la maison: Composition de base, surfaces, orientation, ...}

La maison fait 100\,\si{m^{2}} est est composée de ...
\itodo{Ajouter du bla bla sur la composition de la maison}
% subsubsection description_de_l_enveloppe (end)

% - - - - - - - - - - - - - - - - - - - - - - - - - - - - - - - - - - - - - - -
\subsubsection{Description des scénarios} % (fold)
\label{ssub:description_des_scenarios}
\itodo{Description des différents scénarios et consignes}

Le cas d’étude utilise les scénarios d’une famille moyenne développés au travers de
la création d’un outil pour évaluer l’impact des occupants sur les besoins
énergétiques du bâtiment (\cite{Vorger2014}).
\itodo{Décrire rapidement les travaux d’Éric}

Pour cette étude différents scénarios ont été pris en compte de manière fixe:
\begin{itemize}
    \item Occupation
    \item Charges internes (faibles par rapport à un scénario RT)
    \item Température de consigne
    \item Débits d’extractions d’air
\end{itemize}
\itodo{Décrire les différents scénarios avec des illustrations}

D’autres font partis intégrante de l’optimisation. En effet ils influencent le
fonctionnement du système et donc sa performance. Les scénarios non-fixes sont donc
les suivants:
\begin{itemize}
    \item Occupation
    \item Profil de puisage ECS
    \item Température de consigne solaire
\end{itemize}
\itodo{Décrire les scénarios variables}
% subsection description_de_la_maison (end)


% ------------------------------------------------------------------------------
\subsection{Approche monozone} % (fold)
\label{sub:approche_monozone}
\itodo{Décrire la raison de ce choix}
\itodo{Introduire la validité à l’échelle annuelle par un modèle détaillé (Aurélie)}
\itodo{Décrire les limitations de cette approches grâce au premiers résultats}
% subsection approche_monozone (end)


% ------------------------------------------------------------------------------
\subsection{Approche multizone} % (fold)
\label{sub:approche_multizone}
\itodo{Décrire les outils nécessaire pour l’approche multi-zone: EnergyPlus, FMU,
       Python, Sketchup, OpenStudio}
\itodo{Introduire les avantages au regard de l’approche multizone}
\itodo{Décrire la mise en place du modèle}
% subsection approche_multizone (end)
% section description_du_batiment (end)




% ..............................................................................
% ..............................................................................
\section{Étude paramétrique} % (fold)
\label{sec:etude_parametrique}


% ------------------------------------------------------------------------------
\subsection{Récapitulatif} % (fold)
\label{sub:récapitulatif}
\itodo{Descriptif des simulations réalisées dans le cas Monozone}
% subsection récapitulatif (end)


% ------------------------------------------------------------------------------
\subsection{Analyse détaillée} % (fold)
\label{sub:analyse_detaillee}
\itodo{Évaluation et analyse des résultats}
% subsection analyse_detaillee (end)


% ------------------------------------------------------------------------------
\subsection{Ce qu’il faut retenir} % (fold)
\label{sub:ce_qu_il_faut_retenir}
\itodo{Conclusion de l’étude}

\itodo{Développement de modèles détaillés}
\itodo{Mise en évidence du découplage de la performance entre chauffage et ECS}
\itodo{Nécessité de comprendre les intéractions}
\itodo{Nécessité d’intégrer une estimation du coût ---> Optimisation}
% subsection ce_qu_il_faut_retenir (end)
% section etude_parametrique (end)

