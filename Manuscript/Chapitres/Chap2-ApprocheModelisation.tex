%!TEX root = ../main.tex
% Chapitres/Chap2-ApprocheModelisation.tex


L’objectif de ces travaux a été explicité à travers le chapitre précédent. Il est proposé
de réaliser un outil d’aide à la décision pour explorer les combinaisons pertinentes pour
la conception de maisons solaires à énergie positive. Il est ainsi nécessaire de
présenter le bâtiment et le potentiel du système solaire combiné retenu. Ce chapitre
introduit ainsi le couplage entre une maison à énergie positive et un système solaire
combiné. Dans un premier temps, le choix de la modélisation est argumenté et les outils
retenus explicités. Dans un second temps, le bâtiment et les scénarios de référence sont
introduits, puis, le système solaire combiné est décrit. Son fonctionnement général et les
divers équipements le constituant sont discutés, puis la logique de contrôle proposée est
détaillée. Finalement une étude paramétrique est réalisée afin d’évaluer le potentiel du
système solaire combiné en fonction de différents scénarios et de différents choix
techniques.
\clearpage


% ..............................................................................
% ..............................................................................
\section{Modélisation détaillée d’un système solaire combiné} % (fold)
\label{sec:modelisatioe_detaillee_d_un_systeme_solaire_combine}
% ------------------------------------------------------------------------------
Il existe principalement deux approches permettant d’évaluer le potentiel d’un bâtiment ou
d’un système~: l’approche par expérimentation et l’approche par modélisation. Cette
première section décrit ainsi dans un premier temps les deux méthodes puis les outils
retenus pour l’évaluation du potentiel sont présentés.

\subsection{Le choix de la modélisation} % (fold)
\label{sub:le_choix_de_la_modelisation}
% - - - - - - - - - - - - - - - - - - - - - - - - - - - - - - - - - - - - - - -
\subsubsection{L’expérimentation} % (fold)
\label{ssub:l_experimentation}
L’approche par expérimentation est la seule manière d’évaluer un système dans des
conditions réelles. Elle est ainsi toujours nécessaire afin de confronter les résultats
obtenus par la simulation.
Le fonctionnement d’un bâtiment et de ses systèmes est en effet un système physique
complexe et instationnaire où de nombreuses incertitudes entrent en jeu~:
\begin{itemize}
    \item Conditions externes~: température, rayonnement solaire\dots)
    \item Conditions internes~: occupants, équipements\dots
    \item Performance de l’enveloppe
    \item Performance des systèmes
\end{itemize}
Contrairement à l’expérimentation, la prise en compte de tous les paramètres dans une
modélisation est impossible et il est nécessaire de faire des simplifications. Par exemple
il peut être admis des scénarios types pour l’occupation, une pose uniforme de l’isolant,
ou encore une équilibrage parfait des réseaux. La simulation permet alors de vérifier que
les hypothèses faites sont satisfaisantes au regard de l’objet de l’étude.

L’expérimentation permet aussi d’observer un phénomène et de réaliser des modifications
successives afin de mieux le comprendre. Il est ainsi possible dans certain cas d’obtenir
une relation permettant de simplifier un problème complexe grâce à l’observation et la
prise de mesure pour différentes conditions de fonctionnement. Ces relations empiriques
sont souvent utilisées dans le bâtiment en particulier lorsque le modèle détaillé
nécessite la définition de trop nombreux paramètres qui ne sont pas toujours disponibles.
Par exemple, le rendement des capteurs dans ces travaux (\ref{eq:quasi_complete}) est
définie suivant une relation empirique. Chaque capteur a une géométrie propre mais grâce à
des mesures expérimentales dans des conditions limites identiques, un comportement
caractéristique a été identifié. La relation empirique trouvée permet alors d’approximer le
rendement de tous les capteurs solaires thermiques sans devoir tenir compte des specifités
de leur géométrie.
% subsubsection l_experimentation (end)


% - - - - - - - - - - - - - - - - - - - - - - - - - - - - - - - - - - - - - - -
\subsubsection{La modélisation} % (fold)
\label{ssub:la_modelisation}
La modélisation est un outil puissant qui dépend cependant intrinsèquement du niveau de
détail retenu pour chaque composant. Bien que possible en théorie, une modélisation ne
permet pas de simuler la complexité inhérente à la réalité comme le souligne parfaitement
\textit{Georges Box}~:
\blockquote{All models are wrong but some are useful}
Une modélisation est toujours une approximation simulant de manière plus ou moins précise
un système dans des conditions imposées. Un modèle est donc représenté par des systèmes
d’équation qui admettent des hypothèses et ne tiennent compte que d’une partie des
phénomènes. Un niveau important d’expertise est ainsi nécessaire afin de déterminer le
niveau de détail des modèles permettant de représenter au mieux le système afin d’éviter
des erreurs numériques ou des résultats erronés. Il est aussi clair que les installations
sont en réalité imparfaites et même dans l’hypothèse d’une modélisation complète de
l’ensemble des phénomènes il restera un écart entre le modèle et la réalité car des
facteurs ne sont pas pris en compte~: les imperfections techniques, les
malfonctionnements, ou les erreurs d’installations. Le système considéré peut en effet
comporter des fuites, des problèmes de raccord, un équipement peut avoir des
caractéristiques différentes que ceux de sa fiche technique\dots\
Une modélisation est donc une représentation plus ou moins précise de la réalité
admettant des conditions limites contrôlables.
% subsubsection la_modelisation (end)


% - - - - - - - - - - - - - - - - - - - - - - - - - - - - - - - - - - - - - - -
\subsubsection{Bilan} % (fold)
\label{ssub:bilan_modelisation}
L’expérimentation comme la modélisation reviennent à interpréter des valeurs d’intérêt à
partir d’informations récoltées sur les conditions de fonctionnement et sur les conditions
limites. Les deux approches sont complémentaires et comportent des avantages et des
inconvénients. L’expérimentation permet d’analyser un système dans des conditions réelles
et peut être utilisée pour confronter les résultats théoriques à la réalité. Elle est
cependant couteuse et complexe à mettre en place. À l’inverse, la modélisation est
contrainte par des systèmes d’équations mais offre une liberté plus importante sur les
conditions de fonctionnement. L’ensemble des paramètres étant contrôlable, la
modélisation offre l’opportunité d’évaluer de multiples configurations. Elle est donc
particulièrement adaptée pour des études cherchant à évaluer les effets de la variation
d’un ou plusieurs paramètres comme les études paramétriques ou l’optimisation~:
l’approche par modélisation est donc le choix le plus pertinent pour répondre à la
problématique de ces travaux.

Cette conclusion nécessite cependant d’être ajustée. Bien que seule la simulation puisse
répondre à ces questions, l’expérimentation est complémentaire et permet de vérifier le
modèle numérique. L’approche la plus correcte serait donc de réaliser un modèle numérique
pour ensuite le valider à travers une approche expérimentale. Une fois validé, le modèle
peut être utilisé pour étudier l’impact de diverses variations. Ces travaux n’intègrent
cependant pas de validation expérimentale mais, comme détaillé ci-après, les modèles
retenus ont fait pour la plupart l’objet de vérifications analytiques.
% subsubsection bilan (end)
% subsection le_choix_de_la_modelisation (end)


% ------------------------------------------------------------------------------
\subsection{Le choix de l’outil de modélisation} % (fold)
\label{sub:le_choix_de_l_outil_de_modelisation}
% - - - - - - - - - - - - - - - - - - - - - - - - - - - - - - - - - - - - - - -
\subsubsection{Les contraintes de modélisation} % (fold)
\label{ssub:les_contraintes_de_modelisation}
Ces travaux cherchent à évaluer la performance d’un \abr{SSC} à l’échelle d’une maison
individuelle en particulier à travers le développement et l’évaluation d’un algorithme de
contrôle détaillé. Il est donc nécessaire de sélectionner un outil permettant un accès
complet aux différents composants du modèle. Ce contrôle est nécessaire pour permettre
l’analyse des interactions induites par des variations au niveau du composant et le
fonctionnement global du système au niveau du bâtiment. De plus, l’algorithme étant en
développement le processus est dans un premier temps fortement itératif. L’outil
sélectionné doit par conséquent permettre de modifier, ajouter, et supprimer rapidement et
simplement des composants. Finalement le problème à modéliser est un problème multi-physique
et l’outil doit être adapté à la modélisation d’un système hydraulique (solaire),
d’un système aéraulique (ventilation), d’un algorithme de contrôle (régulation), le tout
couplé à un modèle thermique (bâtiment).
% subsubsection les_contraintes_de_modelisation (end)


% - - - - - - - - - - - - - - - - - - - - - - - - - - - - - - - - - - - - - - -
\subsubsection{Les outils existants} % (fold)
\label{ssub:les_outils_existants}
Au fil des années de nombreuses bibliothèques ou logiciels ont été développés. Parmi ces
approches certaines cherchent à modéliser de manière très précise la dynamique de
l’échange au niveau de capteurs alors que d’autres se focalisent sur une estimation
grossière des performances. Les deux approches permettent d’évaluer de manière plus ou
moins précise un système solaire pour différents pas de temps.

\paragraph{TRNSYS~:} % (fold)
\label{par:trnssys}
Aujourd’hui en version $18$ \parencite{Klein2017} il est développé par l’université de Madison (\abr{USA}).
Disponible dès $1975$ (version $6$), le logiciel est développé à l’origine pour des applications
solaires et reste aujourd’hui encore le plus utilisé. À partir de $1994$, la version
$14$ regroupe cinq programmes sous le méta-programme \textit{TRNSHELL}~:
\begin{itemize}
    \item TRNSYS~: moteur de calcul et bibliothèque de base
    \item PRESIM~: une interface graphique pour la composition des modèles
    \item ONLINE~: visualisation des résultats durant la simulation
    \item PREBID~: aide à la construction de bâtiment
    \item TRNSED~: un éditeur de texte pour créer des modules
\end{itemize}
De part son approche modulaire, le logiciel peut être étendu simplement afin de répondre aux
nouveaux besoins. Cette approche offre ainsi la possibilité aux utilisateurs
d’ajouter de nouveaux modules, expliquant d’une part sa large bibliothèque, et d’autre part
sa longévité. Finalement, il propose aussi le couplage avec divers logiciels tels que
\textit{MATLAB} ou \textit{GenOpt}. Après $40$ d’existence le logiciel profite aujourd’hui
d’une des plus large bibliothèque et permet de modéliser tout type de procédé thermique ou
électrique pour différentes échelles que ce soit un bâtiment ou un simple contrôleur.
L’$IEA$ l’utilise par exemple à travers la tâche $26$ et a participé à l’amélioration de
sa bibliothèque \parencite{Task26B2002}.
% paragraph trnsys(end)

\paragraph{Simulink~:} % (fold)
\label{par:simulink}
Développé par \textit{MathWorks}, c’est un outil de programmation graphique permettant de
créer des modèles complexes par association de blocs causaux. Il est fortement intégré
avec le logiciel \textit{Matlab} qui s’occupe de résoudre les systèmes d’équations. Cette
intégration permet aussi à  \textit{Simulink} d’accéder aux nombreuses fonctions de
l’environnement \textit{Matlab}.
Développé dans les années $1990$, la bibliothèque couvre
aujourd’hui de nombreux domaines de l’ingénierie. Comme \textit{TrnSys}, il permet de
simuler des systèmes linéaires et non-linéaires. Il peut de plus générer du code source C,
permettant d’embarquer le modèle développé dans un procédé pour un fonctionnement en temps
réel. \textcite{Mosallat2013686} l’utilisent afin de modéliser un système solaire pour le
chauffage d’un bâtiment d’entrepreneuriat social et comparent les résultats obtenus par
l’approche dynamique à ceux d’une approche statique obtenue avec le logiciel
\fnref{www.retscreen.net}{\textit{RETScreen}}. Les résultats montrent que la performance
du système solaire est sur-évaluée avec l’approche statique.
% paragraph simulink_ (end)

\paragraph{EnergyPlus~:} % (fold)
\label{par:energyplus}
Développé par le département de l’énergie des états-unies (\abr{DOE}), c’est un logiciel
\enquote{open source} permettant de simuler la consommation de l’énergie dans les
bâtiments. Comme \textit{TrnSys}, il utilise un pas de temps fixe qui doit représenter
$\nicefrac{1}{n}$ d’heure. Il permet soit de simuler le bâtiment avec un système
idéal, soit de le coupler avec des systèmes en proposant un large choix de solutions
techniques. \textit{EnergyPlus} propose une interface graphique de base même si son
utilisation en particulier pour modéliser des systèmes reste complexe. Pour cette raison
des interfaces plus conviviales ont été développées comme \textit{OpenStudio},
\textit{Design Builder}, ou \textit{Simergy}. Un plugin pour le
logiciel de dessin $3$D \textit{Sketchup} a aussi été développé et permet d’importer
directement dans \textit{EnergyPlus} le bâtiment modélisé.
% paragraph energyplus_ (end)

\paragraph{IDA-ICE~:} % (fold)
\label{par:ida_ice}
Développé par \textit{EQUA Simulation AB}, c’est un logiciel permettant le calcul des
consommations mais aussi du confort intérieur avec un pas de temps variable
\parencite{Kalamees2004}. Dans cette optique il offre une solution tout-en-un avec une
interface et des solutions modernes. Le logiciel comprend un moteur de calcul (\abr{IDA}),
un éditeur $3$D, et un outil de post-traitement permettant d’évaluer les résultats à
travers des courbes classiques ou bien des animations $3$D du bâtiment et des systèmes.
L’interface propose trois niveaux de contrôle afin de couvrir les besoins de différents
types d’utilisateurs. Le bâtiment et ces systèmes peuvent ainsi être modélisés, soit à
travers le suivi d’étapes simples, soit par association de composants, soit en créant de nouveaux
composants grâce à un langage de programmation équationnel intégré.
% paragraph ida_ice (end)

\paragraph{Modelica~:} % (fold)
\label{par:modelica_}
\textit{Modelica} est un langage de programmation libre et ouvert
développé pour répondre aux contraintes de la modélisation multi-physique. Il a été pensé
pour être intuitif en offrant une approche équationnelle et orientée objet au développeur
\parencite{Wetter2016290}.
L’approche objet permet d’encapsuler un ensemble de données dont l’accès est restreint par
une interface publique~: l’$API$ (Application Programming Interface). Un modèle est
ainsi une combinaison d’un ensemble de sous-modèles (composition) où certains sous-modèles
peuvent être améliorés afin de leur ajouter une spécialisation (héritage). Cette
approche permet alors une forte réutilisation des modèles déjà existant tout en
simplifiant l’ajout de nouvelles fonctionnalités ou le couplage des deux. Finalement, le
langage est acausal (\defref{def:acausal}), permettant au modeleur de rapidement
créer des prototypes de systèmes complexes sans devoir à chaque modification réécrire
l’ensemble des systèmes d’équations. Il est cependant toujours possible de définir
explicitement un composant comme étant causal en héritant de la classe \textit{Bloc}.

\begin{Def}[Acausal]\label{def:acausal}
Un modèle est dit acausal lorsque le sens de la définition d’une équation n’est pas
imposé (\figref{fig:acausal_vs_causal}). Le solveur va ainsi déterminer par lui-même les
paramètres dont il connaît la valeur et réécrire les systèmes d’équations afin d’exprimer
les inconnues à partir des valeurs connues. À l’inverse un système causal nécessite
que l’inconnu soit isolé a priori par le programmeur.
\end{Def}

\begin{figure}
    \centering
    \includegraphics[width=0.8\textwidth]{Ressources/Images/Modelisation/composant_vs_bloc.png}
    \caption[Différences entre modélisation acausale et causal]
            {Différences entre modélisation acausale (par composant) et causal (par bloc)
             (\href{http://www.wolfram.com/system-modeler/}{Wolfram}).}
    \label{fig:acausal_vs_causal}
\end{figure}
% paragraph modelica(end)

\paragraph{Méthodes de calculs~:} % (fold)
\label{par:methodes_de_calculs}
Elles permettent d’estimer le potentiel solaire mais ne fournissent pas d’informations sur
la dynamique du système. D’après \textcite{Duffie1980} il existe trois catégories de
méthodologie, issues soit~:
\begin{enumerate}
    \item de corrélations entre caractéristiques des capteurs et
          données météorologiques
    \item de corrélations à partir de nombreuses simulations détaillées
    \item du calcul des journées typiques pour un ciel clair et nuageux
          pondérées en fonction de la couverture nuageuse moyenne du site.
\end{enumerate}
La méthode la plus répandue, \textit{F-CHART}, permet d’estimer la performance
d’un \abr{SSC} (vecteur air ou eau) ou bien d’un système produisant uniquement de
l’\abr{ECS}. la méthode utilise une corrélation établie sur la base de systèmes de
références pour lesquels des variations paramétriques ont été réalisées. La méthode
nécessite la connaissance des conditions extérieures comme la température et de l’ensoleillement
sur une base mensuelle et permet d’obtenir l’indicateur $F_{sol}$. \textcite{Cuadros200796} montre que cette
méthode sur-estime la performance du système solaire pour la production d’\abr{ECS} et
propose une autre méthodologie. La nouvelle approche est plus performante lorsque les
besoins d’\abr{ECS} sont importants en particulier pour des capteurs sous-vide. Cependant
lorsque le puisage journalier est inférieur à \SI{\sim 500}{\litre}, l’écart entre les
différentes méthodes est moindre. Finalement l’ensemble des méthodes évaluées sont
comparées avec un logiciel de simulation dynamique~:\textit{TrnSys}. Il est montré que
toutes les approches sur-estiment le potentiel solaire, en particulier lorsque des
capteurs plan vitrés sont considérés.
% paragraph methodes_de_calculs (end)

\paragraph{Bilan~:} % (fold)
\label{par:bilan_logiciel}
Il existe ainsi de nombreux logiciels, langages, et méthodes, permettant d’évaluer la
performance d’un \abr{SSC}. L’approche retenue cherchant à caractériser de manière
détaillée l’algorithme de contrôle, les méthodes issues de corrélations sont écartées au
profit d’une approche par simulation dynamique. Parmi les choix disponibles, le logiciel
\textit{EnergyPlus} est aussi écarté à cause de sa complexité. Bien qu’il soit relativement aisé
de définir l’enveloppe, la logique de contrôle et l’ajout de système n’est pas intuitive.

Ces travaux nécessitant la réalisation de nombreuses modifications durant la phase de
développement, le choix a été fait de retenir \textit{Modelica} ($3.2.2$) pour son
approche acausale facilitant le processus itératif de construction tout en offrant une
liberté de détail sur chaque composant. \textit{Modelica} n’étant qu’un langage de
modélisation, il est aussi retenu \textit{Dymola} (Dynamics MOdeling LAboratory) et la
bibliothèque \textit{Buildings} décrits ci-après.
% paragraph bilan (end)
% subsubsection les_outils_existants (end)
% subsection le_choix_de_l_outil_de_modelisation (end)



% ------------------------------------------------------------------------------
\subsection{Outils complémentaires} % (fold)
\label{sub:outils_complementaires}
% - - - - - - - - - - - - - - - - - - - - - - - - - - - - - - - - - - - - - - -
\subsubsection{Dymola} % (fold)
\label{ssub:dymola}
\textit{Dymola} est une suite de logiciels développée par \textit{Dassault Systèmes}
permettant de simplifier le développement de modèles numériques grâce au langage
\textit{Modelica}. Il offre une interface graphique permettant de connecter les modèles
entre eux de manière intuitive et un outil de post-traitement afin d’évaluer rapidement
les résultats. Il propose aussi de nombreuses fonctions simplifiant le développement afin
de se focaliser sur le problème physique à modéliser~:
\begin{itemize}
    \item \textit{Dymosim}~: un large choix de solveurs pour intégrer les modèles
    \item Un format texte permettant la coopération, le \enquote{versionnage}, et la construction dynamique.
    \item Un langage de script interne permettant l’automatisation.
    \item Le \enquote{refactoring} permettant de renommer de manière dynamique une variable.
    \item Le \enquote{drag and drop} permettant de composer et connecter rapidement des modèles.
    \item Le support de la parallélisation.
    \item La création de \abr{FMU} (Functional Mock-up Unit).
    \item La génération de modèles autonomes.
    \item \dots
\end{itemize}


% - - - - - - - - - - - - - - - - - - - - - - - - - - - - - - - - - - - - - - -
\subsubsection{Buildings} % (fold)
\label{ssub:buildings}
Le couplage de \textit{Modelica} et de \textit{Dymola} permet le développement itératif
nécessaire à notre problème tout en offrant un contrôle total sur chaque partie du modèle.
Bien que largement utilisé dans l’industrie du transport, le langage \textit{Modelica},
en pleine croissance dans le secteur du bâtiment, est cependant encore peu utilisé. De
nombreuses bibliothèques \enquote{open source} ont été développées dont la liste peut être
trouvée sur le site officiel de l’association
\href{https://www.modelica.org/libraries}{\textit{Modelica}}.

Ces travaux s’appuient sur la bibliothèque Buildings \parencite{Wetter2014253} développée
par le laboratoire national Lawrence Berkeley ($LBNL$, Lawrence Berkeley National
Laboratory) dont le développement est le plus avancé et le plus actif. C’est une
bibliothèque libre et ouverte, développée pour la modélisation des systèmes du bâtiment
(hydraulique, aéraulique, électrique, thermique\dots) et son développement est toujours
actif.
Avec le développement important d’outils autour de \textit{Modelica}, l’\abr{IBPSA}
(International Building Performance Simulation Association) développe une bibilothèque de
composants basiques pour l’ensemble de la communauté de l’énergétique du bâtiment~:
\fnref{https://github.com/ibpsa/modelica-ibpsa}{\textit{modelica-ibpsa}}. La biblothèque
n’a pas vocation à remplacer les différentes initiatives existantes mais à proposer un
socle commun de composants afin de simplifier le développement, et le couplage entre les
différents outils. Initiés à travers les travaux de l’Annexe $60$ \parencite{Wetter2015},
les développements continuent dans le cadre du \fnref{https://ibpsa.github.io/project1/}{projet $1$}.
% subsubsection buildings (end)


% - - - - - - - - - - - - - - - - - - - - - - - - - - - - - - - - - - - - - - -
\subsubsection{DymTK} % (fold)
\label{ssub:dymtk}
Afin de permettre la simulation de nombreux modèles de manière concurente, une
bibliothèque, \textit{DymTK}, est développée à l’aide du langage de programmation
\textit{Python}. En plus de permettre la simulation en parallèle de nombreux modèles, elle
propose une \abr{API} permettant de réaliser des traitements en amont et en aval des
simulations. \textit{DymTK} ajoute une couche d’abstraction à l’utilisateur afin que le
lancement et le traitement des données soit simplifié et est basée sur principalement deux
bibliothèques, \textit{BuildingsPy} et \textit{DyMat}. La bibliothèque
\href{http://simulationresearch.lbl.gov/modelica/buildingspy/}{\textit{Buildingspy}} est
développée par le $LBNL$ et permet d’intéragir avec le langage de script de
\textit{Dymola} pour modifier et simuler des modèles sans l’interface graphique. La
bibliothèque \fnref{https://www.j-raedler.de/projects/DyMat/}{\textit{DyMat}}
permet d’extraire les données des fichiers de sorties de
\textit{DymoSim} qui utilisent une version modifiée du format \textit{.mat}.
% subsubsection dymtk (end)
% subsection outils_complementaires (end)


% ------------------------------------------------------------------------------
\subsection{Bilan} % (fold)
\label{sub:bilan_choix_modelisation}
Afin de pouvoir répondre à la problématique, le choix a été fait de retenir une approche
par modélisation et d’utiliser le language \textit{Modelica} pour simuler le comportement
du \abr{SCC}. D’autres outils complémentaires ont aussi été utilisés ou dévelopés afin
d’automatiser l’évaluation de nombreux modèles de manière concurrente. Les outils étant
définis, la section suivante introduit le cas d’étude.
% subsection bilan_choix_modelisation (end)
% section modelisation_detaillee_d_un_systeme_solaire_combine (end)


% ..............................................................................
% ..............................................................................
\section{Application à la modélisation du cas d’étude} % (fold)
\label{sec:application_a_la_modelisation_du_cas_d_etude}
Cette section introduit le cas d’étude et les choix faits au niveau de la modélisation.
Dans un premier temps le bâtiment et ces scénarios sont présentés puis la pertinence
du modèle retenu est discuté. Ensuite, le \abr{SSC} ainsi que les divers équipements le
composant sont détaillés. Finalement le fonctionnement de la logique de contrôle proposée
sont discutés de manière approfondie.


% ------------------------------------------------------------------------------
\subsection{Modélisation de l’enveloppe} % (fold)
\label{sub:modelisation_de_l_enveloppe}
Le bâtiment est une maison de plain-pied (\figref{fig:plan_maison}) avec une surface habitable
de \SI{98.4}{\meter\squared}. Elle comporte trois chambres, une cuisine/salon, et un local
technique où se trouvent les équipements du système solaire combiné (\abr{SSC}). En plus de
l’espace chauffé, les combles et le vide sanitaire ont été modélisés comme des conditions
limites (\figref{fig:modelisation_maison}). Enfin une fenêtre de toit est ajoutée dans
le salon afin d’améliorer le confort lumineux (\tabref{tab:compo_velux}).

Dans le cadre de ces travaux un couplage fort entre bâtiment, systèmes, et algorithme de
contrôle est envisagé. Dans cette optique, un modèle mono-zone validé à travers une suite
de tests issus de l’\textit{ASHRAE} est retenu pour modéliser le bâtiment
\parencite{Wetter2011,Nouidui2012}. Ce choix est discuté ci-après en particulier en
comparant les besoins obtenus entre cette approche et une approche multi-zonale réalisée
à l’aide d’\textit{Energy Plus}. Ainsi la température dans l’ensemble de la zone chauffée
est considérée comme uniforme.

\begin{figure}
    \centering
    \includegraphics[width=0.8\textwidth]{Ressources/Images/Modelisation/Batiment/plan.png}
    \caption[Plan du bâtiment utilisé à travers ces travaux]
            {Plan du bâtiment utilisé à travers ces travaux.}
    \label{fig:plan_maison}
\end{figure}

\begin{figure}
    \centering
    \includegraphics[width=0.7\textwidth]{Ressources/Images/Modelisation/maison.png}
    \caption[Représentation du bâtiment sous Modelica]
            {Représentation du bâtiment sous Modelica.}
    \label{fig:modelisation_maison}
\end{figure}

Les caractéristiques thermiques des parois opaques sont décrites en fonction de
leur orientation (\tabref{tab:perf_parois_opaques}) et une description détaillée et exhaustive est
disponible en annexe (\tabref{tab:compo_parois}). La composition retenue permet ainsi d’obtenir une
enveloppe très performante.

\begin{table}
\centering
\caption[Description de la performance des parois opaques]
        {Description de la performance des parois opaques.}
\label{tab:perf_parois_opaques}
\begin{tabular}{l *{3}{c} r}
    \toprule
                       & Murs           & Plancher     & Plafond & Unité     \\
    \midrule
    $U$                & \num{0.174}    & \num{0.110}  & \num{0.123}  & \si{\watt\per(\meter\squared\period\kelvin)}\\
    $Surface$          & \num{91.17}    & \num{98.40}  & \num{97.06}  & \si{\meter\squared}\\
    $U \times Surface$ &  \num{15.864}  & \num{10.824} & \num{11.938} & \si{\watt\period\kelvin}\\
    \bottomrule
\end{tabular}
\end{table}



% - - - - - - - - - - - - - - - - - - - - - - - - - - - - - - - - - - - - - - -
\subsubsection{Déperditions à travers les fenêtres} % (fold)
\label{ssub:deperditions_a_travers_les_fenetres}
Les vitrages utilisés sont des vitrages doubles avec une lame d’air ou d’argon et une
couche faiblement émissive est ajoutée pour les vitrages des parois verticales.
Il est considéré pour l’ensemble des vitrages, une composition
similaire (Planilux SGG \SI{4}{mm} - Argon - Planitherm XN \SI{4}{mm}), exception
faite de la fenêtre de toit.

Une attention particulière a été apportée à la description des vitrages afin d’être
cohérente avec les fiches techniques et satisfaire les normes de calculs implicites de la
bibliothèque \textit{Buildings}. En effet, le modèle de vitrage de la bibliothèque
\textit{Buildings} sépare le calcul de la part transmise en conductif et en radiatif~: les
caractéristiques des vitrages doivent alors être renseignées de manière détaillée.
Cependant, les fabricants fournissent seulement les caractéristiques globales~: le
coefficient de transmission thermique $U_{g}$ et le facteur solaire $g$. Le logiciel
\textit{Windows 7.4} a donc été utilisé afin d’obtenir la composition détaillée des vitrages
retenus.

Dans un premier temps, le vitrage est construit à partir de la composition décrite dans sa
fiche technique puis les coefficients caractéristiques sont calculés. Afin de pouvoir
comparer avec les données fabricants, les conditions limites issues des normes européennes
sont implémentées dans le logiciel (\tabref{tab:detail_calcul_fenetre}).
En effet, le choix des conditions limites impactent fortement le résultat
et il est observé une variation moyenne de \SI{20}{\percent} entre les normes européennes
et américaines \parencite{RDH2014}.

\begin{table}
\centering
\caption[Détail du calcul du $U_{g}$ et du $g$ selon \textcite{NFEN673} et \textcite{NFEN410}]
        {Détail du calcul du $U_{g}$ et du $g$ selon \textcite{NFEN673} et \textcite{NFEN410}.}
\label{tab:detail_calcul_fenetre}
\begin{tabular}{l *5{c}}
    \toprule
    & $T_{int}$ & $T_{ext}$            & $h_{c}^{int}$ & $h_{c}^{ext}$                                    & Ensoleillement \\
    \addlinespace[\defaultaddspace]
    & \multicolumn{2}{c}{[\si{\kelvin}]} & \multicolumn{2}{c}{[\si{\watt\per(\meter\squared\period\kelvin)}]} & [\si[per-mode=symbol]{W\per\metre\squared}] \\
    \midrule
    Calcul du $U_{g}$       & \num{20}         & \num{0}       & \num{3.6}   & \num{25}    & -    \\
    Calcul du $SHGC$        & \num{30}         & \num{25}       & \num{3.6}   & \num{25}    & \num{500} \\
    \bottomrule
\end{tabular}
\end{table}

Finalement, la bibliothèque \textit{Buildings} ne permettant pas l’ajout de ponts
thermiques aux vitrages, le choix a été fait de les intégrer dans le coefficient thermique
caractéristique du cadre, $U_{f}$ en considérant les coefficients de transmission
thermique linéique ($\psi$ en \si{\watt\per(\metre\period\kelvin)}) suivants~:
\begin{itemize}
    \item $\psi_{mur / appui fenêtre} = \num{0.07}$
    \item $\psi_{mur / seuil porte fenêtre} = \num{0.14}$
\end{itemize}
Une description détaillée et exhaustive de la composition des fenêtres est
disponible en annexe (\tabref{tab:compo_vitrage}, \tabref{tab:compo_velux},
et \tabref{tab:compo_gaz}).
% subsubsection deperditions_a_travers_les_fenetres (end)


% - - - - - - - - - - - - - - - - - - - - - - - - - - - - - - - - - - - - - - -
\subsubsection{Description des infiltrations} % (fold)
\label{ssub:description_des_infiltrations}
Les infiltrations ont été définies en utilisant une perméabilité de
\SI{0.4}{m^{3}\per(\hour\period\meter\squared)}. Le calcul tient compte de la
somme des surfaces dites froides correspondant à la somme des surfaces donnant
vers l’extérieur à l’exception du plancher bas \eqref{eq:infiltrations}.
\begin{equation}
    \begin{aligned}
    Infiltrations &= \num{0.4} \times (Parois_{verticales} + Parois_{velux} + Plafond)\\
    &              \backsimeq \SI{85}{m^{3}/h}
    \label{eq:infiltrations}
    \end{aligned}
\end{equation}
% subsubsection description_des_infiltrations (end)


% - - - - - - - - - - - - - - - - - - - - - - - - - - - - - - - - - - - - - - -
\subsubsection{Limitations du modèle} % (fold)
\label{ssub:limitations_du_modele}
Certains choix ont été faits en raison des limitations de modélisation~:
\begin{itemize}
    \item Le coefficient intérieur d’échanges convectifs est de \SI{7.7}{\watt\per(\meter\squared\period\kelvin)}
          pour toutes les parois et le coefficient extérieur
          est fixé à \SI{25}{\watt\per(\meter\squared\period\kelvin)} (\textcite{NFENISO6946}).
    \item Les ponts thermiques ont été définis comme une surface de déperdition
          équivalente à \SI{13.2}{\meter\squared} en considérant les coefficients de transmission
          thermique linéique ($\psi$ en \si{\watt\per(\metre\period\kelvin)}) suivants~:
          \begin{itemize}
              \item $\psi_{mur / plancher} = \num{0.14}$
              \item $\psi_{mur / mur} = \num{0.13}$
          \end{itemize}
    \item La fenêtre de toit est placée horizontalement et non à \SI{33}{\percent}
          (inclinaison réelle) car la bibliothèque \textit{Buildings} ne supporte pas les
          surfaces vitrées obliques. Le facteur solaire utilisé est cependant celui défini
          pour \SI{33}{\percent} dans sa fiche technique.
\end{itemize}
% subsubsection limitations_du_modele (end)


% - - - - - - - - - - - - - - - - - - - - - - - - - - - - - - - - - - - - - - -
\subsubsection{Comparaison avec un modèle multi-zone} % (fold)
\label{ssub:comparaison_avec_un_modele_multi_zone}
Une comparaison a été réalisée entre un modèle multi-zone développé sous \textit{Energy Plus}
par le \textit{CEA} dans le cadre du proje \abr{COMEPOS} et le modèle mono-zone
retenu dans ces travaux. Le fichier météo utilisé est celui de Bordeaux
(\href{https://www.energyplus.net/weather-download/europe_wmo_region_6/FRA//FRA_Bordeaux.075100_IWEC/all}{IWEC - WMO 075100}).
La consigne de chauffage est de \SI{19}{\celsius} en occupation et de \SI{16}{\celsius} en
inoccupation. L’étude ne portant pas sur l’évaluation du confort estival, seul les
besoins en chauffage ont été comparés. Les résultats pour l’approche mono-zone et de l’approche
multi-zone sont similaires~: respectivement \SI{1153}{\kilo\watt\hour} et
\SI{1189}{\kilo\watt\hour} pour une année complète (\figref{fig:compare_models}).
Il peut être noté que le bâtiment modélisé avec \textit{Modelica} se comporte de
manière similaire au bâtiment modélisé avec \textit{Energy Plus}. Il n’est cependant
pas possible de comparer directement les puissances nécessaires car les deux logiciels
ne considèrent pas les conditions initiales ni le même pas de temps. En effet le solveur
utilisé dans \textit{Dymola} utilise un pas de temps variable là ou \textit{Energy Plus}
utilise un pas de temps fixe. Pour cette raison les appels de puissance sont en moyenne
plus importants sur le modèle réalisé avec \textit{Energy Plus} alors qu’il sont plus
lissés lorsque le modèle est évalué avec \textit{Dymola}. Le choix d’un bâtiment est
donc pertinent dans l’optique de ces travaux et les différents scénarios retenus
sont discutés dans la partie qui suit.

\begin{figure}
    \centering
    \includegraphics[width=\textwidth]{Ressources/Images/Modelisation/Batiment/compare_power.pdf}
    \caption[Évolution des besoins de chauffage entre le modèle mono-zone et multi-zone]
             {Évolution des besoins en chauffage (puissance appelée (haut) et énergie (bas)) entre le modèle mono-zone
              (\textit{Modelica}) et le modèle multi-zone (\textit{EnergyPlus}).}
    \label{fig:compare_models}
\end{figure}
% subsubsection comparaison_avec_un_modele_multi_zone (end)
% subsection modelisation_de_l_enveloppe (end)


% % ------------------------------------------------------------------------------
\subsection{Scénarios de référence} % (fold)
\label{sub:scenarios_de_reference}
% - - - - - - - - - - - - - - - - - - - - - - - - - - - - - - - - - - - - - - -
\subsubsection{Occupation} % (fold)
\label{ssub:profil_d_occupation}
Un profil unique pour l’ensemble de la zone et pour chaque occupant est retenu
(\figref{fig:scenario_reference}). Il est considéré quatre personnes, deux enfants, et
deux parents. Dans ce profil, les occupants sont considérés à la maison durant le week-end
et absent durant les jours ouvrés (travail) exception faite du mercredi après-midi. La
puissance par habitant est de \SI{97.5}{\watt} (\SI{70}{\percent} convective), résultat de
la moyenne pondérée de la puissance dégagée par une personne en fonction de la surface de
chaque pièce dans le modèle multi-zone avec \SI{78}{\watt} pour un enfant et \SI{117}{\watt}
pour un adulte(\tabref{tab:puissance_occupants}).

\begin{table}
\centering
\caption[Récapitulatif des puissances dissipées en fonction des pièces]
        {Récapitulatif des puissances dissipées en fonction des pièces.}
\label{tab:puissance_occupants}
\begin{tabular}{*8{c}}
    \toprule
    Chambre 1 & Chambre 2  & Chambre 3 & Séjour     & Cuisine    & Sanitaire   & SdB         & Cellier     \\
    \midrule
    \num{78}  & \num{78}   & \num{117} & \num{97.5} & \num{97.5} & \num{114.3} & \num{114.3} & \num{114.3} \\
    \bottomrule
\end{tabular}
\end{table}
% subsubsection profil_d_occupation (end)


% - - - - - - - - - - - - - - - - - - - - - - - - - - - - - - - - - - - - - - -
\subsubsection{Charges internes} % (fold)
\label{ssub:charges_internes}
En plus des occupants, les charges internes dues aux équipements sont prises en compte. Il
est entendu comme charges internes, les consommations des équipements électriques
(électroménager, ordinateurs\dots) et la consommation de l’éclairage. Aucune information
n’étant disponible afin d’estimer les consommations réelles de la maison, les
consommations réglementaires, \textit{RT\,2012} \parencite{CSTB2011} ont été retenues.
Dans les deux cas il est considéré une consommation type durant l’occupation
et réduite en inoccupation.

\begin{blockdescription}{Équipements électriques}
    \item[Équipements électriques~:] La consommation est fixée à \SI{5.7}{\watt\per m^{2}} (\SI{80}{\percent}
                                      convective) durant l’occupation et à \SI{1.14}{\watt\per m^{2}} en inoccupation, soit une
                                      réduction de \SI{80}{\percent} (\figref{fig:scenario_reference}).
    \item[Éclairage~:] La consommation est fixée à \SI{1.4}{\watt\per m^{2}} (\SI{42}{\percent} convective) de
                       \SI{7}{\hour} à \SI{9}{\hour} et de \SI{19}{\hour} à \SI{22}{\hour}
                       (\figref{fig:scenario_reference}). Durant le week-end, la consommation est réduite de
                       \SI{50}{\percent}, soit \SI{0.7}{\watt\per m^{2}} de \SI{10}{\hour} à \SI{19}{\hour}.
                       Finalement, en inoccupation la consommation est considérée comme nulle.
\end{blockdescription}
% subsubsection charges_internes (end)


% - - - - - - - - - - - - - - - - - - - - - - - - - - - - - - - - - - - - - - -
\subsubsection{Consigne de chauffage} % (fold)
\label{ssub:consigne_de_chauffage}
Le profil de température de référence (\figref{fig:scenario_reference}) est issu de la
réglementation thermique (RT\,2012) qui prévoit le maintien d’une température de
\SI{19}{\celsius} en occupation, et un réduit de \SI{16}{\celsius} en inoccupation. De plus,
elle autorise un abaissement de la consigne durant la période nocturne~: un réduit à \SI{18}{\celsius} est
retenu.
% subsubsection consigne_de_chauffage (end)

% - - - - - - - - - - - - - - - - - - - - - - - - - - - - - - - - - - - - - - -
\subsubsection{Ventilation} % (fold)
\label{ssub:ventilation_ref}
Finalement le profil de ventilation de référence est considéré à \SI[per-mode=symbol]{90}{\meter\cubed\per\hour}
comme le prévoit l’arrêté du \href{https://www.legifrance.gouv.fr/affichTexte.do?cidTexte=JORFTEXT000000862344}{24 Mars
1982 et du 28 Octobre 1983} pour une maison comportant quatre pièces principales. Il est aussi
considéré une réduction du débit d’air neuf minimal à \SI[per-mode=symbol]{20}{\meter\cubed\per\hour}
en inoccupation.
% subsubsection ventilation (end)

\begin{figure}
    \centering
    \includegraphics[width=0.8\textwidth]{Ressources/Images/Modelisation/Scenario/charges_internes.pdf}
    \caption[Profil de référence pour les charges internes et la consigne de chauffage]
            {Profil de référence pour les charges internes (occupants, équipements et éclairage)
             et la consigne de chauffage.}
    \label{fig:scenario_reference}
\end{figure}
% subsection scenarios_de_reference (end)


% ------------------------------------------------------------------------------
\subsection{Description du système solaire} % (fold)
\label{sub:description_du_systeme_solaire}
Dans un premier temps un modèle existant et innovant basé sur le vecteur eau
de l’entreprise \textit{SolisArt} a été modélisé. Les travaux réalisés sur son
développement sont en partie décrit dans \textcite{Bois2015}. La maison considérée dans
l’étude est du niveau \abr{RT}\,$2005$ et le système modélisé considère des radiateurs
pour émetteurs. Les résultats obtenus montrent clairement le potentiel d’un \abr{SSC}
pour la maison individuelle, indépendamment du climat considéré. Ces travaux
ont servi de base de travail pour le développement du système de ce document et
s’inscrivent dans le cadre du projet \abr{COMEPOS} (\ref{sub:les_initiatives_francaises})
en partenariat avec l’entreprise \abr{IGC}, un constructeur de maisons individuelles
personnalisables implémenté dans \num{12} départements français.

Le système décrit dans ce chapitre est un \abr{SSC} utilisant l’air comme vecteur de chaleur
afin d’éviter l’installation d’un système de chauffage conventionnel (radiateurs, plancher
chauffant\dots). Le système permet donc, par l’intermédiaire du réseau de ventilation, de
couvrir les besoins chauffage (\abr{CH}) pour l’ensemble du bâtiment.
Le vecteur air est retenu afin d’améliorer la réactivité du bâtiment, et donc le
ressenti de confort des occupants. Le système modélisé est composé de trois
parties principales~: la partie hydraulique, la partie aéraulique, et l’algorithme de
contrôle qui orchestre le fonctionnement couplé entre le système solaire et le bâtiment
(\figref{fig:air_complet_mono}). Le modèle correspondant réalisé sous \textit{Dymola} est disponible
en annexe (\figref{fig:dymola_ssc_batiment}).
De plus afin de permettre la reproductibilité des résultats les différents
modèles retenus pour chaque composant du système sont listés et les modèles développés
décrits en annexe (\ref{modele_solaires_list}).

Dans un premier temps le fonctionnement de la partie hydraulique sera détaillé. Dans un
second temps la partie aéraulique sera présentée. Finalement la logique de contrôle sera
décrite et le fonctionnement combiné du système et de la maison explicité.

\begin{figure}
    \centering
    \includegraphics[width=\textwidth]{Ressources/Images/Modelisation/Principe/air_complet_mono.pdf}
    \caption[escription schématique du système solaire couplé à la ventilation]
            {Description schématique du système solaire couplé à la ventilation.}
    \label{fig:air_complet_mono}
\end{figure}

% - - - - - - - - - - - - - - - - - - - - - - - - - - - - - - - - - - - - - - -
\subsubsection{Partie hydraulique} % (fold)
\label{ssub:partie_hyraulique}
Le système hydraulique (\figref{fig:air_complet_mono}) est composé de deux ballons dont les
caractéristiques sont disponibles dans le \tabref{tab:tanks_specs}. Le premier, le ballon
sanitaire, permet de lisser la demande en énergie nécessaire pour couvrir les besoins en
\abr{ECS}. Le système est alors dit à semi accumulation car la réserve d’eau chaude est inférieure
à la quantité totale puisée durant la journée. Le ballon sanitaire est connecté en
continu avec le réseau d’eau froide public. Il est donc nécessaire, soit de maintenir
à une température minimale de \SI{55}{\celsius} l’eau du ballon, soit de réaliser une surchauffe journalière
(\href{https://www.legifrance.gouv.fr/affichTexte.do?cidTexte=JORFTEXT000000423756}{Arrêté
du 30 novembre 2005}). La première option a été retenue afin de garantir le respect de la
réglementation dans les limites techniques imposées par la modélisation. Le second, le
ballon de stockage, est utilisé pour stocker l’énergie accumulée durant la journée afin de
la valoriser en période nocturne. Il est retenu deux ballons afin de pouvoir travailler
à des températures réduites sur le chauffage dans l’optique de valoriser au maximum
l’énergie solaire récupérée.
Dans les deux cas, le volume du ballon est discrétisé en
\num{20} sous-volumes afin de tenir compte de la stratification. Cette stratification est
particulièrement importante pour le ballon sanitaire où l’eau froide en partie basse, et
l’eau à \SI{55}{\celsius} en partie haute, imposent un différentiel de température
important. Enfin, les deux ballons étant dans la partie chauffée de la maison, leurs
déperditions sont considérées comme des charges internes au bâtiment.

Afin de garantir le maintien du confort thermique des occupants, un appoint électrique en
partie haute du ballon sanitaire est aussi ajouté. Ce dernier n’est activé que lorsque
l’énergie solaire disponible n’est pas suffisante. L’énergie solaire est elle transmise à
l’eau par l’intermédiaire de capteurs solaires thermiques (\tabref{tab:idmk_specs}) puis vers le
système aéraulique à l’aide d’un échangeur de chaleur. Finalement, le système comporte trois
pompes~: $S6$, $S5$, et $S2$. Un débit nominal de \SI{40}{\litre\per(\hour\period\meter\squared)}
(capteur) pour les pompes est considéré comme première approximation
\parencite{Peuser2005}. Les pompes utilisées sont des pompes à vitesse variable afin de
limiter les arrêts intempestifs \parencite{Kicsiny20123489} et permettre de maintenir
constant un différentiel de température comme décrit dans la suite du document.


\begin{table}
\centering
\caption[Caractéristiques techniques des ballons de référence (tampon et sanitaire)]
        {Caractéristiques techniques des ballons de référence (tampon et sanitaire).}
\label{tab:tanks_specs}
\begin{tabular}{l*{2}{c}r}
    \toprule
    Paramètre & Ballon tampon & Ballon sanitaire & Unité\\
    \midrule
    Volume                                       & \num{300}   & \num{300}    & \si{\litre}              \\
    Hauteur                                      & \num{1.05}  & \num{1.25}   & \si{\metre}              \\
    Épaisseur isolation                          & \num{100}   & \num{55}     & \si{\milli\metre}             \\
    $\lambda$ isolant                            & \num{0.04}  & \num{0.04}   & \si{W/m^{2}\period K}      \\
    Échangeur haut                               & \num{0.85}  & \num{0.64}   & \si{\metre}              \\
    Échangeur bas                                & \num{0.15}  & \num{0.13}   & \si{\metre}              \\
    Diamètre échangeur (extérieur)               & \num{34.6}  & \num{33.7}   & \si{\milli\metre}             \\
    Chaleur spécifique de échangeur (acier noir) & \num{490}   & \num{490}    & \si{J/kg\period K}         \\
    Puissance nominale                           & \num{103}   & \num{53}     & \si{\kilo\watt}             \\
    Température nominale (ballon)                & \num{10}    & \num{45}     & \si{\celsius} \\
    Température nominale (échangeur)             & \num{45}    & \num{10}     & \si{\celsius} \\
    Débit nominal                                & \num{0.36}  & \num{0.366}  & \si{kg\per\second}           \\
    \bottomrule
\end{tabular}
\end{table}

\begin{table}
\centering
\caption[Caractéristiques du collecteur de référence (\textit{Sonnenkraft IDMK\,25-AL}]
        {Caractéristiques du collecteur de référence (\textit{Sonnenkraft IDMK\,25-AL})
         d’après \figref{fig:caracs_idmk}.}
\label{tab:idmk_specs}
\begin{tabular}{lcr}
    \toprule
    Paramètre                                   & Valeur         & Unité                 \\
    \midrule
    Surface d’entrée                            & \num{2.33}           & \si{m^{2}}            \\
    Poids à vide                                & \num{49}             & \si{kg}               \\
    Contenance                                  & \num{1.35}           & \si{l}                \\
    Rendement optique ($\eta_{0}$)              & \num{76.5}           & \si{\percent}               \\
    Coefficient de pertes linéiques ($a_{1}$)   & \num{3.951}          & \si{W/(m^{2}\period K)}      \\
    Coefficient de pertes surfaciques ($a_{2}$) & \num{0.011}          & \si{W/(m^{2}\period K^{2})}  \\
    Modulation du direct ($b_{0}$)              & \num{-0.1396}        & \si{-}               \\
    Modulation du direct ($b_{1}$)              & \num{-0.0004}        & \si{-}               \\
    Modulation du diffus ($K_{\theta,\, dif}$)  & \num{92}             & \si{\percent}               \\
    \bottomrule
\end{tabular}
\end{table}


% - - - - - - - - - - - - - - - - - - - - - - - - - - - - - - - - - - - - - - -
\subsubsection{Validation du modèle de capteurs solaires thermiques} % (fold)
\label{ssub:validation_du_modele_de_capteurs_solaires_thermiques}
Les capteurs solaires thermiques sont modélisés par l’équation \enquote{quasi-dynamique}
implémentée dans le modèle \textit{Fluid.SolarCollectors.EN12975} de la bibliothèque \textit{Buildings}.
Alors que la norme considère un unique modificateur d’\abr{IAM} ($b_{0}$)
le modèle \textit{Fluid.SolarCollectors.EN12975} considère deux modificateurs ($b_{0}$ et $b_{1}$),
permettant de mieux approcher le comportement réel du capteur \eqref{eq:iam_dir_plan_model}.

\begin{equation}\label{eq:iam_dir_plan_model}
    K_{\theta,\,dir} (\theta) = 1 + b_{0} \times \left(\frac{1}{\cos(\theta)} - 1\right)
                                  + b_{1} \times \left(\frac{1}{\cos(\theta)} - 1\right)^{2}
\end{equation}

Ces modificateurs ont été définis par régression linéaire (\figref{fig:IAM_idmk}) à partir des données
expérimentales disponibles dans le certificat du capteur (\figref{fig:caracs_idmk} en annexe)
pour des angles d’incidences allant de \SIrange{0}{70}{\degree}.

\begin{figure}
    \centering
    \includegraphics[width=0.7\textwidth]{Ressources/Images/Modelisation/Capteurs/IDMK/IAM.pdf}
    \caption[\abr{IAM} pour le rayonnement direct]
             {Évolution de $K_{\theta,\,dir}$ en fonction de l’angle d’incidence.
              Les croix représentent les points expérimentaux, et la courbe le résultat
              de la régression linéaire.}
    \label{fig:IAM_idmk}
\end{figure}

Afin de vérifier la pertinence du modèle des données expérimentales fournies par
l’entreprise \textit{SolisArt} ont été utilisées. Pour ce faire la température, le débit
de l’eau en entrée des collecteurs, et les irradiations directe et indirecte ont été
utilisées comme conditions limites pour le modèle~: la production des capteurs a ainsi pu
être comparée. Les résultats montrent une production similaire sur les trois mois couverts
par les données expérimentales (\figref{fig:compare_capteurs}, gauche). De plus, les
parts respectives $I_{dif,\,inc}$ et $I_{dir,\,inc}$ calculées par le modèle
\textit{Fluid.SolarCollectors.EN12975} sont aussi comparés avec les résultats obtenus avec
le logiciel \textit{TrnSys}. Les deux modèles estiment de manière similaire l’irradiation
sur une surface inclinée (\figref{fig:compare_capteurs}, droite). Au regard des résultats,
le niveau de détail du capteur solaire semble suffisant et peut donc être utilisé dans le
reste du document.

\begin{figure}
    \centering
    \includegraphics[width=0.8\textwidth]{Ressources/Images/Modelisation/compare_capteurs.pdf}
    \caption[Comparaisons de l’irradiation entre \textit{TrnSys}, \textit{Modelica} et des résultats expérimentaux]
             {Irradiation sur les capteurs calculé avec \textit{TrnSys} et \textit{Modelica} (gauche).
             Production des capteurs pour des résultats expérimentaux et le modèle \textit{Modelica} (droite).}
    \label{fig:compare_capteurs}
\end{figure}
\FloatBarrier
% subsubsection validation_du_modele_de_capteurs_solaires_thermiques (end)
% subsubsection partie_hyraulique (end)


% - - - - - - - - - - - - - - - - - - - - - - - - - - - - - - - - - - - - - - -
\subsubsection{Partie aéraulique} % (fold)
\label{ssub:partie_aeraulique}
La partie aéraulique du système est responsable du renouvellement d’air et du chauffage
par l’intermédiaire du solaire thermique ou bien de la batterie électrique en
position terminale. Un caisson de mélange permet de récupérer une partie de l’énergie
sur l’air extrait (\figref{fig:air_complet_mono}). L’air soufflé est un mélange entre
l’air neuf et l’air repris dans le respect de la réglementation en vigueur (\ref{ssub:ventilation}).
Le système de ventilation peut ainsi être apparenté à une ventilation mécanique par insufflation
(\abr{VMI}).


% - - - - - - - - - - - - - - - - - - - - - - - - - - - - - - - - - - - - - - -
\subsubsection{Modèle utilisés~:} % (fold)
\label{ssub:modele_utilises}

% subsubsection modele_utilises (end)
% subsection description_du_systeme_solaire (end)


% ------------------------------------------------------------------------------
\subsection{Logique de contrôle} % (fold)
\label{sub:logique_de_controle}
% - - - - - - - - - - - - - - - - - - - - - - - - - - - - - - - - - - - - - - -
\subsubsection{Fonctionnement global} % (fold)
\label{ssub:fonctionnement_global}
La partie hydraulique est contrôlée par un ensemble hiérarchisé de contrôleurs. Le niveau
le plus élevé de contrôle a pour rôle l’activation ou la désactivation des différents
éléments comme les pompes ou les vannes. Au plus bas niveau chaque pompe et chaque
équipement électrique utilise un contrôleur Proportionnel, Intégral, Dérivé (\abr{PID}). Cette
approche permet d’adapter dynamiquement le comportement du système tout en conservant une
stricte séparation entre la logique de chaque composant. Le système solaire fonctionne
ainsi suivant trois principaux modes ordonnés afin de valoriser l’énergie solaire captée.
(\figref{fig:schema_modes}). La représentation au niveau macro de l’ensemble de l’algorithme sous
\textit{Dymola} est disponible en annexe (\figref{fig:dymola_ssc_controle}).

\begin{figure}[tb]
    \centering
    \includegraphics[width=0.9\textwidth]{Ressources/Images/Modelisation/Principe/air_modes.pdf}
    \caption[Description schématique des différents modes de fonctionnements du \abr{SSC}]
            {Description schématique des différents modes de fonctionnements du \abr{SSC}. Système
             global (a), fonctionnement $Indirect$ (b), fonctionnement $Direct$ avec besoin de
             chauffage (c), et, fonctionnement $Direct$ sans besoin de chauffage (d).}
    \label{fig:schema_modes}
\end{figure}

La priorité revient respectivement au maintien du ballon sanitaire, au respect de la température de
consigne, puis à l’élévation de la température dans le ballon de stockage. Le respect de
ces priorité n’est cependant pas exclusif. En effet, lorsque l’énergie solaire disponible est
suffisante, le chauffage, l’\abr{ECS}, et la charge du ballon tampon peuvent être actifs de
manière simultanée.
Durant la période diurne, l’énergie solaire est récupérée directement au niveau des
capteurs solaires (mode $Direct$) et la vanne trois voies (\abr{V3V}) est ouverte.
L’activation des pompes $S5$, $S2$, et $S6$ permet alors respectivement de charger le
ballon tampon, couvrir les besoins en \abr{ECS}, et les besoins en chauffage. Le système
ajuste alors la vitesse des pompes afin de maintenir une différence de température minimale
($\Delta T_{sol}$) de \SI{10}{\celsius} entre la sortie des capteurs ($T1$) et la
température soit des ballons ($T3$, $T5$), soit de l’eau en sortie de l’échangeur eau/air ($T7$).
La différence de
température entre l’entrée et la sortie du collecteur n’est pas utilisée comme un élément
de régulation car cette approche ne permet pas de tenir compte de manière dynamique
des fluctuations entre besoins, pertes, et énergie disponible \parencite{Mosallat2013686}.
L’approche retenue tient compte des pertes en ligne mais aussi des
fluctuations de la part d’énergie fournie aux ballons ou à l’air. Ainsi dans le cas où
l’énergie solaire est importante, il est possible de la distribuer entre les différentes
cibles. Dans le cas où l’énergie solaire est limitée la différence minimale de température
assure de valoriser cette énergie en contrôlant dynamiquement le nombre de pompes activées.

En dehors des heures d’ensoleillement, le système utilise l’énergie solaire stockée dans
le ballon tampon pour couvrir les besoins de chauffage (mode $Indirect$). Ainsi le sens de
circulation du fluide dans l’échangeur du ballon tampon est déterminé par la position de
la \abr{V3V}. Dans le mode $Direct$, la \abr{V3V} est ouverte vers les capteurs permettant à la
pompe $S6$ de s’activer~: l’eau circule du haut vers le bas du ballon. Dans le mode
$Indirect$ la \abr{V3V} bascule vers le ballon tampon (fermée coté capteurs)~: l’eau circule
du bas vers le haut (inverse) et la pompe $S6$ ne peut donc pas être active.
Dans le cas où l’énergie du ballon sanitaire n’est pas suffisante (température inférieure
à \SI{55}{\celsius}), l’appoint électrique est activé afin de garantir la température
minimale réglementaire. De même, pour le chauffage, une batterie électrique en partie
terminale du réseau aéraulique s’active si la consigne de soufflage n’est pas respectée.
Il est important de noter que les besoins en énergie peuvent être couverts simultanément
par le solaire et par l’appoint électrique si l’énergie solaire disponible n’est pas
suffisante.

Afin d’éviter les instabilités, deux formes d’hystérésis sont utilisés. Certains admettent une borne
min et max fixe (\figref{fig:hysteresis}) comme pour le maintien de la température
d’\abr{ECS} au niveau du ballon sanitaire. La température doit être maintenue à \SI{55}{\celsius}
mais une variation de \SI{5}{\celsius} est autorisée. Lorsque la température descend en dessous de
\SI{55}{\celsius} la demande en énergie est de nouveau active. Lorsque le ballon atteint une
température supérieure à \SI{60}{\celsius} la demande est de nouveau inactive.
L’intervalle de température (\num{55} - \num{60}) représente alors une zone morte assurant
une meilleure stabilité au système. Le même fonctionnement est admis pour les pompes
afin d’éviter une activation trop rapide.
Par exemple, si la température dans le ballon tampon ($T5$) est inférieure à \SI{40}{\celsius},
alors le chauffage solaire $Indirect$ ne peut pas être activé. Aussi, bien que le système
puisse remplir plusieurs fonctions concurrentes, certaines contraintes permettent de
garantir l’ordre des priorités. L’activation de la charge du ballon de stockage est ainsi
seulement autorisée si la \abr{V3V} est ouverte vers les capteurs solaires et si le ballon
sanitaire en partie basse (position de l’échangeur solaire) a atteint au minimum
\SI{30}{\celsius} ($T3$). Cette règle permet de garantir que la couverture des besoins en
\abr{ECS} est prioritaire. Comme mis en avant dans la littérature, la sonde est positionnée
à $\nicefrac{1}{6}$ de la position haute de l’échangeur solaire afin de valoriser
l’énergie solaire sans risquer de refroidir l’eau du ballon.
L’autre type d’hystérésis nécessite le retour d’une sonde de référence et les bornes
min et max sont alors définies relativement à cette référence (\figref{fig:onOff}). Par exemple, la consigne
de chauffage étant variable l’hystérésis est définie relativement à la référence
qui est la température de consigne. De même
le maintien du différentiel de température ($\Delta T_{sol}$) est fonction à la fois
de la température de sortie des capteurs ($T1$) ou du ballon tampon ($T5$), et des
températures en partie basse du ballon sanitaire ($T3$) ou de la sortie de l’échangeur
$T7$. Ainsi il est aussi admis un hystérésis relatif à ce différentiel.

\begin{figure}
    \centering
    \begin{subfigure}[b]{0.35\textwidth}
        \centering
        \includegraphics[width=0.6\textwidth]{Ressources/Images/Modelisation/Regulation/hysteresis.pdf}
        \caption{}
        \label{fig:hysteresis}
    \end{subfigure}
    \quad
    \begin{subfigure}[b]{0.6\textwidth}
        \centering
        \includegraphics[width=0.6\textwidth]{Ressources/Images/Modelisation/onOff_control.pdf}
        \caption{}
        \label{fig:onOff}
    \end{subfigure}
    \caption[Types de Contrôleurs]
             {Principe de fonctionnement d’un contrôleur \enquote{Hystérésis} (a)
              et d’un contrôleur \enquote{OnOff} (b). Dans les deux cas, la sortie
              du contrôleur est \enquote{Vrai} si la couleur est verte, et \enquote{Fausse}
              si la couleur est rouge.}
    \label{fig:types_controleurs}
\end{figure}

\paragraph{Architecture de l’algorithme~:} % (fold)
\label{par:architecture_de_l_algorithme}
Chaque équipement ou entité est dans ces travaux représenté comme un automate fini (\abr{FSM},
Finite State Machine)~: la position de la V3V, l’état des
pompes, et le mode de chauffage actif. Un \abr{FSM} décrit un algorithme comme étant un
ensemble d’états ayant un jeu de conditions et/ou temporisations permettent de passer d’un
état à l’autre~: les transitions.
Dans le cas de la \figref{fig:automate_fini}, le \abr{FSM} est initialisée dans l’\textit{état}
\enquote{Immobile debout} qui admet deux \textit{transitions}, \enquote{Appuis sur touche A}
ou \enquote{Maintient la touche B}. Dès que la touche B est maintenue le \abr{FSM}
passe dans l’\textit{état} \enquote{Se baisse}. Dans ce nouvel état, il n’existe qu’une
unique \textit{transition}~: \enquote{Relâche la touche B}. Ainsi même si la touche A
est pressée, rien ne se passera car l’action n’est pas prévue, il n’existe pas de \textit{transition}
pour cette action. Ainsi il est clair qu’un état est toujours unique, et que chaque état doit
explicitement décrire les transitions permettant de passer de l’un à l’autre.
L’approche est donc robuste, stable, et évite des comportements éradiques pour des configurations
non prévues.
Cependant, la définition explicite des chaque transition est à la fois une force et une faiblesse.
Il est en effet maintenant nécessaire de décrire pour chaque état les transitions
valables, processus rapidement répétitif.
Afin de simplifier sa création, les Automates Finis Hiérarchisés (\abr{HFSM}, Hierarchical
Finite State Machine) ont été développés. Cette formulation admet que chaque état peut
être lui-même un \abr{FSM} dont les sous-états ne sont connus que de l’état maître. Cette
nouvelle formulation permet ainsi de réduire le nombre de conditions de transition et
l’encapsulation des sous-états simplifie la compréhension et la construction de
l’algorithme. Il existe ainsi toujours un unique état maître actif mais celui ci peut avoir
un certain nombre de sous-états avec leur propres transitions.

\begin{figure}
    \centering
    \includegraphics[width=0.75\textwidth]{Ressources/Images/Modelisation/Regulation/exemple_fsm.pdf}
    \caption[Illustration du fonctionnement d’un automate fini]
            {Illustration du fonctionnement d’un automate fini appliqué au contrôle
             d’un personnage fictif grâce à deux boutons~: A et B.}
    \label{fig:automate_fini}
\end{figure}

Le \abr{HFSM} gouvernant le chauffage admet trois états primaires~: $Chauffage_{off}$,
$Chauffage_{on}$, et $Surchauffe_{on}$ (\figref{fig:automate_chauffage}). L’état
$Chauffage_{on}$ admet deux \abr{FSM} fonctionnant en parallèle~: celui du chauffage
solaire, et celui du chauffage électrique. Le chauffage solaire admet ainsi trois sous-états~:
$Solaire_{off}$, $Solaire_{direct}$, ou $Solaire_{indirect}$. Le premier est actif lorsque
il n’y a pas d’énergie solaire disponible ou pas assez pour couvrir les besoins en ECS et
en chauffage. Le second ($Solaire_{direct}$) est actif lorsque il y a une demande en
chauffage et que le système est en mode $Direct$. Finalement le dernier état est atteint
lorsque il existe toujours une demande en chauffage et que le système est en mode
$Indirect$.
Il peut être noté l’ajout d’un hystérésis de \SI{5}{\celsius} mis en place sur la
transition entre $Solaire_{off}$ et les deux autres états. Dans les deux cas, une
temporisation permet d’assurer que l’énergie est vraiment disponible et que ce n’est pas
le résultat d’une variation parasite temporaire (activation d’une pompe, fermeture d’une
vanne\dots). Le chauffage électrique lui admet deux états~: $Appoint_{off}$ et
$Appoint_{on}$ et est temporisée afin de favoriser le solaire thermique dans un premier temps.
L’appoint électrique une fois activé reste cependant actif tant que le besoin de chauffage n’est pas couvert
afin d’assurer le confort des occupants.
Lorsque le besoin de chauffage est couvert, le chauffage passe dans l’état
$Surchauffe_{on}$. Comme décrit ci-avant, cet état autorise une élévation de la température
dans le bâtiment. Finalement, dans le cas où la surchauffe n’est pas autorisée ou que de nouveaux
besoins en chauffage apparaissent, l’état du chauffage est réinitialisé à $Chauffage_{off}$
et la boucle de contrôle recommence.
L’exemple suivant illustre le fonctionnement de l’algorithme~:
\blockquote{L’énergie disponible au niveau des capteurs est suffisante pour couvrir les besoins en chauffage de
la maison. Le système solaire est alors en mode $Direct$ et le chauffage dans l’état
$Chauffage_{on}$ avec le chauffage solaire dans le sous-état $Solaire_{direct}$ et
l’appoint électrique dans le sous-état $Appoint_{off}$.}
Un des habitants décide de prendre sa douche, abaissant la température de l’eau du ballon
sanitaire en dessous du seuil minimal (\SI{55}{\celsius}). Dans cette configuration la
pompe $S5$ va s’enclencher pour fournir de l’énergie au ballon sanitaire. Si l’énergie
disponible au niveau des capteurs est insuffisante pour le chauffage et le maintien du
ballon sanitaire alors la pompe $S2$ (chauffage) va s’arrêter car l’\abr{ECS} est prioritaire.
Le chauffage solaire passe alors dans l’état $Solaire_{off}$ et l’appoint électrique est
en attente d’activation pour permettre de couvrir le chauffage (temporisation de
\SI{3}{\minute}).

\begin{figure}
    \centering
    \includegraphics[width=0.95\textwidth]{Ressources/Images/Modelisation/Regulation/chauffage_fsm.pdf}
    \caption[\abr{HFSM} contrôlant la transition entre les différents modes de chauffage]
            {Description détaillée de l’\abr{HFSM} contrôlant la transition entre
             les différents modes de chauffage.}
    \label{fig:automate_chauffage}
\end{figure}
% paragraph architecture_de_l_algorithme(end)


\paragraph{} % (fold)
\label{par:conclusion_algo}
L’algorithme de contrôle de la partie hydraulique fonctionne ainsi grâce à un algorithme
maître en cascade délégant le contrôle des équipements à une combinaison de
\abr{PID} fonctionnant par rétroaction (feed-back). Cette approche permet au système de
prévenir au niveau global les changements et de s’adapter aux besoins dynamiquement.
% paragraph conclusion_algo (end)
% subsubsection fonctionnement_global (end)


% - - - - - - - - - - - - - - - - - - - - - - - - - - - - - - - - - - - - - - -
\subsubsection{Logique de contrôle spécifique au chauffage solaire} % (fold)
\label{ssub:logique_de_controle_specifique_au_chauffage_solaire}
Afin de favoriser le chauffage par l’énergie solaire, l’échangeur de chaleur entre l’eau
et l’air est placé en amont du terminal électrique. Grâce à une loi d’air, loi
proportionnelle par rapport à la température extérieure (similaire à une loi d’eau), la
température de soufflage nécessaire est déterminée~: $T_{air}^{cons}$ (\tabref{eq:temp_soufflage}).
La température d’air en sortie de l’échangeur est directement liée à la température et au
débit de l’eau en entrée de l’échangeur. Pour cette raison, la différence de température
entre l’air en sortie de l’échangeur et la consigne ($T_{air}^{cons}$), est utilisée
par le \abr{PID} contrôlant la pompe $S2$.

\begin{equation}\label{eq:temp_soufflage}
    T_{air}^{cons} = T_{air}^{ext} + PI_{souff} \times (T_{air}^{max} - T_{air}^{ext})
\end{equation}
avec $PI_{souff}$, $T_{air}^{max}$, et $T_{air}^{ext}$ respectivement l’impact du
contrôleur $PI$ associé, la température maximale de l’air soufflé, et la température
extérieure de l’air. Cette formulation permet d’éviter de souffler un air à une
température élevée en continu. Cependant dans le cas où les besoins sont importants, le
débit hygiénique ($Q_{v}^{min} = \SI[per-mode=symbol]{90}{\meter\cubed\per\hour}$) est insuffisant
et le débit doit être augmenté.

Une temporisation de \SI{10}{min} est alors mis en place lorsque $T_{air}^{cons} = T_{air}^{max}$.
À la fin de la temporisation si la température de la maison est
toujours inférieure à $T_{ins}$, alors le débit soufflé est modulé jusqu’à un débit maximal de
$Q_{v}^{max} = \SI[per-mode=symbol]{400}{\meter\cubed\per\hour}$ permettant de couvrir
l’appel de puissance maximal pour l’ensemble des climats. Ces
temporisations permettent respectivement de favoriser en priorité le solaire et de
garantir le maintien du confort thermique (\figref{fig:chauffage_aeraulique}).

\begin{figure}
    \centering
    \includegraphics[width=0.55\textwidth]{Ressources/Images/Modelisation/Regulation/chauffage_aeraulique.pdf}
    \caption[Fonctionnement de la régulation du chauffage par vecteur air]
            {Fonctionnement de la régulation du chauffage par vecteur air.}
    \label{fig:chauffage_aeraulique}
\end{figure}

Afin d’améliorer la couverture solaire sur le chauffage, l’algorithme autorise la
surchauffe de la maison durant les périodes d’inoccupation diurnes
(\figref{fig:control_air}) mais seul l’état $Solaire_{direct}$ est autorisé. Dans ce
cas, le débit minimum sanitaire est maintenue et seul $T_{air}^{cons}$ varie afin que la
température intérieure atteigne la température limite de surchauffe~: $T^{cons}_{sol}$.
L’augmentation de la température intérieure permet de profiter de l’inertie du bâtiment, et
ainsi retarder ou éviter la relance du chauffage~: l’utilisation potentielle de
l’appoint électrique sur le chauffage est réduite.
\begin{figure}
    \centering
    \includegraphics[width=0.9\textwidth]{Ressources/Images/Modelisation/Regulation/control_air_curve.pdf}
    \caption[Principe de la surchauffe diurne durant une journée hivernale]
            {Principe de la surchauffe diurne durant une journée hivernale.}
    \label{fig:control_air}
\end{figure}
% subsubsection logique_de_controle_specifique_au_chauffage_solaire (end)
% subsection logique_de_controle (end)


% ------------------------------------------------------------------------------
\subsection{Bilan} % (fold)
\label{sub:bilan_cas_etude}
Dans un premier temps le bâtiment et les scénarios internes ont été présentés. Le
choix a été fait d’utiliser un modèle de bâtiment mono-zone et les résultats en
comparaison avec un modèle multi-zone montrent que l’approche est pertinente. Ensuite
le \abr{SSC} et sa logique de fonctionnement a été présentée. Le système s’inspire
des travaux de la littérature afin de proposer une approche simple et robuste.
La section suivante décrit ainsi logiquement son application à travers une étude
paramétrique où des variations à la fois techniques et climatiques sont étudiées.
% subsection bilan_cas_etude (end)
% section application_a_la_modelisation_du_cas_d_etude (end)







% ..............................................................................
% ..............................................................................
\section{Étude paramétrique} % (fold)
\label{sec:etude_parametrique}
% ------------------------------------------------------------------------------
L’étude paramétrique détaillée dans ce chapitre a été réalisée à partir de la version
définitive de l’algorithme et du bâtiment. En effet, ces travaux s’inscrivent dans le
cadre d’un travail en collaboration avec l’entreprise \abr{IGC}, un constructeur de
maisons individuelles dans le sud-ouest de la France, à travers le projet
\abr{COMEPOS}. Les caractéristiques du bâtiment comme du \abr{SSC} ont ainsi évolué
au fur et à mesure de l’avancement du projet.

L’étude est nécessaire afin d’évaluer le potentiel du \abr{SSC}. Dans un premier temps,
les variations paramétriques retenues sont présentées~: variations climatiques, scénarios,
taille des équipements comme les ballons ou les capteurs solaires thermiques, logique de
contrôle\dots\ Ensuite, la méthodologie utilisée pour l’analyse des résultats est
introduite. Finalement, les observations sont ensuite détaillés et les limites de
l’approche discutées.


% ------------------------------------------------------------------------------
\subsection{Climats étudiés} % (fold)
\label{sub:climats_etudies}
Afin d’être représentatif des zones climatiques de la France, l’étude a été réalisée
pour 5 villes~: Bordeaux, Nantes, Strasbourg, Limoges, et Marseille (\figref{fig:carte_france},
\tabref{tab:description_climat}).

\begin{figure}
    \centering
    \includegraphics[width=0.6\textwidth, clip=true, trim=2mm 2mm 2mm 20mm]{Ressources/Images/Modelisation/Batiment/France_map.pdf}
    \caption[Cartographie des villes sélectionnées pour l’étude paramétrique]
            {Cartographie des villes sélectionnées pour l’étude paramétrique.}
    \label{fig:carte_france}
\end{figure}

La ville de Marseille correspond à un climat très ensoleillé avec une faible demande en
chauffage. À l’opposé le climat de Strasbourg est rude avec
une forte demande en chauffage. Le climat de Bordeaux est lui modéré~: l’ensoleillement est correct
et la demande en chauffage faible. Limoges et Nantes se placent entre Bordeaux et
Strasbourg avec un bon ensoleillement mais une demande en énergie pour le chauffage plus
importante. Pour ces différents sites, la température du sol est
considéré comme constante et est fixée à \SI{10}{\celsius}. Bien qu’impactant les besoins
d’un bâtiment, sa détermination de manière précise est un processus complexe spécialement
lorsque de nombreuses variations de l’enveloppe comme des charges internes sont envisagées.

Dans un premier temps, l’impact du climat est investigué puis de nombreuses variations au
niveau des équipements, des scénarios, et de la régulation sont évaluées pour deux
villes~: Bordeaux et Strasbourg.

\begin{table}
\centering
\caption[Description des différents climats retenues]
        {Description des différents climats retenues.}
\label{tab:description_climat}
\begin{tabular}{ l c c  c  c  c  c }
  \toprule
                                          &    & \textbf{Bordeaux} & \textbf{Nantes} & \textbf{Strasbourg} & \textbf{Limoges} & \textbf{Marseille} \\
  \midrule
  \addlinespace[\defaultaddspace]
  \multirow{3}{*}{Irradiation solaire} & $IGH$   & \num{1264}              & \num{1184}               & \num{1091}                & \num{1257}              & \num{1545}              \\
                                       & $IDN$   & \num{929}               & \num{885}               & \num{721}                 & \num{1209}              & \num{1503}              \\
                                       & $IDH$   & \num{712}               & \num{665}               & \num{650}                 & \num{602}              & \num{615}               \\
  \addlinespace[\defaultaddspace]
  \multirow{2}{*}{Température eau froide} & Min     & \num{8.9}               & \num{8.3}               & \num{5.3}                 & \num{7}                 & \num{12}                \\
                                          & Max     & \num{16}                & \num{15}               & \num{14}                  & \num{14}                & \num{19}                \\
  \addlinespace[\defaultaddspace]
  \abr{DJU} (\SI{19}{\celsius})                 & -  & \num{2408}              & \num{2660}               & \num{3360}                & \num{2972}              & \num{2049}              \\
  \bottomrule
\end{tabular}
\end{table}
% subsection climats_etudies (end)


% ------------------------------------------------------------------------------
\subsection{Scénarios étudiés} % (fold)
\label{sub:scenarios_etudies}
% - - - - - - - - - - - - - - - - - - - - - - - - - - - - - - - - - - - - - - -
\subsubsection{Puisage en eau chaude sanitaire} % (fold)
\label{ssub:puisage_en_eau_chaude_sanitaire}
Afin de pouvoir estimer la consommation nécessaire pour la production d’\abr{ECS}, un profil
de puisage type est nécessaire. Pour chaque climat, la température de l’eau du réseau suit
une évolution mensuelle extrapolée durant la simulation
(\tabref{tab:temp_eau}). Il existe donc une forte disparité entre les différents climats
avec des extremums représentés respectivement par Strasbourg et Marseille~:
\begin{itemize}
    \item Les minimums varient entre \SI{5.3}{\celsius} et \SI{12}{\celsius}.
    \item Les maximums varient entre \SI{14}{\celsius} et \SI{19}{\celsius}.
\end{itemize}

\begin{table}
\centering
\caption[Variation de la température de l'eau du réseau en fonction de la position géographique]
        {Variation de la température de l'eau du réseau (\si{\kelvin}) au cours
         de l'année en fonction de la position géographique.}
\label{tab:temp_eau}
\begin{tabular}{l*{12}{c}}
    \toprule
               & Janv. & Fevr. & Mars & Avr. & Mai & Juin & Juil. & Août & Sept. & Oct. & Nov. & Dec. \\
    \midrule
    Strasbourg & \num{5.3}   & \num{5.8}   & \num{7.7}  & \num{9.5}  & \num{11}  & \num{13}   & \num{14}    & \num{14}   & \num{12}    & \num{9.8}  & \num{7.5}  & \num{5.8}  \\
    Limoges    & \num{7}     & \num{7.4}   & \num{9}    & \num{10}   & \num{12}  & \num{14}   & \num{14}    & \num{14}   & \num{13}    & \num{11}   & \num{8.8}  & \num{7.3}  \\
    Toulouse   & \num{8.6}   & \num{9.2}   & \num{11}   & \num{12}   & \num{14}  & \num{16}   & \num{17}    & \num{17}   & \num{16}    & \num{13}   & \num{11}   & \num{9}    \\
    Bordeaux   & \num{8.9}   & \num{9.3}   & \num{11}   & \num{12}   & \num{14}  & \num{15}   & \num{16}    & \num{16}   & \num{15}    & \num{13}   & \num{11}   & \num{9.2}  \\
    Nantes     & \num{8.3}   & \num{8.5}   & \num{9.9}  & \num{11}   & \num{13}  & \num{14}   & \num{15}    & \num{15}   & \num{14}    & \num{12}   & \num{9.8}  & \num{8.6}  \\
    Marseille  & \num{12}    & \num{12}    & \num{13}   & \num{14}   & \num{16}  & \num{18}   & \num{19}    & \num{19}   & \num{18}    & \num{16}   & \num{14}   & \num{12}   \\
    \bottomrule
\end{tabular}
\end{table}


\paragraph{Profils retenus~:} % (fold)
\label{par:profils_retenus}
Cinq scénarios ont été considérés dans ces travaux (\figref{fig:profil_puisage}).
Pour l’ensemble des profils, un volume de puisage typique moyen de
\SI{33}{\litre\per(jour\period pers)}\,(\SI{60}{\celsius}), soit un
volume total de \SI{220}{\litre/jour}\,(\SI{40}{\celsius}) pour les quatre occupants est
retenu. Le profil de référence utilisé est le scénario décrit dans la norme
\textcite{EN129771}, \textbf{EN\,12977}. Trois variantes ont aussi été évaluées favorisant
respectivement un puisage en matinée (\textbf{Matin}), en soirée (\textbf{Soir}) et un
puisage similaire pour les trois pics journaliers (\textbf{Réparti}). Finalement un scénario
issue de données statistiques \parencite{ADEME2016} a aussi été implémenté
(\textbf{Réaliste}). Ce dernier décrit de manière plus représentative le comportement
d’une famille dans une habitation individuelle ou un appartement, et les besoins
journaliers sont pondérées en fonction du jour de la semaine (\tabref{tab:coef_semaine}).
\begin{figure}
    \centering
    \includegraphics[width=0.9\textwidth]{Ressources/Images/Modelisation/Scenario/puisage.pdf}
    \caption[Description des profils de puisage envisagés]
            {Description des profils de puisage envisagés.}
    \label{fig:profil_puisage}
\end{figure}
% paragraph profils_retenus (end)

\paragraph{Pondération de la demande~:} % (fold)
\label{par:ponderation_de_la_demande}
Afin d’évaluer l’impact de la variation mensuelle des besoins en \abr{ECS}, l’ajout d’un
coefficient mensuel est investigué. Les coefficients utilisés sont issues des travaux de
l’\textit{ADEME} \parencite{ADEME2016}. La demande totale reste inchangée~; seule sa
répartition est altérée. Les résultats de l’\textit{ADEME} montrent ainsi que le puisage est plus
important durant la période hivernale (\tabref{tab:coef_mois}).

\begin{table}
\centering
\caption[Détail des coefficients de pondération journaliers pour le profil de puisage Réaliste]
        {Détail des coefficients de pondération journaliers pour le profil de
         puisage \textbf{Réaliste}.}
\label{tab:coef_semaine}
\begin{tabular}{l*{7}{c}}
    \toprule
                & Lundi & Mardi & Mercredi & Jeudi & Vendredi & Samedi & Dimanche \\
    \midrule
    Coefficient & \num{0.97}  & \num{0.95}  & \num{1.00}     & \num{0.97}  & \num{0.96}     & \num{1.02}   & \num{1.13}     \\
    \bottomrule
\end{tabular}
\end{table}

\begin{table}
\centering
\caption[Détail des coefficients de pondération mensuels pour le profil de puisage Réaliste]
        {Détail des coefficients de pondération mensuels pour le profil de
         puisage \textbf{Réaliste}.}
\label{tab:coef_mois}
\begin{tabular}{l*{12}{c}}
    \toprule
                & Janv. & Fevr. & Mars & Avr. & Mai & Juin & Juil. & Août & Sept. & Oct. & Nov. & Dec. \\
    \midrule
    Coefficient & \num{1.11}   & \num{1.2}   & \num{1.11}  & \num{1.06}  & \num{1.03}  & \num{0.93}   & \num{0.84}    & \num{0.72}   & \num{0.92}    & \num{1.03}  & \num{1.04}  & \num{1.01}  \\
    \bottomrule
\end{tabular}
\end{table}
% paragraph ponderation_de_la_demande (end)

\paragraph{Variations paramétriques~:} % (fold)
\label{par:variations_parametriques}
L’évaluation de l’impact de la variation des profils de puisage comme de la quantité des
besoins sur la performance du \abr{SSC} est donc retenue. Les consommations variant fortement
d’une habitation à une autre, des consommations de \num{27}, \num{33}, et
\SI{40}{\litre\per(jour \period  pers)}\,(\SI{60}{\celsius}) sont aussi évaluées.
% paragraph variations_parametriques (end)
% subsubsection puisage_en_eau_chaude_sanitaire (end)


% - - - - - - - - - - - - - - - - - - - - - - - - - - - - - - - - - - - - - - -
\subsubsection{Ventilation} % (fold)
\label{ssub:ventilation}
Deux scénarios de ventilation ont été retenus pour l’étude paramétrique~:
\begin{itemize}
    \item Référence~: $90-20$\,\si[per-mode=symbol]{\meter\cubed\per\hour} (\ref{ssub:ventilation_ref})
    \item Constant~: $90-90$\,\si[per-mode=symbol]{\meter\cubed\per\hour}
\end{itemize}
L’analyse de ces deux scénarios permettra d’identifier l’impact d’un débit réduit sur
les consommations nécessaires afin de maintenir la température de consigne dans la
maison.
\bigskip
% subsubsection ventilation (end)


% - - - - - - - - - - - - - - - - - - - - - - - - - - - - - - - - - - - - - - -
\subsubsection{Température de consigne} % (fold)
\label{ssub:temperature_de_consigne}
\paragraph{Consigne de chauffage~:} % (fold)
\label{par:consigne_de_chauffage}
Afin de couvrir une large palette de scénarios, la consigne en inoccupation, en occupation
diurne, et en inoccupation nocturne sont modifiées indépendamment. La consigne de
référence étant, pour rappel, définie comme étant $19$-$18$-$16$~: \SI{19}{\celsius} durant les
occupations diurnes, \SI{18}{\celsius} durant les occupations nocturnes et
\SI{16}{\celsius} en inoccupation. À partir des règles et contraintes définies dans
\tabref{tab:consigne_chauffage} il est alors possible de construire un lot de scénarios
représentatifs. L’étude cherchera en particulier à évaluer l’impact d’une augmentation de
la consigne de chauffage, ainsi que du réduit en inoccupation et en période
nocturne. Les scénarios construits sont les suivants~: $19$-$18$-$16$, $19$-$19$-$16$, $19$-$19$-$19$,
$20$-$18$-$16$, $20$-$20$-$16$.

\begin{table}
\centering
\caption[Variations envisagées pour la consigne de chauffage en fonction de la période]
        {Variations envisagées pour la consigne de chauffage en fonction de la période.}
\label{tab:consigne_chauffage}
\begin{tabular}{l c c c c}
    \toprule
                           & \textbf{\num{16}}                     & \textbf{\num{18}}                     & \textbf{\num{19}}                     & \textbf{\num{20}}              \\
    \midrule
    Occupation (diurne)    &                             &                             & \cellcolor{SolarizedBrBlue} & \cellcolor{SolarizedBrBlue} \\
    Occupation (nocturne)  &                             & \cellcolor{SolarizedBrBlue} & \cellcolor{SolarizedBrBlue} & \cellcolor{SolarizedBrBlue} \\
    Inoccupation           & \cellcolor{SolarizedBrBlue} &                             & \cellcolor{SolarizedBrBlue} &                     \\
    \bottomrule
\end{tabular}
\end{table}
% paragraph consigne_de_chauffage (end)


\paragraph{Consigne de chauffage solaire~:} % (fold)
\label{par:consigne_de_chauffage_solaire}
L’impact de la température de consigne solaire est aussi discutée. Deux configurations sont
simulées. La première admet une température de consigne de \SI{22}{\celsius} et la seconde
ne considère pas d’élévation de la température par le solaire. Pour rappel l’élévation de
la température n’est possible que par l’énergie solaire provenant des capteurs et donc
uniquement si le système est dans le sous-état $Solaire_{direct}$. Finalement les interactions
entre consigne et consigne solaire de chauffage sont aussi discutées.
% paragraph consigne_de_chauffage_solaire (end)
% subsubsection temperature_de_consigne (end)
% subsection scenarios_etudies (end)


% ------------------------------------------------------------------------------
\subsection{Variations techniques étudiées} % (fold)
\label{sub:variations_techniques_etudiees}
% - - - - - - - - - - - - - - - - - - - - - - - - - - - - - - - - - - - - - - -
\subsubsection{Fluide caloporteur} % (fold)
\label{ssub:fluide_caloporteur}
Afin de pouvoir implémenter le modèle pour tous les climats, il est nécessaire de
tenir compte des risques de gel. Le fluide caloporteur utilisé à travers les capteurs
solaires est donc un mélange~: eau (\SI{70}{\percent}) et éthylène-glycol (\SI{30}{\percent}).

Seul les capteurs solaires sont à l’extérieur, il n’est alors pas nécessaire de protéger
l’ensemble du système contre le gel. En effet, l’eau du ballon sanitaire provient du
réseau d’eau de ville, et le ballon de stockage est dans une zone chauffée. Dans les
deux cas une eau sans glycol est considérée. Ainsi seules les canalisations reliant les
échangeurs contiennent du glycol et la modélisation admet deux fluides différents
(\tabref{tab:fluide_carac}). Enfin, la variation de la capacité massique étant négligeable sur
les plages de variation considérées (\num{0} à \SI{120}{\celsius}), le modèle assume une
capacité massique constante permettant de simplifier les systèmes d’équations.

\begin{table}
\centering
\caption[Caractéristiques des deux fluides caloporteurs utilisés pour la modélisation]
        {Caractéristiques des deux fluides caloporteurs utilisés pour la modélisation.}
\label{tab:fluide_carac}
\begin{tabular}{l *{2}{c} r}
    \toprule
                       & Eau                 & Eau + glycol          & Unité                             \\
    \midrule
    Chaleur massique   & \num{4180}          & \num{3608}            & \si{\joule\per(kg\period\kelvin)} \\
    Masse volumique    & \num{1000}          & \num{1034}            & \si{kg\per\meter\cubed}           \\
    Plage de variation & \num{0} à \num{100} & \num{-20} à \num{110} & \si{\celsius}                     \\
    \bottomrule
\end{tabular}
\end{table}
% subsubsection fluide_caloporteur (end)

% - - - - - - - - - - - - - - - - - - - - - - - - - - - - - - - - - - - - - - -
\subsubsection{Capteurs solaires} % (fold)
\label{ssub:capteurs_solaires}
Au cours de l’étude paramétrique, l’orientation, la surface, et l’inclinaison
des capteurs est évaluée afin de comparer les résultats avec les observations déjà
faites au cours des études antérieures \parencite{Task26C2007,Shariah2002587}.
Les variations suivantes sont retenues~:
\begin{itemize}
  \item Inclinaison~: \SI{18.9}{\degree} (\SI{33}{\percent}), \SI{33}{\degree}, \SI{45}{\degree}, \SI{60}{\degree}
  \item Orientation~: Est, Ouest, Sud
  \item Nombre de capteurs~: \num{2}, \num{4}, \num{6}, \num{8}
\end{itemize}
Finalement, la maison de référence est orientée Sud, comporte $4$ capteurs
de type IDMK\,$25$, et une toiture ayant une pente de \SI{33}{\percent}.
% subsubsection capteurs_solaires (end)


% - - - - - - - - - - - - - - - - - - - - - - - - - - - - - - - - - - - - - - -
\subsubsection{Ballons} % (fold)
\label{ssub:ballons}
L’analyse paramétrique tient aussi compte de variations au niveau des ballons comme
son volume. Celui-ci est discrétisé en \num{20} segments numérotés de haut en bas
permettant de prendre en compte le phénomène de stratification.
Afin d’être cohérent, les caractéristiques des ballons sont proportionnelles à sa taille~:
si le volume du ballon augmente, sa hauteur augmente aussi.
De même, la taille de l’échangeur est adaptée à la nouvelle hauteur du ballon
afin que sa position relative reste identique. De cette manière, le ballon de
stockage conserve un échangeur couvrant la quasi totalité du volume, et
l’échangeur solaire sur le ballon sanitaire reste adapté à la nouvelle taille du
ballon. Les variations retenues pour chaque ballon sont identiques~:
\SI{150}{l}, \SI{300}{l}, et \SI{450}{l}.

La position de l’échangeur solaire est aussi étudiée. Ces travaux ne considèrent que
la position de l’échangeur du ballon sanitaire ($Ech_{sol}^{pos}$) car l’échangeur du
ballon de stockage couvre presque l’intégralité de sa hauteur. De plus, afin d’être
cohérent avec les variations réalisées sur le volume des ballons, la variation de
$Ech_{sol}^{pos}$ est réalisée relativement à sa position d’origine, laissant intact sa
longueur. Il est alors possible d’évaluer l’impact du volume du ballon et de la $Ech_{sol}^{pos}$
de manière combinée ou bien indépendamment. Les coefficients
retenus sont issus des contraintes imposées par les spécificités techniques des ballons~:
\num{0.8}, \num{1}, et \num{1.3}. Avec un coefficient de \num{0.8} l’échangeur sera ainsi
positionné plus bas et avec un coefficient de \num{1.3}, il sera à l’inverse plus haut. La
position haute de référence pour l’échangeur(\num{1}) correspond au $12^{ème}$ segment.
% subsubsection ballons (end)


% ------------------------------------------------------------------------------
\subsubsection{Algorithme} % (fold)
\label{ssub:variations_algorithmiques}
Finalement il est aussi étudié à travers cette étude des variations au niveau de
l’algorithme de contrôle (\tabref{tab:variations_algo}). $\Delta T_{sol}$, la différence de
température entre la source (sortie des capteurs) et la cible (sortie des échangeurs) est
modulée autour du cas de référence (\SI{10}{\celsius}). Augmenter le $\Delta T_{sol}$
revient à attendre plus longtemps avant d’autoriser l’utilisation de l’énergie solaire et
impose une température en sortie des capteurs plus élevée afin de valoriser les apports
solaires. À l’inverse diminuer ce paramètre amène le système a être plus réactif mais le
risque d’instabilité est cependant plus important. Il est aussi évalué l’impact des
temporisations au niveau de la régulation du chauffage par vecteur air~:
\begin{itemize}
  \item Activation de la batterie électrique terminale
  \item Modulation du débit de soufflage
  \item Modulation de la température de soufflage
\end{itemize}
La première temporisation, \SI{3}{min} pour le cas de référence, sera notée
$Tempo_{batterie}$. Elle permet d’éviter d’activer la batterie terminale électrique
directement à chaque demande de chauffage. Elle permet donc de retarder son activation
quand de l’énergie solaire est disponible. La seconde temporisation correspond à la temporisation
entre la modulation de la température, et la modulation du débit pour le chauffage par
l’air. Le cas de référence admet une temporisation de \SI{10}{min} notée $Tempo_{souff}$.

\begin{table}
\centering
\caption[Variations algorithmiques étudiées lors de l’étude paramétrique]
        {Variations algorithmiques étudiées lors de l’étude paramétrique.}
\label{tab:variations_algo}
\begin{tabular}{l c c r}
    \toprule
    Paramètre          & Référence & Variations          & Unité         \\
    \midrule
    $\Delta T_{sol}$     & \num{10}  & \num{5}, \num{15}   & \si{\celsius} \\
    $Tempo_{batterie}$ & \num{3}   & \num{2}, \num{4}    & \si{min}      \\
    $Tempo_{souff}$    & \num{10}  & \num{5}, \num{15}   & \si{min}      \\
    \bottomrule
\end{tabular}
\end{table}
% subsubsection algorithme (end)
% subsection variations_techniques_etudiees (end)



% ------------------------------------------------------------------------------
\subsection{Analyse des résultats} % (fold)
\label{sub:analyse_des_resultats}
% - - - - - - - - - - - - - - - - - - - - - - - - - - - - - - - - - - - - - - -
\subsubsection{Méthodologie} % (fold)
\label{ssub:methodologie}
L’analyse a été réalisée sur la base d’un cas de référence dont les scénarios et les
caractéristiques ont été présentés dans les parties précédentes. Sans mentions explicites, les
simulations utilisent ces paramètres. L’ensemble des variations possibles ainsi que
le cas de référence sont décrits à travers un tableau récapitulatif (\tabref{tab:ref_description})
afin d’offrir une vue globale des variations étudiées.
Enfin afin d’évaluer le système solaire, différents indicateurs sont introduits
\eqref{eq:indicators}. La consommation électrique tient compte de la consommation des deux
appoints (chauffage et \abr{ECS}) mais aussi de la consommation des pompes du \abr{SSC}. Elle est
notée $Conso_{app}$ dans le reste de l’étude. Enfin $F_{sol}^{ECS}$ et $F_{sol}^{CH}$
sont respectivement le taux de couverture sur la production de l’eau chaude sanitaire
et sur le chauffage. Le taux de couverture correspond au rapport entre la production
solaire et la consommation de l’appoint et s’exprime sous la forme d’un pourcentage.


\begin{table}
\centering
\caption[Description de la solution de référence et de variations étudiées]
        {Description de la solution de référence et de variations étudiées.}
\label{tab:ref_description}
  \begin{tabular}{l c C{4cm} r r}
    \toprule
    \addlinespace
                                           & Référence & Variations                             & Unité         & Détails                                               \\
    \multicolumn{5}{l}{\textbf{Scénarios}}                                                                                                                                 \\
    \midrule
    Ventilation                            & $90-20$   & $$90-90$$                                  & \si{m^{3}/h}  & \ref{ssub:ventilation}                                \\
    Chauffage                              & $19$-$18$-$16$  & $19$-$19$-16, $19$-$19$-$19$, $20$-$18$-$16$, $20$-$20$-$16$ & \si{\celsius} & \multirow{2}{*}{\ref{ssub:temperature_de_consigne}}   \\
    Chauffage solaire                      & Avec (22) & Sans                                   & \si{\celsius} &                                                       \\
    $Puisage_{scenario}$                   & EN\,12977 & Soir, Matin, Réparti, Réaliste         & \si{l/h}      & \multirow{3}{*}{\ref{ssub:puisage_en_eau_chaude_sanitaire}}            \\
    $Puisage_{vol}$ (\SI{60}{\celsius})    & \num{33}  & \num{27}, \num{40}                     & \si{\litre/(jour\period pers)}      &                                 \\
    $Puisage_{mod}$                        & Non       & Mensuelle                              & -             &                                                       \\
    \\
    \addlinespace[\defaultaddspace]
    \multicolumn{5}{l}{\textbf{Équipements}}                                                                                                                              \\
    \midrule
    Nombre de capteurs                     & \num{4}   & \num{2}, \num{6}, \num{8}              & -             &                                                       \\
    Inclinaison des capteurs               & \num{33}  & \num{18.9}, \num{45}, \num{60}         & \si{\degree}  &                                                       \\
    Orientation des capteurs               & Sud       & Est, Ouest                             & -             &                                                       \\
    Volume ballon de stockage              & \num{300} & \num{150}, \num{450}                   & \si{\litre}   & \multirow{3}{*}{\ref{ssub:ballons}}                   \\
    Volume ballon \abr{ECS}                    & \num{300} & \num{150}, \num{450}                   & \si{\litre}   &                                                       \\
    $Ech_{sol}^{pos}$                      & \num{1}   & \num{0.8}, \num{1.3}                   & \si{m}        &                                                       \\
    \\
    \addlinespace
    \multicolumn{5}{l}{\textbf{Algorithme}}                                                                                                                                 \\
    \midrule
    $Tempo_{batterie}$                     & \num{3}   & \num{2}, \num{4}                       & \si{\min}     & \multirow{3}{*}{\ref{ssub:variations_algorithmiques}} \\
    $Tempo_{souff}$                        & \num{10}  & \num{5}, \num{15}                      & \si{\min}     &                                                       \\
    $\Delta T_{sol}$                         & \num{10}  & \num{5}, \num{15}                      & \si{\celsius} &                                                       \\
    \addlinespace[\defaultaddspace]
    \bottomrule
  \end{tabular}
\end{table}

\begin{equation}
\label{eq:indicators}
\setlength{\jot}{10pt}  % Change equation vertical space only for this environment
\begin{aligned}
    Conso_{app}   &= Conso_{app}^{ECS} + Conso_{app}^{CH} + Conso_{pompes} \\
    F_{sol}       &= \frac{Prod_{sol}}{Conso_{app}}     \\
    F_{sol}^{ECS} &= \frac{Prod_{sol}^{ECS}}{Conso_{app}^{CH}} \\
    F_{sol}^{CH}  &= \frac{Prod_{sol}^{CH}}{Conso_{app}^{ECS}}
\end{aligned}
\end{equation}
% subsubsection methodologie (end)



% - - - - - - - - - - - - - - - - - - - - - - - - - - - - - - - - - - - - - - -
\subsubsection{Impact des conditions météorologiques} % (fold)
\label{ssub:impact_des_conditions_meteorologiques}
Les résultats obtenus à enveloppe identique pour les différents climats sont disponibles à
travers le \tabref{tab:performance_annuelles}. Une forte disparité dans les consommations
de chauffage est observée. La consommation pour Marseille est la plus faible, et deux fois
plus d’énergie est nécessaire pour couvrir les besoins de Strasbourg. De plus il est aussi
mis en évidence l’impact important de la température de l’eau froide du réseau. En effet
les simulations admettent le même scénario de puisage mais la $Conso_{app}$ varie de
\SI{400}{\kilo\watt\hour} entre Marseille et Strasbourg. Les pertes en ligne à travers
les canalisations ($Pertes_{réseau}$) sont similaires pour les différents climats et la
consommation des pompes ne semble pas être un facteur impactant sauf pour Marseille où la
$Conso_{app}$ cumulée du chauffage et de la production d’\abr{ECS} est presque nulle.
La $Conso_{app}$ (\abr{ECS} + \abr{CH}) à Limoges est inférieure à celle de Nantes alors
que la consommation totale (solaire et appoint) est plus importante sur Limoges.
Les besoins sur Nantes sont inférieurs mais l’ensoleillement est plus important
sur Limoges ce qui permet de valoriser une plus grande part solaire et donc de réduire
la $Conso_{app}$ (\tabref{tab:description_climat}).

La même remarque peut être faite au niveau du
chauffage. La consommation totale à Limoges est supérieure de \SI{200}{kWh} par rapport à
Nantes mais la $Conso_{app}$ est elle seulement supérieure de \SI{100}{kWh}. Pour
Strasbourg, il semble que quatre capteurs solaires soient insuffisants pour permettre
de couvrir majoritairement les besoins par le solaire. Finalement, le système solaire obtient une bonne
performance autant sur le $F_{sol}^{ECS}$ que sur le $F_{sol}^{CH}$ en récupérant une part
similaire d’énergie solaire sur les différents climats.

\begin{table}
\small
\centering
\caption[Performances annuelles du système solaire pour différents climats]
        {Performances annuelles du système solaire pour différents climats.}
\label{tab:performance_annuelles}
\begin{tabular}{l c c c c c c c c c c c c}
    \toprule
               &   \multicolumn{9}{c}{Consommation [\si{\kilo\watt\hour}]} & & \multicolumn{2}{c}{\multirow{2}{*}{\si{\percent}}} \\
    \cmidrule(r){2-10}
               & \multicolumn{2}{c}{\textbf{Totale}} &  \multicolumn{3}{c}{\textbf{Électrique}}  & \multicolumn{4}{c}{\textbf{Solaire}} & \\
    \cmidrule(r){2-3}
    \cmidrule(r){4-6}
    \cmidrule(r){7-10}
    \cmidrule(r){12-13}
               & \abr{ECS}    & \abr{CH}      &  \abr{ECS}        & \abr{CH} & Pompes    & $Solaire_{abs}$  & $Pertes_{réseau}$ & \abr{ECS}  & \abr{CH} & & $F_{sol}^{ECS}$  & $F_{sol}^{CH}$ \\
    \midrule
    Bordeaux   & \num{2983}     & \num{1024}      &  \num{101}          & \num{91}          &  \num{7}                 & \num{3387}                  & \num{219}       & \num{2444}   &  \num{949}    &   & \num{95}         & \num{91}  \\
    Strasbourg & \num{3180}     & \num{1986}      &  \num{479}          & \num{1203}        &  \num{8}                 & \num{3164}                  & \num{199}       & \num{2332}   &  \num{845}    &   & \num{83}         & \num{42}  \\
    Marseille  & \num{2784}     & \num{975}       &  \num{2}            & \num{1}           &  \num{7}                 & \num{3280}                  & \num{218}       & \num{2300}   &  \num{974}    &   & \num{100}        & \num{100} \\
    Nantes     & \num{3044}     & \num{1114}      &  \num{233}          & \num{248}         &  \num{7}                 & \num{3295}                  & \num{209}       & \num{2399}   &  \num{902}    &   & \num{91}         & \num{78}  \\
    Limoges    & \num{3120}     & \num{1311}      &  \num{217}          & \num{359}         &  \num{9}                 & \num{3474}                  & \num{209}       & \num{2502}   &  \num{983}    &   & \num{92}         & \num{73}  \\
    \bottomrule
\end{tabular}
\end{table}

L’évolution des températures des différents ballons durant la période de chauffage pour
les différents climats donne plus d’informations (\figref{fig:temp_ballon_mensuel}). Du
fait des priorités de régulation le ballon de stockage ne peut être chargé si la partie
basse du ballon sanitaire ($T3$) n’est pas à une température suffisante. Ainsi l’évolution
des températures des ballons permet de renseigner sur l’activité du système, et plus
particulièrement sur l’importance de l’énergie solaire récupérée par rapport à la demande.
Sur Limoges, le système utilise principalement l’énergie solaire durant tous les
mois à l’exception de décembre où $T5$ stagne. Durant janvier, l’énergie solaire est
suffisante pour charger le ballon de stockage et réduire la $Conso_{app}$ sur
l’\abr{ECS}. Cependant, $T4$ stagne autour de \SI{55}{\celsius} car l’énergie solaire fournie
ne permet pas de charger le ballon sanitaire suffisamment. Sur Strasbourg, le système ne
collecte pas assez d’énergie de novembre à février afin d’être autonome mais l’énergie
accumulée permet cependant de préchauffer l’eau du réseau et donc réduire la
$Conso_{app}$. Finalement sur Bordeaux le système permet de couvrir la majorité des besoins
même si durant décembre et janvier l’appoint électrique est nécessaire ponctuellement.

\begin{figure}
    \centering
    \includegraphics[width=\textwidth]{Ressources/Images/Parametrique/temp_ballons.pdf}
    \caption[Évolution des température internes des ballons du \abr{SSC}]
            {Évolution des température internes des ballons du \abr{SSC} en fonction du mois
             de l’année avec indication de la moyenne (triangle vert) et
             de la médiane (rectangle horizontal bleu).}
    \label{fig:temp_ballon_mensuel}
\end{figure}


\paragraph{Période de simulation~:} % (fold)
\label{par:periode_de_simulation}
L’optique de cette étude paramétrique étant d’évaluer l’impact de différents éléments du
\abr{SSC} (technique et algorithmique) il n’est pas nécessaire de simuler les périodes où
le système est autonome. Chaque simulation annuelle nécessite en effet entre
\SI{2}{\hour} et \SI{3}{\hour} et réduire la période de simulation permet de réduire la durée
d’intégration. Des simulations comprenant un jeu de paramètres défavorables ont aussi été
réalisées afin d’évaluer la période de simulation minimale nécessaire. Pour ces
simulations, un scénario de chauffage $19$-$19$-$19$, un volume de \SI{450}{\litre} pour les
deux ballons, ainsi que \num{2} capteurs avec une orientation Ouest et une inclinaison de
\SI{18.9}{\degree} sont retenus. Les résultats montrent que durant la
période estivale, le système solaire est autonome avec une $Conso_{app}$ de seulement
\SI[per-mode=symbol]{0.01}{\kilo\watt\hour\per\metre\squared} et
\SI[per-mode=symbol]{0.08}{\kilo\watt\hour\per\metre\squared}
pour respectivement Bordeaux et Strasbourg. Dans le reste de l’étude paramétrique
les résultats sont donc discutés \emph{sur une période s’étendant du $1^{er}$
octobre au $30$ avril} permettant de réduire à \SI{\approx 1}{\hour} par simulation. Les
résultats sont fournis pour la période concernée et doivent donc être vus comme la
\textbf{performance minimale} du \abr{SSC}. En effet, le système est autonome durant le reste
de l’année car les besoins en chauffage et en \abr{ECS} sont complètement couverts par la
production solaire~: le $F_{sol}^{ECS}$ et le $F_{sol}^{CH}$ annuels sont donc
supérieurs car ils tiennent compte de la période d’autonomie complète du \abr{SSC}.
Afin de ne pas pénaliser le système, l’eau du système est à une température
initiale de \SI{55}{\degree}. De plus durant les $12$ premiers jours (en septembre),
l’énergie consommée par le système n’est pas considérée afin de laisser le système
se positionner dans une configuration correspondant à la saison.

% \iunsure{Ajouté évolution annuelle de la puissance nécessaire}
% \begin{figure}
%     \centering
%         \ftodo{Ajouter évolution de la puissance et de l’énergie pour Strasbourg et Bordeaux}
%         % \includegraphics{Ressources/Images/Scenario/puisage.pdf}
%     \caption{Évolution des appels de puissance des appoints électriques et du cumul d’énergie pour
%              une année complète \label{fig:puissance_annuelle_defavorable}}
% \end{figure}
% paragraph période_de_simulation (end)
% subsubsection impact_des_conditions_meteorologiques (end)



% - - - - - - - - - - - - - - - - - - - - - - - - - - - - - - - - - - - - - - -
\subsubsection{Impact des scénarios internes} % (fold)
\label{ssub:impact_des_scenarios_internes}
Dans cette section l’impact des charges internes et de la consigne de température est
investigué. Afin d’éviter la redondance, les différentes variations sont seulement
réalisées pour le climat de Bordeaux et de Strasbourg. L’analyse des résultats montre
que le comportement du système est similaire pour les deux climats~: l’analyse est donc
illustrée uniquement pour le climat de Strasbourg où les modifications ont un impact plus
net.

Afin de pouvoir comparer de manière efficace les variations paramétriques une
représentation graphique est introduite (\figref{fig:impact_temp_consigne}). Elle permet
de visualiser les paramètres caractéristiques du système combiné~:
\begin{itemize}
    \item Le $F_{sol}^{ECS}$~: la longueur des barres horizontales
    \item Le $F_{sol}^{CH}$~: la largeur des barres dont l’étendue de variation quantitative
          est décrite en bas à droite.
\end{itemize}
Il est aussi possible d’évaluer la production solaire valorisée
($Prod_{sol}^{valorisée}$), définie comme la part solaire transmise à l’eau après
soustraction des pertes au niveau des canalisations~; et la part d’énergie consommée
par l’appoint ($Conso_{app}$). Ces deux indications sont fournies à l’extrémité de chaque
barre horizontale et une légende est disponible en dessous.
Enfin la couleur et l’ordre des barres sont définis à partir des valeurs normalisées
du rendement solaire des capteurs ($\eta_{sol}$) dont l’échelle est disponible en partie droite. Le
$\eta_{sol}$ traduit le rapport entre la $Prod_{sol}^{valorisée}$ et l’énergie
incidente sur les capteurs~: il indique donc la part utile de l’énergie récupérée par rapport à
l’énergie solaire incidente.

L’analyse de la \figref{fig:impact_temp_consigne} permet d’obtenir de nombreuses
informations. Augmenter la température de consigne durant le période
d’occupation ($20$-$20$-$16$) augmente de \SI{8}{\percent} la $Conso_{app}$
sur le chauffage, mais n’impacte pas le $F_{sol}^{ECS}$. Il est aussi important de noter que le
réduit de nuit permet de réduire sensiblement la $Conso_{app}$ mais le $F_{sol}^{ECS}$ et le
$F_{sol}^{CH}$ restent inchangés. Le \abr{SSC} fonctionne ainsi aussi bien avec que sans réduit de
nuit et la $Conso_{app}$ est seulement proportionnelle à l’augmentation de la demande en
énergie pour le chauffage. Cependant la performance du \abr{SSC} est impactée si on considère
une consigne de chauffage constante ($19$-$19$-$19$).

L’impact de la ventilation est plus subtil. Le scénario de ventilation $90-90$
augmente le $F_{sol}^{CH}$ (\SI{\approx +3}{\percent}) mais réduit le $F_{sol}^{ECS}$
(\SI{\approx -1}{\percent}) en comparaison avec le scénario de référence qui considère un
réduit en inoccupation ($90-20$). Un débit minimal réduit en inoccupation ($90-20$) augmente
alors la $Conso_{app}$ pour le chauffage (\SI{\approx +15}{kWh}). Cependant la
$Conso_{app}$ diminue (\SI{\approx -8}{\kilo\watt\hour}) car la part
électrique couvrant la production d’\abr{ECS} diminue. En effet, lorsque le débit de ventilation
est plus élevé le \abr{SSC} est capable de fournir plus d’énergie pour le chauffage au
détriment de l’\abr{ECS} expliquant cet équilibre. Il est aussi intéressant de noter que le
rendement des capteurs augmente si on considère un débit de ventilation constant
($90-90$). Au regard des différentes remarques il semble donc que
favoriser en priorité la production d’\abr{ECS} permet de mieux valoriser l’énergie solaire.

Enfin il apparaît sans surprise que la diminution des apports internes, réduction par deux
des charges internes électriques (équipements et éclairage), impacte fortement l’ensemble
des indicateurs. En effet les charges internes représentent la majorité des consommations
pour des bâtiments dont l’enveloppe est très performante (\figref{fig:besoins_charges_internes}).
Il est donc important de tenir compte des scénarios d’éclairage, de l’utilisation des
équipements et des occupants pour analyser la performance d’un \abr{SSC} pour des
\abr{BEPOS}.

\begin{figure}
    \centering
    \includegraphics[width=\textwidth]{Ressources/Images/Parametrique/parametric_consigne.pdf}
    \caption[Impact de la température de consigne sur la performance du \abr{SSC}]
            {Impact de la température de consigne sur la performance
             du \abr{SSC} à Strasbourg durant la période de chauffage (01/10 - 30/04).}
    \label{fig:impact_temp_consigne}
\end{figure}

\begin{figure}
    \centering
    \includegraphics[width=0.9\textwidth]{Ressources/Images/Parametrique/besoins_charges_internes.pdf}
    \caption[Évolution annuelle des charges internes et des besoins du bâtiment]
            {Évolution annuelle des charges internes et des besoins du bâtiment
             pour la simulation de référence~: (haut) Bordeaux, (bas) Strasbourg.}
    \label{fig:besoins_charges_internes}
\end{figure}
% subsubsection impact_des_scenarios_internes (end)


% - - - - - - - - - - - - - - - - - - - - - - - - - - - - - - - - - - - - - - -
\subsubsection{Impact des ballons} % (fold)
\label{ssub:impact_des_ballons}
Outre la surface de capteur, le choix du volume des ballons, stockage comme sanitaire,
reste un élément déterminant lors du dimensionnement d’un \abr{SSC}. Cette section discute
l’impact du volume mais aussi de la position de l’échangeur solaire. Pour rappel, la
position de l’échangeur est définie de manière relative afin d’être indépendante des
variations réalisées sur le volume du ballon.

Le climat de Strasbourg est retenu afin d’illustrer les résultats et ceux sur Bordeaux
sont ensuite discutés. Au regard des résultats, augmenter le volume du ballon sanitaire
(\num{300} à \SI{450}{\litre}) améliore les indicateurs principaux (le $F_{sol}^{ECS}$, le
$F_{sol}^{CH}$, et la $Conso_{app}$). Cependant la part électrique pour le chauffage
augmente. Au niveau du ballon de stockage, un volume plus important améliore le $F_{sol}^{CH}$
et réduit la $Conso_{app}$ ainsi que le $F_{sol}^{ECS}$. Dans les deux cas, le $\eta_{sol}$
augmente traduisant une meilleure valorisation de l’énergie incidente disponible.
De plus les températures des ballons ($T3$, $T4$, $T5$) fluctuent plus doucement lorsque le
volume est important. Réduire le volume des deux ballons (\num{300} à \SI{150}{\litre})
impacte négativement les performances du \abr{SSC} mais l’inverse améliore l’ensemble des
indicateurs. Aussi, le volume du ballon sanitaire importe plus que celui du ballon de
stockage et un volume de \SI{450}{\litre} semble préférable. Pour le ballon de stockage,
un volume compris entre \num{300} et \SI{450}{\litre} est suffisant. La même analyse sur
le climat de Bordeaux traduit un comportement similaire. Cependant, la $Conso_{app}$
varie de manière plus importante.

Il est aussi mis en évidence un élément important lors de l’évaluation d’un \abr{SSC}. La
couverture solaire ($F_{sol}$), ne permet pas de clairement évaluer la performance d’un
système solaire. En effet au regard des résultats, faire varier le volume du ballon de
stockage ou sanitaire n’a pas d’effet (\figref{fig:importance_chauffage_ecs}). Il est
nécessaire, comme cette analyse le fait, de considérer séparément le $F_{sol}^{ECS}$ et
$F_{sol}^{CH}$ afin de comprendre que le système favorise dans un cas le chauffage et dans
l’autre la production en \abr{ECS}. Le système est donc capable de s’adapter à
différentes conditions de fonctionnement.

\begin{figure}
    \centering
    \includegraphics[width=0.9\textwidth]{Ressources/Images/Parametrique/importance_chauffage_ecs.pdf}
    \caption[Évolution mensuelle de la répartition des consommations à Bordeaux]
            {Évolution mensuelle de la répartition des consommations et du taux
             de couverture solaire ($F_{sol}$) à Bordeaux (01/10 - 30/04).}
    \label{fig:importance_chauffage_ecs}
\end{figure}

L’analyse de l’impact de la position de l’échangeur est illustrée à travers la
\figref{fig:impact_pos_ech}. Il est observé que la position de l’échangeur impacte
fortement les performances du \abr{SSC} ($F_{sol}^{CH}$ et $F_{sol}^{ECS}$), en particulier
lorsque le volume du ballon est faible. Pour les deux climats, une position basse
(\num{0.8}) est à favoriser afin d’améliorer le rendement des capteurs et réduire la part
couverte par l’appoint. Il est cependant noté qu’une position plus élevée (\num{1.3})
permet de réduire sensiblement la $Conso_{app}$ sur le chauffage.
Enfin contrairement au volume du ballon dont l’impact est plus significatif sur Bordeaux, le
climat de Strasbourg est plus sensible à la position de l’échangeur solaire du ballon
sanitaire ($Ech_{sol}^{pos}$). En effet, la position de l’échangeur
est un facteur influençant directement la production solaire valorisée ($Prod_{sol}^{valorisée}$). Il est donc clair que le gain potentiel
est plus notable lorsque l’ensoleillement est moins important comme à Strasbourg.
Finalement l’impact du matériau utilisé pour l’échangeur a aussi été évalué et sa
composition n’impacte pas les performances du \abr{SSC}.
\textbf{}
En résumé, la position de l’échangeur et le volume des ballons sont des facteurs
impactant la performance du \abr{SSC}. Il semble aussi plus important de favoriser
le volume du ballon sanitaire et de positionner en partie basse l’échangeur afin
d’améliorer la $Prod_{sol}^{valorisée}$.

\begin{figure}
    \centering
    \includegraphics[width=\textwidth]{Ressources/Images/Parametrique/parametric_echangeur.pdf}
    \caption[Impact de la position de l’échangeur solaire sur la performance du \abr{SSC}]
            {Impact de la position de l’échangeur solaire sur la performance
             du \abr{SSC} à Strasbourg durant la période de chauffage (01/10 - 30/04).}
    \label{fig:impact_pos_ech}
\end{figure}
% subsubsection impact_des_ballons (end)


% - - - - - - - - - - - - - - - - - - - - - - - - - - - - - - - - - - - - - - -
\subsubsection{Impact du profil de puisage} % (fold)
\label{ssub:impact_du_profil_de_puisage}
Dans cette avant dernière partie, l’impact des différents profils de puisage comme du
volume de puisage est discuté. Puis, la prise en compte de coefficients modulateurs,
mensuels et hebdomadaires est aussi détaillée. Pour rappel, les différents profils ne
modifient pas la quantité puisée mais uniquement sa répartition dans le temps.

Les résultats obtenus pour Strasbourg montrent que le scénario de puisage n’impacte pas
les différents indicateurs même si le scénario \textbf{Réaliste} améliore sensiblement la
$Prod_{sol}^{valorisée}$ (\figref{fig:impact_profil_puisage}). Les résultats des
simulations faisant varier le volume puisé montrent que le \abr{SSC} est capable de
fournir plus d’énergie solaire pour couvrir l’augmentation des besoins
en \abr{ECS}. Si on compare les deux consommations extrêmes (\num{27} et
\SI{40}{\litre/(jour\period pers)}\,(\SI{60}{\celsius})), la $Prod_{sol}^{valorisée}$
augmente de \SI{248}{\kilo\watt\hour} et la consommation de l’appoint de \SI{390}{\kilo\watt\hour}.
Ainsi, le système s’adapte sans problème à différent profils de puisage et le
$F_{sol}^{ECS}$ et le $F_{sol}^{CH}$ restent élevés même lors que les besoins en \abr{ECS} augmentent.

L’utilisation des coefficients mensuels n’affecte ni le $F_{sol}^{CH}$, ni la $Conso_{app}$
électrique sur le chauffage. Cependant le $F_{sol}^{ECS}$ est fortement impacté, montrant une
nouvelle fois l’importance d’évaluer de manière séparée le chauffage et la production
d’\abr{ECS}. Sur Bordeaux où l’ensoleillement est plus propice, la variation est moindre. Ces
résultats montrent ainsi que les habitudes des occupants (prendre une douche plus longue
ou plus chaude en hiver\dots), doivent être pris en compte afin d’évaluer correctement la
performance du \abr{SSC}. Ils montrent aussi que le choix du profil journalier n’est pas impactant lors
de l’évaluation du \abr{SSC}.

\begin{figure}
    \centering
    \includegraphics[width=\textwidth]{Ressources/Images/Parametrique/parametric_puisage.pdf}
    \caption[Impact du profil de puisage sur la performance du \abr{SSC}]
            {Impact du profil de puisage sur la performance
             du \abr{SSC} à Strasbourg durant la période de chauffage (01/10 - 30/04).}
    \label{fig:impact_profil_puisage}
\end{figure}
% subsubsection impact_du_profil_de_puisage (end)


% - - - - - - - - - - - - - - - - - - - - - - - - - - - - - - - - - - - - - - -
\subsubsection{Autres variations} % (fold)
\label{ssub:autres_variations}
Cette dernière partie discute des variations restantes qui sont plus communes lors
de l’évaluation d’un \abr{SSC}~: le nombre ou la surface des capteurs, leur
orientation et inclinaison, et leur performance au regard des caractéristiques techniques.
Enfin, les variations algorithmiques sont aussi discutées.

L’inclinaison des capteurs jouent un rôle important dans la caractérisation de la
performance d’un système \abr{SSC} et affectent l’ensemble des indicateurs. La $Conso_{app}$
est ainsi réduite lorsque l’inclinaison des capteurs augmente. Le choix d’une inclinaison
importante est en effet fortement bénéfique. Durant la période hivernale, une inclinaison
importante (\num{45} ou \SI{60}{\degree}) permet d’augmenter l’énergie récupérée par les
capteurs et réduit les risques de surchauffe en période estivale. Sans surprise, le
nombre de capteurs impacte fortement la consommation de l’appoint. À Strasbourg, une
variation de surface de capteurs de \SI{14}{\meter\squared} (\num{2} à \num{8} capteurs)
fait varier le $F_{sol}^{CH}$ de \num{11} à \SI{52}{\percent} et la $Conso_{app}$ varie de
\num{2232} à \SI{1133}{\kilo\watt\hour}. L’orientation joue aussi un rôle important sur la
performance du \abr{SSC}. À surface équivalente, des capteurs orientés au sud produisent deux
fois plus que des capteurs à l’est.

Au niveau des variations algorithmiques, les temporisations n’impactent que faiblement
le \abr{SSC}. Augmenter la $Tempo_{souff}$ (\SI{15}{min}) améliore le $F_{sol}^{ECS}$ mais dégrade
le $F_{sol}^{CH}$. La $Tempo_{batterie}$ elle, n’a pas d’impact notable. Le $\Delta T_{sol}$
impacte le système solaire de manière modérée~: l’augmenter dégrade à la fois le rendement
solaire, le $F_{sol}^{ECS}$, et le $F_{sol}^{CH}$. À l’opposé le réduire permet de valoriser une plus
grande quantité d’énergie solaire. Enfin, autoriser une sur-chauffage par le solaire
durant la journée (consigne solaire à \SI{22}{\celsius}) améliore le $F_{sol}^{CH}$ mais dégrade
de manière plus importante le $F_{sol}^{ECS}$. Ainsi, la $Conso_{app}$ augmente. Ce résultat est
cohérent avec les observations faites lors de l’évaluation des variations sur la
ventilation (\ref{ssub:impact_des_scenarios_internes}) et montrent que l’énergie solaire
est mieux valorisée si elle est attribuée en priorité pour la production d’\abr{ECS}.

En résumé, les capteurs solaires thermiques doivent être placés en priorité sur un pan sud
avec une inclinaison supérieure de \SI{15}{\degree} par rapport à la latitude du lieu afin
d’améliorer les performances du \abr{SSC} sur l’ensemble des indicateurs. Il apparaît aussi
que le $\Delta T_{sol}$ impacte le système et doit donc être considéré. À l’opposé la surchauffe
diurne impacte négativement le bilan global du système. Cependant, elle permet de favoriser le
chauffage ou apporter un confort thermique hivernal plus important sans surcoût.
% subsubsection autres_variations (end)
% subsection analyse_des_resultats (end)
% section etude_parametrique (end)





% ..............................................................................
% ..............................................................................
\section{Vers une méthodologie d’aide à la décision} % (fold)
\label{sec:vers_une_methodologie_d_aide_a_la_decision}
Au cours de ce chapitre, la démarche entreprise a été explicitée. Dans l’optique de mieux
comprendre les interactions entre le bâtiment et le \abr{SSC}, un modèle comprenant
systèmes et bâtiment a été modélisé avec le langage \textit{Modelica}. Pour le bâtiment, un
modèle mono-zone a été retenu suite à une comparaison avec une approche multi-zones.
Pour le \abr{SSC}, une approche détaillée a été retenue afin de tenir compte de la
complexité de la logique de contrôle. En effet un algorithme innovant et robuste a été
utilisé afin d’orchestrer le fonctionnement couplé du bâtiment et du \abr{SSC}.
Finalement, une étude paramétrique a permis de mieux comprendre le comportement du
\abr{SSC}.

De nombreux paramètres dont la littérature avait montré l’impact ont été identifiés dans
ces travaux et les conclusions sont similaires. Cependant d’autres paramètres et d’autres
conclusions ont aussi été obtenues. Ce \abr{SSC} est robuste, faire varier la température de
consigne, supprimer le réduit de nuit, ou encore modifier la répartition des puisages
journaliers impacte de manière très sensible sa performance. Aussi, il a été mis en exergue que
le $F_{sol}^{ECS}$ et le $F_{sol}^{CH}$ ne sont pas influencés de manière similaire en
fonction des variations techniques. Le système est en effet capable de s’adapter~: dans
certains cas le $F_{sol}^{ECS}$ est favorisé, dans d’autre c’est le $F_{sol}^{CH}$. Ce
comportement est aisément observable lorsque le volume des ballons est modifié et que
$F_{sol}$ demeure intact mais où $F_{sol}^{ECS}$ et $F_{sol}^{CH}$ subissent de fortes
variations. Il est donc indispensable de considérer de manière indépendante la performance
du système solaire sur le chauffage ($F_{sol}^{CH}$) et sur la production d’\abr{ECS}
($F_{sol}^{ECS}$). Il a aussi été mis en évidence que l’énergie solaire est mieux
valorisée lorsque elle est en priorité utilisée afin de charger le ballon
sanitaire. Ceci a en effet été noté à plusieurs reprises~: lors de la charge diurne du
bâtiment afin de réduire la $Conso_{app}$, ou bien lors de la variation du débit de
ventilation. Cette étude a donc permis de confronter le modèle aux variations techniques
communes dans l’étude des \abr{SSC} mais aussi d’apprendre plus à travers d’autres variations
au niveau du bâtiment, du \abr{SSC} ou de son algorithme. Les résultats tendent à
montrer que le \abr{SSC} permet d’obtenir une couverture importante sur le chauffage comme la
production \abr{ECS} lorsque un algorithme détaillé est utilisé. Il est donc propice à
l’utilisation pour des bâtiments à faible consommation.

Fort de ces résultats, trois indicateurs principaux ont pu être identifiés. La performance
d’un \abr{SSC} nécessite l’optimisation de la $Prod_{sol}^{valorisée}$ ainsi que de la
$Conso_{app}$ pour le chauffage et la production d’\abr{ECS}. Considérer uniquement la
$Conso_{app}$ ne permet en effet pas de tenir compte des influences réciproques entre le
\abr{SSC} et le bâtiment illustrées à travers l’étude paramétrique. Dans l’optique d’une
aide à la décision, il apparaît donc important de considérer à la fois la $Conso_{app}$,
le $F_{sol}^{ECS}$, et le $F_{sol}^{CH}$. Cependant la méthodologie utilisée pour
l’obtention de ces premiers résultats est guidée par l’expérience et son processus
itératif est couteux en temps humain. De plus cette approche ne permet pas de garantir
l’obtention de solutions optimales. Pour ces raisons, une méthodologie d’aide à la
décision tenant compte du caractère multi-objectifs propre à la conception de \abr{MEPOS}
solaire est introduite dans le chapitre qui suit. L’approche est automatisée et
n’introduit pas de préférence a priori issue de l’expérience. Elle permet donc d’optimiser
la puissance de calcul disponible libérant le temps humain pour la prise de décision.
% subsection bilan (end)
% section vers_une_methodologie_d_aide_a_la_decision (end)
