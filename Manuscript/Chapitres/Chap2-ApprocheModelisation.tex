%!TEX root = ../main.tex
% Chapitres\Chap2-ApprocheModelisation.tex

\iunsure{À mettre dans le premier chapitre}
\itodo{Ajouter citations, voir article BS 2017}
De nombreuses études ont déjà cherché à évaluer la performance d’un système solaire
combiné à l’aide de méthodes plus ou moins détaillées. L’approche la plus répandue utilise
les besoins mensuels de la maison générés grâce à un modèle du bâtiment simplifié
\parencite{Raffenel2009657,Martinopoulos2014130}. Cette approche néglige le comportement dynamique du système
solaire comme du bâtiment. D’autres approches utilisent un modèle de bâtiment plus
détaillé (TrnSys, Energy Plus) permettant d’évaluer plus précisément la couverture du
système \parencite{Glembin2012601}. Ces approches permettent de tenir compte de l’évolution dynamique du système
solaire combiné (SSC) mais négligent les interactions entre le bâtiment et les systèmes. Le
bâtiment est seulement considéré comme une consommation (chauffage et production d’ECS) et
le système solaire comme une source potentielle répondant à cette demande. Cette approche
est aussi celle retenue pour la Task 26 \parencite{Task262003} au cours de laquelle de nombreux modèles
de SSC ont été développés puis validés.

Dans ces travaux, une approche détaillée de l’algorithme de contrôle et des systèmes est retenue. En
effet, plusieurs études ont mis en évidence l’importance de la modélisation du contrôle
sur la performance d’un système solaire combiné \parencite{Kicsiny20123489,Huang20123278}.
Afin de présenter des résultats pouvant être obtenues sur des bâtiment réels pour la
prochaine réglementation thermique (\mtodo{Ajouter citation}{RT 2020}), un algorithme existant et innovant
a été utilisé. Celui-ci est modifié et adapté pour répondre aux contraintes du projet~: obtenir
un bâtiment réactif en utilisant l’énergie solaire.
L’originalité de l’approche réside principalement dans l’évaluation couplée du système
solaire combiné et du bâtiment. Dans cette optique les interactions bâtiment / systèmes et
systèmes / bâtiment sont pris en compte et évaluées.

Dans un premier temps des outils retenues pour la modélisation et l’analyse des résultats.
Dans un second temps le modèle SSC développée ainsi que le bâtiment et les scénarios, sont
discutés. Finalement une étude paramétrique est réalisée. Elle permettra de mieux
comprendre les interactions existantes entre bâtiment et système et de fixer les scénarios
qui seront retenues pour l’aide à la décision. De plus les indicateurs nécessaires à la
caractérisation d’un SSC seront identifiés.


% ..............................................................................
% ..............................................................................
\section{Modélisation détaillée d’un système solaire innovant} % (fold)
\label{sec:modelisation_detaillee_d_un_systeme_solaire_innovant}
% ------------------------------------------------------------------------------
\subsection{Quelles contraintes} % (fold)
\label{sub:quelles_contraintes}
Il existe plusieurs moyens permettant d’évaluer un système énergétique. Le premier
consiste à reproduire expérimentalement le système et son environnement. Ce processus est
coûteux, particulièrement si le système est évalué à l’échelle du bâtiment. De plus les
conditions extérieures ne peuvent être contrôlées et une variation dans la structure du
système est complexe à mettre en place. Pour ces raisons une approche par modélisation a
été retenue. Elle permet de contrôler l’ensemble des composants du système comme son
algorithme de contrôle, les conditions limites (météo) ou encore les composants du
système. Grâce à cette approche il est alors possible d’évaluer le système modélisé de
manière totalement libre. La modélisation est donc un outil adapté pour une
étude de faisabilité, un dimensionnement, ou encore optimiser un système. Il existe de
nombreux langages ou logiciels permettant de réaliser des simulations plus ou moins
complexes. Il est ainsi nécessaire dans un premier temps de définir le niveau de précision
souhaité pour les différentes parties du modèle. Notre cas d’étude se place à
l’échelle du bâtiment mais l’on souhaite aussi conserver un contrôle important sur la
gestion des équipements.

Il est aussi important de déterminer le niveau de contrôle que l’on souhaite avoir sur
chaque composant du modèle. On distingue principalement 2 familles d’outils~: les modèles
opaques (boîtes noires) qui masquent à l’utilisateur le fonctionnement internes, et les
modèles ouverts (boîtes blanches) dont la composition de chaque bloc est connue et est
modifiable. Notre étude portant sur l’évaluation et l’amélioration d’un système solaire
combiné, il est nécessaire de pouvoir contrôler chaque partie du modèle afin de pouvoir
analyser et comprendre la relation entre la variation d’un paramètre au niveau du
composant et son impact global sur le système. La modélisation d’un système détaillé
demande de nombreuses itérations, dans un premier temps pour modéliser le système, et dans
un second temps pour évaluer l’impact d’un ensemble de variation. Le processus de
développement du modèle étant itératif, l’outil sélectionné doit permette de modifier,
ajouter, et supprimer rapidement et simplement des composants. Finalement le problème à
modéliser est un problème multi- physique et l’outil doit être adapté à la modélisation
d’un système hydraulique (solaire), d’un système aéraulique (ventilation), d’un algorithme
de contrôle (régulation) et d’un bilan thermique (bâtiment).
% subsection quelles_contraintes (end)



% ------------------------------------------------------------------------------
\subsection{Quelles solutions} % (fold)
\label{sub:quelles_solutions}
Le langage \textit{Modelica} et le logiciel \textit{Dymola} (Dynamics MOdeling LAboratory)
permettent de répondre à l’ensemble des contraintes décrites dans la section précédente.
\textit{Modelica} est un langage de programmation libre et ouvert développé pour répondre
aux contraintes de la modélisation multi-physique. Il a été pensé pour être intuitif et
offre une approche équationnelle et orientée objet au développeur. L’approche objet est
très intuitive et permet d’encapsuler un ensemble de données et d’offrir des interfaces
pour accéder à ces données. Un modèle est ainsi une combinaison d’un ensemble de sous-
modèles (composition) ou certains sous-modèles peuvent être améliorés afin de leur ajouter
une spécialisation (héritage). Enfin le langage est acausal
(Définition~\ref{def:acausal}), permettant au développeur de de rapidement créer des
prototypes de systèmes complexes sans devoir à chaque modification réécrire les systèmes
d’équations.


\begin{Def}[Acausale]\label{def:acausal}
Un modèle est dit acausal lorsque le sens de la définition d’une équation n’est pas
imposée (Fig~\ref{fig:acausal_vs_causal}). Le solveur va déterminer par lui-même les
paramètres dont il connaît la valeur et réécrire les systèmes d’équations afin d’exprimer
les inconnues.
\end{Def}

\begin{figure}
    \begin{center}
        \includegraphics{Ressources/Images/Modelisation/composant_vs_bloc.png}
    \end{center}
    \caption{Différences entre modélisation acausale (par composant) et causal (par bloc).
             \label{fig:acausal_vs_causal}}
\end{figure}

\subsubsection{Dymola} % (fold)
\label{ssub:dymola}
~
\itodo{http://www.wolfram.com/system-modeler/features/}
\textit{Dymola} est une suite de logiciels développée par \textit{Dassault Systèmes}
permettant de simplifier le développement de modèles numériques grâce au langage
\textit{Modelica}. Il offre une interface graphique permettant de connecter les modèles
entre eux de manière intuitive (Fig~\ref{fig:exemple_modelica}). Il offre aussi un outil
puissant de "refactoring" permettant de rendre transparent la modification du nom d’une
variable et sa propagation sur l’ensemble des modèles de la bibliothèque. Il offre aussi
un outil de post- traitement afin d’évaluer les résultats des simulations. Finalement il
intègre le logiciel \textit{Dymosim} qui offre un large choix de solveurs pour simuler les
modèles. \textit{Dymosim} supporte aussi la parallélisation, les \emph{FMU}
(Functional Mock-up Unit) et la génération de modèles autonomes.
\begin{figure}
    \begin{center}
        \includegraphics{Ressources/Images/Modelisation/exemple.PNG}
    \end{center}
    \caption{Modélisation d’un échange thermique entre deux masses en Modelica.
             \label{fig:exemple_modelica}}
\end{figure}

\iunsure{Parler de buildingspy}

% subsubsection dymola (end)


\subsubsection{Buildings} % (fold)
\label{ssub:buildings}
~
\itodo{Détailler buildings}
Le couplage de Modelica et de Dymola permet le développement itératif nécessaire à notre
problème et offre un contrôle total sur chaque partie du modèle répondant ainsi
parfaitement à nos contraintes. Le langage Modelica étant en pleine croissance dans le
secteur du bâtiment, de nombreuses bibliothèques open source ont été développées dont la
liste peut être trouvée sur le site officiel de l’association Modelica. Dans notre étude
nous avons utilisé la bibliothèque Buildings qui est développée par le Laboratoire
National Lawrence Berkeley (LBNL). C’est une bibliothèque libre et ouverte, développée
pour la modélisation des systèmes du bâtiment (hydraulique, aéraulique, électrique,
thermique, ...).
% subsubsection buildings (end)
% subsection quelles_solutions (end)
% section modelisation_detaillee_d_un_systeme_solaire_innovant (end)




% ..............................................................................
% ..............................................................................
\section{Description du cas d’étude~: le bâtiment} % (fold)
\label{sec:description_du_cas_d_etude_le_batiment}
% ------------------------------------------------------------------------------
\subsection{Modélisation mono-zone} % (fold)
\label{sub:modelisation_monozone}
La maison (Fig~\ref{fig:plan_maison}) a une surface habitable de \SI{98.4}{\meter\squared}
et est de plain-pied.
Elle comporte 3 chambres, une cuisine/salon, et un local technique où se
trouvent les équipements du système solaire.

\itodo{Vérifier ce que je dis}
Le modèle utilisé pour sa modélisation a été validé à travers une suite de test (\mtodo{BestTestCase}).
Ces tests permettent de vérifier que l’évolution des températures au cours de l’année est cohérent.
En plus de l’espace chauffée, les combles et le vide sanitaire ont été modélisés
comme des conditions limites (Fig~\ref{fig:modelisation_maison}). Le modèle utilisé est une version
simplifié du modèle de zone disponible dans la bibliothèque \textit{Buildings}.


La température dans l’ensemble de la zone chauffée est ainsi considérée uniforme.
\begin{figure}
    \begin{center}
        \includegraphics{Ressources/Images/Batiment/plan.png}
    \end{center}
    \caption{Plan du bâtiment utilisé à travers ces travaux.
             \label{fig:plan_maison}}
\end{figure}

\begin{figure}
    \begin{center}
        \includegraphics{Ressources/Images/Modelisation/maison.png}
    \end{center}
    \caption{Représentation du bâtiment sous Modelica.
             \label{fig:modelisation_maison}}
\end{figure}

Les caractéristiques thermiques des parois opaques sont décrites en fonction de
leur orientation (\autoref{tab:paroi_opaque}) et une description détaillée et exhaustive est
disponible en annexe (\mtodo{ref tableau annexe}).
\begin{table}
\centering
\begin{tabular}{l*{3}{c}}
    \toprule
               & $U$                                       & $Surface$             & $U \times Surface$     \\
               & \si{\watt\per(\meter\squared\period\kelvin)} & \si{\meter\squared}    & \si{\watt\period\kelvin}  \\
    \midrule
    Murs       & \num{0.174}                                     & \num{91.17}                 & \num{15.864}                 \\
    Plancher   & \num{0.110}                                     & \num{98.40}                 & \num{10.824}                 \\
    Plafond    & \num{0.123}                                     & \num{97.06}                 & \num{11.938}                 \\
    \bottomrule
\end{tabular}
\caption{Description de la performance des parois opaques.}
         \label{tab:paroi_opaque}
\end{table}


Dans le cadre de ces travaux un couplage fort entre bâtiment, systèmes, et algorithme
de contrôle est envisagé. Dans cette optique, il est nécessaire de déterminer quel
est le niveau de détail nécessaire pour chaque élément. Dans le cas du bâtiment, une
approche mono-zone a été retenue après comparaison avec les besoins calculés
avec un modèle zonal dont le détail des résultats sont présentés à la fin de cette section.



% - - - - - - - - - - - - - - - - - - - - - - - - - - - - - - - - - - - - - - -
\subsubsection{Déperditions à travers les fenêtres} % (fold)
\label{ssub:déperditions_à_travers_les_fenêtres}
Une attention particulière a été apportée à la description des vitrages afin
d’être cohérent entre le modèle multi-zone et mono-zone et de satisfaire les normes
de calculs implicites de la bibliothèque \textit{Buildings}.

Le modèle de zone de la bibliothèque \textit{Buildings} sépare le calcul de la part
transmise en conductif et en radiatif~: les caractéristiques des vitrages doivent alors
être renseignées de manière détaillées. Les différents vitrages sont des vitrages doubles
avec une lame d’air ou d’argon et une couche faiblement émissive est ajoutée pour les
vitrages des parois verticales (\mtodo{Ajouter ref tab}). Il est considéré pour l’ensemble
des vitrages, une composition similaire (Planilux SGG \SI{4}{mm} - Argon - Planitherm XN
\SI{4}{mm}), exception faite de la fenêtre de toit. Cependant les fabricants fournissent
seulement les caractéristiques globales~: le coefficient de transmission thermique $U_{g}$
et le facteur solaire $g$. Le logiciel \textit{Windows 7.4} a donc été utilisé afin
d’obtenir la composition détaillée des vitrages retenus.

Dans un premier temps, le vitrage est construit à partir de la composition décrite dans sa
fiche technique puis les coefficients caractéristiques sont calculés. Afin de pouvoir
comparer avec les données fabricants, les conditions limites des normes européennes
\mtodo{Ref}{EN 673-2011} et \mtodo{Ref}{EN 410-2011} sont implémentées dans le logiciel.
En effet, le choix des conditions limites (normes) impactent fortement le résultat
(\mtodo{Ajouter ref tab}) et il est observé une variation moyenne de \SI{20}{\percent}
entre les normes européennes et états-uniennes (\mtodo{Ajouter ref}).
\ttodo{Ajouter comparaison entre calcul selon diverses normes}

\itodo{Demander à Aurélie les informations relatives aux ponts thermiques}
Finalement, la bibliothèque \textit{Buildings} ne permettant pas l’ajout de ponts
thermiques, le choix a été fait de les intégrer dans le coefficient thermique
caractéristique du cadre, $U_{f}$. Une description détaillée et exhaustive de la
composition des fenêtres est disponible en annexe (\mtodo{ref tableau annexe}).
% subsubsection déperditions_à_travers_les_fenêtres (end)


% - - - - - - - - - - - - - - - - - - - - - - - - - - - - - - - - - - - - - - -
\subsubsection{Description des infiltrations} % (fold)
\label{ssub:description_des_infiltrations}
Les infiltrations ont été définies en utilisant une perméabilité de
\SI{0.4}{m^{3}\per(\hour\period\meter\squared)}. Le calcul tient compte de la
somme des surfaces dites froides correspondant à la somme des surfaces donnant
vers l’extérieur à l’exception du plancher bas \eqref{eq:infiltrations}.

\begin{align}
    Infiltrations &= 0.4 \times (Parois_{Verticales} + Parois_{velux} + Plafond)\\
    &              \backsimeq \SI{85}{m^{3}/h}
    \label{eq:infiltrations}
\end{align}
% subsubsection description_des_infiltrations (end)


\subsubsection{Limitations du modèle} % (fold)
\label{ssub:limitations_du_modele}
Certains choix ont été fait du aux limitations de modélisation~:
\begin{itemize}
    \item Les coefficients de convection intérieurs utilisés sont les mêmes pour toutes
          les parois (\SI{7.7}{\watt\per(\meter\squared\period\kelvin)}) et le coefficient extérieur
          est fixé à \SI{25}{\watt\per(\meter\squared\period\kelvin)}, une valeur couramment retenu
          dans la littérature (\mtodo{Ajouter ref}).
    \item Les ponts thermiques ont été définies comme une surface de déperdition
          équivalente à \SI{13.2}{\meter\squared} (\mtodo{Récup détail Aurélie}).
    \item La fenêtre de toit est placée horizontalement et non à \SI{33}{\degree}
          (inclinaison réelle) car la bibliothèque \textit{Buildings} ne supporte pas les
          surfaces vitrés inclinées. Le facteur solaire utilisé est cependant celui défini
          dans la fiche technique du fabricant (\munsure{Ajouter ref velux}).
\end{itemize}
% subsubsection limitations_du_modele (end)


% - - - - - - - - - - - - - - - - - - - - - - - - - - - - - - - - - - - - - - -
\subsubsection{Comparaison avec un modèle multi-zone} % (fold)
\label{ssub:comparaison_avec_un_modele_multi_zone}
~
\iunsure{À mettre après présentation des scénarios}
Une comparaison a été réalisée
entre un modèle multi-zone développé sous \textit{Energy Plus} par le
\textit{CEA} et un modèle mono-zone développé sous \textit{Dymola} en utilisant
la bibliothèque \textit{Buildings}. Le fichier météo utilisé est celui de
Bordeaux et est de type IWEC (\mtodo{Ajouter lien}{IWEC - WMO 075100}). Une
consigne commune de chauffage de \SI{19}{\celsius} a été utilisée et les scénarios
harmonisés. L’étude ne portant pas sur l’évaluation du confort estival des
occupants, seule les besoins ont été comparés. Les résultats montrent que les
besoins de l’approche mono-zone et de l’approche multi-zone sont similaires
(\mtodo{Ajouter ref table}).
\itodo{Ajouter résultats entre les deux. Ajouter comparatif des puissances}.
% subsubsection comparaison_avec_un_modele_multi_zone (end)
% subsection modelisation_monozone (end)


% % ------------------------------------------------------------------------------
\subsection{Scénarios de référence} % (fold)
\label{sub:scénarios_de_référence}
% - - - - - - - - - - - - - - - - - - - - - - - - - - - - - - - - - - - - - - -
\subsubsection{Occupation} % (fold)
\label{ssub:profil_d_occupation}
~
\itodo{Récupérer plus d’informations auprès de Aurélie sur le profil d’occupation}
Un profil unique pour l’ensemble de la zone et pour chaque occupant est retenu (\mtodo{ref
fig}). Il est considéré quatre personnes, deux enfants, et deux parents. Dans ce profil,
les occupants sont considérés à la maison durant le week-end complet. Durant les jours
ouvrés, les habitants sont soit à l’école (enfants) soit au travail (parents) exception
faite du mercredi après-midi où la maison est considéré comme occupée.
La puissance par habitant est de \SI{97.5}{\watt} (\SI{70}{\percent} convective), résultat
de la moyenne pondérée de la puissance dégagée par une personne en fonction de la surface
de chaque pièce du modèle multi-zone (\ref{tab:puissance_occupants}).

\ftodo{Profil d’occupation retenue}

\begin{table}
\begin{tabular}{*8{c}}
    \toprule
    Chambre 1 & Chambre 2  & Chambre 3 & Séjour & Cuisine & Sanitaire & SdB   & Cellier \\
    \midrule
    \num{78}        & \num{78}         & \num{117}       & \num{97.5}   & \num{97.5}    & \num{114.3}     & \num{114.3} & \num{114.3}   \\
    \bottomrule
\end{tabular}
\caption{Récapitulatif des puissances dissipées en fonction des pièces.}
\label{tab:puissance_occupants}
\end{table}
% subsubsection profil_d_occupation (end)


% - - - - - - - - - - - - - - - - - - - - - - - - - - - - - - - - - - - - - - -
\subsubsection{Charges internes} % (fold)
\label{ssub:charges_internes}
En plus des occupants, les charges internes dues aux équipements sont pris en compte.
Il est entendu comme charges internes, les consommations des équipements électriques
(électroménager, ordinateurs, ...) et la consommation de l’éclairage. Aucunes informations
n’étant disponible afin d’estimer une consommation représentative de la maison réelle,
les consommations réglementaires issues de la réglementation thermique (\emph{RT 2012}, \cite{CSTB2011})
ont été retenues. Pour les deux cas il est ainsi considéré un consommation type durant
l’occupation et une consommation réduite en inoccupation.

\paragraph{Équipements électriques~:} % (fold)
\label{par:equipements_electriques}
La consommation est fixée à \SI{5.7}{\watt\per m^{2}} (\SI{80}{\percent}
convective) et à \SI{1.14}{\watt\per m^{2}} durant l’inoccupation, soit une
réduction de \SI{80}{\percent} (\mtodo{Ref fig}).
% paragraph equipements_electriques (end)

\paragraph{Éclairage~:} % (fold)
\label{par:eclairage}
La consommation est fixée à \SI{1.4}{\watt\per m^{2}} (\SI{42}{\percent}
convective) de \SI{7}{\hour} à \SI{9}{\hour} et de \SI{19}{\hour} à \SI{22}{\hour}
(\mtodo{Ref fig}). Durant le week-end, la consommation est réduit de \SI{50}{\percent},
soit
\SI{0.7}{\watt\per m^{2}} comme le prévoit la réglementation thermique (RT 2012)
de \SI{10}{\hour} à \SI{19}{\hour}. Finalement, en inoccupation la consommation est
considérée comme nulle.
% paragraph eclairage (end)

\ftodo{Ajouter figure des profil de charges internes (équipements et éclairage)}
% subsubsection charges_internes (end)


% - - - - - - - - - - - - - - - - - - - - - - - - - - - - - - - - - - - - - - -
\subsubsection{Consigne de chauffage} % (fold)
\label{ssub:consigne_de_chauffage}
Le profil de température de référence (\mtodo{ref profil chauffage}) est issue de la réglementation thermique (RT
2012) qui prévoit le maintien d’une température de \SI{19}{\celsius} en occupation, un
réduit de \SI{16}{\celsius} en inoccupation, et autorise un abaissement de la consigne durant la
période nocturne~: \SI{18}{\celsius} est retenue. Le profil de la consigne de chauffage
est donc fonction du scénario d’occupation retenu (\mtodo{ref occupation}).
% subsubsection consigne_de_chauffage (end)

% - - - - - - - - - - - - - - - - - - - - - - - - - - - - - - - - - - - - - - -
\subsubsection{Ventilation} % (fold)
\label{ssub:ventilation_ref}
Le profil de ventilation de référence est considéré à \SI[per-mode=symbol]{90}{\meter\cubed\per\hour}
comme le prévoit l’arrêté du \href{https://www.legifrance.gouv.fr/affichTexte.do?cidTexte=JORFTEXT000000862344}{24 Mars
1982 et du 28 Octobre 1983} pour une maison comportant 4 pièces principales.
% subsubsection ventilation (end)
% subsection scénarios_de_référence (end)
% section description_du_cas_d_etude_le_batiment (end)



% ..............................................................................
% ..............................................................................
\section{Description du cas d’étude : le système solaire} % (fold)
\label{sec:description_du_cas_d_etude_le_systeme_solaire}
% ------------------------------------------------------------------------------
\iunsure{Parler du système solaire par vecteur eau}
Dans un premier temps un modèle existant basé sur le vecteur eau a été modélisé.
Le comportement du modèle a ainsi pu être validé par l’entreprise (SolisArt) à
l’origine du système.

\iunsure{Mentionner IGC ou en parler dans le cadre du projet COMEPOS (sous-entendu)}
Le système décrit dans ce chapitre est un SSC utilisant l’air comme vecteur de chaleur
afin d’éviter l’installation d’un système de chauffage conventionnel (vecteur eau). Le
système permet ainsi de couvrir les besoins en $ECS$ et en chauffage pour l’ensemble du
bâtiment par l’intermédiaire du réseau de ventilation. De plus le choix de l’air comme
vecteur de chaleur est motivé par la volonté de l’entreprise \emph{IGC} d’obtenir un
système réactif à travers le projet \textit{COMEPOS}.
Le système modélisé est composé de 3 parties principales~: la partie
hydraulique, la partie aéraulique, et la partie contrôle qui orchestre le fonctionnement
couplé entre le système solaire et le bâtiment (Fig~\ref{fig:air_complet_mono}).

\begin{figure}
    \begin{center}
        \includegraphics{Ressources/Images/Modelisation/air_complet_mono.pdf}
    \end{center}
    \caption{Description schématique du système solaire couplé à la ventilation.
             \label{fig:air_complet_mono}}
\end{figure}

Dans un premier temps le fonctionnement de la partie hydraulique sera détaillé. Dans un
second temps la partie aéraulique sera présentée. Finalement la partie contrôle sera
décrite et le fonctionnement combiné du système et de la maison explicité.


% ------------------------------------------------------------------------------
\subsection{Partie hydraulique} % (fold)
\label{sub:partie_hydraulique}
% - - - - - - - - - - - - - - - - - - - - - - - - - - - - - - - - - - - - - - -
\subsubsection{Description} % (fold)
\label{ssub:description_hydraulique}
~
\itodo{Ajouter description du choix des pompes et de l’équilibrage}
\itodo{Description du fonctionnement des pompes et des débits}
Le système hydraulique (Fig~\ref{fig:air_complet_mono}) est composé de 2 ballons dont les
caractéristiques sont disponibles dans le \autoref{tab:tanks_specs}. Le
premier, le ballon sanitaire, permet de lisser la demande en énergie fou. Le système est
dit semi-accumulée car la réserve d’eau chaude est inférieure à la quantité totale puisée
durant la journée. Le ballon sanitaire est donc connecté en continue avec le réseau d’eau
froide public et certaines contraintes doivent être respectées. Ainsi le ballon doit soit~;
être maintenu à une température minimale de \SI{55}{\celsius}, soit une surchauffe
journalière doit être réalisée
(\href{https://www.legifrance.gouv.fr/affichTexte.do?cidTexte=JORFTEXT000000423756}{Arrêté
du 30 novembre 2005}). La première option a été retenue afin de garantir le respect de la
réglementation dans les limites techniques imposées par la modélisation. Le second, le
ballon de stockage, est utilisé pour stocker l’énergie accumulée durant la journée afin de
la valoriser en période nocturne.

Les deux ballons sont considérés dans la partie chauffée de la maison et leurs
déperditions sont ainsi considérées comme des charges internes au bâtiment. Afin de
garantir le maintien du confort thermique des occupants, un appoint électrique en partie
haute du ballon sanitaire est aussi ajouté. Ce dernier n’est activé que lorsque l’énergie
solaire disponible n’est pas suffisante. L’énergie du soleil est quand à elle transmise à
l’eau par l’intermédiaire de panneaux solaires (\autoref{tab:idmk_specs}) puis vers le
système aéraulique à l’aide d’un échangeur de chaleur.

\begin{table}
\centering
\caption{Caractéristiques techniques des ballons de référence (tampon et sanitaire).}
\label{tab:tanks_specs}
\begin{tabular}{l*{2}{c}r}
    \toprule
    Paramètre & Ballon tampon & Ballon ECS & Unité\\
    \midrule
    Volume                                       & \num{300}   & \num{300}    & \si{\litre}              \\
    Hauteur                                      & \num{1.05}  & \num{1.25}   & \si{\metre}              \\
    Épaisseur isolation                          & \num{100}   & \num{55}     & \si{\milli\metre}             \\
    $\lambda$ isolant                            & \num{0.04}  & \num{0.04}   & \si{W/m^{2}\period K}      \\
    Échangeur haut                               & \num{0.85}  & \num{0.64}   & \si{\metre}              \\
    Échangeur bas                                & \num{0.15}  & \num{0.13}   & \si{\metre}              \\
    Diamètre échangeur (extérieur)               & \num{34.6}  & \num{33.7}   & \si{\milli\metre}             \\
    Chaleur spécifique de échangeur (acier noir) & \num{490}   & \num{490}    & \si{J/kg\period K}         \\
    Puissance nominale                           & \num{103}   & \num{53}     & \si{\kilo\watt}             \\
    Température nominale (ballon)                & \num{10}    & \num{45}     & \si{\celsius} \\
    Température nominale (échangeur)             & \num{45}    & \num{10}     & \si{\celsius} \\
    Débit nominal                                & \num{0.36}  & \num{0.366}  & \si{kg\per\second}           \\
    \bottomrule
\end{tabular}
\end{table}

\begin{table}
\centering
\caption{Caractéristique du collecteur de référence (modèle IDMK 25 de chez Sonnenkraft).}
\label{tab:idmk_specs}
\begin{tabular}{lcr}
    \toprule
    Paramètre                                & Valeur         & Unité                 \\
    \midrule
    Surface nette                            & \num{2.32}           & \si{m^{2}}            \\
    Poids à vide                             & \num{54}             & \si{kg}               \\
    Contenance                               & \num{1.35}           & \si{l}                \\
    $\eta_{0}$                               & \num{78}             & \si{\%}               \\
    Décroissance de performance              & \num{-5.103}         & -                     \\
    Coefficient $a_{1}$ (pertes linéiques)   & \num{3.796}          & \si{W/(m^{2}.K)}      \\
    Coefficient $a_{2}$ (pertes surfaciques) & \num{0.013}          & \si{W/(m^{2}.K^{2})}  \\
    Modulation diffus ($IMDiff$)             & \num{100}            & \si{\%}               \\
    \bottomrule
\end{tabular}
\end{table}


Enfin, le système comporte 3 pompes~: S6, S5, et S2. Il est considéré un débit nominale de \SI{40}{\litre\per(\meter\squared\period\hour)}
pour les pompes S6 et S5 et de \SI{70}{\litre\per(\meter\squared\period\hour)} pour la pompe S2 comme
première approximation \parencite{Peuser2005}. Les pompes utilisées sont des pompes à vitesse variable
afin de réduire les arrêts intempestifs des pompes \parencite{Kicsiny20123489}.

\ftodo{Ajouter lien en schéma et modélisation Modelica. Discuter la simplicité visuelle}

% subsubsection description (end)

% - - - - - - - - - - - - - - - - - - - - - - - - - - - - - - - - - - - - - - -
\subsubsection{Vérification du modèle de capteur solaire} % (fold)
\label{ssub:verification_du_modele_de_capteur_solaire}
Dans l’optique de l’évaluation de la performance  du système solaire couplé à un bâtiment, une
attention particulière doit être apporté au modèle de capteur thermiques. Il est
nécessaire d’utiliser un modèle étant à la fois représentatif du comportement réel
et il doit permettre l’implémentation de différents capteurs en utilisant les informations
disponibles sur les fiches techniques. Ce dernier point exclue une modélisation détaillée
des transferts thermiques. L’approche retenue utilise le modèle disponible dans la
bibliothèque \textit{Buildings} basant son système d’équations sur les relations
empiriques décrites dans le volet 2 de la norme EN\,12975 (\mtodo{Ajouter Ref}{EN\,12975-2}).
Dans cette approche, deux coefficient
permettent de caractériser les pertes thermiques du capteurs : les coefficients de pertes linéiques
et quadratiques notés respectivement $a_{1}$ et $a_{2}$. Deux autre coefficients permettent de
caractériser la part solaire absorbée :  le rendement solaire et le coefficient de modulation
de la part solaire issue du diffus notés respectivement $\eta_{0}$ et $IMDiff$.

Afin de vérifier la pertinence du modèle des données expérimentales ont été utilisées
permettant de vérifier les caractéristiques techniques du capteur et la performance du
modèle. Pour ce faire la température de l’eau en entrée des collecteurs et les
irradiations directe et indirecte ont été utilisées comme conditions limites pour le
modèle. La température de sortie entre les conditions réelles et le modèle a ainsi
pu être comparée et les résultats montrent que sur les deux mois couverts par les données
expérimentales le modèle reflète correctement le comportement réel (\mtodo{Ajouter ref fig}).

Enfin le modèle utilisé pour convertir l’irradiation disponible à l’horizontal (diffus)
et à la normale (direct) sur une surface inclinée (capteur) a aussi été comparé
avec les résultats d’une modèle (\mtodo{Ajouter type} implémenté dans TrnSys. Les deux
modèles estime de manière similaire l’irradiation sur une surface inclinée.

\ftodo{Ajouter résultats de comparaison expérimentation}
\ftodo{Ajouter résultats de comparaison avec Trnsys}
% subsubsection verification_du_modele_de_capteur_solaire (end)
% subsection partie_hydraulique (end)


% ------------------------------------------------------------------------------
\subsection{Partie aéraulique} % (fold)
\label{sub:partie_aeraulique}
~
\itodo{Partie aéraulique ...}
La partie aéraulique du système est responsable du renouvellement d’air et du chauffage
par l’intermédiaire du solaire thermique ou bien de la batterie électrique en
position terminale. Un caisson de mélange permet de récupérer une partie de l’énergie
sur l’air extrait (Fig~\ref{fig:air_complet_mono}). L’air soufflé est un mélange entre de l’air neuf et de l’air repris
dans le respect de la réglementation en vigueur (\ref{ssub:ventilation}).
Le système de ventilation peut ainsi être apparenté à une ventilation mécanique par insufflation
(VMI).

\begin{figure}
    \begin{center}
        \includegraphics{Ressources/Images/Modelisation/air_aeraulique_mono.pdf}
    \end{center}
    \caption{Description schématique de la partie aéraulique du système solaire.
             \label{fig:schema_aeraulique}}
\end{figure}
% subsection partie_aeraulique (end)


% ------------------------------------------------------------------------------
\subsection{Algorithme de contrôle} % (fold)
\label{sub:algorithme_de_controle}
~
\itodo{Ajouter description de l’état de l’art et du blabla sur ce qui a été fait et
       ce qui manque. Voir Article}
\itodo{Remplacer les acronymes FR par des US}

% - - - - - - - - - - - - - - - - - - - - - - - - - - - - - - - - - - - - - - -
\subsubsection{Fonctionnement global} % (fold)
\label{ssub:fonctionnement_global}
La partie hydraulique est contrôlée par un ensemble hiérarchisé de contrôleurs. Le niveau
le plus élevé de contrôle s’occupe d’activer ou de désactiver les différents éléments
comme les pompes ou les vannes. Au plus bas niveau chaque pompes et chaque équipement
électrique utilise un contrôleur PID. Cette approche permet d’adapter dynamiquement le
comportement du système tout en conservant un stricte séparation entre la logique de
chaque composant. Le système solaire fonctionne suivant trois principaux modes
(Fig~\ref{fig:schema_modes}).
\begin{figure}
    \begin{center}
        \includegraphics{Ressources/Images/Modelisation/air_modes.pdf}
    \end{center}
    \caption{Description schématique des différents modes de fonctionnements. Système
    global (a), fonctionnement $Indirect$ (b), fonctionnement $Direct$ avec besoin de
    chauffage (c), et, fonctionnement $Direct$ sans besoin de chauffage (d).
             \label{fig:schema_modes}}
\end{figure}

Afin de valoriser l’énergie solaire les différents modes sont ordonnées. La \textbf{priorité} revient au maintien du ballon
sanitaire, au respect de la température de consigne, puis à l’élévation de la température
dans le ballon de stockage. Le respect de ces priorité n’est pas pour autant restrictif~:
le système peut remplir plusieurs fonctions en parallèles. Lorsque l’énergie solaire
disponible est suffisante, le chauffage, l’$ECS$, et la charge du ballon tampon
peuvent être réalisées de manière simultanées.

Durant la période diurne, l’énergie solaire est récupérée directement au niveau des
capteurs solaires (mode $Direct$) et la vanne 3 voies est alors ouverte. L’activation des
pompes $S5$, $S2$, et $S6$ permet alors respectivement de charger le ballon tampon,
couvrir les besoins en $ECS$, et couvrir les besoins en chauffage. Ces trois pompes
sont des pompes à vitesse variable permettant d’adapter le débit en fonction des besoins
grâce à un contrôleur PID. Le système ajuste la vitesse des pompes afin de maintenir une
différence de température minimale de \SI{10}{\celsius} entre la sorties des échangeurs et
la sortie des capteurs solaires. Contrairement aux approches classiques, la différence de
température entre l’entrée et la sortie du collecteur n’est pas utilisée comme un élément
de régulation. En effet, cette approche ne permet pas de tenir compte de manière dynamique
des fluctuations entre besoins, pertes, et énergie disponible (\mtodo{Ajouter sources}).
En considérant la différence de température entre la source (sortie des capteurs) et la
cible (sortie des échangeurs) on tient compte des pertes en lignes mais aussi des
fluctuations de la part d’énergie fournie aux ballons ou à l’air. Ainsi dans le cas où
l’énergie solaire est importante, il est possible de la distribuer
entre les différentes cibles. Dans le cas où l’énergie solaire est limitée la différence
minimale de température assure de valoriser cette énergie en contrôlant dynamiquement le
nombre de pompes activés grâce à un PID associé à chaque pompe.

En dehors des heures ensoleillées, le système utilise l’énergie solaire stockée dans le
ballon tampon pour couvrir les besoins de chauffage (mode $Indirect$). Ainsi le sens de
circulation du fluide dans l’échangeur du ballon tampon est déterminé par la position de
la vanne 3 voies. Dans le mode $Direct$ la vanne 3 voies est ouverte vers les capteurs
permettant à la pompe $S6$ de s’activer et l’eau circule du haut vers le bas du ballon.
Dans le mode $Indirect$ la vanne 3 voies bascule vers le ballon tampon (fermée coté
capteurs) et l’eau circule du bas vers le haut (inverse). Dans le mode $Indirect$ la pompe
$S6$ ne peut donc pas être activée. L’énergie accumulée dans le ballon tampon est réservée
à la couverture des besoins en chauffage, les besoins en $ECS$ sont couverts par
l’énergie stockée dans le ballon sanitaire. Dans le cas où l’énergie du ballon sanitaire
n’est pas suffisante (température inférieure à \SI{50}{\celsius}, l’appoint électrique est
activé.

Finalement lorsque la production solaire est insuffisante les besoins sont couverts à
\SI{100}{\percent} par les appoints électriques. Le premier permettant de couvrir les
besoins d’ECS se trouve dans le ballon sanitaire, et celui permettant de couvrir les
besoins en chauffage se trouve en partie terminale du réseau aéraulique. Il est important
de noter que les besoins en énergie peuvent être couverts simultanément par
le solaire et par l’appoint électrique si l’énergie solaire disponible n’est pas
suffisante.

Afin d’éviter les instabilités, des hystérésis sont utilisés. Sur le ballon sanitaire par
exemple, la température à atteindre est de \SI{55}{\celsius} mais une variation de
\SI{5}{\celsius} est autorisée. Lorsque la température descend en dessous de
\SI{55}{\celsius} la demande en énergie est active. Lorsque le ballon atteint une
température supérieure à \SI{60}{\celsius} la demande est de nouveau inactive.
L’intervalle de température (55 - 60) représente alors une zone morte.
% \begin{figure}
%     \begin{center}
%         \includegraphics{Ressources/Images/Regulation/hysteresis.pdf}
%     \end{center}
%     \caption{État du chauffage sans et avec un hystérésis en fonction d’une condition
%              d’activation.
%              \label{fig:hysteresis}}
% \end{figure}

De plus l’activation d’une pompe nécessite que la température de la source, sortie des
capteurs ou du ballon de stockage, est atteint un seuil. Ce seuil minimal permet d’éviter
l’activation trop rapide d’une pompe. Par exemple, si la température dans le ballon tampon
(stockage) est inférieure à \SI{45}{\celsius} alors le chauffage solaire $Indirect$ ne
peut pas être activé. Bien que le système puissent remplir plusieurs fonctions
simultanément, le respect de certaines conditions est nécessaire afin de garantir l’ordre
des priorités. L’activation de la charge du ballon de stockage est ainsi autorisée si la
vanne 3 voies est ouverte vers les capteurs solaires et si le ballon sanitaire en partie
basse (position de l’échangeur solaire) a atteint au minimum \SI{30}{\celsius}. Cette
règle permet ainsi d’ajouter une priorité sur la couverture des besoins en ECS.

L’algorithme a été formulé grâce à des automates finis ($FSM$, Finite State Machine) (Fig~\ref{fig:automate_fini}).

Un $FSM$ est composé de plusieurs états et des conditions, des temporisations, permettent
de passer d’un état à l’autre. Cette formulation permet de garantir un état unique à chaque
instant. Chaque équipement ou entité est dans ces travaux représentés comme un $FSM$~: la
position de la V3V, l’état des pompes, et le mode de chauffage actif.

\begin{figure}
    \begin{center}
        \includegraphics{Ressources/Images/Regulation/exemple_fsm.pdf}
    \end{center}
    \caption{Illustration du fonctionnement d’un automate finie appliqué au contrôle
             d’un personnage fictif grâce à deux boutons~: A et B.
             \label{fig:automate_fini}}
\end{figure}

L’unicité garantie par cette formulation permet d’éviter une incohérence structurelle
résultant en un comportement erratique ou instable. Il impose cependant que chaque état
soit capable de réaliser la transition vers le bon état et ce pour toutes les conditions
existantes. Afin de simplifier sa création les Automates Fines Hiérarchisés (\textbf{HFSM}
pour Hierarchical Finite State Machine) ont été développés. Cette formulation admet que
chaque état peut être lui-même un \textbf{FSM} dont les sous-états ne sont connus que de
l’état maître. Cette nouvelle formulation permet ainsi de réduire le nombre de conditions
de transition et l’encapsulation des sous-états simplifie la compréhension et construction
de l’algorithme.


Le $FSM$ gouvernant le chauffage admet trois états primaires~: $Chauffage_{OFF}$,
$Chauffage_{ON}$, et $Surchauffe_{ON}$ (\ref{fig:automate_chauffage}). L’état
$Chauffage_{ON}$ admet deux $FSM$ fonctionnant en parallèle~: celui du chauffage
solaire, et celui du chauffage électrique. Le chauffage solaire admet ainsi trois états~:
$Solaire_{OFF}$, $Solaire_{Direct}$, ou $Solaire_{Indirect}$. Le premier est actif lorsque
il n’y a pas d’énergie solaire disponible ou pas assez pour couvrir les besoins en ECS et
en chauffage. Le second ($Solaire_{Direct}$) est actif lorsque il y a une demande en
chauffage et que le système est en mode $Direct$. Finalement le dernier état est atteint
lorsque il existe toujours une demande en chauffage et que le système est en mode
$Indirect$.
Il peut être noté l’ajout d’un hystérésis de \SI{5}{\celsius} mis en place sur la
transition entre $Solaire_{OFF}$ et les deux autres états. Dans les deux cas, une
temporisation permet d’assurer que l’énergie est vraiment disponible et que ce n’est pas
le résultat d’une variation parasite temporaire (activation d’une pompe, fermeture d’une
vanne, ...). Le chauffage électrique lui admet deux états~: $Appoint_{OFF}$ et
$Appoint_{ON}$. Seul une temporisation est utilisée pour favoriser le solaire thermique.
L’appoint électrique reste ainsi actif tant que le besoin de chauffage existe. Finalement
lorsque le besoin de chauffage est couvert, le chauffage passe dans l’état
$Surchauffe_{ON}$. Cet état autorise comme décrit ci-avant une élévation de la température
dans le bâtiment. Dans le cas où la surchauffe n’est pas autorisée ou que de nouveaux
besoins en chauffage apparaissent l’état du chauffage est réinitialisé à $Chauffage_{OFF}$
et la boucle de contrôle recommence.

\begin{figure}
    \begin{center}
        \includegraphics{Ressources/Images/Regulation/chauffage_fsm.pdf}
    \end{center}
    \caption{Description détaillée de l’$FSM$ contrôlant la transition entre
             les différents modes de chauffage.
             \label{fig:automate_chauffage}}
\end{figure}


Afin d’illustrer le fonctionnement de l’algorithme prenons le cas suivant: "L’énergie
disponible au niveau des capteurs est suffisante pour couvrir les besoins en chauffage de
la maison." Le système solaire est alors en mode $Direct$ et le chauffage dans l’état
$Chauffage_{ON}$ avec le chauffage solaire dans le sous-état $Solaire_{Direct}$ et
l’appoint électrique dans le sous-état $Appoint_{OFF}$". Un des habitants décide de
prendre sa douche, abaissant la température de l’eau du ballon sanitaire en dessous du
seuil minimal (\SI{55}{\celsius}). Dans cette configuration la pompe $S5$ va s’enclencher
pour fournir de l’énergie au ballon sanitaire. Si l’énergie disponible au niveau des
capteurs est insuffisante pour le chauffage et le maintien du ballon sanitaire alors la
pompe $S2$ (chauffage) va s’arrêter car l’ECS est prioritaire. Le chauffage solaire passe
alors dans l’état $Solaire_{OFF}$ et l’appoint électrique est en attente d’activation pour
permettre de couvrir le chauffage (temporisation de \SI{3}{\minute}).


L’algorithme de contrôle de la partie hydraulique fonctionne ainsi grâce à un algorithme
maître en cascade utilisant une approche par prospection (feed-forward) et délégant le
contrôle des équipements à une combinaison de $PID$ fonctionnant par rétroaction (feed-
back). C’est approche permet au système de prévenir au niveau global les changements et de
s’adapter aux besoins au niveau local.
% subsubsection fonctionnement_global (end)


% - - - - - - - - - - - - - - - - - - - - - - - - - - - - - - - - - - - - - - -
\subsubsection{Contrôle du chauffage solaire (partie aéraulique)} % (fold)
\label{ssub:controle_du_chauffage_solaire}
Afin de favoriser le chauffage par l’énergie solaire, l’échangeur de chaleur entre l’eau
et l’air est placé en amont du terminal électrique. Grâce à une loi d’air , loi
proportionnelle par rapport à la température extérieure (similaire à une loi d’eau), la
température de soufflage nécessaire est fixée~: $T_{air}^{cons}$ (\autoref{eq:temp_soufflage}).

\begin{equation}\label{eq:temp_soufflage}
    T_{air}^{cons} = T_{air}^{ext} + PI_{souff} \times (T_{air}^{max} - T_{air}^{ext})
\end{equation}
avec $PI_{souff}$, $T_{air}^{max}$, et $T_{air}^{ext}$ respectivement l’impact du
contrôleur PI associé, la température maximale de l’air soufflé, et la température
extérieure de l’air. Cette formulation permet d’éviter de souffler un air à une
température élevée en continue. Cependant dans le cas où les besoins sont importants, le
débit hygiénique ($Q_{v}^{min} = $\SI[per-mode=symbol]{90}{\meter\cubed\per\hour}) peut être insuffisant. Une
temporisation de \SI{10}{min}) est alors mis en place lorsque $T_{air}^{cons} =
T_{air}^{max}$. Si à la fin de cette temporisation si la température de la maison est
toujours inférieure à $T_{ins}$ le débit soufflé est modulé afin de répondre à la demande
de besoin avec un débit maximal de $Q_{v}^{max} = $\SI[per-mode=symbol]{900}{\meter\cubed\per\hour}. Ces
temporisations permettent respectivement de favoriser en priorité le solaire et de
garantir le maintien du confort thermique (Fig~\ref{fig:chauffage_aeraulique}).

\begin{figure}
    \begin{center}
        \includegraphics{Ressources/Images/Regulation/chauffage_aeraulique.pdf}
    \end{center}
    \caption{Fonctionnement de la régulation du chauffage par vecteur air.
             \label{fig:chauffage_aeraulique}}
\end{figure}

La température d’air en sortie de l’échangeur est directement liée à la température et au
débit de l’eau en entrée de l’échangeur. C’est pourquoi, la différence de température
entre l’air en sortie de l’échangeur et la consigne ($T_{air}^{cons}$), est utilisé comme
variable d’apprentissage par le PID contrôlant la pompe $S2$.



Afin d’améliorer la couverture solaire sur le chauffage, l’algorithme autorise la
surchauffe de la maison durant les périodes diurnes d’inoccupation
(Fig~\ref{fig:control_air}) mais seul le chauffage solaire $Direct$ est autorisé. Dans ce
cas, le débit minimum sanitaire est maintenue et seul $T_{air}^{cons}$ varie afin que la
température intérieure atteigne la température limite de surchauffe~:
$T^{cons}_{sol}$. L’augmentation de la température intérieure permet de profiter de
l’inertie du bâtiment et de retarder, voir éviter la relance du chauffage et donc
l’utilisation potentielle de l’appoint électrique.
\begin{figure}
    \begin{center}
        \includegraphics{Ressources/Images/Regulation/control_air_curve.pdf}
    \end{center}
    \caption{Principe de la surchauffe diurne durant une journée hivernale.
             \label{fig:control_air}}
\end{figure}
% paragraph couplage_aeraulique (end)
% subsubsection controle_du_chauffage_solaire (end)
% subsection algorithme_de_controle (end)
% section description_du_cas_d_etude_le_systeme_solaire (end)








% ..............................................................................
% ..............................................................................
\section{Étude paramétrique~:} % (fold)
\label{sec:etude_parametrique_}
% ------------------------------------------------------------------------------
L’étude paramétrique détaillée dans ce chapitre a été réalisée à partir de la
version définitive de l’algorithme et du bâtiment. En effet, le projet étant
porté à travers le projet COMEPOS (\munsure{Plus interne IGC mais possible de le dire ?})
il est le fruit d’un travail itératif. Le bâtiment a ainsi évolué au fur et à mesure
de l’avancement du projet. De même l’algorithme présenté ci-avant est le fruit d’un
travail itératif sur la base d’un algorithme existant et innovant dont
ces travaux se sont inspirés.

Dans un premier temps les différents scénarios puis les variations techniques
seront présentées. Ensuite les résultats seront ensuite détaillés et les limites
de cette approche discutées.


\subsection{Climats étudiés} % (fold)
\label{sub:climats_etudies}
~
\mtodo{Choix plus orienté par IGC que pour couverture ... virer Toulouse ?}
Afin d’être représentatif des zones climatiques de la France, l’étude a été réalisée
pour 5 villes~: Bordeaux, Nantes, Strasbourg, Limoges, et Marseille (Fig~\ref{fig:carte_france},
\autoref{tab:description_climat}).
\begin{figure}
    \begin{center}
        \includegraphics{Ressources/Images/Batiment/France_map.pdf}
    \end{center}
    \caption{Cartographie des villes sélectionnées pour l’étude paramétrique.
             \label{fig:carte_france}}
\end{figure}
La ville de Marseille correspond à un climat très ensoleillé avec une faible demande
en chauffage. À l’opposée le climat de Strasbourg est rude avec un ensoleillement faible
et une forte demande en chauffage. Le climat de Bordeaux est modéré, l’ensoleillement
est bon et la demande en chauffage faible. Limoges et Nantes se placent entre Bordeaux et Strasbourg
avec un bon ensoleillement mais une demande en énergie pour le chauffage plus importante
que pour Bordeaux ou Toulouse.
Pour ces différents sites, la température du sol est considéré comme constante et
est fixée à \SI{10}{\celsius}. Bien que impactant les besoins
d’un bâtiment, sa détermination de manière précise est un processus complexe
spécialement lorsque de nombreuses variations de l’enveloppe comme des charges internes
est étudié.

\begin{table}
\centering
\caption{Description des différents climats retenues}
\label{tab:description_climat}
\begin{tabular}{ l c c  c  c  c  c  c }
  \toprule
                                        &    &    & \textbf{Bordeaux} & \textbf{Nantes} & \textbf{Strasbourg} & \textbf{Limoges} & \textbf{Marseille} \\
  \cmidrule[\lightrulewidth](){4-8}
  \addlinespace[\defaultaddspace]
  \multirow{2}{*}{Température eau froide} &
  \multirow{2}{*}{\si{\celsius}}             & Min     & \num{8.9}               & \num{8.3}               & \num{5.3}                 & 7                 & 12                \\
                                  &          & Max     & 16                & 15               & 14                  & 14                & 19                \\
  \addlinespace[\defaultaddspace]
  DJU (\SI{19}{\celsius}) &
  \si{\celsius}                              & -  & 2408              & 2660               & 3360                & 2972              & 2049              \\
  \addlinespace[\defaultaddspace]
  \multirow{3}{*}{Irradiation solaire} &
  \multirow{3}{*}{\si{kWh/m^{2}}}            & $IGH$   & 1264              & 1184               & 1091                & 1257              & 1545              \\
                                  &          & $IDN$   & 929               & 885               & 721                 & 1209              & 1503              \\
                                  &          & $IDH$   & 712               & 665               & 650                 & 602              & 615               \\
  \bottomrule
\end{tabular}
\end{table}



Dans un premier temps, l’impact du climat est investigué puis deux villes sont
sélectionnées d’une large palette de variations plus détaillée. Dans cette
étude le comportement des occupants, des variations techniques et algorithmiques sont
discutés.
% subsection climats_etudies (end)


% ------------------------------------------------------------------------------
\subsection{Scénarios étudiés} % (fold)
\label{sub:scenarios_etudies}
% - - - - - - - - - - - - - - - - - - - - - - - - - - - - - - - - - - - - - - -
\subsubsection{Puisage en eau chaude sanitaire} % (fold)
\label{ssub:puisage_en_eau_chaude_sanitaire}

\itodo{À mettre dans l’étude paramétrique car propre aux fichiers météos}
Afin de pouvoir estimer la consommation nécessaire pour la production d’$ECS$, un profil
de puisage type est nécessaire. Dans ces travaux le choix du profil de puisage est analysé
à travers les différents climats retenues. Pour chaque climat, la température de l’eau du
réseau est définie suivant une évolution mensuelle, et est extrapolée durant la simulation
(\ref{tab:temp_eau}). On remarque une forte disparité entre les différents climats où les
minimums varient entre \SI{5.3}{\celsius} et
\SI{12}{\celsius}, et les maximums entre \SI{14}{\celsius} et \SI{19}{\celsius},
représentés respectivement par Strasbourg et Marseille.
\begin{table}
\centering
\caption{Température de l'eau du réseau au cours de l'année en fonction de la
         position géographique.}
\label{tab:temp_eau}
\begin{tabular}{l*{12}{c}}
    \toprule
               & Janv. & Fevr. & Mars & Avr. & Mai & Juin & Juil. & Août & Sept. & Oct. & Nov. & Dec. \\
    \cmidrule[\lightrulewidth](){2-13}
    Strasbourg & \num{5.3}   & \num{5.8}   & \num{7.7}  & \num{9.5}  & \num{11}  & \num{13}   & \num{14}    & \num{14}   & \num{12}    & \num{9.8}  & \num{7.5}  & \num{5.8}  \\
    Limoges    & \num{7}     & \num{7.4}   & \num{9}    & \num{10}   & \num{12}  & \num{14}   & \num{14}    & \num{14}   & \num{13}    & \num{11}   & \num{8.8}  & \num{7.3}  \\
    Toulouse   & \num{8.6}   & \num{9.2}   & \num{11}   & \num{12}   & \num{14}  & \num{16}   & \num{17}    & \num{17}   & \num{16}    & \num{13}   & \num{11}   & \num{9}    \\
    Bordeaux   & \num{8.9}   & \num{9.3}   & \num{11}   & \num{12}   & \num{14}  & \num{15}   & \num{16}    & \num{16}   & \num{15}    & \num{13}   & \num{11}   & \num{9.2}  \\
    Nantes     & \num{8.3}   & \num{8.5}   & \num{9.9}  & \num{11}   & \num{13}  & \num{14}   & \num{15}    & \num{15}   & \num{14}    & \num{12}   & \num{9.8}  & \num{8.6}  \\
    Marseille  & \num{12}    & \num{12}    & \num{13}   & \num{14}   & \num{16}  & \num{18}   & \num{19}    & \num{19}   & \num{18}    & \num{16}   & \num{14}   & \num{12}   \\
    \bottomrule
\end{tabular}
\end{table}


\paragraph{Profils retenus~:} % (fold)
\label{par:profils_retenus}
Cinq scénarios ont été ont été considérés dans ces travaux (\ref{fig:profil_puisage}). Pour
l’ensemble des profils, un volume de puisage typique moyen de
\SI{33}{\litre\per(jour\period pers)}\,(\SI{60}{\celsius}), soit un
volume total de \SI{220}{\litre/jour}\,(\SI{40}{\celsius}) pour les 4 occupants est
retenu. Le profil de référence utilisé est le scénario décrit dans la norme
\textbf{EN\,12977}. Trois variantes de celui-ci ont aussi été évaluées favorisant
respectivement un puisage en matinée (\textbf{Matin}), en soirée (\textbf{Soir}) et un
puisage similaire pour les 3 pics journaliers (\textbf{Réparti}). Finalement un scénario
issue de données statistiques \parencite{ADEME2016} a aussi été implémenté
(\textbf{Réaliste}). Ce dernier décrit de manière plus représentative le comportement
d’une famille dans une habitation individuelle ou un appartement et les besoins
journaliers sont pondérées en fonction du jour de la semaine (\ref{tab:coef_semaine}).
\begin{figure}
    \begin{center}
        \includegraphics{Ressources/Images/Scenario/puisage.pdf}
    \end{center}
    \caption{Description des profils de puisage envisagés.
             \label{fig:profil_puisage}}
\end{figure}
% paragraph profils_retenus (end)

\paragraph{Pondération de la demande~:} % (fold)
\label{par:ponderation_de_la_demande}
Afin d’évaluer l’impact de la variation mensuelle des besoins en $ECS$, un
coefficient pondérateur est utilisé. Comme pour les coefficients utilisés pour pondérer
les besoins hebdomadaires, les besoins totaux ne varient pas, seul la répartition est
modifiée. Les coefficients utilisés sont issues des travaux de l’\textit{ADEME}
\parencite{ADEME2016}, qui montre par exemple que les besoins sont plus importants durant
la période hivernale que pendant la période estivale (\ref{tab:coef_mois}).

\begin{table}
\centering
\begin{tabular}{l*{7}{c}}
    \toprule
                & Lundi & Mardi & Mercredi & Jeudi & Vendredi & Samedi & Dimanche \\
    \cmidrule[\lightrulewidth](){2-8}
    Coefficient & \num{0.97}  & \num{0.95}  & \num{1.00}     & \num{0.97}  & \num{0.96}     & \num{1.02}   & \num{1.13}     \\
    \bottomrule
\end{tabular}
\caption{Détail des coefficients de pondération journaliers pour le profil de
         puisage Réaliste.}
         \label{tab:coef_semaine}
\end{table}

\begin{table}
\centering
\begin{tabular}{l*{12}{c}}
    \toprule
                & Janv. & Fevr. & Mars & Avr. & Mai & Juin & Juil. & Août & Sept. & Oct. & Nov. & Dec. \\
    \cmidrule[\lightrulewidth](){2-13}
    Coefficient & \num{1.11}   & \num{1.2}   & \num{1.11}  & \num{1.06}  & \num{1.03}  & \num{0.93}   & \num{0.84}    & \num{0.72}   & \num{0.92}    & \num{1.03}  & \num{1.04}  & \num{1.01}  \\
    \bottomrule
\end{tabular}
\caption{Détail des coefficients de pondération mensuels pour le profil de
         puisage Réaliste.}
         \label{tab:coef_mois}
\end{table}
% paragraph ponderation_de_la_demande (end)

\paragraph{Variations paramétriques~:} % (fold)
\label{par:variations_parametriques}
L’étude paramétrique permettra d’évaluer l’impact de la variation des profils de puisage
comme de la quantité des besoins sur la performance du $SSC$. Les consommations variant
fortement d’une habitation à une autre, des consommations de 27, 33, et
\SI{40}{\litre\per(jour \period  pers)}\,(\SI{60}{\celsius}) seront aussi évaluées.
% paragraph variations_parametriques (end)
% subsubsection puisage_en_eau_chaude_sanitaire (end)


% - - - - - - - - - - - - - - - - - - - - - - - - - - - - - - - - - - - - - - -
\subsubsection{Ventilation} % (fold)
\label{ssub:ventilation}
Deux scénarios de ventilation ont été retenus pour l’étude paramétrique~:
\begin{itemize}
    \item Référence~: \SI[per-mode=symbol]{90}{\meter\cubed\per\hour} (\ref{ssub:ventilation_ref}).
    \item Réduit~: Débit minimal de \SI[per-mode=symbol]{20}{\meter\cubed\per\hour} en inoccupation et de
          \SI[per-mode=symbol]{90}{\meter\cubed\per\hour} en occupation.
\end{itemize}.
Dans le cas d’une ventilation L’analyse de ces deux scénarios permettra de quantifier
l’impact apportée par une ventilation Hygro B sur la performance du système solaire.
% subsubsection ventilation (end)


% - - - - - - - - - - - - - - - - - - - - - - - - - - - - - - - - - - - - - - -
\subsubsection{Température de consigne} % (fold)
\label{ssub:temperature_de_consigne}
\paragraph{Consigne de chauffage:} % (fold)
\label{par:consigne_de_chauffage}
L’analyse des variations de la consigne de chauffage permet d’évaluer les performance du
système solaire et évaluer l’évolution de l’économie sur les consommations de la maison.
Afin de couvrir une large palette de scénarios, la consigne en inoccupation, en occupation
diurne, et en inoccupation nocturne sont modifiées indépendamment. La consigne de
référence étant, pour rappel, définie comme étant 19-18-16~: \SI{19}{\celsius} durant les
occupations diurnes, \SI{18}{\celsius} durant les occupations nocturnes et
\SI{16}{\celsius} en inoccupation. À partir des règles et contraintes définies dans
\autoref{tab:consigne_chauffage} il est alors possible de construire un lot de scénarios
représentatifs. L’étude cherchera en particulier à évaluer l’impact d’une augmentation de
la consigne de chauffage mais aussi l’impact du réduit en inoccupation et durant la
période nocturne. Les scénarios construits sont les suivants~: 19-18-16, 19-19-16,
19-19-19, 20-18-16, 20-20-16.

\begin{table}
\centering
\begin{tabular}{| l | c | c | c | c |}
    \hline
                    & \bf{16}             & \bf{18}             & \bf{19}             & \bf{20}              \\
    \hline
Occupation (diurne)    &                     &                     & \cellcolor{SolarizedBrBlue} & \cellcolor{SolarizedBrBlue} \\
    \hline
Occupation (nocturne)  &                     & \cellcolor{SolarizedBrBlue} & \cellcolor{SolarizedBrBlue} & \cellcolor{SolarizedBrBlue} \\
    \hline
Inoccupation         & \cellcolor{SolarizedBrBlue} &                     & \cellcolor{SolarizedBrBlue} &                     \\
    \hline
\end{tabular}
\caption{Variations envisagées pour la consigne de chauffage en fonction de la période.}
         \label{tab:consigne_chauffage}
\end{table}
% paragraph consigne_de_chauffage (end)


\paragraph{Consigne de chauffage solaire :} % (fold)
\label{par:consigne_de_chauffage_solaire}
L’impact de la température de consigne solaire est aussi discuté. Deux configurations sont
simulées. La première admet une température de consigne de \SI{22}{\celsius}. La
seconde ne considère pas d’élévation de la température par le solaire. Pour rappel
l’élévation de la température n’est réalisé que par l’énergie solaire provenant des
capteurs, soit l’état $Solaire_{Direct}$. Finalement les interactions entre consigne et
consigne solaire de chauffage sont aussi discutées
% paragraph consigne_de_chauffage_solaire (end)

\iunsure{Ajouter variation charges internes. N’apporte pas grand chose.}
% subsubsection temperature_de_consigne (end)
% subsection scenarios_etudies (end)


% ------------------------------------------------------------------------------
\subsection{Variations techniques étudiées} % (fold)
\label{sub:variations_techniques_etudiees}
% - - - - - - - - - - - - - - - - - - - - - - - - - - - - - - - - - - - - - - -
\subsubsection{Fluide caloporteur} % (fold)
\label{ssub:fluide_caloporteur}
Afin de pouvoir implémenter le modèle pour tous les climats, il est nécessaire de
tenir compte des risques de gel. Le fluide caloporteur utilisé à travers les capteurs
solaires et donc un mélange~: eau (\SI{70}{\percent}) et éthylène-glycol (\SI{30}{\percent}).

Cependant, il n’est pas nécessaire que l’ensemble du système soit protégé contre
le gel car seul les capteurs solaires sont à l’extérieur. L’eau des ballons de stockage
et sanitaire seront modélisés avec de l’eau sans glycol pour les raisons suivantes car
ces ballons sont en zone chauffée et l’eau du ballon sanitaire provient du réseau
d’eau froide. Ainsi seules les canalisations contiennent du glycol.
La modélisation admet ainsi deux fluides différents (\autoref{tab:fluide_carac}).

\begin{table}
\centering
\begin{tabular}{l*{3}{c}}
    \toprule
                 & Chaleur massique              & Masse volumique         & Plage de variation \\
                 & \si{\joule\per(kg\period\kelvin)} & \si{kg\per\meter\cubed} & \si{\celsius}      \\
    \midrule
    Eau          & 4180                          & 1000                    & 0 à 100            \\
    Eau + glycol & 3608                          & 1034                    & -20 à 110          \\
    \bottomrule
\end{tabular}
\caption{Caractéristiques des deux fluides caloporteurs utilisés pour la modélisation.}
         \label{tab:fluide_carac}
\end{table}

Enfin, la variation de la capacité massique étant négligeable sur les plages de variation
considérées (0 à \SI{120}{\celsius}), le modèle assume une capacité massique constante
permettant de simplifier les systèmes d’équations.
% subsubsection fluide_caloporteur (end)

% - - - - - - - - - - - - - - - - - - - - - - - - - - - - - - - - - - - - - - -
\subsubsection{Capteurs solaires} % (fold)
\label{ssub:capteurs_solaires}
~
Afin d’évaluer l’impact du choix du capteur, 3 capteurs ont été retenus
(\autoref{tab:capteurs_specs}). Le capteur de référence (IDMK 25), un capteur sous-vide
(12 CPC58) et un autre capteur plan très performant (308C HP). Les caractéristiques du
premier (IDMK) sont issues de sa fiche technique
(\ref{ssub:capteurs_solaires}). Les deux autres types sont issue d’une source
commune~: \href{www.solar-rating.org}{ICC SRCC}. Cette source a été sélectionnée car elle
regroupe les résultats de tests sur différents capteurs en fournissant des informations
supplémentaires qui ne sont pas forcement disponibles sur les fiches techniques. En effet,
le modèle de capteur retenu nécessite les conditions dans lesquels les essais ont été réalisés.
Enfin les modèles retenues ont une surface d’absorption similaire facilitant la comparaison
entre eux.

Les fiches techniques des capteurs sont disponible en annexe (\mtodo{Ajout annexe et ref})

\begin{table}
\centering
\begin{tabular}{l | c | ccc}
    \toprule
                                             & Unité                   & IDMK 25 (référence)  & 308C HP              & 12 CPC58    \\
    \midrule
    Fabricant                                & -                       & Sonnenkraft          & Radco                & Sky Pro   \\
    Type                                     & -                       & Plan vitrée          & Plan vitrée          & Tubulaire \\
    Surface nette                            & \si{m^{2}}              & \num{2.32}           & \num{2.193}          & \num{2.28}      \\
    Poids à vide                             & \si{kg}                 & \num{54}             & \num{36}             & \num{53}        \\
    Contenance                               & \si{\litre}             & \num{1.35}           & \num{3.5}            & \num{1.83}      \\
    $\eta_{0}$                               & \si{\%}                 & \num{78}             & \num{83.4}           & \num{63}        \\
    Décroissance de performance              & -                       & \num{-5.103}         & \num{-4.777}         & \num{-0.975}    \\
    $a_{1}$                                  & \si{W/(m^{2}\period K)}      & \num{3.796}          & \num{1.4539}         & \num{0.9249}    \\
    $a_{2}$                                  & \si{W/(m^{2}\period K^{2})}  & \num{0,013}          & \num{0.0589}         & \num{0.00069}   \\
    Modulation diffus ($IMDiff$)             & \si{\%}                 & \num{100}            & \num{96}             & \num{102}       \\
    \bottomrule
\end{tabular}
\caption{Caractéristique du collecteur de référence (modèle IDMK 2.5 de chez Sonnenkraft).
         \label{tab:capteurs_specs}}
\end{table}

Au cours de l’étude paramétrique, l’orientation, la surface d’absorption, et l’inclinaison
des capteurs sera aussi évaluée afin de comparer les résultats avec les observations déjà
relevées au cours des études existantes \parencite{Task262003,Shariah2002587}.
Il est donc retenue les variations suivantes~:
\begin{itemize}
  \item Inclinaison~: \SI{18.9}{\degree} (\SI{33}{\percent}), \SI{33}{\degree}, \SI{45}{\degree}, \SI{60}{\degree}
  \item Orientation~: Est, Ouest, Sud
  \item Nombre de capteurs~: 2, 4, 6, 8
\end{itemize}
Il est important de noter que la maison de référence est orientée Sud, comporte 4 capteurs
IDMK25, et une toiture ayant une pente de \SI{33}{\percent}.
% subsubsection capteurs_solaires (end)


% - - - - - - - - - - - - - - - - - - - - - - - - - - - - - - - - - - - - - - -
\subsubsection{Ballons} % (fold)
\label{ssub:ballons}
L’analyse paramétrique tient aussi compte de variations au niveau des ballons.
Pour les deux ballons, l’impact du volume sera pris en compte. Afin d’être
représentatif les caractéristiques du ballon seront proportionnelles à la taille
du ballon. Lorsque le volume du ballon augmente la hauteur de augmentera aussi.
De même la taille de l’échangeur sera adaptée à la nouvelle hauteur du ballon
afin que sa position relative reste identique. De cette manière, le ballon de
stockage conserve un échangeur couvrant la quasi totalité du volume, et
l’échangeur solaire sur le ballon sanitaire reste adapté à la nouvelle taille du
ballon. Les variations retenues pour chaque ballon sont identiques :
\SI{150}{l}, \SI{300}{l}, et \SI{450}{l}.

Ensuite l’impact de la position de l’échangeur solaire est aussi étudiée. L’échangeur du
ballon de stockage couvrant presque l’intégralité de sa hauteur seule la position de
l’échangeur solaire sur le ballon sanitaire est analysée. Afin d’être compatible avec les
variations réalisées sur le volume des ballons, la variation de la position de l’échangeur
sera réalisée relativement à sa position d’origine sans modifier sa longueur. Il est alors
possible d’évaluer l’impact du volume du ballon et de la position relative de l’échangeur
de manière combinés ou bien séparément. Les coefficients ont été sélectionnés en fonction
des limitations techniques des ballons qui imposent ne hauteur minimale et maximale car la
hauteur de l’échangeur est fixe. Les coefficients retenues sont alors les suivants :
\num{0.8}, \num{1}, et \num{1.3}. Avec un coefficient de \num{0.8} l’échangeur sera ainsi
positionné plus bas et avec un coefficient de \num{1.3}, il sera à l’inverse plus haut.
% subsubsection ballons (end)



% ------------------------------------------------------------------------------
\subsubsection{Circulateurs} % (fold)
\label{ssub:circulateurs}
L’étude cherchera aussi à évaluer l’impact du débit nominal sur la performance du
système. Comme il a été décrit le débit nominal est dépendant de la surface de capteur
et la surface de capteur est elle aussi un élément variant au cours de l’étude. Il a donc
été retenue de moduler le débit nominale par capteur.
\mtodo{Pour mémo}{\SI[per-mode=symbol]{}{\litre\per(\meter\squared\period\hour)}}

Le système décrit est complexe et plusieurs circulateurs en série peuvent être active en même temps. Par défaut, les
modèles hydrauliques de la bibliothèque \textit{Buildings} considère un cas idéal
sans pertes de charges.
Ainsi si on considère deux circulateurs en série ayant toutes les deux pour consigne un débit de \SI[per-mode=symbol]{2}{\meter\cubed\per\hour}
le débit résultant sera de \SI[per-mode=symbol]{4}{\meter\cubed\per\hour}. Afin de corriger ce
comportement, il est nécessaire d’équilibrer l’ensemble du réseau et de sélectionner
une pompe adapté. Le choix a été fait d’imposer la vitesse de rotation au niveau
des circulateurs mais il est aussi envisageable d’imposer une différence de pression en
sortie de pompe.

\itodo{Liste les variations retenues}
Afin de pouvoir réaliser les variations paramétriques tout en conservant un équilibre
entre les différentes branches du réseau, l’équilibrage est réalisé automatiquement
durant l’initialisation du modèle. Dans un premier temps la surface totale de capteur et
les débit nominaux surfaciques permettent de calculer le débit nominal de chaque pompe. Ensuite les pertes
de charges sont réparties en suivant les règles décrites à travers le schéma \mtodo{Ref fig pdc}.
Les pertes de charge sont alors calculé pour un certain nombre de débit permettant de construire
la courbe de fonctionnement des circulateurs.

Une pompe à vitesse variable (adaptée aux températures élevées d’un réseau solaire) a été sélectionnée (\mtodo{Ajouter ref carac pompe})
car elle est adapté pour toutes les configurations envisagées (\mtodo{Ajouter ref tab variations}).
\ttodo{Variations du débit surfacique nominal}
% subsubsection circulateurs (end)



\subsubsection{Algorithme} % (fold)
\label{ssub:variations_algorithmiques}
Finalement il est aussi étudié à travers cette étude des variations au niveau
de l’algorithme de contrôle.

$DeltaT_{sol}$, la différence de température entre la source (sortie des capteurs) et la cible (sortie des
échangeurs) est modulée autour du cas de référence (\SI{10}{\celsius}) avec deux nouvelles
variations~: \SI{5}{\celsius} et \SI{15}{\celsius}. Augmenter le $DeltaT_{sol}$ revient à
attendre plus longtemps avant d’autoriser l’utilisation de l’énergie solaire et impose une
température en sortie des capteurs plus élevée pour valoriser les apports solaires. À
l’inverse diminuer ce paramètre amène le système a être plus réactif.

Il est aussi évalué l’impact des temporisations au niveau de la régulation du chauffage
par vecteur air~:
\begin{itemize}
  \item Activation de la batterie électrique terminale
  \item Modulation du débit de soufflage
  \item Modulation du de la température de soufflage
\end{itemize}

La première temporisation est sur le cas de référence de \SI{3}{min} et sera notée
$Tempo_{batterie}$. Elle permet d’éviter d’activer la batterie terminale d’appoint à
chaque demande de chauffage en retardant son activation lorsque il y a de l’énergie
solaire de disponible. La seconde temporisation correspond à la temporisation entre la
modulation de la température, et la modulation du débit. Le cas de référence admet une
temporisation de \SI{10}{min} et elle sera notée $Tempo_{souff}$.

\begin{table}
\centering
\begin{tabular}{l c r}
    \toprule
    Paramètre            & Unité         & Variations \\
    \midrule
    $DeltaT_{sol}$   & \si{\celsius} & 5, 10, 15 \\
    $Tempo_{batterie}$   & \si{min}      & 2, 3, 4 \\
    $Tempo_{souff}$  & \si{min}      & 5, 10, 15   \\
    \bottomrule
\end{tabular}
\caption{Variations algorithmiques étudiées lors de l’étude paramétrique.
         \label{tab:variations_algo}}
\end{table}
% subsubsection algorithme (end)
% subsection variations_techniques_etudiees (end)



% ------------------------------------------------------------------------------
\subsection{Analyse des résultats} % (fold)
\label{sub:analyse_des_resultats}
% - - - - - - - - - - - - - - - - - - - - - - - - - - - - - - - - - - - - - - -
\subsubsection{Méthodologie} % (fold)
\label{ssub:methodologie}
L’analyse a été réalisée sur la base d’un cas de référence dont les scénarios et les
caractéristiques ont été présentés dans les parties précédentes. Sans mentions explicites les
simulations utilisent ces paramètres. L’ensemble des variations possibles ainsi que
le cas de référence sont décrits à travers un tableau récapitulatif (\autoref{tab:ref_description}).

Le vocabulaire utilisé lors de l’analyse des résultats est décrit, en particulier la
définition des principaux indicateurs.




\begin{table}
\centering
\caption{Description de la solution de référence et de variations étudiées.}
  \label{tab:ref_description}
  \begin{tabular}{l r c C{4cm} r}
    \toprule
    \addlinespace
                                     & Unité & Référence & Variations & Détails  \\
    \multicolumn{5}{l}{\bf{Scénarios}}                                                                                       \\
    \midrule
    Puisage                          & \si{l/h}      & EN12977   & Soir, Matin, Réparti, Réaliste & \ref{ssub:puisage_en_eau_chaude_sanitaire}            \\
    Ventilation                      & \si{m^{3}/h}  & 90-20     & 90-90 & \ref{ssub:ventilation}                                \\
    Chauffage                        & \si{\celsius} & 19-18-16  & 19-19-16, 19-19-19, 20-18-16, 20-20-16 & \multirow{2}{*}{\ref{ssub:temperature_de_consigne}}   \\
    Chauffage solaire                & \si{\celsius} & Avec (22) & Sans &                                                       \\
    \\
    \addlinespace[\defaultaddspace]
    \multicolumn{5}{l}{\bf{Équipements}}                                                                                          \\
    \midrule
    Modèle des capteurs              & -             & IDMK 25   & 308C HP, 12 CPC58  & \multirow{3}{*}{\ref{ssub:capteurs_solaires}}         \\
    Inclinaison des capteurs         & \si{\degree}  & 33        & \num{18.9}, 45, 60 &                                                       \\
    Orientation des capteurs         & -             & Sud       & Est, Ouest         &                                                       \\
    Volume ballon de stockage        & \si{\litre}   & 300       & 150, 450 & \multirow{3}{*}{\ref{ssub:ballons}}                   \\
    Volume ballon $ECS$              & \si{\litre}   & 300       & 150, 450 &                                                       \\
    Position Échangeur solaire       & \si{m}        & 1         & \num{0.8}, \num{1.3} &                                                       \\
    \\
    \addlinespace
    \multicolumn{5}{l}{\bf{Algorithme}}                                                                                      \\
    \midrule
    $Tempo_{batterie}$               & \si{\min}     & 3         & 2, 4 & \multirow{3}{*}{\ref{ssub:variations_algorithmiques}} \\
    $Tempo_{souff}$              & \si{\min}     & 10        & 5, 15 &                                                       \\
    $DeltaT_{sol}$               & \si{\celsius} & 10        & 5, 15 &                                                       \\
    \bottomrule
  \end{tabular}
\end{table}

% subsubsection methodologie (end)


% - - - - - - - - - - - - - - - - - - - - - - - - - - - - - - - - - - - - - - -
\subsubsection{Impact du climat} % (fold)
\label{ssub:impact_du_climat}
Les résultats obtenues pour les différents climats sont disponibles à travers le
\autoref{tab:performance_annuelles}. Les consommations électriques tiennent compte
de la consommation des appoints (chauffage et ECS) mais aussi de la consommation des
pompes du système solaire combiné. Il peut ainsi être observé que le système solaire
permet d’obtenir une bonne performance pour tous les climats autant sur $F_{ECS}$

\begin{table}
\small
\centering
\caption{Performances annuelles du système solaire pour différents climats.}
\label{tab:performance_annuelles}
\begin{tabular}{l c c c c c c c c c c c c}
    \toprule
               &   \multicolumn{9}{c}{Consommation [\si{\kilo\watt\hour}]} & & \multicolumn{2}{c}{\multirow{2}{*}{\%}} \\
    \cmidrule(r){2-10}
               & \multicolumn{2}{c}{Total} &  \multicolumn{3}{c}{Électrique}  & \multicolumn{4}{c}{Solaire} & \\
    \cmidrule(r){2-3}
    \cmidrule(r){4-6}
    \cmidrule(r){7-10}
    \cmidrule(r){12-13}
               & $ECS$    & $CH$      &  $ECS$        & $CH$ & Pompes    & $Solaire_{abs}$ & $Pertes_{reseau}$ & $ECS$  & $CH$ & & $F_{ECS}$  & $F_{CH}$ \\
    \midrule
    Bordeaux   & 2983     & 1024      &  101          & 91          &  7             & 3387                  & 219       & 2444   &  949    &   & 95         & 91  \\
    Strasbourg & 3180     & 1986      &  479          & 1203        &  8             & 3164                  & 199       & 2332   &  845    &   & 83         & 42  \\
    Marseille  & 2784     & 975       &  2            & 1           &  7             & 3280                  & 218       & 2300   &  974    &   & 100        & 100 \\
    Nantes     & 3044     & 1114      &  233          & 248         &  7             & 3295                  & 209       & 2399   &  902    &   & 91         & 78  \\
    Limoges    & 3120     & 1311      &  217          & 359         &  9             & 3474                  & 209       & 2502   &  983    &   & 92         & 73  \\
    \bottomrule
\end{tabular}
\end{table}

% subsubsection impact_du_climat (end)

% - - - - - - - - - - - - - - - - - - - - - - - - - - - - - - - - - - - - - - -
\subsubsection{Impact des scénarios internes} % (fold)
\label{ssub:impact_des_scenarios_internes}

% subsubsection impact_des_scenarios_internes (end)


% - - - - - - - - - - - - - - - - - - - - - - - - - - - - - - - - - - - - - - -
\subsubsection{Impact des caractéristiques des ballons} % (fold)
\label{ssub:impact_des_caracteristiques_des_ballons}

% subsubsection impact_des_caracteristiques_des_ballons (end)


% - - - - - - - - - - - - - - - - - - - - - - - - - - - - - - - - - - - - - - -
\subsubsection{Impact du volume de puisage en ECS} % (fold)
\label{ssub:impact_du_volume_de_puisage_en_ecs}

% subsubsection impact_du_volume_de_puisage_en_ecs (end)


% - - - - - - - - - - - - - - - - - - - - - - - - - - - - - - - - - - - - - - -
\subsubsection{Autres variations} % (fold)
\label{ssub:autres_variations}

% subsubsection autres_variations (end)
% subsection analyse_des_resultats (end)
% section etude_parametrique_ (end)






% ..............................................................................
% ..............................................................................
\section{Vers une méthodologie d’aide à la décision} % (fold)
\label{sec:vers_une_methodologie_d_aide_a_la_decision}

% section vers_une_methodologie_d_aide_a_la_decision (end)
























% % ..............................................................................
% % ..............................................................................
% \section{Description du bâtiment} % (fold)
% \label{sec:description_du_batiment}

% % ------------------------------------------------------------------------------
% \subsection{Description du site} % (fold)
% \label{sub:description_du_site}
% \itodo{Description du site étudié: climat, données metéos, ...}
% \itodo{
% Le fichier météo utilisé est celui de Bordeaux et est de type \emph{IWEC}
% et a pour code d’identification \href{http://www.ladybug.tools/epwmap/}{{IWEC - WMO 075100}}.
% }
% L’étude porte sur les climats de Limoges, Bordeaux, Toulouse, Marseille, et Strasbourg
% afin de couvrir les différentes configuration qui peuvent être rencontrées en
% France (\ref{tab:description_site}).
% \scriptsize
% \begin{table}
%     \begin{tabular}{c c | c c c c c}
%                             &                     & \textbf{Bordeaux}         & \textbf{Marseille} & \textbf{Toulouse}       & \textbf{Limoges}            & \textbf{Strasbourg}         \\
%         \toprule
%         Température         & Min                 & \cellcolor{Amaranth}-8,2  &                    & -5,4                    & -7,2                        &  \\
%         extérieure          & Max                 & 34                        &                    & 35,6                    & 33,7                        &  \\
%         \si{\degreeCelsius} & Moy                 & 13,2                      &                    & 13,8                    & 11,4                        &  \\
%         \midrule
%         Température         & Min                 & 8,9                       &                    & 8,6                     & \cellcolor{Amaranth}7       &  \\
%         eau froide          & Max                 & 16                        &                    & 17                      & 14                          &  \\
%         \si{\degreeCelsius} & Moy                 & 12,5                      &                    & 12,8                    & \cellcolor{Amaranth}10,6    &  \\
%         \midrule
%         DJU (19)            & \si{\degreeCelsius} & 2408                      &                    & 2321                    & \cellcolor{Amaranth}2972    &  \\
%         \midrule
%         Ensoleillement      &  Direct             &                           &                    &                         &                             &  \\
%         \si{kWh/m^{2}}      &  Diffus             &                           &                    &                         &                             &  \\
%         \bottomrule
%     \end{tabular}
%     \caption{Description des différentes sites.}
%     \label{tab:description_site}
% \end{table}
% \normalsize

% Marseille correspond à un climat très ensoleillé avec peu de demande en chauffage
% durant l’année entière. Il est alors à priori propice à une installation solaire.
% À l’opposé le climat de Strasbourg est rude avec une forte demande en chauffage et
% un ensoleillement très faible particulièrement durant les mois hivernaux. Il est alors
% à priori non-propice à une installation solaire. Bordeaux et Toulouse propose tous deux
% un climat favorable, l’ensoleillement est bon et les besoins de chauffage sont peu
% importants. Limoges est entre le climat Bordelais et Strasbourgeois et correspond
% ainsi à un cas d’étude intéressant pour évaluer l’impact d’une modification au niveau
% composant sur les performances du système au niveau global.
% La température du sol est considérée constante durant l’ensemble de la simulation
% et identique pour chaque site site. Il est important de noter que la variation de
% la température du sol est un élément important pour l’évaluation des besoins en chauffage
% du bâtiment, mais est fortement dépendant de la composition du plancher et du niveau
% d’isolation de celui-ci.

% \itodo{Ajouter un tableau récap des besoins en énergie}
% \ftodo{Ajouter une carte pour positionner les villes}
% % subsection description_du_site (end)



%%%%%%%%%%%%%%%%%%%%%%%%%%%%%%%%%%%%%%%%%%%%%%%%
% PAROIS
%%%%%%%%%%%%%%%%%%%%%%%%%%%%%%%%%%%%%%%%%%%%%%%%


% \itodo{Tableau des parois}
% \begin{table}
%     \begin{tabular}{l *4{c}}
%         \toprule
%         Matériaux         & e         & $\lambda$      & $C_{p}$         & $\rho$          \\
%                           & \si{m}  & \si{W/(m.k)} & \si{J/(kg.k)} & \si{kg/m^{3}} \\
%         \midrule
%         Enduit            & 0.01      & 1.15           & 850             & 2000            \\
%         Optibric (R=1.32) & 0.2       & 0.1515         & 1000            & 685             \\
%         Laine de verre    & 0.14      & 0.03218        & 840             & 20              \\
%         Plâtre            & 0.01      & 0.25           & 1000            & 820             \\
%         \bottomrule
%     \end{tabular}
%     \caption{Description des parois verticales.}
%     \label{tab:compo_mur}
% \end{table}

% \begin{table}
%     \begin{tabular}{l *4{c}}
%         \toprule
%         Matériaux              & e         & $\lambda$        & $C_{p}$         & $\rho$          \\
%                                & \si{m}  & \si{W/(m.k)}   & \si{J/(kg.k)} & \si{kg/m^{3}} \\
%         \midrule
%         Chape béton            & 0.05      &  0.92            & 880             & 2300            \\
%         Polyuréthane           & 0.1       &  0.0215          & 1590            & 35              \\
%         Hourdis + polystyrène  & 0.12      &  0.0276          & 1450            & 35              \\
%         Chape béton            & 0.05      &  0.92            & 880             & 2300            \\
%         \bottomrule
%     \end{tabular}
%     \caption{Description du plancher.}
%     \label{tab:compo_plancher}
% \end{table}

% \begin{table}
%     \begin{tabular}{l *4{c}}
%         \toprule
%         Matériaux               & e         & $\lambda$      & $C_{p}$         & $\rho$          \\
%                                 & \si{m}  & \si{W/(m.k)}   & \si{J/(kg.k)}    &\si{kg/m^{3}}  \\
%         \midrule
%         Laine de roche soufflée & 0.365     & 0.045        & 1030               & 150                 \\
%         Plâtre                  & 0.01      & 0.25         & 1000               & 820                 \\
%         \bottomrule
%     \end{tabular}
%     \caption{Description du plafond sous combles.}
%     \label{tab:compo_plafond}
% \end{table}

% \begin{table}
%     \begin{tabular}{l *4{c}}
%         \toprule
%         Matériaux & e         & $\lambda$      & $C_{p}$         & $\rho$          \\
%                   & \si{m}  & \si{W/(m.k)} & \si{J/(kg.k)} &\si{kg/m^{3}}  \\
%         \midrule
%         Tuile     & 0.04      & 1              & 800             & 2000            \\
%         \bottomrule
%     \end{tabular}
%     \caption{Description de la toiture.}
%     \label{tab:compo_toiture}
% \end{table}

% \begin{table}
%     \begin{tabular}{l *4{c}}
%         \toprule
%         Matériaux         & e         & $\lambda$      & $C_{p}$         & $\rho$          \\
%                           & \si{m}  & \si{W/(m.k)} & \si{J/(kg.k)} &\si{kg/m^{3}}  \\
%         \midrule
%         Laine de verre    & 0.1       & 0.03218        & 840             & 20              \\
%         Laine de verre    & 0.1       & 0.03218        & 840             & 20              \\
%         Plâtre            & 0.01      & 0.25           & 1000            & 820             \\
%         \bottomrule
%     \end{tabular}
%     \caption{Description des parois du puits de jour.}
%     \label{tab:compo_puits}
% \end{table}

% \begin{table}
%     \begin{tabular}{l *4{c}}
%         \toprule
%         Matériaux         & e         & $\lambda$      & $C_{p}$         & $\rho$          \\
%                           & \si{m}  & \si{W/(m.k)} & \si{J/(kg.k)} &\si{kg/m^{3}}  \\
%         \midrule
%         Plâtre            & 0.025     & 0.25           & 1000            & 900             \\
%         Air (R=0.15)      & -         & -              & -               & -               \\
%         Plâtre            & 0.025     & 0.25           & 1000            & 900             \\
%         \bottomrule
%     \end{tabular}
%     \caption{Description des murs de partitions.}
%     \label{tab:compo_partition}
% \end{table}

% \begin{table}
%     \begin{tabular}{l *4{c}}
%         \toprule
%         Matériaux         & e         & $\lambda$      & $C_{p}$         & $\rho$          \\
%                           & \si{m}  & \si{W/(m.k)} & \si{J/(kg.k)} &\si{kg/m^{3}}  \\
%         \midrule
%         Béton             & 0.2       & 1.13           & 1000            & 2000            \\
%         Terre             & 4         & 0.52           & 50              & 2050            \\
%         \bottomrule
%     \end{tabular}
%     \caption{Description des parois du vide sanitaire.}
%     \label{tab:compo_VS}
% \end{table}


% \itodo{Tableau des vitrages}
% \begin{table}
%     \begin{tabular}{l *8{c}}
%         \toprule
%         Matériaux & $e$       & $\lambda$      & $\tau_{sol}$ & $\tau_{IR}$ & Émis$_{ext}$ & Émis$_{int}$ & Réflec$_{ext}$ & Réflec$_{int}$ \\
%                   & \si{m}  & \si{W/(m.k)} & -              & -         & -                & -                & -                & -                \\
%         \midrule
%         SGG Planilux       & 0.04  & 1          & 0.849      & 0           & 0.837              & 0.837              & 0.076              & 0.076                   \\
%         Argon              & 0.016 & -          & -          & -           & -                  & -                  & -                  & -                       \\
%         SGG PlaniTherm One & 0.04  & 1          & 0.591      & 0           & 0.037              & 0.837              & 0.312              & 0.264                   \\
%         \bottomrule
%     \end{tabular}
%     \caption{Description des vitrages en façades verticales ($U_{f}$ de 1.4~\si{W/m^{2}.k}).}
%     \label{tab:compo_vitrage}
% \end{table}

% \begin{table}
%     \begin{tabular}{l *8{c}}
%         \toprule
%         Matériaux & $e$       & $\lambda$      & $\tau_{solaire}$ & $\tau_{IR}$ & Émis$_{ext}$ & Émis$_{int}$ & Réflec$_{ext}$ & Réflec$_{int}$  \\
%                   & \si{m}  & \si{W/(m.k)} & -              & -         & -          & -          & -            & -             \\
%         \midrule
%         Verre ext & 0.01      & 1              & 0.44             & 0           & 0.88        & 0.88         & 0.075          & 0.075           \\
%         Air       & 0.016     & -              & -                & -           & -           & -            & -              & -               \\
%         Verre int & 0.01      & 1              & 0.44             & 0           & 0.88        & 0.88         & 0.075          & 0.075           \\
%         \bottomrule
%     \end{tabular}
%     \caption{Description de la fenêtre de toit ($U_{f}$ de 1.8~\si{W/m^{2}.k}).}
%     \label{tab:compo_velux}
% \end{table}

% \begin{table}
%     \raggedright
%     \begin{tabular}{l *5{c}}
%         \toprule
%         Gaz   & $a_{k}$              & $b_{k}$              & $a_{mu}$             & $b_{mu}$             & $a_{c}$           \\
%               & \si{W/(m.k)}       & \si{W/(m.k^{2})}   & Pa.s               & \si{N.s/(m^{2}.k)} & \si{J/(kg.k)}   \\
%         \midrule
%         Argon & $2.285\times10^{-3}$ & $5.149\times10^{-5}$ & $3.379\times10^{-6}$ & $6.451\times10^{-8}$ & 521.9285          \\
%         Air   & $2.873\times10^{-3}$ & $7.760\times10^{-5}$ & $3.723\times10^{-6}$ & $4.940\times10^{-8}$ & 1002.737          \\
%         \bottomrule
%     \end{tabular}
%     \bigskip
%     \begin{tabular}{l *3{c}}
%               & $b_{c}$                & $MM$                  & $P0$       \\
%               & \si{J/(kg.k^{2})}    & \si{kg/mol}         & \si{bar} \\
%         \midrule
%         Argon & 0                      & $39.948\times10^{-3}$ & 101325     \\
%         Air   & $1.2324\times10^{-2}$  & $28.97\times10^{-3}$  & 101325     \\
%         \bottomrule
%     \end{tabular}
%     \caption{Description des caractéristiques des gaz}
%     \label{tab:compo_gaz}
% \end{table}


% % subsection description_de_l_enveloppe (end)


