% Chap4-OptimisationSystemeSolaire

\section{Construction d’un système solaire: Un problème d’optimisation multicritère} % (fold)
\label{sec:construction_d_un_systeme_solaire_un_probleme_d_optimisation}
-----------------------------------
% Rappeler ce qui différencie cette approche des autres issues de la littérature:
%  - Approche couplée système/enveloppe
%  - Optimisation complète systèmes/contrôle/enveloppe

% Aide à la décision nécessite en set de solutions optimales:
%  - Génération d’un jeu de solutions optimales

L’optimisation multi-critère est aujourd’hui largement utilisé dans le bâtiment.
On a par exemple \cite{} qui a développé une méthode d’optimisation pour les
construction bois en utilisant un meta-modèle de bâtiment. Elle utilise un meta-heuristique
à population (Particule Swarm optimization) et évalue les besoins en énergie, le confort des
occupants, la sécurité de l’ouvrage et de l’impact environnemental. \cite{} a quand à lui
utilisé une méthode approchée (NSGA-II) et une exacte (programmation dynamique)
pour identifier des programmes séquentiels efficaces de réhabilitation énergétique.
Il a ainsi optimisé la combinaison des modifications pour chaque phase mais aussi l’ordre
dans lequel ces améliorations doivent être réalisées afin d’être le plus optimal
possible.
Pour les différentes solutions l’impact environnemental, le confort des occupants en
période estivale, et le coût ont été évaluées.

\NOTE{Ajouter des sources vers des exemples d’optimisation de bâtiment}


On a aussi de nombreux exemples d’optimisation de système énergétique.

\NOTE{Ajouter des sources vers des exemples d’optimisation de système}

On voit donc que l’optimisation est un outil qui a été largement utilisé dans
le bâtiment comme pour l’amélioration ou l’identification de solutions performantes
pour les systèmes.
L’originalité de ce travail provient principalement du couplage entre le bâtiment
et ses systèmes. La plupart des optimisations de bâtiment évalues les besoins
du bâtiment et considère donc un système de chauffage et de ventilation idéaux.
Dans ces travaux on cherche à optimiser la partie système et son algorithme de
contrôle en même temps que l’enveloppe du bâtiment. On évalue donc plus un besoin
en énergie mais une consommation qui est fonction de la performance des systèmes
envisagées. Les travaux se concentre sur l’évaluation de solutions utilisant fortement
l’énergie solaire comme vecteur énergétique. On cherche donc à couvrir les besoins
de chauffage, d’eau chaude sanitaire (ECS) et d’électricité.
Cette approche permettra d’évaluer le potentiel d’autonomie solaire disponible
pour différentes combinaisons systèmes/enveloppe.

\NOTE{Ajouter du bla bla sur l’originalité et les perspectives de ces travaux}

Enfin ces travaux vise à l’élaboration d’un outil d’aide à la décision. Il existe
diverses methodes pour faire de l’aide à la décision comme décrites dans le chapitre
précédent~\autoref{sec:multi_critere}. L’approche choisie est de déterminer un
ensemble de solutions non-dominées dans un premier temps, puis d’utiliser des
outils d’aide à la décision pour réduire le nombre de solution. On se trouve donc
dans une approche par front de Pareto.

\NOTE{Ajouter du bla bla sur le choix de l’ordre entre optimisation et aide à la décision}


Dans les sections suivantes nous décrirons dans un premier temps les hypothèses
retenues pour l’étude. Dans un second temps nous décrirons le processus retenu
pour l’outil d’aide à la décision.

\NOTE{Reformuler tout ça pour mieux introduire les parties qui suivent}

% section construction_d_un_systeme_solaire_un_probleme_d_optimisation (end)




\section{Cas d’étude: Un système solaire combiné couplée à une maison passive} % (fold)
\label{sec:cas_d_etude_un_systeme_solaire_combine_couplee_a_une_maison_passive}
-----------------------------------
\subsection{Hypothèses retenues} % (fold)
\label{sub:hypotheses_retenues}
\subsubsection{Description du site étudié} % (fold)
\label{ssub:description_du_site_etudie}
% Description du site étudié:
%  - Climat
%  - Données météos
%  - ...
Afin d’évaluer et d’optimiser la performance d’un système solaire, il est nécessaire
que les conditions extérieures ne soit pas fortement favorable.
Les climats méditerranéens ne sont donc pas des cas d’étude intéressant du fait
de la forte couverture solaire.
À l’opposé le climat de Limoges est assez rude et l’ensoleillement durant la période
hivernale est faible. C’est donc un cas d’étude intéressant.

\NOTE{Ajouter une carte pour localiser la ville}\\
\NOTE{Ajouter une description plus complète avec DJU, extrèmes, ...}\\
\NOTE{Refaire complètement ce paragraphe car il fait pitié actuellement}
% subsubsection description_du_site_etudie (end)


\subsubsection{Description de la maison etudiée} % (fold)
\label{ssub:description_de_la_maison_etudiee}
% Description de la maison:
%  - Composition de base
%  - Surface
%  - ...
La maison fait 100\,\si{m^{2}} est est composée de ...

\NOTE{Ajouter du bla bla sur la composition de la maison}
% subsubsection description_de_la_maison_etudiee (end)


\subsubsection{Description des paramètres fixes} % (fold)
\label{ssub:description_des_parametres_fixes}
% Descriptions des paramètres fixes:
%  - L’occupation
%  - La température de consigne
%  - Charges internes
%  - ...

\NOTE{Décrire les différents scénarios avec des illustrations}

% subsubsection description_des_parametres_fixes (end)
% subsection hypotheses_retenues (end)
% section cas_d_etude_un_systeme_solaire_combine_couplee_a_une_maison_passive (end)



\section{Formulation du problème d’optimisation} % (fold)
\label{sec:formulation_du_probleme_d_optimisation}
-----------------------------------
Choix des variables de décisions:
 - Propre au bâtiment
 - Propre au système
 - Propre au contrôle

Définir chaque variables:
 - Type
 - Plage de variation
 - Unité
 - Description


Définition des objectifs:
 - Couverture solaire chauffage
 - Couverture solaire ECS
 - Coût de l’installation


Définition des contraintes:
 - Surface de toiture à partager (Photovoltaïque)
 - Équilibre entre production/consommation (Primaire)


Réduction du nombre de variables par l’étude de sensibilité:
 - Méthodes locales
 - Méthodes globales
 - Screening (Morris)

Nécessiter de limiter le nombre de variables à évaluer pour réduire la cardinalité
du problème.
% section formulation_du_probleme_d_optimisation (end)



\section{Méthode retenue pour le dimensionnement du système} % (fold)
\label{sec:methode_retenue_pour_le_dimensionnement_du_systeme}
-----------------------------------
Faire un comparatif des solutions existantes et des performances de celles-ci.
Mettre en avant le faible nombre de paramètres nécessaires (bien pour un outil bureau d’étude)
No-free-lunch-theorem est aussi un argument
Mettre en exergue le problème de temps de calcul qui sera le facteur influant
sur le choix de l’approche d’optimisation.

Description du Meta-heuristique choisi:

 - Origine de l’algorithme (inspiration des abeilles):
Décrire l’origine de cet algorithme. Décrire les recherches qui ont été faites sur
les abeilles et ensuite les conclusions tirées par l’inventeur pour formuler ce
meta-heuristique.
Les sources suivantes non pas encore été lues: \cite{Camazine_1991547}, \cite{Wisdom_of_the_Hive}
mais traitent du comportement des abeilles.

 - Formulation théorique:

    + Description des différentes étape de résolution de la méthode:
      Détails pour onlookers, employed, scout, ...
      Détails pour les paramètres nécessaires au fonctionnement de l’heuristique.

    + Description de la mise à jour de la position d’une source:
      Dans le cas de variables continues la formulation de base est applicable et sera
      donc conservée. Dans le cas de variables discrètes on utilisera une méthode alternative
      développée dans `10.1177/0021998308097681` et `tel-01234197, version 1`.

Description de la mise à jour de l’archive choisie:

 - Formulation théorique
 - Graphiquement

Analyse de sensibilité retenue:

 - Formulation théorique
 - Graphiquement

Ajouter une vision globale du processus d’optimisation sous forme de graphique
servant de résumé du chapitre.
% section methode_retenue_pour_le_dimensionnement_du_systeme (end)

