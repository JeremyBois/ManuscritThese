%!TEX root = ../main.tex
% Chapitres\Chap4-OptimisationSystemeSolaire.tex

% Plan temporaire
\itodo{Plan}
\itodo{Présentation \\ sensibilité \\ chaos \\ optimisation \\ résultats \\ discussions}


% ..............................................................................
% ..............................................................................
\section{Formulation du problème d’optimisation} % (fold)
\label{sec:formulation_du_probleme_d_optimisation}
\itodo{Décrire objectifs, contraintes, variables ...}
% ------------------------------------------------------------------------------
\subsection{Définition des objectifs et des contraintes} % (fold)
\label{sub:definition_des_objectifs_et_des_contraintes}

% - - - - - - - - - - - - - - - - - - - - - - - - - - - - - - - - - - - - - - -
\subsubsection{Les objectifs de l’étude} % (fold)
\label{ssub:les_objectifs_de_l_etude}
\itodo{Les objectifs: \\
       - Maximiser couverture solaire sur l’ECS \\
       - Maximiser couverture solaire sur le chauffage \\
       - Minimiser le coût de l’installation \\
       - Minimiser le retour sur investissement}
Dans cette étude on considère trois fonctions objectifs. On cherche dans un premier
temps à évaluer la performance du système solaire. Pour ce faire on évalue sa
performance sur le chauffage et sur la production d’ECS séparément. Il en a enfin
été noté dans ~\autoref{sub:approche_monozone} que certaines variations impactent
de manière différentes la part de chauffage et d’ECS. Enfin on a vu dans
~\autoref{sec:modelisation_des_systemes} que la production d’ECS reste prioritaire
sur le chauffage, la modification du profil de puisage aura donc un impact différent
sur le chauffage et l’ECS.
\itodo{Il est aussi nécessaire de mettre en valeur l’impact de l’algorithme sur
      les rendements}

Le dernier objectif qui sera pris en compte est l’\textbf{impact économique}. Il
est important de ne pas seulement se focaliser sur la performance du système afin
de filtrer les solutions certes très performantes mais non-réalisable dans les années
proches. Il est ainsi important de rappeler que ces travaux se focalise sur une
technologie innovante mais pouvant être implémentées aujourd’hui.
Ainsi ce facteur bien que discutable du fait de son caractère changeant est indispensable
dans notre étude pour guider la recherche et donner un ordre de prix pour une solution
type.
L’optimisation se portera ainsi sur la maximisation de la couverture solaire pour
(i) le chauffage, (ii) la production d’eau chaude sanitaire, et la minimisation
du coût d’installation et du temps de retour sur investissement.
\itodo{Décrire les différents objectifs retenues avec plus de bla bla}
% subsubsection les_objectifs_de_l_etude (end)

% - - - - - - - - - - - - - - - - - - - - - - - - - - - - - - - - - - - - - - -
\subsubsection{Le choix des variables de décision} % (fold)
\label{ssub:le_choix_des_variables_de_decision}
\itodo{Choix des variables de décisions: \\
       - Propre au bâtiment \\
       - Propre au système \\
       - Propre à l’algorithme}

Dans cette partie sera décrit les différentes variables qui ont été sélectionnées
en amont du **screening**.
\itodo{Décrire l’ensemble des variables considérées en amont de l’étude de sensibilité
      classée par groupe, système, contrôle, enveloppe, scénarios}
\itodo{Définir chaque variables avec type, plage de variation, unité, description}
\itodo{Dresser une liste des caractéristiques de chaque critère}
\itodo{Faire apparaître clairement (graphiquement) l’impact de chaque variables
      sur les différents objectifs peut être pas à mettre ici mais plus dans
      ~\autoref{sub:reduction_de_la_cardinalite_par_screening}}
% subsubsection le_choix_des_variables_de_decision (end)

% - - - - - - - - - - - - - - - - - - - - - - - - - - - - - - - - - - - - - - -
\subsubsection{Les contraintes de l’étude} % (fold)
\label{ssub:les_contraintes_de_l_etude}
\itodo{Définition des contraintes: \\
       - Équilibre prod/conso \\
       - Toiture limité entre photo et thermique}
Les objectifs sont maintenant clairement définies mais l’étude comporte plusieurs
contraintes qui doivent être prises en compte durant le processus d’optimisation.
La première contrainte est la surface de toiture qui est limitée et doit donc être
partagée entre les panneaux photovoltaïques et thermiques. Comme nous l’avons définie
(~\autoref{sec:modelisation_des_systemes}) le cas d’étude étudié comporte des pans
de toiture avec diverses orientations ce qui complète la première contrainte.
On a donc une double contrainte, à savoir
partager la surface de toiture pour chaque orientation entre les différents capteurs.
Des études ont déjà été réalisées pour évaluer le meilleur ratio entre photovoltaïque
et thermique\mtodo{Ajouter citation} mais traite le problème sans tenir compte des combinaisons
entre les équipements, la structure, la régulation, ... Ces travaux permettront donc
d’obtenir plus d’informations sur cet aspect encore aujourd’hui faiblement exploré.
\itodo{Retrouver la publication qui parle de ratio thermique/photovoltaïque}

La seconde contrainte est la place disponible dans la maison pour accueillir les
équipements.
\itodo{Bla bla bla}

Enfin la contrainte principale est sur l’équilibre entre production et consommation
d’énergie primaire. En effet l’approche MEPOS ou encore NZEB définit comme requit
pour une maison passive d’avoir un équilibre entre production et consommation.
On a donc ici une contrainte forte au niveau énergétique.
\itodo{Bla bla bla}
\itodo{Décrire les contraintes et la méthode utilisée pour les traiter}
% subsubsection les_contraintes_de_l_etude (end)
% subsection definition_des_objectifs_et_des_contraintes (end)
% section formulation_du_probleme_d_optimisation (end)




% ..............................................................................
% ..............................................................................
\section{Étude de sensibilité} % (fold)
\label{sec:etude_de_sensibilite}
\itodo{Présenter l’état avant avec les critères choisis}
\itodo{Décrire le processus}
\itodo{Sélection des critères en aval de l’étude}
% section etude_de_sensibilite (end)




% ..............................................................................
% ..............................................................................
\section{Optimisation} % (fold)
\label{sec:optimisation}
\itodo{Description résumé de la methode}
\itodo{Discussion sur le front de Pareto obtenu: Analyse répartition, convergence, ...}
% section optimisation (end)

