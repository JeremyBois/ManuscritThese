%!TEX root = ../main.tex
% Chapitres/Chap4-OptimisationSystemeSolaire.tex


\iunsure{Description commune de Bordeaux et Strasbourg pour éviter répétition (section 2, 3)}
\iunsure{Relancer les simulations Pareto optimales avec Dymola pour comparaison}
\iunsure{Nécessaire de garder des facteurs avec $sigma = \mu^{*}$ même si distance faible}


% ..............................................................................
% ..............................................................................
\section{Description de l’étude de cas} % (fold)
\label{sec:description_de_l_etude_de_cas}
L’étude de cas est réalisée sur le bâtiment décrit dans le chapitre II et les scénarios de
charges internes (équipements, éclairage, et occupants) sont issues de la simulation de
référence (\ref{sub:scenarios_de_reference}). De même, le scénario de référence est
retenue pour la consigne de chauffage ($19$-$18$-$16$) comme pour la ventilation
($90-20$). Pour le puisage en $ECS$, le profil $Réaliste$ est retenue tout en tenant
compte des variations hebdomadaires et mensuelles
(\ref{ssub:puisage_en_eau_chaude_sanitaire}). En effet, comme il a été montré, il est
important de tenir compte de la variation du profil de puisage au cours de l’année afin de
ne pas sur-estimer la performance du $SSC$. Enfin, l’inclinaison des capteurs est définie
suivant la pente de la toiture du bâtiment. Pour Bordeaux la pente du bâtiment existant
est de \SI{33}{\percent} soit \SI{18.9}{\degree}. Pour Strasbourg, il est retenu une pente
de \SI{54}{\degree} typique d’une architecture alsacienne.

L’optimisation est réalisée dans un premier temps sur Bordeaux, dont le climat
est doux et l’ensoleillement important. Dans un second temps, sur Strasbourg caractérisé
par un climat rude et un ensoleillement limité durant la période de chauffage bien que
important durant la période estival (\figref{fig:diff_ensoleillement_bor_stras}).


\begin{figure}
    \centering
    \ftodo{Ajouter comparatif de l’évolution mensuelle de l’énergie disponible pour les deux.}
    % \includegraphics{Ressources/Images/Modelisation/Capteurs/IAM_all.pdf}
    \caption[Évolution mensuel du potentiel solaire sur Bordeaux et Strasbourg]
            {Évolution mensuel du potentiel solaire sur Bordeaux et Strasbourg}
    \label{fig:diff_ensoleillement_bor_stras}
\end{figure}

L’optimisation multi-objectifs est réalisée à la fois sur l’enveloppe, le \abr{SSC},
et sur l’algorithme de contrôle. Dans un premier temps, les hypothèses
et paramètres a priori sont explicités, puis les contraintes et objectifs retenus
discutés. Afin de limiter la
cardinalité du problème, une étude de sensibilité est réalisée en utilisant la méthode de
\textit{Morris}. Comme explicité précédemment elle permet d’identifier les facteurs les
plus influents de manière qualitative. Le but étant de réduire le nombre de variables de
décision pour l’optimisation. Une fois les variables influentes
identifiées, un échantillon représentatif pour chaque climat est construit grâce à une
méthode de \textit{quasi-Monte-Carlo} afin que les solutions soient uniformément réparties.
Ces échantillons sont ensuite utilisés afin de construire un modèle de substitution pour chaque objectif ou
contraintes. Ensuite l’optimisation par colonie d’abeilles virtuelles est réalisée
grâce à la bibliothèque \textit{pyMayBee}.
Une fois le front de Pareto obtenu, deux approches sont explorées~:
\begin{itemize}
  \item L’utilisation d’une méthode interactive d’aide à la décision avec le
        logiciel \textit{Xdat} afin de guider le choix d’une solution finale.
  \item L’utilisation de l’indicateur $FSC$ afin d’évaluer la capacité du \abr{SSC}
        à valoriser l’énergie solaire disponible.
\end{itemize}
Les deux approches sont complémentaires et ne visent pas le même public. La première
est plus utile pour faire le choix lors de la réalisation d’une construction, alors que
la seconde permet d’évaluer la capacité du \abr{SSC} à valoriser l’énergie solaire
disponible.
L’utilisation de modèles de substitution introduisant toujours des approximations,
les solutions non-dominées sont simulées avec le modèle original. Les résultats obtenues
avec les deux approches sont alors discutés.



% ------------------------------------------------------------------------------
\subsection{Hypothèses retenus} % (fold)
\label{sub:hypotheses_retenus}
L’objectif de ces travaux est de déterminer les configurations existantes qui permettent
de garantir l’obtention d’une \abr{MEPOS} dont les besoins sont majoritairement couverts
par l’énergie solaire. Le terme \abr{MEPOS} considère ici quatre usages, les consommation
propres à la climatisation n’est ainsi pas discutées~:
\begin{itemize}
  \item La chauffage
  \item La production d’\abr{ECS}
  \item L’éclairage
  \item Les équipements électro-domestiques
\end{itemize}



% - - - - - - - - - - - - - - - - - - - - - - - - - - - - - - - - - - - - - - -
\subsubsection{Répartition de la toiture} % (fold)
\label{ssub:repartition_de_la_toiture}
Afin de pouvoir couvrir les consommations électriques, éclairage ($Conso_{éclairage}$) et
électro-domestique ($Conso_{électroménager}$), une production photovoltaïque ($Prod_{PV}$)
est retenue. La toiture doit donc être partagée avec d’une part les capteurs thermiques, et
d’autre part les capteurs photovoltaïques (\abr{PV}). Comme le montre la littérature et les résultats
de l’étude paramétrique l’orientation est pour le \abr{SSC} un des facteurs les plus
important. Le choix est donc fait de favoriser en priorité les capteurs solaires
thermiques sur le pan sud. Afin d’être le plus représentatif possible la géométrie
de la toiture et des capteurs est prise en compte.
En effet la maison comporte une toiture quatre pans, chaque pan est donc triangulaire~; par
contre les capteurs sont eux rectangulaires. Il apparaît donc clairement que la surface
de toiture réellement disponible ne peut pas être approximée par sa surface totale.
De plus, chaque capteur admet une surface totale supérieure à sa surface d’entrée,
les proportions variants fortement en fonction du capteur considéré.
Ainsi afin d’estimer la surface réellement disponible un algorithme de \enquote{packaging}
a été développé (annexe \ref{cha:repartition_des_capteurs}). À partir des dimensions
du capteur (largeur et longueur de la surface totale), l’algorithme identifie le
nombre maximal pouvant loger sur chaque pan de toiture sans superpositions.
Ainsi la surface de capteurs photovoltaïques ($Surf_{PV}$) est définie suivant
\eqref{eq:repartition_toiture} en tenant compte de leur géométrie propre.

\begin{equation}\label{eq:repartition_toiture}
  \begin{aligned}
    &Surf_{PV}^{sud}   &=& \max\left[0,\quad \min \left(Surf_{PV}^{tot},\quad Surf_{PV}^{capteur} \times
                                                      \Bigg\lfloor\frac{Surf_{dispo}^{sud} - Surf_{TH}^{tot}}{Surf_{PV}^{capteur}}\Bigg\rfloor\right)
                                   \right] \\
    &Surf_{PV}^{ouest} &=& \max \left[0,\quad \min\left(Surf_{dispo}^{ouest},\quad
                                                      (Surf_{PV}^{tot} - Surf_{PV}^{sud}\right)
                                    \right] \\
    &Surf_{PV}^{est} &=& \max \left[0,\quad \min\left(Surf_{dispo}^{est},\quad
                                                    (Surf_{PV}^{tot} - Surf_{PV}^{sud} - Surf_{PV}^{ouest}) \right) \right] \\
  \end{aligned}
\end{equation}

Avec $Surf_{PV}^{tot}$ la surface totale de capteur \abr{PV} a installée,
$Surf_{PV}^{capteur}$ la surface extérieure d’un capteur \abr{PV}, $Surf_{dispo}$ la
surface totale disponible sur le pan de toiture considéré, et $Surf_{TH}^{tot}$ la surface
installée de capteurs solaires thermiques. Sur le pan sud, il est admis que
la $Surf_{dispo}$ est calculée en fonction de la géométrie des capteurs solaires
thermiques. Les capteurs thermiques étant prioritaires sur ce pan de toiture, la surface
disponible pour les capteurs \abr{PV} est définie par $Surf_{dispo} - Surf_{TH}^{tot}$.
Sur les autres pans où il n’y a pas de capteurs thermiques, la $Surf_{dispo}$ est
définie en fonction de la géométrie (largeur et hauteur) des capteurs \abr{PV}. En effet,
les capteurs solaires thermiques et photovoltaïques n’ont pas les mêmes dimensions.
Dans tout les cas, si un capteur ne loge pas complètement sur le pan de toiture,
il est ajouté au pan suivant. Afin de favoriser au maximum la production solaire
les capteurs sont en priorité mis sur le pan sud, puis ouest, puis est.
% subsubsection repartition_de_la_toiture (end)


% - - - - - - - - - - - - - - - - - - - - - - - - - - - - - - - - - - - - - - -
\subsubsection{Conditions limites} % (fold)
\label{ssub:conditions_limites}
Comme pour l’étude paramétrique le coût important d’une simulation impose de simuler
sur une \textbf{période s’étendant du $\bm{1^{er}}$ octobre au $\bm{30}$ avril}. Le \abr{SSC}
couvre en effet les besoins sur le reste de l’année comme explicité dans le chapitre II.
Il est donc important de garder à l’esprit que la performance du \abr{SSC} sur l’année
complète est plus importante que celle obtenue sur la période la plus défavorable en
particulier sur la couverture des besoins en \abr{ECS} qui restent important durant
l’été.

Ainsi l’ensemble de solution non dominées obtenu au cours du processus d’optimisation
seront simulées sur une année complète afin d’obtenir la performance annuelle du \abr{SSC}.

La production d’énergie par les capteurs photovoltaïques est elle pré-calculée pour
les différentes combinaisons existantes. En effet les capteurs solaires thermiques
étant prioritaires sur le pan sud, la surface de capteurs \abr{PV} n’influence
pas la performance du \abr{SSC}. De la même manière, les charges internes propres
aux occupants sont fixes et suivent un scénario prédéfini. Ainsi la production
\abr{PV} et les charges internes sont évaluées sur \textbf{l’année complète}.

En couplant les résultats des deux simulations, il est donc possible d’évaluer
la consommation totale sur les quatre usages considérés. Le \abr{SSC} couvrant en
totalité les besoins sur la période non simulée, la consommation de l’appoint
est équivalente à la consommation des pompes. Il est donc possible d’approximer $Conso_{app}$
par la consommation de l’appoint sur la période simulée. Ainsi il peut être
obtenue la consommation totale suivant \eqref{eq:conso_totale}.

\begin{align} \label{eq:conso_totale}
  Conso_{tot} &= Conso_{app} + Conso_{usages} - Prod_{PV}  \\
              &= Conso_{app} + Conso_{électroménager} + Conso_{éclairage} - Prod_{PV} \\
\end{align}
% subsubsection conditions_limites (end)


% - - - - - - - - - - - - - - - - - - - - - - - - - - - - - - - - - - - - - - -
\subsubsection{Définition des contraintes} % (fold)
\label{ssub:definition_des_contraintes}
Afin de pouvoir proposer un ensemble d’alternatives, la maison est considérée
à énergie positive si le bilan charge / production (\ref{ssub:la_methodologie_de_calcul})
est proche de $0$~: $\abs{Conso_{tot}}   \leq  8 \quad (\si{kWh_{ep}\per\metre\squared})$

Ainsi les solutions produisant trop d’énergie comme celle produisant trop peu d’énergie
seront écartées.
% subsubsection definition_des_contraintes (end)
% subsection hypotheses_retenus (end)



% ------------------------------------------------------------------------------
\subsection{Définition des paramètres a priori} % (fold)
\label{sub:definition_des_parametres_a_priori}
Comme décrit dans le chapitre précédent, une analyse de sensibilité est nécessaire
afin de réduire la cardinalité du problème et mieux guider l’exploration durant l’optimisation.
Couvrant à la fois les caractéristiques de l’enveloppe et du \abr{SSC}, l’ensemble des
paramètres retenus a priori sont disponibles dans le \tabref{tab:parametre_a_priori}.

\begin{table}
\centering
\caption{Liste des paramètres a priori utilisés pour l’analyse de sensibilité.}
\label{tab:parametre_a_priori}
\begin{tabular}{l c c l}
  \toprule
  \addlinespace
                                               & Borne min     & Borne max   & Remarques                                                            \\
  \addlinespace
  \multicolumn{4}{l}{\textbf{\abr{SSC}}}                                                                           \\
  \midrule
  Nombre capteurs \abr{TH}                     & \num{2}       & \num{5} ou \num{7}* & \SIrange{4.6}{11.6}{\metre\squared} ou \SI{16.2}{\metre\squared}                            \\
  Type de capteurs \abr{TH}                    & -             & -           & Voir \tabref{tab:capteurs_specs_optimisation}   \\                                                                   \\
  $Ech_{sol}^{pos}$                            & \num{0.8}     & \num{1.3}   & Position relative à la hauteur du ballon                              \\
  Volume ballon tampon                         & \num{100}     & \num{500}   & \multirow{2}{*}{Géométrie adaptée proportionnellement}             \\
  Volume ballon $ECS$                          & \num{100}     & \num{500}   &                                                                      \\
  $Isolant_{ballon,\, ep}$ tampon                & \num{0.055}   & \num{0.12}  &  \multirow{2}{*}{Résistance dépendante du volume du ballon}   \\
  $Isolant_{ballon,\, ep}$ \abr{ECS}             & \num{0.055}   & \num{0.12}  &                                                           \\
  $Isolant_{réseau,\, ep}$                       & \num{0.013}   & \num{0.04}  & Résistance dépendante du nombre de capteurs                           \\
  $\Delta T_{sol}$                             & \num{5}       & \num{20}    &  -                                                                  \\
  $\Delta T min_{capteur}$                     & \num{0}       & \num{30}    &  -                                                                   \\
  $\Delta T min_{tampon}$                      & \num{0}       & \num{30}    &  -                                                                   \\
  $T3_{min}$                                   & \num{15}      & \num{40}    &  -                                                                   \\
  \\
  \addlinespace[\defaultaddspace]
  \multicolumn{4}{l}{\textbf{Enveloppe du bâtiment}}                                                                              \\
  \midrule
  Type de vitrage                              & -             & -           &  Voir \tabref{tab:carac_vitrages}                                             \\
  $R$ murs                                     & \num{4}       & \num{7}     &  -                                                                   \\
  $R$ plafond                                  & \num{6}       & \num{10}    &  -                                                                   \\
  $R$ plancher                                 & \num{6}       & \num{10}    &  -                                                                   \\
  Surface vitrée est                           & \num{4.3}     & \num{6.46}  & \multirow{4}{*}{Surface totale du mur \SI{26.4}{\metre\squared}}            \\
  Surface vitrée nord                          & \num{0.46}    & \num{0.684} &                                                                      \\
  Surface vitrée sud                           & \num{5.42}    & \num{8.13}  &                                                                      \\
  Surface vitrée ouest                         & \num{2.6}     & \num{3.89}  &                                                                      \\
  \\
  \addlinespace[\defaultaddspace]
  \multicolumn{4}{l}{\textbf{Production d’électricité}}                                                                     \\
  \midrule
  Nombre de capteurs $PV$                      & \num{8}       &  \num{24}   &  Capteurs thermiques prioritaires sur le pan Sud \\                                                             \\
  \bottomrule
  \end{tabular}
  \raggedright
  *  $5$ pour Bordeaux et $7$ pour Strasbourg.
\end{table}


% - - - - - - - - - - - - - - - - - - - - - - - - - - - - - - - - - - - - - - -
\subsubsection{Logique de contrôle} % (fold)
\label{ssub:logique_de_controle}
Au cours de l’étude paramétrique, $\Delta T_{sol}$ a été évalué comme impactant et
est donc retenue lors de l’application de la méthodologie d’aide à la décision.
Dans l’optique d’une valorisation maximale de l’énergie solaire pour le chauffage, il est aussi
investigué $3$ variations algorithmiques supplémentaires.
Le fonctionnement de base de l’algorithme admet une priorité sur la production d’\abr{ECS}
car les travaux de la littérature ont montré que l’énergie était mieux valorisée de cette
manière. Pour rappel, la température au niveau de l’échangeur solaire ($T3$) doit être
supérieure à \SI{30}{\degree} afin de pouvoir charger le ballon tampon. Le choix de cette
borne ($T3_{min}$) est arbitraire et l’impact de sa variation est donc investigué.
L’activation du chauffage solaire ($Solaire_{direct}$ ou $Solaire_{indirect}$) est aussi
contrôlé par une borne fixe \figref{fig:automate_chauffage}. Dans le cas du
$Solaire_{direct}$, la température en sortie des capteurs $T1$ doit être supérieure ou
égale au maximum entre \SI{40}{\degree} et la température de l’eau en sortie de
l’échangeur eau/air ($T7$). Dans le cas du $Solaire_{indirect}$, la température du ballon
tampon ($T5$) doit être supérieure à \SI{40}{\degree}. Afin de chercher à valoriser au
maximum l’énergie solaire pour le chauffage, il est proposé non pas une borne fixe mais un
différentiel de température minimal. Dans les deux modes de chauffage, la consigne
d’activation s’adapte ainsi au rapport entre la production et la demande. Il est
alors introduit deux nouvelles variables~: $\Delta min_{capteur}$ et $\Delta min_{tampon}$.
Le chauffage peut ainsi être en $Solaire_{direct}$ lorsque, la différence de température entre la sortie des capteurs
($T1$) et l’air après le caisson de mélange $T_{air}^{mix}$ est plus importante que
$\Delta min_{capteur}$. De même, le chauffage peut être en $Solaire_{indirect}$, lorsque
la différence de température entre le haut du ballon tampon et $T_{air}^{mix}$ est
supérieure à $\Delta min_{tampon}$. Afin d’éviter des instabilité, il est de plus ajouté
un hystérésis de \SI{5}{\degree} sur le différentiel de référence.
% subsubsection logique_de_controle (end)



% - - - - - - - - - - - - - - - - - - - - - - - - - - - - - - - - - - - - - - -
\subsubsection{Variables qualitatives} % (fold)
\label{ssub:variables_qualitatives}
Le choix a aussi été fait d’introduire deux capteurs plans et deux capteurs sous-vides avec
des caractéristiques optiques et thermiques hétérogènes. La recherche a été faite
à travers deux bases de données (\fnref{http://www.sunwindenergy.com}{\textit{Sun\&Wind Energy}} et
\fnref{http://www.solar-rating.org/}{\textit{ICC-SRCC}}) et seul des capteurs récents sont considérés.
Les capteurs ont tous une \textbf{surface totale} similaire afin de pouvoir évaluer
aisément leurs performances en fonction de la surface installée. Par contre, les caractéristiques
de chaque capteur sont décrites suivant la surface considérée dans leur certification \enquote{Solar Keymark},
indiquée par un \emph{*} (\tabref{tab:capteurs_specs_optimisation}). Les capteurs
retenus sont ainsi dans la mesure du possible, similaires d’un point de vue géométrique
mais offre une diversité d’un point de vue des caractéristiques de performance. Bien que
la surface totale soit similaire, la largeur et la hauteur de chaque capteur varie, et
l’algorithme de \enquote{packaging} est utilisé pour chaque variation afin d’évaluer le
nombre maximal de capteur pouvant loger sur chaque pan de toiture. Il est observée que
malgré les différences géométriques (capteur de chez \textit{Ritter}), il est possible de
loger le même nombre de capteur sur le pan sud, même sur Bordeaux où la surface disponible
est réduite. Comme pour le capteur solaire thermique \textit{IDMK\,25-AL}, une régression
linéaire est réalisée afin d’identifier les coefficients d’\abr{IAM} $b_{0}$ et $b_{1}$
(\figref{fig:correlation_IAM_all}). Il peut être noté que les capteurs sous-vides
(\textit{CPC $14$ Star}, \textit{SKY PRO $12$ CPC $58$}) sont beaucoup moins sensible à la
variation de l’angle d’incidence expliquant que leurs performances soient constantes sur
la journée comme montré dans le premier chapitre (\ref{fig:compare_static_quasi_dyn})

\begin{figure}
    \centering
    \includegraphics{Ressources/Images/Modelisation/Capteurs/IAM_all.pdf}
    \caption[Évolution des \abr{IAM}s en fonction de l’angle d’incidence]
    {Évolution des \abr{IAM}s respectifs des quatre capteurs considérés en fonction
     de l’angle d’incidence.}
    \label{fig:correlation_IAM_all}
\end{figure}

Finalement la performance des vitrages est aussi variable et $3$ variantes sont proposées
dont le détail est disponible dans le \tabref{tab:carac_vitrages}. Les caractéristiques
des vitrages sont ici présentés avec les coefficient caractéristiques mais les vitrages
sont implémentés de manière détaillés \ref{ssub:deperditions_a_travers_les_fenetres}.
Le type de vitrage s’applique sur l’ensemble des parois verticales, la fenêtre verticale
conserve ses caractéristiques propres.

Dans la suite du document l’optimisation est réalisée pour chaque type de capteur solaire
mais le type de vitrage est un paramètre variable. Ainsi le front de \textit{Pareto}
obtenu pour chaque capteur est discuté.

\begin{table}
\centering
\caption{Caractéristiques des capteurs solaires thermiques sélectionnés. La présence d’un astérisque (*)
         indique la surface utilisée pour les corrélations.}
\label{tab:capteurs_specs_optimisation}
\begin{tabular}{l c c c c r}
    \toprule
                                 & IDMK\,$25$-AL              & CPC $14$ Star            & SKY PRO $12$ CPC $58$      & Energy + ECO 25         & Unité                       \\
    \midrule
    Fabricant                    & Sonnenkraft                & Ritter                   & Kloben                     & Dima                    & -                           \\
    Type                         & Plan vitrée                & Sous-vide                & Sous-vide                  & Plan vitrée             & -                           \\
    \addlinespace[\defaultaddspace]
    Surface totale               & \num{2.52}                 & \num{2.62}               & \num{2.59}*                & \num{2.53}              & \si{\metre\squared}                  \\
    Surface d’entrée             & \num{2.33}*                & \num{2.33}*              & \num{2.28}                 & \num{2.31}*              & \si{\metre\squared}                  \\
    Longueur totale              & \num{2.061}                & \num{1.622}              & \num{1.927}                & \num{2.008}             & \si{\metre}                  \\
    Largeur totale               & \num{1.225}                & \num{1.616}              & \num{1.342}                & \num{1.258}             & \si{\metre}                  \\
    Poids à vide                 & \num{49}                   & \num{41.2}               & \num{51}                   & \num{34}                & \si{kg}                     \\
    Contenance                   & \num{1.7}                  & \num{2.31}               & \num{1.76}                 & \num{1.9}               & \si{\litre}                 \\
    \addlinespace[\defaultaddspace]
    $\eta_{0}$                   & \num{76.5}                 & \num{0.644}              & \num{64.1}                 & \num{74.0}              & \si{\percent}                     \\
    $a_{1}$                      & \num{3.951}                & \num{0.749}              & \num{0.935}                & \num{5.116}             & \si{W/(m^{2}\period K)}     \\
    $a_{2}$                      & \num{0,011}                & \num{0.005}              & \num{0.004}                & \num{0.023}             & \si{W/(m^{2}\period K^{2})} \\
    $b_{0}$                      & \num{-0.1396}              & \num{-0.0709}            & \num{-0.1050}              & \num{-0.1284}           & -                     \\
    $b_{1}$                      & \num{-0.0004}              & \num{0.0006}             & \num{0.0114}               & \num{-0.0008}           & -                     \\
    $K_{\theta,\, dif}$          & \num{93.0}                 & \num{93.0}               & \num{97.2}                 & \num{93.0}              & \si{\percent}  \\
    \addlinespace[\defaultaddspace]
    Fiche technique              & \figref{fig:caracs_idmk}   & \figref{fig:caracs_star} & \figref{fig:caracs_skypro} & \figref{fig:caracs_eco} & - \\
    \bottomrule
\end{tabular}
\end{table}

\begin{table}
\centering
\caption{Descriptif des caractéristiques (suivant \cite{NFEN410} et \cite{NFEN673})
         des différents vitrages envisagés.}
\label{tab:carac_vitrages}
\begin{tabular}{l c c c r}
  \toprule
                     & Planitherm XN       & Planitherm ONE       & OptiwhiteKGlass       & Unité                        \\
  \midrule
  Fabricant    & \fnref{http://fr.saint-gobain-glass.com/product/2422/sgg-planitherm-xn}{%
                       St Gobain}
               & \fnref{http://eg.saint-gobain-glass.com/product/1659/}{%
                       St Gobain}
               & \fnref{https://www.pilkington.com/en-gb/uk/products/product-categories/thermal-insulation/pilkington-k-glass-range/pilkington-k-glass}{%
                       Pilkington}                                                              & -                             \\
  Construction & \num{4}-\num{16}-\num{4}  & \num{4}-\num{16}-\num{4} & \num{4}-\num{16}-\num{4} & -                             \\
  Gaz          & Argon                     & Argon                    & Argon                    & -                             \\
  $U_{g}$      & \num{1.1}                 & \num{1.0}                & \num{1.5}                & \si{W/(m^{2}\period \kelvin)} \\
  $g$          & \num{82}                  & \num{49}                 & \num{78}                 & \si{\percent}                 \\
  \bottomrule
    \end{tabular}
\end{table}
% subsubsection variables_qualitatives (end)


% - - - - - - - - - - - - - - - - - - - - - - - - - - - - - - - - - - - - - - -
\subsubsection{Autres paramètres} % (fold)
\label{ssub:autres_parametres}
Les bornes inférieures et supérieures retenus pour les isolants représentent respectivement les valeurs typiques observés
pour des bâtiment du niveau \abr{RT\,$2012$} et des bâtiment à énergie positive.
La plage de variation pour les surfaces vitrées a été fixée arbitrairement à plus ou moins
\SI{20}{\percent} de leur valeur d’origine et les proportions de cadre sont ajustées
en fonction de la nouvelle surface (les caractéristiques du cadre et des ponts thermiques
restent constantes).
Le nombre minimal de $PV$ a été définie à $8$ afin de couvrir \SI{80}{\percent} (\SI{\approx
2408}{\kWh}) de la consommation des équipements internes (\SI{\approx 3082}{\kWh}) à minima.
Pour les capteurs solaires thermiques, il est considéré au minimum $2$ capteurs et respectivement
$5$ et $7$ pour Bordeaux et Strasbourg. Une distinction est faite entre les deux climats afin
de tenir compte des spécificités de chaque climats.
% subsubsection autres_parametres (end)
% subsection definition_des_parametres_a_priori (end)



% ------------------------------------------------------------------------------
\subsection{Définition des objectifs} % (fold)
\label{sub:definition_des_objectifs}
% - - - - - - - - - - - - - - - - - - - - - - - - - - - - - - - - - - - - - - -
\subsubsection{Approche initiale} % (fold)
\label{ssub:approche_initiale}
\noindent Dans un premier temps, fort de l’expérience acquise à travers l’étude paramétrique,
les objectifs sont formalisés de la manière suivante~:
\begin{itemize}
  \item Maximiser le $F_{sol}^{ECS}$
  \item Maximiser le $F_{sol}^{CH}$
  \item Minimiser la $Conso_{app}$
\end{itemize}

Formulé de cette manière, l’approche comporte plusieurs limites. Il est clair que même si
les \num{3} objectifs ne sont pas impactés par les mêmes paramètres, une tendance commune
existe et l’optimisation risque de converger vers un ensemble de solution très réduit voir
une solution unique.

Concernant les deux premiers objectifs, $F_{sol}^{ECS}$ et $F_{sol}^{CH}$, un autre
problème est soulevé. Comme identifié dans la littérature, ces indicateurs évaluent la
part solaire respectivement pour la production d’\abr{ECS} et le chauffage. Cependant une
augmentation de la part solaire ne signifie pas forcement une diminution de la part de
l’appoint en particulier durant la période estivale où le solaire est abondant. Même si
dans ces travaux, la de simulation est limitée à la période s’étendant du $1^{er}$ octobre
au $30$ avril, il est toujours possible que certains apports solaires soient inutiles, en
particulier les apports passifs au niveau des ballons.

Finalement, le troisième objectif introduit un biais important sur la surface de $PV$
limitant l’exploratoire de l’algorithme. En effet, l’approche retenue dans ces travaux
cherche à caractériser l’ensemble des combinaisons existantes permettant d’obtenir
une \abr{MEPOS} solaire. Ainsi deux configurations extrêmes sont identifiés~:
\begin{itemize}
  \item Avoir beaucoup de $PV$ et peu de capteurs thermiques
  \item Avoir beaucoup de capteurs thermiques et suffisamment de $PV$ pour couvrir
        les consommations des équipements internes.
\end{itemize}
Cependant la surface de capteurs \abr{PV} est uniquement prise en compte que dans le
calcul de la contrainte. Ainsi, lorsque l’optimisation trouve des solutions minimisant au
maximum la $Conso_{tot}$, la surface de capteurs \abr{PV} ne peut plus augmenter sans que
les solutions violent la borne inférieure de la contrainte. En effet d’après
\defref{def:dominance_de_pareto}, une solution est meilleure qu’une autre si et seulement
si elle est meilleure sur un objectif et au moins aussi bonne sur tous les autres. Il est
donc clair que comme la surface de \abr{PV} n’impacte pas les objectifs, des solutions
avec une surface de capteurs \abr{PV} importantes ne sont pas atteignables. Il pourrait
être considéré la minimisation de la $Conso_{tot}$ en place de la $Conso_{app}$ afin qu’un
objectif tiennent compte des capteurs \abr{PV}. Cependant toujours d’après
\defref{def:dominance_de_pareto}, le problème reste entier.
% subsubsection approche_initiale (end)


% - - - - - - - - - - - - - - - - - - - - - - - - - - - - - - - - - - - - - - -
\subsubsection{Approche retenue} % (fold)
\label{ssub:approche_retenue}
\noindent Afin de palier aux difficultés rencontrées, une autre formulation est proposée~:
\begin{itemize}
  \item Maximiser le $F_{sav}^{ECS}$
  \item Maximiser le $F_{sav}^{CH}$
  \item Maximiser la $Prod_{PV}$
  \item Minimiser la $Nombre_{PV}$
\end{itemize}

Dans cette nouvelle formulation, l’ensemble des paramètres influence au minimum un des
objectifs. Il est aussi clair que les différents objectifs sont antinomique.
En effet, réduire la surface de capteur thermique permet d’améliorer la production des
$PV$ (plus de capteurs au sud) mais impacte négativement les autres objectifs.
Inversement, une surface de capteur thermique importante implique une production des $PV$
plus faible. Ainsi l’ajout d’une maximisation de la $Prod_{PV}$, permet d’obtenir des
solutions variées~: avec une surface de $PV$ importante et une surface de capteur
thermique faible et inversement.

De plus en remplaçant les indicateurs $F_{sol}^{CH}$ et $F_{sol}^{ECS}$ par respectivement
$F_{sav}^{CH}$ et $F_{sav}^{ECS}$, l’algorithme favorise non plus les solutions
augmentant la part solaire, mais les solutions réduisant la part de l’appoint au
profit d’une part plus importante de solaire \eqref{eq:taux_economie_opti}. Cependant ces nouveaux objectifs
nécessitent un modèle de référence afin d’évaluer la réduction de la consommation
induite par l’ajout d’un \abr{SSC}. Comme il est évalué conjointement des variations
sur l’enveloppe et sur le \abr{SSC}, il est nécessaire d’obtenir une consommation
de référence pour chaque performance d’enveloppe afin d’évaluer uniquement
l’amélioration induite par le \abr{SSC}. Un modèle de référence sans apports solaires
(tous électrique) est donc introduit. Il admet un unique ballon de \SI{200}{\litre} pour le stockage
d’\abr{ECS} et une batterie électrique en terminal pour le chauffage.

\begin{align}\label{eq:taux_economie_opti}
  F_{sav}^{CH}   &= 1 - \frac{Conso_{app}^{CH}}{Conso_{ref}^{CH}} \\
  F_{sav}^{ECS}  &= 1 - \frac{Conso_{app}^{ECS}}{Conso_{ref}^{ECS}} \\
  F_{sav}        &= 1 - \frac{Conso_{app} + Conso_{aux}}{Conso_{ref}}
\end{align}

Finalement le dernier objectif, minimiser la $Nombre_{PV}$ , est introduit afin de proposer des solutions non-dominées
sur toute l’intervalle définie par la contrainte. En effet, sans cet objectif les solutions
retenues sont toutes proches de la borne inférieure (maximisation de la surface de capteurs \abr{PV})
car c’est l’espace de solution maximisant la production des capteurs \abr{PV}. Cependant ces travaux considèrent que
toutes les solutions comprises dans l’intervalle sont admissibles. Il existe en effet,
une forte incertitude sur la performance réelle à la fois du bâtiment et des équipements. Ces
incertitudes introduites par le choix des hypothèses sont inhérente à toutes simulations numériques.
De plus le parti pris de cette approche est de proposer une large variété de solutions dans le
respect de la contrainte de \enquote{bilan positif}~: proposer des solutions
sur l’ensemble de l’intervalle répond à cet objectif.

Le problème est maintenant correctement formulé afin de permettre d’explorer l’espace de décision et
répondre au problème initial~: explorer l’ensemble de solutions non-dominées permettant
d’obtenir une \abr{MEPOS} solaire.
% subsubsection approche_retenue (end)
% subsection definition_des_objectifs (end)
% section description_de_l_etude_de_cas (end)



% ..............................................................................
% ..............................................................................
\section{Méthode de criblage de Morris} % (fold)
\label{sec:methode_criblage_de_morris}
Afin de réduire la cardinalité du problème, la méthode de \enquote{screening} de
\textit{Morris} est retenue. L’analyse a été réalisée sur Bordeaux et Strasbourg en
considérant \num{20} trajectoires uniques et \num{4} niveaux. Dans un premier temps les
résultats sont discutés pour les principaux indicateurs d’un \abr{SSC}. Puis un graphe
d’influence est réalisé sur les indicateurs $F_{sol}^{CH}$, $F_{sol}^{ECS}$,
$Conso_{app}$, et $Conso_{tot}$ permettant de représenter les facteurs influents
considérés dans le reste de l’étude.

\subsection{Hypothèses} % (fold)
\label{sub:hypotheses}
La méthode de Morris nécessite que l’ensemble des variables soient continues
et puissent varier indépendamment les unes des autres. Ainsi pour des variables
qualitatives comme le type de capteurs solaires thermiques et le type de vitrage,
les caractéristiques principales ont été considérées comme indépendantes uniquement
lors de l’étude de sensibilité.
Les capteurs sont ainsi exprimées en fonction des coefficient $a_{1}$, $a_{2}$, et
du rendement optique ($\eta_{0}$). Ainsi, il est uniquement considéré le capteur
\textit{IDMK\,25-AL} auquel les variations sont appliquées.
Les vitrages sont eux exprimés en fonction de l’émissivité
du verre intérieur ($Émis_{int}$) et de son coefficient de transmission solaire ($\tau_{sol}$).
Ainsi il est considéré uniquement le vitrage \textit{Planitherm XN} auquel les variations
sont assignées. Les bornes inférieures et supérieures retenues sont issues des caractéristiques
des différents capteurs et vitrages considérés (\tabref{tab:variabilite_capteur_vitrage}).
Bien que la méthode permette de regrouper les paramètres afin d’évaluer
l’influence de l’ensemble et non de chaque élément le composant, les éléments groupés
nécessitent de pouvoir varier indépendamment. Cet option n’est donc pas applicable.
Ce choix est fait afin d’évaluer de possibles interactions existantes avec d’autres facteurs.
Bien entendu, dans le reste de l’étude (optimisation), les variables sont
considérées comme qualitatives et admettent uniquement un des choix existants.

Moins contraignant, l’échantillon créé par la méthode de \textit{Morris} suppose que
la variable est continue. Dans notre cas le nombre de capteurs solaires thermiques
et de capteurs \abr{PV} sont cependant obligatoirement des entiers. Afin de contourner
ce problème, la plage de variation de ces deux facteurs a été choisie comme~:
\begin{itemize}
  \item \SIrange{2}{5}{} capteurs solaires thermiques soit une variation de \SI{7}{\metre\squared}
  \item \SIrange{8}{14}{} capteurs \abr{PV} soit une variation de \SI{9}{\metre\squared}
\end{itemize}
Ce choix est fait afin d’obtenir la surface équivalente la plus proche possible, tout
en assurant que les valeurs prises lors de la création de l’échantillon sont des entiers
lorsque la méthode considère quatre niveaux.


\begin{table}
\centering
\caption{Variabilité des caractéristiques des capteurs et des vitrages pour l’étude
         de sensibilité.}
\label{tab:variabilite_capteur_vitrage}
\begin{minipage}[t][][b]{0.45\linewidth}
\begin{tabular}{l c c}
  \toprule
  \addlinespace
                                               & Borne min     & Borne max    \\
  \addlinespace[\defaultaddspace]
  \multicolumn{2}{l}{\textbf{Vitrages}}                                       \\
  \midrule
  $Émis_{int}$                                 & \num{0.047}   & \num{0.837}  \\
  $\tau_{sol}$                                 & \num{0.643}   & \num{0.849}  \\
  \bottomrule
  \end{tabular}
\end{minipage}%
\begin{minipage}[t][][b]{0.45\linewidth}
\begin{tabular}{l c c}
  \toprule
  \addlinespace
                                               & Borne min     & Borne max    \\
  \addlinespace[\defaultaddspace]
  \multicolumn{3}{l}{\textbf{Capteurs solaires}}                              \\
  \midrule
  $a_{1}$                                      & \num{0.749}   &  \num{5.116} \\
  $a_{2}$                                      & \num{0.004}   &  \num{0.023} \\
  $\eta_{0}$                                   & \num{0.644}   &  \num{0.764} \\
  \bottomrule
  \end{tabular}
\end{minipage}
\end{table}
% subsection hypotheses (end)



% - - - - - - - - - - - - - - - - - - - - - - - - - - - - - - - - - - - - - - -
\subsection{Analyse des résultats} % (fold)
\label{sub:analyse_des_resultats_morris}
% - - - - - - - - - - - - - - - - - - - - - - - - - - - - - - - - - - - - - - -
\subsubsection{Couverture solaire sur l’eau chaude sanitaire} % (fold)
\label{ssub:couverture_solaire_sur_l_ECS}
Les mêmes facteurs influents sont identifiés pour l’indicateur $F_{sol}^{ECS}$ que ce soit
pour Bordeaux ou Strasbourg, cependant les interactions et l’ordre d’importance diffèrent
(\figref{fig:objectifs_mu_star}). Dans les deux cas le nombre de capteurs \abr{TH} et le
volume du ballon \abr{ECS} sont les facteurs les plus influents. Sur Bordeaux, les
facteurs $\Delta T_{sol}$, $\Delta min_{capteur}$, et $\Delta min_{tampon}$ ont des effets
non-linéaires ou avec des interactions alors que les autres ont des impacts linéaires. Sur
Strasbourg, chaque facteur a un impact linéaire à l’exception de $\Delta T_{sol}$ dont
l’influence peut être estimée comme non-linéaire ou avec des interactions. Bien que
n’ayant pas d’action directe sur la production d’\abr{ECS}, la performance thermique des
vitrages est évaluée comme faiblement influente. Le \abr{SSC} répartissant l’énergie
solaire disponible entre chauffage et production d’\abr{ECS}, réduire les besoins de
chauffage permet de valoriser une plus grande part de l’énergie solaire disponible pour la
production d’\abr{ECS}.
% subsubsection couverture_solaire_sur_l_ECS (end)


% - - - - - - - - - - - - - - - - - - - - - - - - - - - - - - - - - - - - - - -
\subsubsection{Couverture solaire sur le chauffage} % (fold)
\label{ssub:couverture_solaire_sur_le_chauffage}
La même tendance que pour l’indicateur $F_{sol}^{ECS}$ est observée~: les facteurs
influents ont plus d’effets non-linéaires ou avec des interactions pour le climat de
Bordeaux que pour Strasbourg. En effet sur Bordeaux le volume du ballon \abr{ECS}, le
$\Delta min_{capteur}$, et la $Ech_{sol}^{pos}$ ont tous des effets non linéaires ou avec
interaction alors que sur Strasbourg, seul le $\Delta T_{sol}$ est identifié.
Pour les deux climats, le volume du ballon
tampon et le nombre de capteurs solaires thermiques sont les plus influents. Il apparaît
que les caractéristiques de l’enveloppe, exception faite des vitrages, ne soit pas parmi
les plus influentes même sur le climat strasbourgeois où les besoins en chauffage sont
importants. Il est même complètement exclut des facteurs influent la résistance thermique du plancher.
Sur Bordeaux et sur Strasbourg les paramètres liés au réglage de l’algorithme de contrôle
($\Delta T_{sol}$, $\Delta min_{capteur}$, $\Delta min_{tampon}$) sont aussi identifiés
comme fortement influents. Il est aussi noté pour Bordeaux un faible impact de la
$Ech_{sol}^{pos}$ et de la surface vitrée à l’est. L’épaisseur de l’isolant du ballon
tampon semble aussi être influent sur Bordeaux.

Afin de mieux comprendre pourquoi ces paramètres sont influents, l’analyse de
\textit{Morris} est réalisée en distinguant la part active ($Prod_{sol}^{CH}$ active) de
la part passive (pertes des ballons, $Prod_{sol}^{CH}$ passive) (annexes,
\figref{fig:prod_sol_chauffage_mu}). Pour Bordeaux et Strasbourg, il est ainsi mis en
évidence que l’amélioration de l’isolation des ballons à un effet négatif sur la part
passive mais aucuns effets sur la part active. À l’inverse lorsque le paramètre
$Émis_{int}$ augmente ($U_{g}$ diminue) la part solaire active est plus importante mais
aucuns effets n’est observé sur la part passive. Il semble donc qu’il existe un compromis
entre la qualité de l’enveloppe et la part des consommations couverte par le solaire. De
plus pour Bordeaux, si seule la part active est considérée il est noté que les facteurs
influents sont principalement, le nombre de capteur, la performance des vitrages, et le
$\Delta min_{tampon}$. En effet, augmenter le volume du ballon \abr{ECS} améliore la part
solaire passive mais diminue la part active, expliquant pourquoi ce facteur n’est pas
parmi les plus influent uniquement sur la part active. Par contre l’augmentation du volume
du ballon tampon améliore à la fois la part active et la part passive même si le facteur
est plus influent sur cette dernière.

Il apparaît que l’analyse des facteurs liés à l’algorithme de contrôle soit plus complexe.
Augmenter le $\Delta min_{tampon}$ semble influencer négativement la part active sur
Bordeaux et Strasbourg, mais permet d’améliorer la part passive uniquement sur
Strasbourg. Les conditions d’ensoleillement étant meilleures et les besoins moins
important sur Bordeaux, le ballon tampon est en moyenne à une température plus élevée et
faire varier le $\Delta min_{tampon}$ est donc moins impactant. Aussi il est observé que
augmenter le $\Delta min_{capteur}$ impacte négativement la part solaire active sur
Strasbourg et est fortement non-linéaire ou avec des interactions sur Bordeaux (annexe,
\figref{fig:prod_sol_chauffage_mu_star}). Finalement le $\Delta T_{sol}$ a un impact non-
linéaire ou avec des interactions sur Strasbourg pour la part solaire active mais ne fait
pas parti des facteurs influent sur Bordeaux.

\paragraph{Bilan} % (fold)
\label{par:bilan_prod_sol_chauff}
Il est exposer la complexité liée à l’évaluation couplée d’un \abr{SSC}
et d’un bâtiment. Bien que les facteurs identifiés comme influents soit similaires
entre Bordeaux et Strasbourg, l’analyse détaillée considérant séparément la part solaire
active et passive met en évidence que le \abr{SSC} ne se comporte pas de la même manière
sur les deux climats. Ainsi en plus d’une forte interaction
identifiée entre la $F_{sol}^{ECS}$ et la $F_{sol}^{CH}$, il est mis en évidence
une forte interaction entre le chauffage solaire actif et passif.
Pour cette raison il est préféré les indicateurs $F_{sav}^{ECS}$ et $F_{sav}^{CH}$
dans l’optimisation. En effet il n’existe pas à la connaissance de l’auteur de moyen
d’identifier la part passive utile de la part inutile. En considérant la part
économisé d’appoint, il est pris indirectement en compte de l’utilité de l’énergie
solaire apportée au bâtiment, que ce soit la part active ou passive.
Finalement un l’existence d’un compromis entre qualité de l’enveloppe et performance du \abr{SSC}
sur le chauffage est aussi identifié.
% paragraph bilan_prod_sol_chauff (end)
% subsubsection couverture_solaire_sur_le_chauffage (end)


% - - - - - - - - - - - - - - - - - - - - - - - - - - - - - - - - - - - - - - -
\subsubsection{Consommation de l’appoint} % (fold)
\label{ssub:consommation_de_l_appoint}
Sur la $Conso_{app}$ (\figref{fig:objectifs_mu_star}) les facteurs les plus influents sont
pour les deux climats~: le nombre de capteurs solaires thermiques et l’$Émis_{int}$ des
vitrages. Les autres caractéristiques de l’enveloppe ont aussi des influences linéaires
($R$ plafond, $R$ murs) sur Bordeaux et Strasbourg. La qualité du plancher est cependant
identifiée comme impactant uniquement sur Strasbourg. Les caractéristiques de l’enveloppe
sont donc sur Strasbourg toutes influentes et la performance du vitrage est le facteur
prédominant.
L’algorithme semble aussi influencer directement la consommation de l’appoint avec
$\Delta min_{tampon}$ sur Bordeaux et $\Delta min_{capteur}$ sur Strasbourg, les deux
ayant une influence non-linéaire ou avec des interactions. Il est aussi observé que
le volume du ballon \abr{ECS} est un facteur important dans les deux climats alors
que le volume de l’appoint n’est influent que sur Strasbourg.

Afin de mieux comprendre, l’analyse est détaillée en séparant la part attribuée au
chauffage et à la production d’\abr{ECS} (annexe, \figref{fig:conso_app_mu_star}).
Si on considère uniquement les consommations sur le chauffage ($Conso_{app}^{CH}$),
la qualité thermique du vitrage arrive largement devant les autres facteurs que ce
soit pour Bordeaux ou Strasbourg. Sur Bordeaux le nombre de capteurs solaires
thermiques est aussi fort impactant par rapport aux autres facteurs. Par contre
sur Strasbourg, la performance thermique des murs est aussi importante que le nombre
de capteurs solaires expliqué par une part des besoins en chauffage plus important
pour ce climat.

$\Delta min_{tampon}$ et $\Delta min_{capteur}$ sont influents pour les deux climats
mais uniquement sur la $Conso_{app}^{CH}$. Sur la $Conso_{app}^{ECS}$ la $\Delta min_{tampon}$ et le seul
facteur impactant pour Bordeaux alors que sur Strasbourg, seul $\Delta min_{capteur}$ est influent.
Cette remarque permet d’expliquer pourquoi uniquement un des deux facteurs est influent lorsque
on considère la consommation totale $Conso_{app}$. La même observation peut être faite
pour le ballon tampon dont le volume impacte uniquement la $Conso_{app}^{CH}$ sur Bordeaux.
Enfin les caractéristiques des capteurs sont influentes à la fois sur la
$Conso_{app}^{CH}$ et la $Conso_{app}^{ECS}$.

\paragraph{Bilan} % (fold)
\label{par:bilan_conso_app}
L’analyse met en évidence que certain facteurs influencent soit la $Conso_{app}^{ECS}$,
soit pour la $Conso_{app}^{CH}$. Les besoins en chauffage sur Strasbourg étant plus
important que sur Bordeaux, les facteurs caractérisants l’enveloppe sont tous influent à a
fois sur $Conso_{app}^{CH}$ et la $Conso_{app}$ sur Strasbourg. Sur Bordeaux ou le
chauffage représente une part moindre par rapport aux besoins en \abr{ECS}, les facteurs
peu impactant sur le chauffage n’apparaissent pas lorsque la consommation totale est
considérée. Cette tendance est de plus renforcée par l’ensoleillement plus important sur
Bordeaux, permettant au
\abr{SSC} de couvrir un part plus importante des besoins et donc de diminuer
la variation de la consommation du chauffage lorsque l’enveloppe est modifiée. Ces
résultats confirment la nécessité de considérer les systèmes et l’enveloppe de concert
afin de proposer une solution adaptée aux contraintes du climat.
% paragraph bilan_conso_app (end)
% subsubsection consommation_de_l_appoint (end)


% - - - - - - - - - - - - - - - - - - - - - - - - - - - - - - - - - - - - - - -
\subsubsection{Consommation totale} % (fold)
\label{ssub:consommation_totale}
Concernant la consommation totale, le nombre de capteur \abr{PV} est le plus influent.
En effet, les consommations propres aux charges internes (électro-domestique et éclairage)
représentent une part importante des consommations. De plus, la surface de capteurs
thermiques minimale permet déjà de couvrir une partie des besoins, notamment sur la
production d’\abr{ECS}. Il est aussi important de garder à l’esprit que la surface de
capteur \abr{PV} varie de \SI{2}{\metre\squared} de plus que celle des capteurs solaires
thermiques pour les raisons explicitées en introduction de section. Ces observations permettent
d’expliquer pourquoi le nombre de capteur \abr{PV} est le facteur le plus influent.
Cependant, sur Strasbourg ou les besoins en \abr{ECS} et en chauffage sont prédominants,
la performance des vitrages et le nombre de capteurs solaires thermiques sont aussi
fortement influents.
% subsubsection consommation_totale (end)


% - - - - - - - - - - - - - - - - - - - - - - - - - - - - - - - - - - - - - - -
\subsubsection{Valorisation de l’énergie solaire~:} % (fold)
\label{ssub:valorisation_de_l_energie_solaire}
Cette dernière partie décrit les facteurs influençant la part solaire absorbée
au niveau des capteurs ($Prod_{sol}$), les pertes en ligne ($Pertes_{réseau}$),
et la part solaire valorisée ($Prod_{sol}^{valorisée} = Prod_{sol} - Pertes_{réseau}$)
(annexe, \figref{fig:prod_sol_valorisee_mu_star}).
Pour Bordeaux comme Strasbourg le volume des ballons, le type de capteur, et le
nombre de capteurs solaires influencent la $Prod_{sol}^{valorisée}$.
Comme observée pour la $Conso_{app}$, le $\Delta min_{capteur}$ n’est influent
que sur le climat de Strasbourg, et le $\Delta min_{tampon}$ ne l’est que sur Bordeaux. Ces deux
facteurs ayant une influence non-linéaire ou avec des interactions.
Sur Bordeaux, il est aussi mis en évidence que la performance des vitrages a un impact
non linéaire ou avec des interaction importantes. Afin de l’expliquer, il est nécessaire
de croiser plusieurs informations. Il a été vu qu’un compromis existe entre qualité
de l’enveloppe et performance du \abr{SSC} sur le chauffage (\ref{ssub:couverture_solaire_sur_le_chauffage}).
De plus, les résultats de l’étude paramétrique montrent que sur le climat de Bordeaux,
les besoins en chauffage sont faibles et que le solaire peut couvrir la majeure partie.
Avec un climat disposant d’un ensoleillement important, le \abr{SSC} sur Bordeaux
peut fournir l’énergie supplémentaire nécessaire lorsque les besoins de chauffage
augmente (augmentation de l’$Émis_{int}$). Sur Strasbourg ce facteur n’est donc
pas influent car le \abr{SSC} couvre moins bien les besoins de chauffage plus importants.
De plus l’ensoleillement est moindre et l’augmentation des pertes thermiques induites
par une diminution de la performance des vitrages et plus grande.

Enfin, les $Pertes_{réseau}$ sont fortement impactées par la surface totale de capteur
comme de l’épaisseur de l’isolant au niveau des canalisations. Le volume du ballon
sanitaire a aussi un impact important. En comparant l’effet élémentaire sur la moyenne
(\figref{fig:prod_sol_valorisee_mu}) et sur la moyenne pondérée, il apparaît que augmenter
le volume du ballon réduise les pertes en ligne. Les variations algorithmiques impacte
aussi les $Pertes_{réseau}$. $\Delta min_{capteur}$ a en effet une influence non-linéaire
ou avec interaction pour les deux climats et $\Delta min_{tampon}$ uniquement pour
Bordeaux. La part solaire perdue en ligne ($Pertes_{réseau}$) étant faible au regard des
apports solaires ($Prod_{sol}$), l’isolation des canalisations ($Isolant_{réseau}^{ep}$)
n’est pas influent sur la $Prod_{sol}^{valorisée}$. Par extension, il n’est pas influent
non plus sur les indicateurs $F_{sol}^{ECS}$ ou $F_{sol}^{CH}$.
% subsubsection valorisation_de_l_energie_solaire (end)


\begin{figure}
    \centering
    \includegraphics{Ressources/Images/Sensibilite/sigma_mu_star_1.pdf}
    \caption{Résultat de l’analyse de \textit{Morris} pour les indicateurs principaux
             ($f(\mu^{*}) = \sigma$).}
    \label{fig:objectifs_mu_star}
\end{figure}
% subsection analyse_des_resultats (end)



% - - - - - - - - - - - - - - - - - - - - - - - - - - - - - - - - - - - - - - -
\subsection{Paramètres retenus} % (fold)
\label{sub:parametres_retenus}
L’analyse de sensibilité a permis de réduire le nombre de critères de décision
pour Bordeaux et Strasbourg. Ainsi sur les $22$ facteurs a priori, il en est retenu
$14$ pour Bordeaux et $15$ pour Strasbourg.  Il est en effet retenue à la fois des variables discrètes, continues,
et qualitatives dont les bornes de variation sont décrites dans \tabref{tab:facteur_retenues}.
Aussi, suite aux observations faites dans l’analyse des résultats, la borne maximale pour le volume des
ballons a été réajustée à \SI{400}{\litre} maximum par ballon contre \SI{500}{\litre}
auparavant.
Les résultats du criblage sont aussi présentés sous une forme synthétique grâce à un
graphe d’influence qui décrit le lien entre chaque facteur, et chaque indicateur
(\figref{fig:graphe_influence_objectifs}). La visualisation permet de mettre en évidence
les facteurs influents peu (Surfaces vitrées, isolation du ballon tampon\dots) ou beaucoup
d’indicateurs (type de vitrage, nombre de capteurs \abr{PV}\dots).

L’analyse a aussi permis de mettre en évidence plusieurs remarques~:
\begin{itemize}
  \item Le \abr{SSC} est fortement influencé par des éléments de l’enveloppe, des équipements,
        mais aussi par les variations algorithmiques.
  \item $16$ facteurs différents ont été identifiés comme influents sur les objectifs de
        l’optimisation pour Bordeaux et Strasbourg cumulés.
  \item Augmenter l’isolation des canalisations n’a pas d’intérêts.
  \item Les variations du \abr{SSC} sont plus influente lorsque la $Conso_{app}$
        est plus importante que la consommation des équipements internes.
  \item Le potentiel de réduction de la consommation de l’appoint est plus important
        pour Strasbourg que Bordeaux.
  \item Il existe un compromis entre qualité de l’enveloppe, la performance du \abr{SSC}
        et la surface installée de capteur \abr{PV}.
  \item Le dimensionnement d’un bâtiment doit être le résultat d’une évaluation
        couplée du bâtiment et du système installée, en particulier lorsque le système
        utilise des énergies renouvelables.
\end{itemize}

Malgré une réduction du nombre de paramètres de décisions, la temps de simulation
important est toujours un frein à l’application d’une méthode d’optimisation
par méta-heuristique. De plus, il est nécessaire d’évaluer en parallèle le modèle
de référence pour chaque variation de l’enveloppe afin de pouvoir calculer l’économie
d’appoint permise par le \abr{SSC}~: $F_{sav}^{CH}$ et $F_{sav}^{ECS}$. Il est donc
proposer d’utiliser des modèles de substitutions.

\begin{figure}
    \centering
    \includegraphics{Ressources/Images/Sensibilite/graphInfluence.pdf}
    \caption[Graphe d’influence du $SSC$ pour Bordeaux et Strasbourg]
            {Graphe d’influence du $SSC$ pour Bordeaux (gauche) et Strasbourg (droite) où les
             relations linéaires sont de couleur noire, et les relations non-linéaires ou
             avec interactions de couleur bleue.}
    \label{fig:graphe_influence_objectifs}
\end{figure}



\begin{table}
\centering
\caption[Description des paramètres retenus pour l’optimisation]
         {Description des paramètres retenus pour l’optimisation. Les facteurs uniquement retenus
          sur Bordeaux sont sous un fond crème, et ceux uniquement pour Strasbourg sous un fond gris.}
\label{tab:facteur_retenues}
\begin{tabular}{l c c c c l}
  \toprule
  \addlinespace
                       & Min        & Max         & Catégorie  & Pas        & Remarques                                \\
  \addlinespace
  \multicolumn{5}{l}{\bm{$SSC$}}         \\
  \midrule
  \rowcolor{SolarizedBrWhite}
  Nombre capteurs \abr{TH}    & \num{2}    & \num{5} & Discrète    & \num{1}    & \SIrange{4.6}{11.6}{\metre\squared}  \\
  \rowcolor{SolarizedBrCyan}
  Nombre capteurs \abr{TH}    & \num{2}    & \num{7} & Discrète    & \num{1}    & \SIrange{4.6}{16.2}{\metre\squared}   \\
  Type capteurs \abr{TH}      & -          &  -      & Qualitative & -          & Voir \tabref{tab:capteurs_specs_optimisation}   \\

  $Ech_{sol}^{pos}$           & \num{0.8}  &  \num{1.3}  & Continue    & -          & Position relative à la hauteur du ballon     \\
  Volume ballon tampon        & \num{100}  &  \num{400}  & Discrète    & \num{50}   & \multirow{2}{*}{Dimensions adaptées proportionnellement}   \\
  Volume ballon $ECS$         & \num{100}  &  \num{400}  & Discrète    & \num{50}   &    \\
  \rowcolor{SolarizedBrWhite}
  $Isolant_{ballon}$ tampon   & \num{0.055} &  \num{0.12} & Discrète    & \num{50}   &  Résistance dépendante du volume du ballon  \\
  $\Delta T_{sol}$            & \num{5}    &  \num{20}   & Continue    & -          &  -      \\
  $\Delta min_{capteur}$      & \num{0}    &  \num{30}   & Continue    & -          &  -      \\
  $\Delta min_{tampon}$       & \num{0}    &  \num{30}   & Continue    & -          &  -      \\
  \\
  \addlinespace[\defaultaddspace]
  \multicolumn{4}{l}{\textbf{Enveloppe du bâtiment}}             \\
  \midrule
  $R$ murs             & \num{4}    &  \num{7}    & Discrète    & \num{0.5}  & -                                  \\
  $R$ plafond          & \num{6}    &  \num{10}   & Discrète    & \num{0.5}  & -                                                                      \\
  \rowcolor{SolarizedBrCyan}
  $R$ plancher         & \num{6}    &  \num{10}   & Discrète    & \num{0.5}  & -                                                                     \\
  \rowcolor{SolarizedBrCyan}
  Surface vitrée sud   & \num{5.42} &  \num{8.13} & Continue    &  -          & -       \\
  Surface vitrée est   & \num{4.3}  &  \num{6.46} & Continue    &  -          & - \\
  Type de vitrage      & -          &  -          & Qualitative &  -         & Voir \tabref{tab:carac_vitrages} \\
  \\
  \addlinespace[\defaultaddspace]
  \multicolumn{5}{l}{\textbf{Production d’électricité}}      \\
  \midrule
  Nombre capteurs \abr{PV}         & \num{14}   &  \num{24}   & Discrète    &  \num{1}   & Capteurs thermiques prioritaires sur le pan Sud   \\
  \bottomrule
\end{tabular}
\end{table}
% subsection parametres_retenus (end)
% section methode_criblage_de_morris (end)




% ..............................................................................
% ..............................................................................
\section{Construction d’un modèle de substitution} % (fold)
\label{sub:construction_d_un_modele_de_substitution}
\subsection{Création de l’échantillon} % (fold)
\label{sub:creation_de_l_echantillon}
Grâce à une approche par criblage, les variables de décisions non influentes ont pu être
écartées. La combinatoire du problème s’en voit ainsi réduit et permet d’éviter
l’évaluation de variations non influentes.
Cependant le temps de simulation est toujours un facteur limitant et ne permet pas
de réaliser l’optimisation dans des temps raisonnables. En effet, il est nécessaire de
compter \SIrange{1}{3}{h} pour simuler le \abr{SSC} et \SIrange{10}{30}{min} pour
le système de référence.
Ainsi des modèles de substitutions sont utilisés sur Bordeaux et sur Strasbourg
soit au total $8$ méta-modèles couvrant les indicateurs suivants~:
\begin{itemize}
  \item $Conso_{app}^{CH}$ la consommation de l’appoint pour le chauffage
  \item $Conso_{ref}^{CH}$ la consommation de référence sur le chauffage (sans solaire)
  \item $Conso_{app}^{ECS}$ la consommation de l’appoint pour la production d’\abr{ECS}
  \item $Conso_{app}$ la consommation cumulée de l’appoint pour le chauffage et l’\abr{ECS}
\end{itemize}

Ces indicateurs ne correspondent pas directement aux objectifs mais permettent de les construire.
En suivant \eqref{eq:taux_economie_opti} il est en effet possible d’obtenir $F_{sav}^{CH}$ et $F_{sav}^{ECS}$.
En suivant \eqref{eq:conso_totale}, il est possible de calculer la $Conso_{tot}$.
Ce choix est préféré car il est plus modulaire. En considérant séparément les consommations
du système de référence et du \abr{SSC}, les deux modèles restent indépendant. Il est
ainsi possible de modifier un des modèles sans impacter le second. Pour rappel, le
modèle de référence est estimée pour chaque variations de l’enveloppe afin de considérer
uniquement le gain apportée par le solaire et non par l’enveloppe.

Concernant les autres objectifs, un modèle de substitution n’est pas nécessaire.
La production des capteurs \abr{PV} est pré-calculée en considérant la répartition
décrite dans \eqref{eq:repartition_toiture} et la $Conso_{ref}^{ECS}$ est considérée
comme constante et est fixée à \SI{1739}{kWh} pour Bordeaux et \SI{1933}{kWh} pour Strasbourg.
En effet les variations de l’enveloppe n’affectent que très peu la consommation de
référence~: \SIrange{1735}{1743}{kWh} pour Bordeaux et \SIrange{1931}{1935}{kWh}
pour Strasbourg.

La méthodologie décrite dans \ref{sub:modeles_de_substitution} est appliquée pour
construire les $4$ échantillons nécessaires (référence et \abr{SSC} pour chaque climat) à
la définition des $8$ modèles de substitution. Les solutions sont générées
pseudo-aléatoirement à partir de la suite de Halton qui permet d’obtenir un échantillon
équitablement réparties afin d’obtenir une bonne représentativité de l’espace de décision.
Une fois l’échantillon simulé, la bibliothèque \fnref{http://openturns.org/}{\textit{OpenTurns}}
et les travaux de \textcite{Rania2013} sont utilisés pour construire les modèles de substitutions.
Étant en phase de conception et cherchant à explorer l’ensemble du domaine de recherche,
une loi uniforme est considérée pour chaque paramètre.
La précision des modèles de substitutions par rapport au modèle original est évaluée sur
deux critères~: l’erreur quadratique moyenne ($RMSE$, Root Mean Square Error) et l’erreur absolue maximale ($MAE$,
Maximal Absolute Error). La $RMSE$ permet d’évaluer l’écart moyen entre,
les sorties du méta-modèle ($\mathcal{M}$), et celles du modèle original ($f$)
\eqref{eq:rmse}. La $MAE$ permet d’obtenir l’écart maximal existant entre le
méta-modèle et le modèle de référence \eqref{eq:mae}.

Pour le système de référence, l’échantillon est construit en faisant varier les différents
paramètres de l’enveloppe soit $6$ facteurs. Pour le système solaire l’ensemble des
variables influentes est considéré, soit respectivement $14$ et $15$ facteurs pour
Bordeaux et Strasbourg (\tabref{tab:facteur_retenues}).
La qualité de l’approximation est discutée pour différentes tailles d’échantillons
dans la section qui suit.

\begin{align}
  \label{eq:rmse}
  RMSE &= \sqrt{\frac{1}{N}\sum^{N}_{i=1} \left[ f(\vec{x}_{i}) - \mathcal{M}(\vec{x}_{i}) \right]^{2} } \\
  \label{eq:mae}
  MAE  &= max \left( \abs{f(\vec{x}_{i}) - \mathcal{M}(\vec{x}_{i})}, \: x = 1, 2, \dotsc, N \right)
\end{align}


\subsubsection{Sélection des méta-modèles} % (fold)
\label{ssub:selection_des_meta_modeles}
\iunsure{Décrire prise en compte capteurs et vitrages}
Afin d’offrir une meilleure lisibilité, seuls les principaux résultats sont intégrés dans
le cœur du document. Cependant comme pour l’étude de sensibilité, tous
les résultats sont disponibles en annexes (\ref{cha:creation_des_meta_modeles}).

Pour l’indicateur $Conso_{ref}^{CH}$, peu de simulations ont été nécessaires afin
d’obtenir une bonne approximation (annexes, \figref{fig:mae_rmse_qualite_ref}). En effet,
à partir d’un échantillon de taille $100$, le méta-modèle est d’ordre $3$ et précis à
\SI{+- 1.5}{kWh}. À partir de $400$ simulations l’erreur quadratique moyenne et l’erreur
absolue stagne tout les deux en dessous de \SI{0.5}{kWh} (\tabref{tab:meta_result_bilan}).
Au regard des résultats, un échantillon de taille $400$ est donc suffisant pour obtenir
une bonne approximation.

Le modèle du \abr{SSC} étant plus complexe, un nombre plus important de simulation est
nécessaire. Premièrement car le nombre de combinaisons est plus important et modifie à la
fois le système et l’enveloppe. Deuxièmement, à cause des fortes interactions identifiées
entre les parts solaires respectives sur le chauffage et l’\abr{ECS}. Ainsi, il est à
minima nécessaire de considérer un échantillon de taille $600$ afin d’approcher
correctement le modèle original (\figref{fig:rmse_mae}).
Entre $600$ et $1000$, la qualité de l’approximation oscille pour les trois indicateurs
sur Bordeaux comme sur Strasbourg. Enfin au delà de $1000$, l’approximation s’améliore
lentement même si sur Strasbourg, l’erreur absolue maximale continue d’osciller. Ainsi,
pour les $3$ méta-modèlesm un échantillon de taille $600$ est pertinent. En effet,
l’erreur moyenne quadratique est pour Bordeaux inférieure à
\SI{3.5}{\percent} et l’erreur maximale absolue inférieure à \SI{8}{\percent}. Sur
Strasbourg l’erreur moyenne quadratique est inférieure à
\SI{2.7}{\percent} et l’erreur absolue maximale inférieure à \SI{6.8}{\percent}.
% subsubsection selection_des_meta_modeles (end)


\begin{figure}
    \centering
    \includegraphics{Ressources/Images/MetaModele/RMSE.pdf}
    \caption[Évolution de la $RMSE$ en fonction de l’échantillon]
            {Évolution de la $RMSE$ pour les \num{3} méta-modèles
             en fonction de la taille de l’échantillon.}
    \label{fig:rmse_mae}
\end{figure}

Comme explicité dans le chapitre précédent,  \SI{90}{\percent} est utilisé pour construire un
modèle de substitution et \SI{10}{\percent} de l’échantillon est uniquement
utilisée pour la validation et donc le calcul de la $RMSE$ et la $MAE$.
Les résultats obtenus (\figref{fig:validite_meta_ssc}) mettent en évidence
la qualité d’approximation des modèles de substitutions retenus pour évaluer la
performance du \abr{SSC}. Les points sont en effet quasiment tous compris dans la
droite d’équation $x = y$ qui indique une approximation parfaite du modèle original.

\begin{figure}
    \centering
    \includegraphics{Ressources/Images/MetaModele/validite_meta_ssc_600.pdf}
    \caption[Évaluation de la validité des méta-modèles du \abr{SSC}]
            {Évolution des résultats obtenus par le méta-modèle (axe Y) et du modèle
             original (axe X). La ligne noire représente une approximation parfaite.}
    \label{fig:validite_meta_ssc}
\end{figure}

Finalement, \tabref{tab:meta_retenus} récapitule les informations concernant les modèles
de substitutions retenus mais aussi la variabilité de chaque indicateurs observée sur les
$2000$ simulations. Sur les $8$ meta-modèles, $7$ sont d’ordre $2$ et un seul, le modèle
substituant le calcul de $Conso_{ref}^{CH}$ sur Strasbourg, est d’ordre $3$. Il est
clairement mis en évidence l’écart de consommation important entre Bordeaux et Strasbourg
pour l’appoint en particulier sur le chauffage où la consommation maximale sur Bordeaux et
similaire à la consommation minimale sur Strasbourg.

\begin{table}
\centering
\caption{Erreurs caractéristiques ($RMSE$ et $MAE$) obtenues pour les \num{10} méta-modèles
         construits)
\label{tab:meta_result_bilan}}
\begin{tabular}{l c c c c c c c c c c}
    \toprule
                    & \multicolumn{4}{c}{Bordeaux} & & \multicolumn{4}{c}{Strasbourg} &
                      Taille \\
                    \cmidrule(r){2-5}
                    \cmidrule(r){7-10}
                    & $RMSE$ & $MAE$  & Ordre & Variation  &       & $RMSE$ & $MAE$ & Ordre & Variation & échantillon \\                        \\
    \midrule
    $Conso_{ref}^{CH}$  & 0.2  & 0.6  & 2 & \numrange{228}{620}&   & 0.19   & 0.93  & 3     & \numrange{1363}{2110} & \num{400}  \\
    \addlinespace[\defaultaddspace]
    $Conso_{app}^{CH}$  & 7.7  & 22.0 & 2 & \numrange{84}{440} &   & 16.9   & 38.0  & 2     & \numrange{612}{1861}       & \num{600} \\
    \addlinespace[\defaultaddspace]
    $Conso_{app}^{ECS}$ & 15.7 & 56.4 & 2 & \numrange{156}{1043}&  & 21.3   & 70.8  & 2     & \numrange{397}{1360}       & \num{600} \\
    \addlinespace[\defaultaddspace]
    $Conso_{app}$       & 16.6 & 52.4 & 2 & \numrange{324}{1372}&  & 22.3   & 65.1  & 2     & \numrange{1222}{3117}       & \num{600} \\
    \bottomrule
\end{tabular}
\end{table}


% subsection creation_des_modeles_de_substitution (end)
% section construction_d_un_modele_de_substitution (end)




% ..............................................................................
% ..............................................................................
\section{Vers une solution adaptée} % (fold)
\label{sec:vers_une_solution_adaptee}
% ------------------------------------------------------------------------------
\subsection{Optimisation multi-objectif} % (fold)
\label{sub:optimisation_multi_objectif}
~
\itodo{Décrire l’évolution du front et le panel de solution obtenue}
\itodo{Décrire l’exploration de l’espace}
\ftodo{Représentation 2D}
\ftodo{Évolution du front en fonction des itérations}
\ftodo{Identification de sous-groupe par couleur}

Dans l’optique de ces travaux, il a été proposé d’évaluer la performance couplée
du système et du bâtiment à travers un processus d’optimisation. Dans un premier temps
un ensemble de paramètre a priori couvrant à la fois des variations
au niveau du système (logique de contrôle et systèmes), et des variations de l’enveloppe,
a été proposé. Grâce à une méthode de criblage le nombre de facteurs a été limité
aux plus impactants puis des méta-modèles ont été construits afin de se substituer
aux modèles originaux dont le temps d’évaluation est important.
Cette partie s’intéresse à l’analyse des solutions obtenues à travers l’optimisation.
Dans un premier temps le front complet est présenté. Ensuite une analyse statistique
est réalisée afin de mieux comprendre les interactions principales existantes~:
\begin{itemize}
  \item répartition qualitative des critères de décisions
  \item influence du niveau d’isolation
  \item influence du nombre et du type de capteur solaire thermique
  \item influence de la taille des ballons, tampon comme sanitaire
  \item influence des variations algorithmiques
  \item variation du nombre de capteurs \abr{PV}
\end{itemize}


% - - - - - - - - - - - - - - - - - - - - - - - - - - - - - - - - - - - - - - -
\subsubsection{Paramétrage de l’optimisation} % (fold)
\label{ssub:parametrage_de_l_optimisation}
L’algorithme de colonie d’abeilles virtuelles (\ref{sub:description_de_l_approche_globale})
est utilisé afin de réaliser l’optimisation multi-objectifs. L’optimisation est réalisée
sur $4$ objectifs et est contraint afin d’obtenir uniquement des solutions à énergie
positives (\ref{ssub:approche_retenue})~:
\begin{itemize}
  \item Maximiser le $F_{sav}^{ECS}$ ($\epsilon = 0.005$)
  \item Maximiser le $F_{sav}^{CH}$ ($\epsilon = 0.005$)
  \item Maximiser la $Prod_{PV}$ ($\epsilon = 5$)
  \item Minimiser la $Nombre_{PV}$ ($\epsilon = 0.1$)
  \item Contraint par $\abs{Conso_{tot}}   \leq  8 \ \si{kWh_{ep}\per\metre\squared}$.
\end{itemize}
Une archive par $\epsilon$-dominance est utilisée pour stocker les solutions optimales
durant le processus afin d’améliorer à la fois l’exploitation et l’exploration, évitant
ainsi de converger vers des optimums locaux. L’archive impose la définition d’un
hyperplan, où chaque hypercube le formant contient une unique solution. Les dimensions de chaque
hypercube est définie suivant les valeurs d’$\epsilon$ respectives de chaque objectif.
Ainsi, l’hyperplan est fixe et n’est pas relatif aux valeurs des objectifs.
Pour les objectifs $F_{sav}^{ECS}$ et de $F_{sav}^{CH}$, une taille de \SI{0.5}{\percent}
est retenue, indiquant que deux solutions ayant une différence de plus de
\SI{0.5}{\percent} sont forcement dans un hypercube différent. Dans le cas de l’objectif $Prod_{PV}$, la
valeur retenue n’est que peu importante car la variation de la production des capteurs
\abr{PV} est discrète et est directement liée au nombre de capteur \abr{PV} ($Nombre_{PV}$).
Dans les deux cas la valeur de l’$\epsilon$ est alors choisie inférieur au saut discret
mais est arbitraire.

Il est important de rappeler que pour appartenir au même hypercube, deux solutions
doivent obligatoirement avoir un écart sur chaque objectif inférieur à la valeur d’$\epsilon$.
Cependant l’hyperplan étant fixe, respecter ces conditions n’est pas suffisant, certaines solutions
très similaires peuvent en effet se trouver respectivement sur les bornes inférieures
et supérieures de deux hypercube adjacents. Dans ce cas les deux solutions sont retenus
car bien que très similaires, elles appartiennent pas au même hypercube et ne violent donc
pas la contrainte d’unicité au sein d’un même hypercube.
Par contre si la solution appartient au même hypercube alors, seule la solution la plus
optimale est retenue. Dans ce cas de figure l’optimalité d’une solution est défini
comme la solution minimisant la distance normalisée au point idéal de l’hypercube.


L’algorithme est définie avec les paramètres suivants~:
\begin{itemize}
  \item $IT = 400$
  \item $NP = 90$
  \item $Max_{Echec} = 2 \times \text{nombre de variables de décisions} = 32$
\end{itemize}

Dans cette configuration, l’algorithme réalise au minimum \num{36000} évaluations ($NP *
IT$) mais ce nombre est en pratique supérieur car le nombre d’évaluation total est aussi
dépendant de $Max_{Echec}$, le nombre maximal de variations consécutives que peut subir
une source avant d’être abandonnée. Son abandon entraine le tirage d’une nouvelle source
aléatoirement puis de sa position opposée. Ainsi si une source n’est pas améliorée au bout
de $32$ essais alors elle est abandonnée et l’algorithme évalue donc une solution
supplémentaire pour cette itération (position opposée). Sur Bordeaux, $386$ sources ont
été abandonnées durant le processus d’optimisation contre $383$ sur Strasbourg.

\itodo{Ajouter description rapide hypervolume et schott}
L’évolution de l’optimisation est décrit par \figref{fig:hypervolume_schott_front} à
travers deux métriques~: la \textit{métrique S} aussi appelée hypervolume, et la métrique
de \textit{Schott}. Une augmentation de l’hypervolume traduit une meilleure convergence et
couverture de l’espace des objectifs par l’ensemble des solutions. Une valeur proche de
zéro pour la métrique de \textit{Schott} indique que les solutions sont réparties
uniformément dans le domaine des objectifs. Ainsi, l’algorithme converge rapidement en
début d’optimisation puis se diversifie rapidement dans un premier temps puis lentement
pour finalement stagner avec obtention d’un front uniformément répartis.

\begin{figure}
    \centering
    \begin{subfigure}[b]{0.48\textwidth}
        \centering
        \ftodo{Ajouter évolution hypervolume et Schott pour optimisation Bordeaux}
        % \includegraphics{Ressources/Images/EtudeDeCas/fsavext_pv_bor.pdf}
        \caption{}
        \label{fig:hypervolume_schott_bor}
    \end{subfigure}
    \quad
    \begin{subfigure}[b]{0.48\textwidth}
        \centering
        \ftodo{Ajouter évolution hypervolume et Schott pour optimisation Strasbourg}
        % \includegraphics{Ressources/Images/EtudeDeCas/fsavext_pv_stras.pdf}
        \caption{}
        \label{fig:hypervolume_schottstras}
    \end{subfigure}
    \caption[Évolution de la convergence et de la diversification de l’optimisation]
             {Évolution de l’hypervolume et de la métrique de \textit{Schott}
              durant le processus d’optimisation en fonction du nombre d’itérations pour
              Bordeaux (a) et Strasbourg (b).}
    \label{fig:hypervolume_schott_front}
\end{figure}
% subsubsection parametrage_de_l_optimisation (end)


% - - - - - - - - - - - - - - - - - - - - - - - - - - - - - - - - - - - - - - -
\subsubsection{Analyse globale des solutions non-dominées} % (fold)
\label{ssub:analyse_globale_des_solutions_non_dominees}
\itodo{Remarques et observations}
Le front de Pareto obtenu comprend \num{120} pour Bordeaux (\figref{fig:front_pareto_bordeaux})
et \num{219} solutions pour Strasbourg (\figref{fig:front_pareto_strasbourg}). Cet écart s’explique
principalement par la plage de variation optimale de la surface de capteurs \abr{PV}, plus grande
sur Strasbourg. En effet la la $Conso_{ref}$ pour le climat de Strasbourg est supérieure, que ce
soit sur le chauffage ou sur la production d’\abr{ECS}.
De plus la variation de la performance de l’enveloppe entraine une variation de
la $Conso_{ref}^{CH}$ de \SI{400}{kWh} sur Bordeaux contre \SI{750}{kWh} sur Strasbourg.
Ainsi la surface de capteurs \abr{PV} maximale nécessaire est supérieure sur Strasbourg ($24$),
contre $16$ sur Bordeaux.

La borne inférieure est elle définie en fonction de deux élément. Le premier est la
consommation des usages spécifiques ($Conso_{usages}$) qui
est la valeur minimale que doit couvrir la production des capteurs \abr{PV}. Le second
est la performance du \abr{SSC} et plus particulièrement, la différence
de consommation entre l’appoint et le système de référence ($Conso_{ref} - Conso_{app}$).
Dans un cas idéal, le nombre de capteurs \abr{PV} nécessaires est donc définie uniquement
en fonction des usages spécifiques. Dans notre cas, le taux d’économie du \abr{SSC}
n’atteint pas \SI{100}{\percent} et donc afin d’atteindre un bilan positif, la
production des capteurs \abr{PV} ($Prod_{PV}$) doit aussi couvrir la consommation
de l’appoint. Ainsi, plus le \abr{SSC} est performant plus la part à couvrir par les capteurs \abr{PV}
diminue.


\begin{figure}
% pair_grid_plot
    \centering
    \includegraphics{Ressources/Images/EtudeDeCas/Bordeaux_front.pdf}
    \caption[Front de Pareto sur Bordeaux pour les $4$ objectifs après $400$ itérations]
             {Front de Pareto sur Bordeaux pour les $4$ objectifs après $400$ itérations.
              Chaque graphique est la projection des objectifs deux à deux.
              Le quatrième objectif, le nombre de capteurs \abr{PV}, est décrit par la couleur des points.}
    \label{fig:front_pareto_bordeaux}
\end{figure}

\begin{figure}
% pair_grid_plot
    \centering
    \includegraphics{Ressources/Images/EtudeDeCas/Strasbourg_front.pdf}
    \caption[Front de Pareto sur Strasbourg pour les $4$ objectifs après $400$ itérations]
             {Front de Pareto sur Strasbourg pour les $4$ objectifs après $400$ itérations.
              Chaque graphique est la projection des objectifs deux à deux.
              Le quatrième objectif, le nombre de capteurs \abr{PV}, est décrit par la couleur des points.}
    \label{fig:front_pareto_strasbourg}
\end{figure}


\itodo{Diff F sav et économie de conso}
Il est observé un taux d’économie maximal plus important sur Bordeaux autant sur le
chauffage que sur la production d’\abr{ECS}, cependant la plage de variation du
nombre de capteurs \abr{PV} est plus grande sur Strasbourg. Bien que la part relative
maximale économisée ($F_{sav,\,ext}$) soit inférieure sur Strasbourg, l’économie
énergétique absolue ($Conso_{ref} - Conso_{app}$) est plus importante. De ce fait,
le \abr{SSC} permet de substituer un nombre plus important de capteurs \abr{PV}.

\itodo{Plus de capteurs sur Strasbourg}



Dans les
deux cas, il est capable d’offrir une large diversité sur les différents objectifs en
proposant à la fois des solutions favorisant le solaire thermique, d’autres le
photovoltaïque, et des combinaisons des deux. De plus, le fait de considérer la couverture
du chauffage et de l’\abr{ECS} de manière distinctes permet de renforcer le nombre de
compromis. Le front pour le climat de Strasbourg est plus fournit que celui de Bordeaux
car les consommations à couvrir sont plus importantes, en particulier sur le chauffage.
De ce fait, le nombre de capteurs \abr{PV} nécessaire est plus important se traduisant par
un nombre plus important de combinaisons.
De plus la consommation maximale sur
Bordeaux est similaire


Les extremums respectifs couverts par le front de Pareto sont décrits dans \tabref{tab:bornes_front_pareto}.
Le \abr{SSC} permet ainsi d’offrir le
\itodo{Discuter l’étendu des résultats}

\begin{table}
\centering
\caption[Performance maximale pouvant être obtenue pour différents indicateurs]
         {Variation de la performance obtenue pour chaque indicateur en fonction du climat}
\label{tab:bornes_front_pareto}
\begin{tabular}{L{2cm} c c c c c c c c c c}
    \toprule
                & $F_{sav,\,ext}$ & $F_{sav}^{ECS}$ & $F_{sav}^{CH}$ & & $Conso_{app}$ & $Conso_{app}^{ECS}$ & $Conso_{app}^{CH}$ & & $Prod_{PV}$ & $Nbr_{PV}$ \\
    \addlinespace
    \multicolumn{11}{l}{\textbf{Bordeaux}} \\
    \midrule
    Max & 90  & 94  & 84  &   & 1379  & 921 & 470 &   & 4743  & 16  \\
    Min & 57  & 64  & 22  &   & 278 & 164 & 48  &   & 3239  & 11  \\
    \addlinespace
    \multicolumn{11}{l}{\textbf{Strasbourg}} \\
    \midrule
    Max & 74  & 89  & 62  &   & 2780  & 1191  & 1746  &   & 6021  &   \\
    Min & 42  & 58  & 8 &   & 1118  & 322 & 508 &   & 4014  & 16  \\
    \bottomrule
\end{tabular}
\end{table}


\subsubsection{Évaluation de la performance globale} % (fold)
\label{ssub:evaluation_de_la_performance_globale}
Il est de plus important de rappeler que la production des capteurs \abr{PV} est estimée
sur l’année (bilan importation / exportation, \ref{ssub:la_methodologie_de_calcul}) et ne
tient donc pas compte de l’adéquation entre les besoins et la demande
(\figref{fig:evolution_usages_pv}). Ainsi une forte production durant la période estivale
compense une faible production durant la période hivernale. À l’opposée, l’économie
engendrée par le \abr{SSC} est calculée par rapport à une solution de référence et tient
donc implicitement compte de l’adéquation entre besoins et demande (bilan charge /
production). C’est d’ailleurs pour cette raison que l’indicateur $F_{sav}$ a été préféré
au taux de couverture ($F_{sol}$) qui permet de tenir compte de l’adéquation entre besoins
et demande \textbf{uniquement} si il est calculé sur une base mensuelle.

\begin{figure}
    \centering
    \ftodo{Ajouter évolution apports PV sur l’année vs besoins spécifiques}
    % \includegraphics{Ressources/Images/EtudeDeCas/Strasbourg_front.pdf}
    \caption[None]
             {None.}
    \label{fig:evolution_usages_pv}
\end{figure}


Afin d’évaluer les solutions du front de Pareto obtenu en tenant mieux compte de
l’adéquation entre besoins et demande, un nouvel indicateur, $F_{sol.\, tot}$ est construit
en se basant sur le taux de couverture \eqref{eq:f_sol_mensuel_total}. Il permet
de réaliser un bilan mensuel du rapport entre la part solaire utile et la demande en
énergie nécessaire pour couvrir l’ensemble des besoins de la maisons, usages spécifiques,
chauffage, et production d’\abr{ECS}. Étant réaliser à une fréquence mensuelle, l’indicateur
tient compte en parti de l’adéquation avec le réseau, à la fois pour le thermique et
pour le photovoltaïque.


\begin{align}\label{eq:f_sol_mensuel_total}
        F_{sol,\, tot} &= \frac{\sum_{i=1}^{12}\left(Gain_{TH}^{i} + Gain_{PV}^{i} \right)}{Conso_{usages} + Conso_{app}} \\
        \shortintertext{avec~:}
        Gain_{TH}^{i} &= \min \left(Prod_{TH}^{i},\ Conso_{app} \right) \\
        Prod_{PV}^{i} &= \min \left(Prod_{PV}^{i},\ Conso_{usages}^{i} + \max\left(0,\ \left[Conso_{app}^{i} - Prod_{TH}^{i} \right]\right) \right) \\
\end{align}

Le bâtiment et ses équipements est ainsi évaluée en traçant l’évolution de l’indicateur $F_{sol,\, tot}$,
en fonction du nombre de capteurs \abr{PV} installés. Pour chaque variation du nombre de capteurs \abr{PV},
il est retenu uniquement la solution ayant la meilleure performance globale sur le \abr{SSC}, la solution
maximisant $F_{sav,\,ext}$.
\itodo{Discuter les résultats}

\begin{figure}
    \centering
    \ftodo{Ajouter F sol tot en fonction du nombre de capteurs PV}
    % \includegraphics{Ressources/Images/EtudeDeCas/Strasbourg_front.pdf}
    \caption[None]
             {None.}
    \label{fig:fsoltot_vs_pvnumber}
\end{figure}
% subsubsection evaluation_de_la_performance_globale (end)

\ftodo{Countplot pour répartition des critères}
\ttodo{Récapitulatif des maximums et minimums obtenues}

Les résultats montrent que le nombre de capteur \abr{PV} est inversement proportionnel
à la performance du \abr{SSC}. Lorsque elle augmente, la production photovoltaïque
diminue la production solaire thermique augmente. Ainsi sur Strasbourg, augmenter
le nombre de capteur thermiques de $5$ permet de réduire de $8$ le nombre de
capteurs \abr{PV} nécessaire (\figref{fig:fsav_pv_stras}). Sur Bordeaux une variation de $3$ capteurs
thermique permet de réduire de $5$ le nombre de capteurs \abr{PV} (\figref{fig:fsav_pv_bor}).

\itodo{Discuter figure}
Pour le nombre de capteurs \abr{PV} considérés, les solutions  il n’existe pas de solutions
sur la borne inférieure de la contrainte



\begin{figure}
% perf_th_vs_nb_pv
    \centering
    \begin{subfigure}[b]{0.48\textwidth}
        \centering
        \includegraphics{Ressources/Images/EtudeDeCas/fsavext_pv_bor.pdf}
        \caption{}
        \label{fig:fsav_pv_bor}
    \end{subfigure}
    \quad
    \begin{subfigure}[b]{0.48\textwidth}
        \centering
        \includegraphics{Ressources/Images/EtudeDeCas/fsavext_pv_stras.pdf}
        \caption{}
        \label{fig:fsav_pv_stras}
    \end{subfigure}
    \caption[À faire]
             {À faire}
    \label{fig:fsav_pv_bor_stras}
\end{figure}




Il est important de noter que les capteurs solaires thermiques
apportent une réduction
effective de la part de l’appoint basée sur un bilan mensuel alors que la

% subsubsection analyse_globale_des_solutions_non_dominees (end)


\subsubsection{Corrélation entre performance de l’enveloppe et \abr{SSC}} % (fold)
\label{ssub:correlation_entre_performance_de_l_enveloppe_et_ssc}
\itodo{Influence isolation}
Cette section s’intéresse plus particulièrement aux corrélations existantes entre la
performance de l’enveloppe et la performance et la performance du \abr{SSC}. Ainsi $3$
indicateurs sont introduits, les coefficients de corrélation de \textit{Pearson}
($pearsonr$) et \textit{Spearman} ($spearmanr$), mais aussi $p$, la probabilité qu’une
corrélation existe entre deux variables. $pearsonr$ permet d’évaluer le niveau de
corrélation linéaire existant entre deux variables et $spearmanr$ si il existe une
relation monotone. Leurs signes renseignent sur le sens de la relation, une valeur
positive indiquant que si une des variables augmente, l’autre aussi. La valeur indique le
niveau de corrélation, $-1$ ou $1$ signifiant qu’il existe une corrélation totale alors
que $0$ indique qu’il n’existe pas de corrélation du tout. Le second paramètre $p$ indique
lui la probabilité que le système ne soit pas corrélé, $0$ indiquant qu’il l’est.

\figref{fig:conso_ref_vs_app} montre qu’il existe une corrélation linéaire forte entre la
consommation du système de référence ($Conso_{ref}^{CH}$) et la consommation de l’appoint sur
le \abr{SSC} ($Conso_{app}^{CH}$). Ainsi améliorer l’isolation du bâtiment semble permettre
de réduire la $Conso_{app}^{CH}$. Maintenant si on cherche à évaluer la performance du
\abr{SSC} de manière globale, il est utile dévaluer la relation existante entre
le taux d’économie en tenant compte de la consommation des pompes ($F_{sav,\,ext}$)
et la qualité de l’enveloppe. Au regard des résultats (\figref{fig:conso_ref_vs_f_sav}),
il apparaît que la corrélation n’est pas linéaire mais l’évolution peut être définie
comme monotone ($spearmanr = \num{-0.63}$). Le système est donc en moyenne plus performant
lorsque la qualité de l’enveloppe augmente. Ainsi, les solutions optimales
considèrent majoritairement une enveloppe performante qui permet à la fois de réduire
$Conso_{app}^{CH}$ et d’augmenter $F_{sav,\,ext}$. En améliorant l’enveloppe, les besoins
sur le chauffage sont réduits et donc une part solaire plus importante peut être
utilisée pour couvrir les besoins en \abr{ECS}.

Il serait alors tentant de conclure que améliorer l’isolation au maximum est
toujours une bonne recommandation mais la \figref{fig:conso_ref_vs_diffconso} permet
de modérer cette conclusion. Dans un premier temps, il est clairement mis en évidence
qu’il n’existe pas de corrélation entre $Conso_{ref}^{CH} - Conso_{app}^{CH}$ et
$Conso_{ref}^{CH}$. Ainsi sur le front de Pareto, il existe des solutions diverses
et variées pour la même performance d’enveloppe. De plus il existe un saut (\SI{110}{kWh})
net lorsque l’écart de consommation est inférieur à $1550$. Renforcer l’isolation
au delà impacte ainsi négativement la performance du \abr{SSC} au regard de l’économie
absolue. Cependant, retenir $Conso_{ref}^{CH} - Conso_{app}^{CH}$ comme objectif
pour l’optimisation n’est pas judicieux. Formulé ainsi, le front de Pareto
comportera des solutions favorisant une enveloppe peu performante afin que le gain
absolue potentiel soit plus important. On obtiendra donc des solutions
maximisant $F_{sav}^{ECS}$ (car $Conso_{ref}^{ECS}$ est fixe) mais aucunes ne
favorisant $F_{sav}^{CH}$.


\itodo{Reformuler}
Ainsi, le choix du niveau d’isolation doit bien être déterminé parallèlement au
choix des équipements du \abr{SSC}. Il est de plus nécessaire de considérer un
autre indicateur afin d’arbitrer ce choix. Ce nouvel indicateur peut par exemple
être le coût de l’installation, ou le temps de retour sur investissement.
Cependant comme le montre la littérature, l’évaluation du coût est très délicat
car il fait intervenir de nombreux facteurs. La condition essentielle pour estimer
correctement le coût de l’installation est de travailler en collaboration avec
des industriels pour avoir une idée réaliste du coût des équipements, et avec
des installateurs pour obtenir une indication quand au coût d’installation,
et surtout de la main d’oeuvre qui est souvent exprimée de manière approximative comme un simple
ratio. De même sur l’enveloppe, une intervention d’un expert est nécessaire afin
d’évaluer le coût réel. Par exemple, il est souvent considéré pour l’isolant, un coût
dépendant uniquement de son épaisseur. Cette approximation ne permet cependant pas
de considérer le bâtiment dans son ensemble. En effet en fonction
de la structure du bâtiment des surcoûts \emph{importants} peuvent survenir lorsque on passe
d’une épaisseur à une autre de part les contraintes de support, de hauteur\dots\ De
plus si l’objectif est de calculer un temps de retour
sur investissement, alors un nombre conséquent de facteur entre en jeu, le taux
d’intérêt, l’inflation, l’évolution du coût des énergies, la valeur ajoutée au bâtiment,
la durée de vie des équipements, le coût de maintenance
\dots\ Ainsi l’évaluation économique nécessite un large panel de connaissances
afin de fournir des résultats pertinents et les travaux de la littératures mettent
en avant cette difficulté à travers des temps de retour variant de quelques années
à plus de $100$ ans.
Dans le cadre d’une \abr{MEPOS}, il est aussi possible de diriger son choix en tenant
compte des résultats d’une Analyse de Cycle de Vie (\abr{ACV}) ou bien du confort des occupants.
L’utilisation de modèle de substitution permet en effet de simuler très rapidement
un ensemble de solution et ne nécessite pas de connaissances sur le modèle, ou bien le
logiciel utilisé pour construire le modèle original. Ainsi de part sa modularité
et bien que non traité dans ces travaux, l’ajout de nouveaux objectifs est trivial.


\begin{figure}
% conso_chauffage_joinplot
    \centering
    \begin{subfigure}[b]{0.31\textwidth}
        \centering
        \includegraphics{Ressources/Images/EtudeDeCas/correlation/conso_ref_vs_app.pdf}
        \caption{}
        \label{fig:conso_ref_vs_app}
    \end{subfigure}
    \quad
    \begin{subfigure}[b]{0.31\textwidth}
        \centering
        \includegraphics{Ressources/Images/EtudeDeCas/correlation/conso_ref_vs_f_sav.pdf}
        \caption{}
        \label{fig:conso_ref_vs_f_sav}
    \end{subfigure}
    \quad
    \begin{subfigure}[b]{0.31\textwidth}
        \centering
        \includegraphics{Ressources/Images/EtudeDeCas/correlation/conso_ref_vs_diffconso.pdf}
        \caption{}
        \label{fig:conso_ref_vs_diffconso}
    \end{subfigure}
    \caption[Corrélation entre la qualité de l’enveloppe et la performance du \abr{SSC}]
             {Corrélations entre la qualité de l’enveloppe et la performance du \abr{SSC}
              en fonction des indicateurs suivants~: (a) $Conso_{app}^{CH}$, (b) $F_{sav,\, ext}$,
              (c) $Conso_{ref}^{CH} - Conso_{app}^{CH}$.}
    \label{fig:conso_ref_vs_app_f_sav}
\end{figure}
% subsubsection correlation_entre_performance_de_l_enveloppe_et_ssc (end)


\subsubsection{Influence du type de capteur} % (fold)
\label{ssub:influence_du_type_de_capteur}
Dans le processus d’aide à la décision retenue, il est dans un premier
une optimisation multi-objectifs est réalisée puis le front obtenu est ensuite
évalué de manière interactive grâce à une méthode paramétrique. Cependant,
cette approche comme toutes approches impliquant une optimisation contraint à retenir
uniquement les meilleures solutions. De ce fait, elle ne permet pas d’obtenir les
solutions qui ne sont pas optimales au sens stricte du terme mais qui ont une
performance équivalente, peut être pour des combinaisons de paramètres plus intéressantes.
Ainsi afin de pouvoir analyser l’influence du type de capteur sur la performance du système,
il est nécessaire de chercher à évaluer la performance maximale pour chaque
capteurs suivant un processus d’optimisation indépendant.

Profitant d’un temps d’évaluation très faible grâce aux modèles de substitutions,
une approche itérative exploratoire est retenue.
En effet, l’optimisation a été réalisée dans un premier
temps en considérant l’ensemble des paramètres et montre une majorité écrasante de
solutions qui considèrent un capteur sous-vide \textit{SkyPro} (\figref{fig:occurence_type_capteur}).
La question qui se pose est donc de savoir si cette solution est retenue car elle
apporte une amélioration de la performance importante ou bien si son apport est
marginal. Dans le premier cas, il est logique de favoriser ce type de capteur,
mais dans le second cas, d’autres indicateurs sont nécessaires pour définir quel
est le choix le plus pertinent.

\begin{figure}
% count_collector_type
    \centering
    \includegraphics{Ressources/Images/EtudeDeCas/occurence_type_capteur.pdf}
    \caption[Occurrences de chaque type de capteurs pour les solutions du front de Pareto]
             {Occurrences de chaque type de capteurs pour les solutions du front de Pareto
              en fonction du nombre installés.}
    \label{fig:occurence_type_capteur}
\end{figure}

Afin de pouvoir comparer la performance maximale de chaque capteurs, \num{4} optimisations
successives sont alors réalisées et les fronts sont regroupés dans un archive commune.
Cette archive ne considère pas de relation de dominance, l’ensemble des solutions sont
conservées. Les résultats (\tabref{tab:capteur_perf_variation}) sont cohérents avec les
proportions obtenues lorsque le type de capteur est un critère de décision. Le capteur
\textit{SkyPro} est en effet le plus performant devant le capteur \textit{IDMK} mais
l’écart de performance est peu impactant. Par contre pour le capteur \textit{EnergyECO},
l’écart de performance est plus important (\SI{170}{kWh} sur la $Conso_{app}$). Ces résultats
indiquent que lors du choix d’un capteur, une attention particulière doit être portée sur~:
\begin{itemize}
  \item la valeur du coefficient $a_{1}$ (en particulier pour les capteurs plans où la variation est plus importante)
  \item la surface de référence utilisée pour déterminer le rendement ($\eta_{0}$)
\end{itemize}

\begin{table}
\centering
\caption[Performance maximale pouvant être obtenue pour les différents capteurs solaires]
         {Évolution de la performance maximale obtenue sur l’indicateur
          $F_{sav,\, ext}$ pour les \num{4} capteurs considérés. Les informations entre
          parenthèses indiquent le maximum obtenus sur l’indicateur.}
\label{tab:capteur_perf_variation}
\begin{tabular}{l c c c c c c c c c}
    \toprule
              & $F_{sav,\,ext}$ & $F_{sav}^{ECS}$ & $F_{sav}^{CH}$ & &  $Conso_{app}$ & & $Conso_{ref}$ & $Conso_{ref}^{ECS}$ & $Conso_{ref}^{CH}$ \\
    \midrule
    SkyPro    & \num{74} & \num{86}(\num{89}) & \num{54} (\num{62}) & & \num{1133} & & \num{4200} & \num{2866} & \num{1334} \\
    IDMK      & \num{72} & \num{85}(\num{87}) & \num{47} (\num{60}) & & \num{1166} & & \num{4236} & \num{2866} & \num{1360} \\
    CPCStar   & \num{71} & \num{84}(\num{86}) & \num{46} (\num{57}) & & \num{1208} & & \num{4200} & \num{2866} & \num{1334} \\
    EnergyEco & \num{69} & \num{81}(\num{84}) & \num{51} (\num{55}) & & \num{1301} & & \num{4220} & \num{2866} & \num{1354} \\
    \bottomrule
\end{tabular}
\end{table}


\begin{figure}
% count_collector_type
    \centering
    \includegraphics{Ressources/Images/EtudeDeCas/evolution_capteur_perf.pdf}
    \caption[Occurences de chaque type de capteurs pour les solutions du front de Pareto]
             {Évolution de la performance du \abr{SSC} selon l’indicateur $F_{sav,\,ext}$
              en fonction du nombre de capteurs (colonnes), du type de capteurs (axe des y).
              La couleur indique le nombre de capteurs \abr{PV} nécessaires pour respecter
              la contrainte de bilan positif.}
    \label{fig:occurence_type_capteur}
\end{figure}

\ftodo{Évolution en fonction du volume du ballon}
\ftodo{Évolution en fonction du nombre de capteur}

Afin de mieux comprendre les intéractions






% subsubsection influence_du_type_de_capteur (end)




% ------------------------------------------------------------------------------
\subsection{Aide à la décision a posteriori} % (fold)
\label{sub:aide_a_la_decision_a_posteriori}
~
\itodo{Description des critères qui rentre en jeu~: coût, Suface disponible,
       fournisseurs locaux, type de clientèle, réglementation, ...}
\itodo{Ajouter un exemple avec XDAT}
\itodo{Tableau ou figure}
\ftodo{Ajouter screenshot de XDAT avec la/les solutions retenues}
\ttodo{Ajouter tableau avec caractéristiques des solutions retenues}

% subsection aide_a_la_decision_a_posteriori (end)
% subsection optimisation_multi_objectif (end)
% section vers_une_solution_adaptee (end)
