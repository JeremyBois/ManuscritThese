%!TEX root = ../main.tex
% Chapitres/Chap4-OptimisationSystemeSolaire.tex

\iunsure{Description commune de Bordeaux et Strasbourg pour éviter répétition (section 2, 3)}
\iunsure{Relancer les simulations Pareto optimales avec Dymola pour comparaison}

% Pour optimisation
% Par exemple un pas de \num{0.1} avec des bornes min de \num{0} et max de \num{0.5} donnera
% les variations possibles suivantes~: (\num{0} - \num{0.1} - \num{0.2} - \num{0.3} - \num{0.4} - \num{0.5}).



% ..............................................................................
% ..............................................................................
\section{Description de l’étude de cas} % (fold)
\label{sec:description_de_l_etude_de_cas}
% ------------------------------------------------------------------------------
\subsection{Scénarios} % (fold)
\label{sub:scenarios}
L’étude de cas est réalisée sur le bâtiment utilisée pour l’étude paramétrique.
Les scénarios pour les charges internes (équipements, éclairage, et occupants) sont
issues de la simulation de référence (\ref{sub:scenarios_de_reference}). De même
le scénario de référence est retenue pour la consigne de chauffage ($19$-$18$-$16$)
et le scénario de ventilation ($90-20$).
Pour le puisage en $ECS$, le profil $Réaliste$ est retenue avec la prise en compte des
variations hebdomadaires et mensuelles (\ref{ssub:puisage_en_eau_chaude_sanitaire}).
En effet, comme il a été montré, il est important de tenir compte de la variation
du profil de puisage au cours de l’année afin de ne pas sur-estimer la performance
du $SSC$.
% subsection scenarios (end)



% ------------------------------------------------------------------------------
\subsection{Paramètres a priori} % (fold)
\label{sub:parametres_a_priori}
~
\iunsure{Ajouter explication variables discrètes}
\ifix{Vérifier valeurs bornes sensibilité}
\ifix{Vérifier capteurs utilisés dans l’otpimisation}
\itodo{Ajouter différence Strasbourg et Bordeaux}


Comme décrit dans le chapitre précédent, une analyse de sensibilité est nécessaire
afin de réduire la cardinalité du problème, évitant ainsi de simuler des variations
non influentes relativement aux autres paramètres.
L’étude de cas est réalisée pour deux climat différents, Bordeaux et Strasbourg,
et un ensemble de \num{22} paramètres a priori est retenue (\tabref{tab:facteur_sensibilite})
en faisant varier à la fois la performance de l’enveloppe et les caractéristiques du $SSC$.
\itodo{Vérifier sources}
Au niveau des isolants, les bornes inférieures et supérieures représentent respectivement
le niveau recommandé pour la \textit{RT\,2005} et le niveau \textit{MEPOS}.
La plage de variation pour les surfaces vitrées a été fixée arbitrairement à plus ou moins \SI{20}{\percent}
de leur valeur d’origine. L’étude ne cherche pas à évaluer l’impact de l’inclinaison
des capteurs qui est imposée par la géométrie du bâtiment, elle est donc fixée à
\SI{33}{\percent} soit \SI{18.9}{\degree}.


% - - - - - - - - - - - - - - - - - - - - - - - - - - - - - - - - - - - - - - -
\subsubsection{Production d’électricité} % (fold)
\label{ssub:production_d_electricite}
L’étude de cas cherchant à obtenir un bâtiment ayant une consommation très faible, il est
nécessaire d’introduire une production locale d’électricité. Cette production permettra de
couvrir l’énergie consommée par les équipements internes (électroménager et éclairage)
ainsi que la $Conso_{app}$.

La même variation de surface est considérée pour les capteurs $PV$ et thermiques~:
\SI{\approx 7}{\metre\squared}. En effet, seul la production de $PV$ peut
compenser les consommations de charges internes et il est plus intéressant d’évaluer son
influence par rapport à une surface équivalent en solaire thermique. De plus la surface
minimale de $PV$ a été définie afin de couvrir \SI{75}{\percent} (\SI{\approx 2330}{\kWh})
de la consommation des équipements internes (\SI{\approx 3082}{\kWh}).

Afin de d’évaluer correctement la production solaire, la surface disponible en toiture
doit être cependant être correctement partagée. Comme il a été vu au chapitre 2, la
$Prod_{sol}$ dépend fortement de l’orientation des capteurs solaires thermiques. Ils sont
donc prioritaire sur le pan Sud. De plus, il est aussi nécessaire de tenir compte de la
géométrie à la fois de la toiture et des capteurs. La maison comporte une toiture quatre
pans, chaque pan est donc triangulaire, et les capteurs eux sont de forme rectangulaire.
Il apparaît donc clairement que l’approche naïve qui consiste à considérer la surface
totale de la toiture comme disponible n’est pas valable~: seule une partie est réellement
utilisable. Un algorithme de \enquote{packaging} a donc été développé afin d’évaluer le
nombre de capteurs pouvant loger sur chaque pan de toiture et le détail du calcul est
disponible en annexe (\ref{cha:repartition_des_capteurs}).

Finalement, contrairement au $SSC$ qui est évalué uniquement \emph{sur une période
s’étendant du $1^{er}$ octobre au $30$ avril}, la production $PV$ doit être évalué sur
l’année complète. Pour rappel, la période de simulation est limitée à la période de
l’année où le $SSC$ n’est pas complètement indépendant pour les combinaisons de paramètres
les plus défavorables (\ref{par:periode_de_simulation}). Ainsi sur le reste de l’année la
$Conso_{app}$ est équivalente à la $Conso_{pompes}$ qui est négligeable au regard des
résultats de l’analyse paramétrique. L’ensemble des combinaisons existantes entre $PV$ et
capteurs thermiques est donc simulée sur l’année en amont.
% subsubsection production_d_electricite (end)


% - - - - - - - - - - - - - - - - - - - - - - - - - - - - - - - - - - - - - - -
\subsubsection{Variables qualitatives} % (fold)
\label{ssub:variables_qualitatives}
La méthode de Morris nécessite que l’ensemble des variables soient quantitatives et chaque
paramètre varie indépendamment. Il n’est donc pas possible d’évaluer directement
l’influence des capteurs solaires ou des vitrages qui sont des variables qualitatives.

Afin de tenir compte de ces paramètres, leurs caractéristiques principales sont considérés
comme des paramètres indépendants. Dans le cas des capteurs solaires, le rendement optique
$\eta_{0}$ et les coefficients $a_{1}$ et $a_{2}$ sont retenues. Dans le cas des vitrages
les caractéristiques principales souvent utilisées sont le $U_{g}$ et le $g$ permettant
respectivement d’évaluer la résistance au transfert thermique et au passage de l’énergie
solaire. Les vitrages étant définies de manière détaillé, les paramètres $Émis_{ext}$
et $\tau_{sol}$ sont retenues. Bien que la méthode de Morris permettent de
regrouper un jeu de paramètre en un paramètre unique (groupe), les paramètres restent
indépendants et l’approche n’est donc pas retenue. Ainsi les combinaisons réalisées
représentent des capteurs / vitrages hypothétiques dont la faisabilité technique n’est pas
garantie. L’approche permet cependant de mettre en exergue l’importance relative de chaque
caractéristique sur la performance du capteur ou du vitrage et son impact global au niveau
du $SSC$. De plus, les autres paramètres définies a priori peuvent tenir compte de la
forte variabilité des capteurs et des vitrages. En effet, il est possible qu’il existe des
interactions entre les vitrages ou capteurs solaires et les autres facteurs quantitatifs.
% subsubsection variables_qualitatives (end)

\begin{table}
\centering
\caption{Liste des paramètres a priori utilisés pour l’analyse de sensibilité.}
\label{tab:facteur_sensibilite}
\begin{tabular}{l c c l}
  \toprule
  \addlinespace
                                               & Borne min     & Borne max   & Remarques                                                            \\
  \addlinespace
  \multicolumn{4}{l}{\bm{$SSC$}}                                                                           \\
  \midrule
  Nombre capteurs                              & \num{2}       & \num{5}     & \num{4.64} -- \SI{11.6}{\metre\squared}                              \\
  $\eta_{0}$                                   & \num{0.63}    & \num{0.84}  & \multirow{3}{*}{Diversité issue de \href{www.solar-rating.org}{ICC-SRCC}}   \\
  $a_{1}$                                      & \num{0.65}    & \num{6.7}   &                                                                      \\
  $a_{2}$                                      & \num{0.00069} & \num{0.29}  &                                                                      \\
  $Ech_{sol}^{pos}$                            & \num{0.8}     & \num{1.3}   & Position relative à la taille du ballon                              \\
  Volume ballon tampon                         & \num{100}     & \num{500}   & \multirow{2}{*}{Dimensions adaptées proportionnellement}             \\
  Volume ballon $ECS$                          & \num{100}     & \num{500}   &                                                                      \\
  $R$ ballon sanitaire                         & \num{7}       & \num{10}    & \multirow{2}{*}{Variation uniquement de l’épaisseur de l’isolant}    \\
  $R$ ballon tampon                            & \num{7}       & \num{10}    &                                                                      \\
  $Isolant_{réseau}^{épaisseur}$               & \num{0.013}   & \num{0.04}  & Résistance dépendant du nombre de capteurs                           \\
  Consigne solaire                             & \num{18}      & \num{24}    &  -                                                                   \\
  $DeltaT_{sol}$                               & \num{5}       & \num{15}    &  -                                                                   \\
  \\
  \addlinespace[\defaultaddspace]
  \multicolumn{4}{l}{\textbf{Enveloppe du bâtiment}}                                                                              \\
  \midrule
  $R$ plancher                                 & \num{6}       & \num{10}    &  -                                                                   \\
  $R$ murs                                     & \num{4}       & \num{7}     &  -                                                                   \\
  $R$ plafond                                  & \num{6}       & \num{10}    &  -                                                                   \\
  $\tau_{sol}$                                 & \num{0.643}   & \num{0.849} & \multirow{2}{*}{Variation des vitrages Nord et Ouest uniquement}     \\
  $Émis_{ext}$                                 & \num{0.037}   & \num{0.837} &                                                                      \\
  Surface vitrée Est                           & \num{4.3}     & \num{6.46}  & \multirow{4}{*}{Surface totale \SI{26.4}{\metre\squared}}            \\
  Surface vitrée Nord                          & \num{0.46}    & \num{0.684} &                                                                      \\
  Surface vitrée Sud                           & \num{5.42}    & \num{8.13}  &                                                                      \\
  Surface vitrée Ouest                         & \num{2.3}     & \num{3.89}  &                                                                      \\
  \\
  \addlinespace[\defaultaddspace]
  \multicolumn{4}{l}{\textbf{Production d’électricité}}                                                                     \\
  \midrule
  Surface $PV$                                 & \num{12}       &  \num{19}   &  Capteurs thermiques prioritaires sur le pan Sud                                                             \\
  \bottomrule
  \end{tabular}
\end{table}
% subsection parametres_a_priori (end)



% ------------------------------------------------------------------------------
\subsection{Objectifs et contraintes} % (fold)
\label{sub:objectifs_et_contraintes}
% - - - - - - - - - - - - - - - - - - - - - - - - - - - - - - - - - - - - - - -
\subsubsection{Approche naïve} % (fold)
\label{ssub:approche_naive}
Fort de l’expérience acquise à travers l’étude paramétrique, les objectifs sont formalisés
de la manière suivante~:
\begin{itemize}
  \item Augmenter le $F_{sol}^{ECS}$
  \item Augmenter le $F_{sol}^{CH}$
  \item Réduire la $Conso_{app}$
\end{itemize}
De plus une contrainte est introduite afin de vérifier que le bâtiment final
est un bâtiment passif~: $|Conso_{totale}| \leq \SI{300}{kWh}$ \eqref{eq:conso_totale}.
De cette manière, les bâtiments produisant localement respectivement trop ou pas assez d’énergie sont
écartés. De plus cette restriction permet de guider la recherche vers un le front de
Pareto.

\begin{equation} \label{eq:conso_totale}
  Conso_{totale} = Conso_{app} + Conso_{équipements} + Conso_{éclairage} - Prod_{PV}
\end{equation}

Cependant formulé de cette manière, l’approche comporte plusieurs failles. Premièrement,
il est clair que même si les \num{3} objectifs ne sont pas impactés par les mêmes
paramètres, une tendance commune est identifiable. Ainsi, sans objectifs évoluant de
manière \enquote{contradictoires} l’optimisation risque de converger vers un ensemble
de solution très réduites voir une solution unique.

Deuxièmement, l’approche introduit un biais important sur la surface de $PV$. Afin de
respecter la contrainte, deux configurations extrêmes sont possibles~:
\begin{itemize}
  \item Avoir beaucoup de $PV$ et peu de capteurs thermiques
  \item Avoir beaucoup de capteurs thermiques et suffisamment de $PV$ pour couvrir
        les consommations des équipements internes.
\end{itemize}
Cependant, une fois que la surface de $PV$ permettant d’obtenir une $Conso_{totale}$
a atteinte la limite basse imposée par la contrainte (\SI{300}{kWh}), elle ne peut plus
augmenter car la contrainte serait violée (production trop importante d’électricité). De plus,
augmenter la surface de $PV$ impacte aucun des objectifs, les solutions avec une surface
de $PV$ plus importante seront toujours rejetées.

Afin que la surface de $PV$ soit influente sur au moins un objectif il est possible
de minimiser la $Conso_{totale}$ au lieu de minimiser la $Conso_{app}$.
Cependant, il a été vu au chapitre précédent que la dominance de Pareto implique que une solution
est meilleure qu’une autre si et seulement si elle est meilleure sur un objectif et au moins
aussi bonne sur les autres (\defref{def:dominance_de_pareto}). Il est donc clair que
cette modification ne permet pas de corriger ce biais qui réduit fortement l’espace
de décision comme celui des solutions.
% subsubsection approche_naive (end)


% - - - - - - - - - - - - - - - - - - - - - - - - - - - - - - - - - - - - - - -
\subsubsection{Approche retenue} % (fold)
\label{ssub:approche_retenue}
Afin de palier aux difficultés rencontrées, une autre formulation est proposée où les objectifs
sont~:
\begin{itemize}
  \item Augmenter le $F_{sol}^{ECS}$
  \item Augmenter le $F_{sol}^{CH}$
  \item Réduire la $Prod_{PV}$
\end{itemize}
De plus la contrainte est exprimée comme~: $|Conso_{totale}| \leq 300$

Dans cette nouvelle formulation, l’ensemble des paramètres influence au minimum un des
objectifs. Il est aussi clair que les deux premiers objectifs sont en contradiction nette
avec le dernier et le biais n’existe donc plus. En effet, il est maintenant possible en
théorie d’obtenir des solutions avec une surface de $PV$ importante et une surface de
capteur thermique faible et inversement.
Le problème est maintenant correctement formulé pour explorer l’espace de décision et
répondre au problème initial~: réaliser une maison passive dont les besoins sont couverts
par l’énergie solaire.
% subsubsection approche_retenue (end)
% subsection objectifs_et_contraintes (end)
% section description_de_l_etude_de_cas (end)





% ..............................................................................
% ..............................................................................
\section{Vers un modèle simplifié} % (fold)
\label{sec:vers_un_modele_simplifie}
% ------------------------------------------------------------------------------
\subsection{Réduction de la cardinalité} % (fold)
\label{sub:reduction_de_la_cardinalite}
~
\itodo{Décrire les résultats obtenues Bordeaux et Strasbourg}
\iunsure{Faire en même temps ou faire des sections}

L’analyse de Morris a été réalisée en considérant \num{15} trajectoires uniques à travers
\num{4} niveaux. Les résultats sont analysés sur les indicateurs caractéristiques d’un $SSC$
(le $F_{sol}^{CH}$, le $F_{sol}^{ECS}$, et la $Conso_{app}$) mais aussi sur les parts
actives et passives du chauffage solaire, respectivement notées $Prod_{sol}^{active}$ et
$Prod_{sol}^{passive}$. En effet, les ballons étant dans le bâtiment, une partie de
l’énergie solaire est fournie par leurs déperditions.

\itodo{Faire en détail la description avec des chiffres pour moyenne et écart type}
\itodo{$Conso_{app}$}
Les résultats (\figref{fig:morris_analysis_indicateurs}) sur la $Conso_{app}$ montrent
que les paramètres $a_{1}$, $a_{2}$, $\eta_{0}$, $Émis_{ext}$, le volume du ballon $ECS$,
la surface vitrée à l’Est et au Sud, la résistance thermique des murs et du plafond, ainsi que le nombre de capteurs thermiques,
sont tous influents. L’ensemble des paramètres influents on un impact linéaire à l’exception
du volume du ballon $ECS$ dont l’influence est ou non-linéaire ou avec des interactions.
Il apparaît aussi que le niveau d’isolation des ballons comme des canalisations ne soit pas
très impactant.

Ainsi au regard de ces premiers résultats le performance thermique des vitrage doit être
considéré tout comme le type de capteur solaire utilisé car ces trois indicateurs caractéristiques
sont influents.

\itodo{$F_{sol}^{ECS}$}
Sur l’indicateur $F_{sol}^{ECS}$ les facteurs ayant une influence linéaire sont
peu nombreux~: la surface de capteur et ces caractéristiques ($a_{1}$, $a_{2}$, $\eta_{0}$)
ainsi que le volume du ballon sanitaire. La $Ech_{sol}^{pos}$ quand à elle a une influence
non-linéaire ou avec des interactions.


\itodo{$F_{sol}^{CH}$}
Sur l’indicateur $F_{sol}^{CH}$ les facteurs ayant une influence linéaire sont~: la
surface des capteurs thermiques, le volume des deux ballons, les \num{3} coefficients
caractéristiques des capteurs solaires ($a_{1}$, $a_{2}$, et $\eta_{0}$), la performance
des vitrages ($Émis_{ext}$), la résistance thermique des murs et du plafond, ainsi que le
$DeltaT_{sol}$. Le coefficient $a_{1}$ est important car il a un $\sigma$ et un $\mu^{*}$
moyens alors que les autres sont caractérisés par un $\mu^{*}$ important mais un $\sigma$
faible. Le volume du ballon sanitaire et le $DeltaT_{sol}$ ont quand à eux une influence
non linéaire ou avec des interactions.
Il est aussi observé que les paramètres n’influence pas de la même manière le $F_{sol}^{CH}$.
Certains paramètres comme la performance thermique des mur, du plancher et de la consigne solaire
influencent uniquement la $Prod_{sol}^{active}$ alors que
d’autres comme la résistance thermique des ballons influence uniquement la $Prod_{sol}^{passive}$.
Il peut aussi être noté que le volume du ballon tampon influence plus fortement la
$Prod_{sol}^{passive}$ alors que l’on observe l’inverse pour la performance thermique
des vitrages.


\itodo{$Conso_{totale}$}
Si on s’intéresse à l’indicateur de $Bilan_{pos}$ ($\nicefrac{Prod_{PV}}{Conso_{totale}}$)
il est observé une très forte influence linéaire de la surface de $PV$. Les autres paramètres
ayant une influence modéré~: surface des capteurs, isolation des vitrages, $a_{1}$, $\eta_{0}$
et le ballon sanitaire qui encore une fois a un impact non-linéaire ou avec des interactions.
En effet, la surface de $PV$ est le seul facteur permettant de couvrir les consommations
électriques des équipements internes qui représente la majorité des consommations. Il est donc
évident que ce paramètre est fortement influent.


Au regard de cette analyse il apparaît important de considérer les caractéristiques des capteurs et
des vitrages. Ainsi, pour l’optimisation un jeu de capteurs solaires et de vitrages représentatifs
des variations étudiées est retenu (\tabref{tab:capteurs_specs}, \tabref{tab:carac_vitrages}).
L’ensemble des paramètres conservés est récapitulé dans la \figref{fig:graphe_influence} où il
apparaît que les variations retenus sont à la fois sur l’enveloppe et le système.
\itodo{Expliquer que c’est intéressant donc de les étudier ensembles}
Il existe ainsi des compromis intéressants entre la qualité de l’enveloppe, la surface de $PV$, et
les caractéristiques du $SSC$. De plus cette méthode de criblage a permise de passer de
\num{19} paramètres à \num{12} réduisant ainsi fortement la cardinalité du problème.




\begin{figure}
    \begin{center}
        \ftodo{$\sigma$ et $\mu^{*}$ colonne Bordeaux et Strasbourg}
        % \includegraphics{Ressources/Images/Modelisation/air_modes.pdf}
    \end{center}
    \caption{Résultat de l’analyse de Morris pour différent indicateurs.
             \label{fig:morris_analysis_indicateurs}}
\end{figure}


\begin{landscape}
    \begin{figure}
      \begin{center}
          \ftodo{Ajouter le graphe d’influence complet}
          \includegraphics{Ressources/Images/Sensibilite/graphInfluence.pdf}
      \end{center}
      \caption{Résultat de l’analyse de Morris pour différent indicateurs.
               \label{fig:graphe_influence}}
  \end{figure}
\end{landscape}

\begin{table}
\centering
\caption{Liste des paramètres retenus pour l’optimisation.}
\label{tab:facteur_retenues}
\begin{tabular}{l c c c c l}
  \toprule
  \addlinespace
                       & Min        & Max         & Catégorie  & Pas        & Remarques                                \\
  \addlinespace
  \multicolumn{5}{l}{\bm{$SSC$}}         \\
  \midrule
  Nombre capteurs      & \num{2}    & \num{5}     & Discrète    & \num{1}    & \num{4.64} -- \SI{11.6}{\metre\squared}   \\
  Type de capteur      & -          &  -          & Qualitative & -          & Voir \tabref{tab:capteurs_specs}   \\
  $Ech_{sol}^{pos}$    & \num{0.8}  &  \num{1.3}  & Continue    & -          & Position relative à la taille du ballon     \\
  Volume ballon tampon & \num{100}  &  \num{500}  & Discrète    & \num{50}   & \multirow{2}{*}{Dimensions adaptées proportionnellement}   \\
  Volume ballon $ECS$  & \num{100}  &  \num{500}  & Discrète    & \num{50}   &    \\
  $DeltaT_{sol}$       & \num{5}    &  \num{15}   & Continue    & -          &        \\
  \\
  \addlinespace[\defaultaddspace]
  \multicolumn{4}{l}{\textbf{Enveloppe du bâtiment}}             \\
  \midrule
  $R$ murs             & \num{4}    &  \num{7}    & Discrète    & \num{0.5}  & -                                  \\
  $R$ plafond          & \num{6}    &  \num{10}   & Discrète    & \num{0.5}  & -                                                                      \\
  Surface vitrée Sud  & \num{5.42} &  \num{8.13} & Continue    &  -         & \multirow{2}{*}{Surface totale \SI{26.4}{\metre\squared}}       \\
  Surface vitrée Est  & \num{4.3}  &  \num{6.46} & Continue    &  -         &   \\
  Type de vitrage      & -          &  -          & Qualitative &  -         & Voir \tabref{tab:carac_vitrages} \\
  \\
  \addlinespace[\defaultaddspace]
  \multicolumn{5}{l}{\textbf{Production d’électricité}}      \\
  \midrule
  Surface PV           & \num{14}   &  \num{30}   & Discret    &  \num{1}   & Capteurs thermiques prioritaires sur le pan Sud   \\
  \bottomrule
\end{tabular}
\end{table}

\begin{table}
\itodo{Ajouter variations des capteurs DIMA voir optimisation}
\centering
\caption{Caractéristiques des panneaux solaires.
\label{tab:capteurs_specs}}
\begin{tabular}{l c c c c r}
    \toprule
                                 & IDMK\,25             & 308C\,HP             & 12\,CPC58      & ECO 25        & Unité                       \\
    \midrule
    Fabricant                    & Sonnenkraft          & Radco                & Sky Pro        & Dima          & -                           \\
    Type                         & Plan vitrée          & Plan vitrée          & Tubulaire      & Plan vitrée   & -                           \\
    Surface nette                & \num{2.32}           & \num{2.193}          & \num{2.28}     & \num{2.312}   & \si{m^{2}}                  \\
    Poids à vide                 & \num{54}             & \num{36}             & \num{53}       & \num{41}      & \si{kg}                     \\
    Contenance                   & \num{1.35}           & \num{3.5}            & \num{1.83}     & \num{1.9}     & \si{\litre}                 \\
    $\eta_{0}$                   & \num{78}             & \num{83.4}           & \num{63}       & \num{66.4}    & \si{\percent}                     \\
    $a_{1}$                      & \num{3.796}          & \num{1.4539}         & \num{0.9249}   & \num{4.9510}  & \si{W/(m^{2}\period K)}     \\
    $a_{2}$                      & \num{0,013}          & \num{0.0589}         & \num{0.00069}  & \num{0.01527} & \si{W/(m^{2}\period K^{2})} \\
    $IMDiff$                     & \num{100}            & \num{96}             & \num{102}      & \num{93}      & \si{\percent}                     \\
    \bottomrule
\end{tabular}
\end{table}

\begin{table}
\centering
\begin{tabular}{l c c c r}
  \toprule
                     & Planitherm XN       & Planitherm ONE       & OptiwhiteKGlass       & Unité                        \\
  \midrule
  Fabricant    & \href{http://fr.saint-gobain-glass.com/product/2422/sgg-planitherm-xn}{%
                       St Gobain}
               & \href{http://eg.saint-gobain-glass.com/product/1659/}{%
                       St Gobain}
               & \href{https://www.pilkington.com/en-gb/uk/products/product-categories/thermal-insulation/pilkington-k-glass-range/pilkington-k-glass}{%
                       Pilkington}                                                              & -                             \\
  Construction & \num{4}-16-4              & \num{4}-16-4            & \num{4}-16-4             & -                             \\
  Gaz          & Argon                     & Argon                   & Argon                    & -                             \\
  $U_{g}$      & \num{1}.1                 & \num{1}.0               & \num{1}.5                & \si{W/(m^{2}\period \kelvin)} \\
  $g$          & \num{82}                  & \num{49}                & \num{78}                 & \si{\percent}                 \\
  \bottomrule
    \end{tabular}
\caption{Descriptif des caractéristiques (suivant \cite{NFEN410} et \cite{NFEN673}) des différents vitrages envisagés.
         \label{tab:carac_vitrages}}
\end{table}

% subsection reduction_de_la_cardinalite (end)

% ------------------------------------------------------------------------------
\subsection{Construction d’un modèle de substitution} % (fold)
\label{sub:construction_d_un_modele_de_substitution}
Afin de réduire la durée de simulation le modèle détaillée peut être substitué à un méta-modèle. Comme décrit dans le
chapitre précédent un ensemble représentatif des combinaisons
possibles est nécessaire afin de construire un modèle valide. Il est donc important
spécialement dans notre cas de réduire le nombre de variables en amont grâce à une
méthode de criblage afin de réduire
la taille de l’échantillon nécessaire pour construire le modèle.
L’échantillon a été construit par une méthode de pseudo-Monte-Carlo avec la suite
de Halton comme générateur. Les solutions construites sont donc équitablement réparties
, assurant une bonne représentativité de l’espace de décision.

\itodo{Décrire la création de l’échantillon et résultats (erreur relative)}
\itodo{Expliquer la variable utilisé pour vitrages et capteurs}
\ftodo{Régression entre modèle et méta-modèle}

% subsection construction_d_un_modele_de_substitution (end)
% section vers_un_modele_simplifie (end)





% ..............................................................................
% ..............................................................................
\section{Vers une solution adaptée} % (fold)
\label{sec:vers_une_solution_adaptee}
% ------------------------------------------------------------------------------
\subsection{Optimisation multi-objectif} % (fold)
\label{sub:optimisation_multi_objectif}
~
\itodo{Décrire l’évolution du front et le panel de solution obtenue}
\itodo{Décrire l’exploration de l’espace}
\ftodo{Représentation 3D}
\ftodo{Représentation 2D}
\ftodo{Évolution du front en fonction des itérations}
\ftodo{Identification de sous-groupe par couleur}
% ------------------------------------------------------------------------------
\subsection{Aide à la décision a posteriori} % (fold)
\label{sub:aide_a_la_decision_a_posteriori}
~
\itodo{Description des critères qui rentre en jeu~: coût, Suface disponible,
       fournisseurs locaux, type de clientèle, réglementation, ...}
\itodo{Ajouter un exemple avec XDAT}
\itodo{Tableau ou figure}
\ftodo{Ajouter screenshot de XDAT avec la/les solutions retenues}
\ttodo{Ajouter tableau avec caractéristiques des solutions retenues}

% subsection aide_a_la_decision_a_posteriori (end)
% subsection optimisation_multi_objectif (end)
% section vers_une_solution_adaptee (end)













% \section{Formulation du problème d’optimisation} % (fold)
% \label{sec:formulation_du_probleme_d_optimisation}
% \itodo{\num{Décrire.objectifs},\num{.contraintes,} variables ...}
% % ------------------------------------------------------------------------------
% \subsection{Définition des objectifs et des contraintes} % (fold)
% \label{sub:definition_des_objectifs_et_des_contraintes}

% % - - - - - - - - - - - - - - - - - - - - - - - - - - - - - - - - - - - - - - -
% \subsubsection{Les objectifs de l’étude} % (fold)
% \label{ssub:les_objectifs_de_l_etude}
% \itodo{Les objectifs: \\
%        - Maximiser couverture solaire sur l’ECS \\
%        - Maximiser couverture solaire sur le chauffage \\
%        - Minimiser le coût de l’installation \\
%        - Minimiser le retour sur investissement}
% Dans cette étude on considère trois fonctions objectifs. On cherche dans un premier
% temps à évaluer la performance du système solaire. Pour ce faire on évalue sa
% performance sur le chauffage et sur la production d’ECS séparément. Il en a enfin
% été noté dans ~\autoref{sub:approche_monozone} que certaines variations impactent
% de manière différentes la part de chauffage et d’ECS. Enfin on a vu dans
% ~\autoref{sec:modelisation_des_systemes} que la production d’ECS reste prioritaire
% sur \num{le.chauffage}, la modification du profil de puisage aura donc un impact différent
% sur le chauffage et l’ECS.
% \itodo{Il est aussi nécessaire de mettre en valeur l’impact de l’algorithme sur
%       les rendements}

% Le dernier objectif qui sera pris en compte est l’\textbf{impact économique}. Il
% est important de ne pas seulement se focaliser sur la performance du système afin
% de filtrer les solutions certes très performantes mais non-réalisable dans les années
% proches. Il est ainsi important de rappeler que ces travaux se focalise sur une
% technologie innovante mais pouvant être implémentées aujourd’hui.
% Ainsi ce facteur bien que discutable du fait de son caractère changeant est indispensable
% dans notre étude pour guider la recherche et donner un ordre de prix pour une solution
% type.
% L’optimisation se portera ainsi sur la maximisation de la couverture solaire pour
% (i) \num{le.chauffage}, (ii) la production d’eau \num{chaude.sanitaire}, et la minimisation
% du coût d’installation et du temps de retour sur investissement.
% \itodo{Décrire les différents objectifs retenues avec plus de bla bla}
% % subsubsection les_objectifs_de_l_etude (end)

% % - - - - - - - - - - - - - - - - - - - - - - - - - - - - - - - - - - - - - - -
% \subsubsection{Le choix des variables de décision} % (fold)
% \label{ssub:le_choix_des_variables_de_decision}
% \itodo{Choix des variables de décisions: \\
%        - Propre au bâtiment \\
%        - Propre au système \\
%        - Propre à l’algorithme}

% Dans cette partie sera décrit les différentes variables qui ont été sélectionnées
% en amont du **screening**.
% \itodo{Décrire l’ensemble des variables considérées en amont de l’étude de sensibilité
%       classée \num{par.groupe},\num{.système,\num}{.contrôle,\num}{.enveloppe,} scénarios}
% \itodo{Définir chaque variables \num{avec.type}, plage \num{de.variation},\num{.unité,} description}
% \itodo{Dresser une liste des caractéristiques de chaque critère}
% \itodo{Faire apparaître clairement (graphiquement) l’impact de chaque variables
%       sur les différents objectifs peut être pas à mettre ici mais plus dans
%       ~\autoref{sub:reduction_de_la_cardinalite_par_screening}}
% % subsubsection le_choix_des_variables_de_decision (end)

% % - - - - - - - - - - - - - - - - - - - - - - - - - - - - - - - - - - - - - - -
% \subsubsection{Les contraintes de l’étude} % (fold)
% \label{ssub:les_contraintes_de_l_etude}
% \itodo{Définition des contraintes: \\
%        - Équilibre prod/conso \\
%        - Toiture limité entre photo et thermique}
% Les objectifs sont maintenant clairement définies mais l’étude comporte plusieurs
% contraintes qui doivent être prises en compte durant le processus d’optimisation.
% La première contrainte est la surface de toiture qui est limitée et doit donc être
% partagée entre les panneaux photovoltaïques et thermiques. Comme nous l’avons définie
% (~\autoref{sec:modelisation_des_systemes}) le cas d’étude étudié comporte des pans
% de toiture avec diverses orientations ce qui complète la première contrainte.
% On a donc une \num{double.contrainte}, à savoir
% partager la surface de toiture pour chaque orientation entre les différents capteurs.
% Des études ont déjà été réalisées pour évaluer le meilleur ratio entre photovoltaïque
% et thermique\mtodo{Ajouter citation} mais traite le problème sans tenir compte des combinaisons
% entre \num{les.équipements}, \num{la.structure}, \num{la.régulation}, ... Ces travaux permettront donc
% d’obtenir plus d’informations sur cet aspect encore aujourd’hui faiblement exploré.
% \itodo{Retrouver la publication qui parle de ratio thermique/photovoltaïque}

% La seconde contrainte est la place disponible dans la maison pour accueillir les
% équipements.
% \itodo{Bla bla bla}

% Enfin la contrainte principale est sur l’équilibre entre production et consommation
% d’énergie primaire. En effet l’approche MEPOS ou encore NZEB définit comme requit
% pour une maison passive d’avoir un équilibre entre production et consommation.
% On a donc ici une contrainte forte au niveau énergétique.
% \itodo{Bla bla bla}
% \itodo{Décrire les contraintes et la méthode utilisée pour les traiter}
% % subsubsection les_contraintes_de_l_etude (end)
% % subsection definition_des_objectifs_et_des_contraintes (end)
% % section formulation_du_probleme_d_optimisation (end)




% % ..............................................................................
% % ..............................................................................
% \section{Étude de sensibilité} % (fold)
% \label{sec:etude_de_sensibilite}
% \itodo{Présenter l’état avant avec les critères choisis}
% \itodo{Décrire le processus}
% \itodo{Sélection des critères en aval de l’étude}
% % section etude_de_sensibilite (end)




% % ..............................................................................
% % ..............................................................................
% \section{Optimisation} % (fold)
% \label{sec:optimisation}
% \itodo{Description résumé de la methode}
% \itodo{Discussion sur le front de Pareto obtenu: \num{Analyse.répartition},\num{.convergence,} ...}
% % section optimisation (end)

