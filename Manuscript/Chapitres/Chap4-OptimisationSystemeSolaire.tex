%!TEX root = ../main.tex
% Chapitres\Chap4-OptimisationSystemeSolaire.tex

\iunsure{Description commune de Bordeaux et Strasbourg pour éviter répétition (section 2, 3)}
\iunsure{Relancer les simulations Pareto optimales avec Dymola pour comparaison}

% ..............................................................................
% ..............................................................................
\section{Description de l’étude de cas} % (fold)
\label{sec:description_de_l_etude_de_cas}
% ------------------------------------------------------------------------------
\subsection{Paramètres a priori} % (fold)
\label{sub:parametres_a_priori}
~
\itodo{Mettre en exergue que l’étude est sur Bordeaux et Strasbourg}
\itodo{Faire la liste des paramètres retenues a priori}
\itodo{Expliquer variation du vitrage}
\itodo{Expliquer la surface PV et TH}
\itodo{Expliquer la répartition des capteurs sur la toiture}
\itodo{Expliquer pourquoi les scénarios sont fixes}
\itodo{Décrire scénario retenues et éléments fixes}
\ifix{Vérifier valeurs bornes sensibilité}
\ifix{Vérifier capteurs utilisés dans l’otpimisation}

L’étude de cas est réalisé pour deux climat différents~: Bordeaux et Strasbourg.
Suite à l’étude paramétrique, il est aussi retenue un ensemble de paramètres a
priori ().

Par exemple un pas de \num{0.1} avec des bornes min de \num{0} et max de \num{0.5} donnera
les variations possibles suivantes~: (\num{0} - \num{0.1} - \num{0.2} - \num{0.3} - \num{0.4} - \num{0.5}).

% \begin{landscape}
\begin{table}
\centering
\caption{Liste des paramètres a priori utilisés pour l’analyse de sensibilité.}
\label{tab:facteur_sensibilite}
\begin{tabular}{l c c l}
  \toprule
  \addlinespace
                                               & Borne min     & Borne max   & Remarques                                                            \\
  \addlinespace
  \multicolumn{4}{l}{\bm{$SSC$}}                                                                           \\
  \midrule
  Nombre Capteurs                              & \num{2}       & \num{5}     & \num{4.64} -- \SI{11.6}{\metre\squared}                              \\
  $\eta_{0}$                                   & \num{0.63}    & \num{0.84}  & \multirow{3}{*}{Variation provenant de \href{www.solar-rating.org}{ICC-SRCC}}   \\
  $a_{1}$                                      & \num{0.65}    & \num{6.7}   &                                                                      \\
  $a_{2}$                                      & \num{0.00069} & \num{0.29}  &                                                                      \\
  $Ech_{sol}^{pos}$                            & \num{0.8}     & \num{1.3}   & Position relative à la taille du ballon                              \\
  Volume ballon tampon                         & \num{100}     & \num{500}   & \multirow{2}{*}{Dimensions adaptées proportionnellement}             \\
  Volume ballon $ECS$                          & \num{100}     & \num{500}   &                                                                      \\
  $R$ Ballon sanitaire                         & \num{7}       & \num{10}    & \multirow{2}{*}{Variation uniquement de l’épaisseur de l’isolant}    \\
  $R$ Ballon tampon                            & \num{7}       & \num{10}    &                                                                      \\
  $Isolant_{r\acute eseau}^{\acute epaisseur}$ & \num{0.013}   & \num{0.04}  & Résistance dépendant du nombre de capteurs                           \\
  Chauffage solaire                            & \num{18}      & \num{24}    &  -                                                                   \\
  $DeltaT_{sol}$                               & \num{5}       & \num{15}    &  -                                                                   \\
  \\
  \addlinespace[\defaultaddspace]
  \multicolumn{4}{l}{\textbf{Enveloppe du bâtiment}}                                                                              \\
  \midrule
  $R$ Plancher                                 & \num{6}       & \num{10}    &  -                                                                   \\
  $R$ Murs                                     & \num{4}       & \num{7}     &  -                                                                   \\
  $R$ Plafond                                  & \num{6}       & \num{10}    &  -                                                                   \\
  $\tau_{sol}$                                 & \num{0.643}   & \num{0.849} & \multirow{2}{*}{Variation des vitrages Nord et Ouest uniquement}     \\
  $\acute Emis_{ext}$                          & \num{0.037}   & \num{0.837} &                                                                      \\
  Surface vitrées Est                          & \num{4.3}     & \num{6.46}  & \multirow{4}{*}{Surface totale \SI{26.4}{\metre\squared}}            \\
  Surface vitrées Nord                         & \num{0.46}    & \num{0.684} &                                                                      \\
  Surface vitrées Sud                          & \num{5.42}    & \num{8.13}  &                                                                      \\
  Surface vitrées Ouest                        & \num{2.3}     & \num{3.89}  &                                                                      \\
  \\
  \addlinespace[\defaultaddspace]
  \multicolumn{4}{l}{\textbf{Production d’électricité}}                                                                     \\
  \midrule
  Surf $PV$   & \num{0} &  \num{15} &  Capteurs thermiques prioritaires sur le pan Sud                                                             \\
  \bottomrule
  \end{tabular}
\end{table}


\begin{table}
\centering
\caption{Liste des paramètres retenus pour l’optimisation.}
\label{tab:facteur_retenues}
\begin{tabular}{l c c c c l}
  \toprule
  \addlinespace
                       & Min        & Max         & Catégorie  & Pas        & Remarques                                \\
  \addlinespace
  \multicolumn{5}{l}{\bm{$SSC$}}         \\
  \midrule
  Nombre Capteurs      & \num{2}    & \num{5}     & Discrete    & \num{1}    & \num{4.64} -- \SI{11.6}{\metre\squared}   \\
  Type de capteur      & -          &  -          & Qualitative & -          & Variation de la performance \href{www.solar-rating.org}{ICC-SRCC}   \\
  $Ech_{sol}^{pos}$    & \num{0.8}  &  \num{1.3}  & Continue    & -          & Position relative à la taille du ballon     \\
  Volume ballon tampon & \num{100}  &  \num{500}  & Discrete    & \num{50}   & \multirow{2}{*}{Dimensions adaptées proportionnellement}   \\
  Volume ballon $ECS$  & \num{100}  &  \num{500}  & Discrete    & \num{50}   &    \\
  $DeltaT_{sol}$       & \num{5}    &  \num{15}   & Continue    & -          &        \\
  \\
  \addlinespace[\defaultaddspace]
  \multicolumn{4}{l}{\textbf{Enveloppe du bâtiment}}             \\
  \midrule
  $R$ Murs             & \num{4}    &  \num{7}    & Discrete    & \num{0.5}  & -                                  \\
  $R$ Plafond          & \num{6}    &  \num{10}   & Discrete    & \num{0.5}  & -                                                                      \\
  Surface vitrées Sud  & \num{5.42} &  \num{8.13} & Continue    &  -         & \multirow{2}{*}{Surface totale \SI{26.4}{\metre\squared}}       \\
  Surface vitrées Est  & \num{4.3}  &  \num{6.46} & Continue    &  -         &   \\
  Type de vitrage      & -          &  -          & Qualitative &  -         & Variation couplée du $\tau_{sol}$ et du $\acute Emis_{ext}$\\
  \\
  \addlinespace[\defaultaddspace]
  \multicolumn{5}{l}{\textbf{Production d’électricité}}      \\
  \midrule
  Surface PV           & \num{14}   &  \num{30}   & Discret    &  \num{1}   & Capteurs thermiques prioritaires sur le pan Sud   \\
  \bottomrule
\end{tabular}
\end{table}


\begin{table}
\itodo{Ajouter variations des capteurs DIMA voir optimisation}
\centering
\caption{Caractéristiques des panneaux solaires.
\label{tab:capteurs_specs}}
\begin{tabular}{l c c c c r}
    \toprule
                                 & IDMK\,25             & 308C\,HP             & 12\,CPC58      & DIMA           & Unité                       \\
    \midrule
    Fabricant                    & Sonnenkraft          & Radco                & Sky Pro        & DIMA           & -                           \\
    Type                         & Plan vitrée          & Plan vitrée          & Tubulaire      & Plan vitréec   & -                           \\
    Surface nette                & \num{2.32}           & \num{2.193}          & \num{2.28}     &  & \si{m^{2}}                  \\
    Poids à vide                 & \num{54}             & \num{36}             & \num{53}       &  & \si{kg}                     \\
    Contenance                   & \num{1.35}           & \num{3.5}            & \num{1.83}     &  & \si{\litre}                 \\
    $\eta_{0}$                   & \num{78}             & \num{83.4}           & \num{63}       &  & \si{\%}                     \\
    Pente                        & \num{-5.103}         & \num{-4.777}         & \num{-0.975}   &  & -                           \\
    $a_{1}$                      & \num{3.796}          & \num{1.4539}         & \num{0.9249}   &  & \si{W/(m^{2}\period K)}     \\
    $a_{2}$                      & \num{0,013}          & \num{0.0589}         & \num{0.00069}  &  & \si{W/(m^{2}\period K^{2})} \\
    $IMDiff$                     & \num{100}            & \num{96}             & \num{102}      &  & \si{\%}                     \\
    \bottomrule
\end{tabular}
\end{table}

\begin{table}
\centering
\begin{tabular}{l c c c r}
  \toprule
                     & Planitherm XN       & Planitherm ONE       & OptiwhiteKGlass       & Unité                        \\
  \midrule
  Fabricant    & \href{http://fr.saint-gobain-glass.com/product/2422/sgg-planitherm-xn}{%
                       St Gobain}
               & \href{http://eg.saint-gobain-glass.com/product/1659/}{%
                       St Gobain}
               & \href{https://www.pilkington.com/en-gb/uk/products/product-categories/thermal-insulation/pilkington-k-glass-range/pilkington-k-glass}{%
                       Pilkington}                                                              & -                             \\
  Construction & \num{4}-16-4              & \num{4}-16-4            & \num{4}-16-4             & -                             \\
  Gaz          & Argon                     & Argon                   & Argon                    & -                             \\
  $U_{g}$      & \num{1}.1                 & \num{1}.0               & \num{1}.5                & \si{W/(m^{2}\period \kelvin)} \\
  $g$          & \num{82}                  & \num{49}                & \num{78}                 & \si{\percent}                 \\
  \bottomrule
    \end{tabular}
\caption{Descriptif des caractéristiques (suivant \cite{NFEN410} et \cite{NFEN673}) des différents vitrages envisagés.
         \label{tab:carac_vitrages}}
\end{table}
% \end{landscape}

% subsection parametres_a_priori (end)

% ------------------------------------------------------------------------------
\subsection{Objectifs et contraintes} % (fold)
\label{sub:objectifs_et_contraintes}
~
\itodo{Décrire l’approche naïve}
\itodo{Décrire l’approche retenue}
% subsection objectifs_et_contraintes (end)
% section description_de_l_etude_de_cas (end)





% ..............................................................................
% ..............................................................................
\section{Simplification du modèle} % (fold)
\label{sec:simplification_du_modele}
% ------------------------------------------------------------------------------
\subsection{Réduction de la cardinalité} % (fold)
\label{sub:reduction_de_la_cardinalite}
~
\itodo{Décrire les résultats obtenues}
\ftodo{$\sigma$ et $\mu^{*}$}
\ftodo{Distances normalisées sur les facteurs retenues}
\ftodo{Ajouter le graphe d’influence}
% subsection reduction_de_la_cardinalite (end)

% ------------------------------------------------------------------------------
\subsection{Construction d’un modèle de substitution} % (fold)
\label{sub:construction_d_un_modele_de_substitution}
~
\itodo{Décrire la création de l’échantillon et résultats (erreur relative)}
\ftodo{Régression entre modèle et meta-modèle}
% subsection construction_d_un_modele_de_substitution (end)
% section simplification_du_modele (end)





% ..............................................................................
% ..............................................................................
\section{Vers une solution adaptée} % (fold)
\label{sec:vers_une_solution_adaptee}
% ------------------------------------------------------------------------------
\subsection{Optimisation multi-objectif} % (fold)
\label{sub:optimisation_multi_objectif}
~
\itodo{Décrire l’évolution du front et le panel de solution obtenue}
\itodo{Décrire l’exploration de l’espace}
\ftodo{Représentation 3D}
\ftodo{Représentation 2D}
\ftodo{Évolution du front en fonction des itérations}
% ------------------------------------------------------------------------------
\subsection{Aide à la décision a posteriori} % (fold)
\label{sub:aide_a_la_decision_a_posteriori}
~
\itodo{Description des critères qui rentre en jeu~: coût, Suface disponible,
       fournisseurs locaux, type de clientèle, réglementation, ...}
\itodo{Ajouter un exemple avec XDAT}
\itodo{Tableau ou figure}
\ftodo{Ajouter screenshot de XDAT avec la/les solutions retenues}
\ttodo{Ajouter tableau avec caractéristiques des solutions retenues}

% subsection aide_a_la_decision_a_posteriori (end)
% subsection optimisation_multi_objectif (end)
% section vers_une_solution_adaptee (end)













% \section{Formulation du problème d’optimisation} % (fold)
% \label{sec:formulation_du_probleme_d_optimisation}
% \itodo{\num{Décrire.objectifs},\num{.contraintes,} variables ...}
% % ------------------------------------------------------------------------------
% \subsection{Définition des objectifs et des contraintes} % (fold)
% \label{sub:definition_des_objectifs_et_des_contraintes}

% % - - - - - - - - - - - - - - - - - - - - - - - - - - - - - - - - - - - - - - -
% \subsubsection{Les objectifs de l’étude} % (fold)
% \label{ssub:les_objectifs_de_l_etude}
% \itodo{Les objectifs: \\
%        - Maximiser couverture solaire sur l’ECS \\
%        - Maximiser couverture solaire sur le chauffage \\
%        - Minimiser le coût de l’installation \\
%        - Minimiser le retour sur investissement}
% Dans cette étude on considère trois fonctions objectifs. On cherche dans un premier
% temps à évaluer la performance du système solaire. Pour ce faire on évalue sa
% performance sur le chauffage et sur la production d’ECS séparément. Il en a enfin
% été noté dans ~\autoref{sub:approche_monozone} que certaines variations impactent
% de manière différentes la part de chauffage et d’ECS. Enfin on a vu dans
% ~\autoref{sec:modelisation_des_systemes} que la production d’ECS reste prioritaire
% sur \num{le.chauffage}, la modification du profil de puisage aura donc un impact différent
% sur le chauffage et l’ECS.
% \itodo{Il est aussi nécessaire de mettre en valeur l’impact de l’algorithme sur
%       les rendements}

% Le dernier objectif qui sera pris en compte est l’\textbf{impact économique}. Il
% est important de ne pas seulement se focaliser sur la performance du système afin
% de filtrer les solutions certes très performantes mais non-réalisable dans les années
% proches. Il est ainsi important de rappeler que ces travaux se focalise sur une
% technologie innovante mais pouvant être implémentées aujourd’hui.
% Ainsi ce facteur bien que discutable du fait de son caractère changeant est indispensable
% dans notre étude pour guider la recherche et donner un ordre de prix pour une solution
% type.
% L’optimisation se portera ainsi sur la maximisation de la couverture solaire pour
% (i) \num{le.chauffage}, (ii) la production d’eau \num{chaude.sanitaire}, et la minimisation
% du coût d’installation et du temps de retour sur investissement.
% \itodo{Décrire les différents objectifs retenues avec plus de bla bla}
% % subsubsection les_objectifs_de_l_etude (end)

% % - - - - - - - - - - - - - - - - - - - - - - - - - - - - - - - - - - - - - - -
% \subsubsection{Le choix des variables de décision} % (fold)
% \label{ssub:le_choix_des_variables_de_decision}
% \itodo{Choix des variables de décisions: \\
%        - Propre au bâtiment \\
%        - Propre au système \\
%        - Propre à l’algorithme}

% Dans cette partie sera décrit les différentes variables qui ont été sélectionnées
% en amont du **screening**.
% \itodo{Décrire l’ensemble des variables considérées en amont de l’étude de sensibilité
%       classée \num{par.groupe},\num{.système,\num}{.contrôle,\num}{.enveloppe,} scénarios}
% \itodo{Définir chaque variables \num{avec.type}, plage \num{de.variation},\num{.unité,} description}
% \itodo{Dresser une liste des caractéristiques de chaque critère}
% \itodo{Faire apparaître clairement (graphiquement) l’impact de chaque variables
%       sur les différents objectifs peut être pas à mettre ici mais plus dans
%       ~\autoref{sub:reduction_de_la_cardinalite_par_screening}}
% % subsubsection le_choix_des_variables_de_decision (end)

% % - - - - - - - - - - - - - - - - - - - - - - - - - - - - - - - - - - - - - - -
% \subsubsection{Les contraintes de l’étude} % (fold)
% \label{ssub:les_contraintes_de_l_etude}
% \itodo{Définition des contraintes: \\
%        - Équilibre prod/conso \\
%        - Toiture limité entre photo et thermique}
% Les objectifs sont maintenant clairement définies mais l’étude comporte plusieurs
% contraintes qui doivent être prises en compte durant le processus d’optimisation.
% La première contrainte est la surface de toiture qui est limitée et doit donc être
% partagée entre les panneaux photovoltaïques et thermiques. Comme nous l’avons définie
% (~\autoref{sec:modelisation_des_systemes}) le cas d’étude étudié comporte des pans
% de toiture avec diverses orientations ce qui complète la première contrainte.
% On a donc une \num{double.contrainte}, à savoir
% partager la surface de toiture pour chaque orientation entre les différents capteurs.
% Des études ont déjà été réalisées pour évaluer le meilleur ratio entre photovoltaïque
% et thermique\mtodo{Ajouter citation} mais traite le problème sans tenir compte des combinaisons
% entre \num{les.équipements}, \num{la.structure}, \num{la.régulation}, ... Ces travaux permettront donc
% d’obtenir plus d’informations sur cet aspect encore aujourd’hui faiblement exploré.
% \itodo{Retrouver la publication qui parle de ratio thermique/photovoltaïque}

% La seconde contrainte est la place disponible dans la maison pour accueillir les
% équipements.
% \itodo{Bla bla bla}

% Enfin la contrainte principale est sur l’équilibre entre production et consommation
% d’énergie primaire. En effet l’approche MEPOS ou encore NZEB définit comme requit
% pour une maison passive d’avoir un équilibre entre production et consommation.
% On a donc ici une contrainte forte au niveau énergétique.
% \itodo{Bla bla bla}
% \itodo{Décrire les contraintes et la méthode utilisée pour les traiter}
% % subsubsection les_contraintes_de_l_etude (end)
% % subsection definition_des_objectifs_et_des_contraintes (end)
% % section formulation_du_probleme_d_optimisation (end)




% % ..............................................................................
% % ..............................................................................
% \section{Étude de sensibilité} % (fold)
% \label{sec:etude_de_sensibilite}
% \itodo{Présenter l’état avant avec les critères choisis}
% \itodo{Décrire le processus}
% \itodo{Sélection des critères en aval de l’étude}
% % section etude_de_sensibilite (end)




% % ..............................................................................
% % ..............................................................................
% \section{Optimisation} % (fold)
% \label{sec:optimisation}
% \itodo{Description résumé de la methode}
% \itodo{Discussion sur le front de Pareto obtenu: \num{Analyse.répartition},\num{.convergence,} ...}
% % section optimisation (end)

