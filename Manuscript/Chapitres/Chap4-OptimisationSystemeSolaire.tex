%!TEX root = ../main.tex
% Chapitres/Chap4-OptimisationSystemeSolaire.tex


\itodo{Faire introduction chapitre}
\itodo{Virer les répétitions}
\itodo{Tenir compte des corrections de Laurent}


Dans ce chapitre, la méthodologie développée dans le chapitre III est appliquée
à travers l’étude combinée du \abr{SSC} et d’une maison à énergie positive afin de trouver
un ensemble de solutions optimales pour deux climats différents. Ces solutions permettent de caractériser le potentiel
du \abr{SSC} mais aussi d’étudier les interactions existantes entre enveloppe du bâtiment et \abr{SSC}.
Dans un premier temps, les hypothèses, les paramètres a priori,
les contraintes, et objectifs retenus sont explicités. Dans un second temps la combinatoire
du problème est limitée grâce à la méthode de \textit{Morris}, puis le temps nécessaire
pour l’évaluation d’une solution réduite, grâce à l’utilisation de méta-modèles.
Suit une description détaillée des résultats obtenus par l’optimisation multi-objectifs
à travers l’analyse du comportement du méta-heuristique et d’une étude détaillée
de l’ensemble des solutions non-dominées. Finalement le processus d’aide à la décision est
illustré.
\clearpage


% ..............................................................................
% ..............................................................................
\section{Description de l’étude de cas} % (fold)
\label{sec:description_de_l_etude_de_cas}
% Le \abr{SSC} et le bâtiment considérés ont été décrits à travers le chapitre II et une étude
% paramétrique a permis de mettre en avant son potentiel. Afin de trouver les solutions
% optimales, une méthodologie d’aide à la décision a été développée dans le chapitre III.
% Dans ce chapitre, la méthodologie est appliquée à travers l’étude combinée du \abr{SSC}
% et d’une maison à énergie positive afin de trouver un ensemble de solutions optimales.
% Ces solutions permettront de caractériser le potentiel du \abr{SSC} mais aussi d’étudier
% les interactions existantes entre la performance de l’enveloppe et l’efficacité du \abr{SSC}.


% Afin de limiter la
% cardinalité du problème, une étude de sensibilité est réalisée en utilisant la méthode de
% \textit{Morris}. Comme explicité précédemment elle permet d’identifier les facteurs les
% plus influents de manière qualitative. Le but étant de réduire le nombre de variables de
% décision pour l’optimisation. Une fois les variables influentes
% identifiées, un échantillon représentatif pour chaque climat est construit grâce à une
% méthode de \textit{quasi-Monte-Carlo} afin que les solutions soient uniformément réparties.
% Ces échantillons sont ensuite utilisés afin de construire un modèle de substitution pour chaque objectif ou
% contraintes. Ensuite l’optimisation par colonie d’abeilles virtuelles est réalisée
% grâce à la bibliothèque \textit{pyMayBee}.
% Une fois le front de Pareto obtenu, les résultats sont dans un premier temps analysés~:
% \begin{itemize}
%   \item Vérification de la validité des méta-modèles
%   \item Analyse de la population optimale
% \end{itemize}
% Enfin une approche interactive d’aide à la décision avec le logiciel \textit{Xdat} est
% utilisée afin d’illustrer le processus.

L’optimisation multi-objectifs est réalisée à la fois sur l’enveloppe, le \abr{SSC},
et sur l’algorithme de contrôle.
À travers cette section, les hypothèses et paramètres a priori sont explicités, puis, les
contraintes et objectifs retenus sont discutés.


% ------------------------------------------------------------------------------
\subsection{Hypothèses} % (fold)
\label{sub:hypotheses_optimization}
L’objectif de ces travaux est de déterminer les configurations permettant
de garantir l’obtention d’une \abr{MEPOS} dont les besoins sont majoritairement couverts
par l’énergie solaire. Le terme \abr{MEPOS} considère ici quatre usages, la consommation
propre à la climatisation n’est pas discutée~:
\begin{itemize}
  \item Le chauffage
  \item La production d’\abr{ECS}
  \item L’éclairage
  \item Les équipements électro-domestiques
\end{itemize}

L’étude de cas est réalisée sur le bâtiment décrit dans le chapitre II et les scénarios de
charges internes (équipements, éclairage, et occupants) sont issues de la simulation de
référence (\ref{sub:scenarios_de_reference}). De même, le scénario de référence est
retenue pour la consigne de chauffage ($19$-$18$-$16$) comme pour la ventilation
($90-20 \si{m^{3}\per h}$). Pour le puisage en $ECS$, le profil $Réaliste$ est retenue tout en tenant
compte des variations hebdomadaires et mensuelles
(\ref{ssub:puisage_en_eau_chaude_sanitaire}). En effet, comme il a été montré, il est
important de tenir compte de la variation du profil de puisage au cours de l’année afin de
ne pas sur-estimer la performance du \abr{SSC}.

L’optimisation est réalisée simultanément sur Bordeaux et Strasbourg. Bordeaux profite
d’un climat doux et d’un ensoleillement important. Strasbourg est caractérisé par un
climat rude et un ensoleillement limité durant la période de chauffage bien que importante
durant la période estival (\figref{fig:diff_ensoleillement_bor_stras}). Ce comportement
est particulièrement mis en évidence lorsque uniquement l’irradiation directe est comparée
(droite). Enfin, l’inclinaison des capteurs est définie suivant la pente de la toiture du
bâtiment. Pour Bordeaux la pente du bâtiment existant est de \SI{33}{\percent} soit
\SI{18.9}{\degree}. Pour Strasbourg, il est retenu une pente de \SI{54}{\degree} typique
d’une architecture alsacienne.

\begin{figure}
    \centering
    \includegraphics[width=\textwidth]{Ressources/Images/EtudeDeCas/irradiations_climats.pdf}
    \caption[Évolution mensuelle du potentiel solaire sur Bordeaux et Strasbourg]
            {Évolution mensuelle du potentiel solaire sur Bordeaux et Strasbourg avec l’irradiation
            globale à gauche ($I$) et la l’ensoleillement direct à droite ($I_{dir,\,nor}$)}
    \label{fig:diff_ensoleillement_bor_stras}
\end{figure}


% - - - - - - - - - - - - - - - - - - - - - - - - - - - - - - - - - - - - - - -
\subsubsection{Répartition de la toiture} % (fold)
\label{ssub:repartition_de_la_toiture}
Afin de pouvoir couvrir les consommations électriques, éclairage ($Conso_{éclairage}$) et
électro-domestique ($Conso_{électroménager}$), une production photovoltaïque ($Prod_{PV}$)
est retenue. La toiture doit donc être partagée avec d’une part les capteurs thermiques, et
d’autre part les capteurs photovoltaïques (\abr{PV}). Comme le montre la littérature et les résultats
de l’étude paramétrique l’orientation est pour le \abr{SSC} un des facteurs les plus
important. Le choix est donc fait de favoriser en priorité les capteurs solaires
thermiques sur le pan sud. Afin d’être le plus représentatif possible la géométrie
de la toiture et des capteurs est prise en compte.
En effet la maison comporte une toiture quatre pans, chaque pan est donc triangulaire~; par
contre les capteurs sont eux rectangulaires. Il apparaît donc clairement que la surface
de toiture réellement disponible ne peut pas être approximée par sa surface totale.
De plus, chaque capteur admet une surface totale supérieure à sa surface d’entrée,
les proportions variants fortement en fonction du capteur considéré.
Ainsi, afin d’estimer la surface réellement disponible un algorithme de \enquote{packaging}
a été développé (annexe \ref{cha:repartition_des_capteurs}). À partir des dimensions
du capteur (largeur et longueur de la surface totale), l’algorithme identifie le
nombre maximal pouvant loger sur chaque pan de toiture sans superpositions et sans rotations.
Ainsi la surface de capteurs photovoltaïques ($Surf_{PV}$) est définie suivant
\eqref{eq:repartition_toiture} en tenant compte de leur géométrie propre.

\begin{equation}\label{eq:repartition_toiture}
  \begin{aligned}
    &Surf_{PV}^{sud}   &=& \max\left[0,\quad \min \left(Surf_{PV}^{tot},\quad Surf_{PV}^{capteur} \times
                                                      \Bigg\lfloor\frac{Surf_{dispo}^{sud} - Surf_{TH}^{tot}}{Surf_{PV}^{capteur}}\Bigg\rfloor\right)
                                   \right] \\
    &Surf_{PV}^{ouest} &=& \max \left[0,\quad \min\left(Surf_{dispo}^{ouest},\quad
                                                      (Surf_{PV}^{tot} - Surf_{PV}^{sud}\right)
                                    \right] \\
    &Surf_{PV}^{est} &=& \max \left[0,\quad \min\left(Surf_{dispo}^{est},\quad
                                                    (Surf_{PV}^{tot} - Surf_{PV}^{sud} - Surf_{PV}^{ouest}) \right) \right] \\
  \end{aligned}
\end{equation}

Avec $Surf_{PV}^{tot}$ la surface totale de capteur \abr{PV} a installé,
$Surf_{PV}^{capteur}$ la surface extérieure de chaque capteur \abr{PV}, $Surf_{dispo}$ la
surface totale disponible sur le pan de toiture considéré, et $Surf_{TH}^{tot}$ la surface
installée de capteurs solaires thermiques. Sur le pan sud, il est admis que
la $Surf_{dispo}$ est calculée en fonction de la géométrie des capteurs solaires
thermiques. Les capteurs thermiques étant prioritaires sur ce pan de toiture, la surface
disponible pour les capteurs \abr{PV} est définie par $Surf_{dispo} - Surf_{TH}^{tot}$.
Sur les autres pans où il n’y a pas de capteurs thermiques, la $Surf_{dispo}$ est
définie en fonction de la géométrie (largeur et hauteur) des capteurs \abr{PV}. En effet,
les capteurs solaires thermiques et photovoltaïques n’ont pas les mêmes dimensions.
Dans tout les cas, si un capteur ne loge pas complètement sur le pan de toiture,
il est ajouté au pan suivant. Afin de favoriser au maximum la production solaire
les capteurs sont en priorité mis sur le pan sud, puis ouest, puis est.
% subsubsection repartition_de_la_toiture (end)


% - - - - - - - - - - - - - - - - - - - - - - - - - - - - - - - - - - - - - - -
\subsubsection{Conditions limites} % (fold)
\label{ssub:conditions_limites}
Comme pour l’étude paramétrique le coût important d’une simulation impose de simuler
sur une \textbf{période s’étendant du $\bm{1^{er}}$ octobre au $\bm{30}$ avril}. Le \abr{SSC}
couvre en effet les besoins sur le reste de l’année comme explicité dans le chapitre II.

La production d’énergie par les capteurs photovoltaïques est elle pré-calculée pour
les différentes combinaisons existantes. En effet les capteurs solaires thermiques
étant prioritaires sur le pan sud, la surface de capteurs \abr{PV} n’influence
pas la performance du \abr{SSC}. De la même manière, les charges internes propres
aux occupants sont fixes et suivent un scénario prédéfini. Ainsi la production
\abr{PV} et les charges internes sont évaluées sur \textbf{l’année complète}
a priori.

En couplant les résultats des deux simulations, il est donc possible d’évaluer la
consommation totale sur les quatre usages considérés. Le \abr{SSC} couvrant en totalité
les besoins sur la période non simulée, la consommation de l’appoint est équivalente à la
consommation des pompes. Il est donc possible d’approximer la $Conso_{app}$ annuelle par
la consommation de l’appoint et des pompes sur la période simulée. La consommation totale
est alors obtenu par \eqref{eq:conso_totale}.

\begin{align} \label{eq:conso_totale}
  Conso_{tot} &= Conso_{app} + Conso_{usages} - Prod_{PV}  \\
              &= Conso_{app} + Conso_{électroménager} + Conso_{éclairage} - Prod_{PV} \\
\end{align}
% subsubsection conditions_limites (end)


% - - - - - - - - - - - - - - - - - - - - - - - - - - - - - - - - - - - - - - -
\subsubsection{Définition des contraintes} % (fold)
\label{ssub:definition_des_contraintes}
Afin de pouvoir proposer un ensemble d’alternatives, la maison est considérée
à énergie positive si le bilan charge / production (\ref{ssub:la_methodologie_de_calcul})
est proche de $0$~: $\abs{Conso_{tot}}   \leq  8 \quad (\si{kWh_{ep}\per\metre\squared})$.
Pour le solaire thermique, la production est en adéquation avec la demande. Pour le
photovoltaïque cependant le bilan est uniquement réalisé sur une base mensuelle. Il
est en effet considéré que la maison est raccordée au réseau électrique et que le surplus
de production photovoltaïque est envoyé vers le réseau.
Ainsi les solutions produisant trop d’énergie comme celle produisant trop peu d’énergie
seront écartées.
% subsubsection definition_des_contraintes (end)
% subsection hypotheses_optimization (end)



% ------------------------------------------------------------------------------
\subsection{Définition des paramètres a priori} % (fold)
\label{sub:definition_des_parametres_a_priori}
Comme décrit dans le chapitre précédent, une analyse de sensibilité est nécessaire
afin de réduire la cardinalité du problème et mieux guider l’exploration durant l’optimisation.
Couvrant à la fois les caractéristiques de l’enveloppe et du \abr{SSC}, l’ensemble des
paramètres retenus a priori est disponible dans le \tabref{tab:parametre_a_priori}.
La description des facteurs est réalisée ci-après à travers $3$ sous-parties. Dans un premier
temps les critères propres à l’algorithme sont explicités, puis les contraintes imposées
par les variables qualitatives. Enfin les autres paramètres sont présentés.

\begin{table}
\centering
\caption[Liste des paramètres a priori utilisés pour l’analyse de sensibilité]
        {Liste des paramètres a priori utilisés pour l’analyse de sensibilité.}
\label{tab:parametre_a_priori}
\begin{tabular}{l c c l}
  \toprule
  \addlinespace
                                               & Borne min     & Borne max   & Remarques                                                            \\
  \addlinespace
  \multicolumn{4}{l}{\textbf{\abr{SSC}}}                                                                           \\
  \midrule
  Nombre capteurs \abr{TH}                     & \num{2}       & \num{5} ou \num{7}* & \SIrange{4.6}{11.6}{\metre\squared} ou \SI{16.2}{\metre\squared}                            \\
  Type de capteurs \abr{TH}                    & -             & -           & Voir \tabref{tab:capteurs_specs_optimisation}   \\                                                                   \\
  $Ech_{sol}^{pos}$                            & \num{0.8}     & \num{1.3}   & Position relative à la hauteur du ballon                              \\
  Volume ballon tampon                         & \num{100}     & \num{500}   & \multirow{2}{*}{Géométrie adaptée proportionnellement}             \\
  Volume ballon $ECS$                          & \num{100}     & \num{500}   &                                                                      \\
  $Isolant_{ballon,\, ep}$ tampon              & \num{0.055}   & \num{0.12}  &  \multirow{2}{*}{Résistance dépendante du volume du ballon}   \\
  $Isolant_{ballon,\, ep}$ \abr{ECS}           & \num{0.055}   & \num{0.12}  &                                                           \\
  $Isolant_{réseau,\, ep}$                     & \num{0.013}   & \num{0.04}  & Résistance dépendante du nombre de capteurs                           \\
  $\Delta T_{sol}$                             & \num{5}       & \num{20}    &  -                                                                  \\
  $\Delta T min_{capteur}$                     & \num{0}       & \num{30}    &  -                                                                   \\
  $\Delta T min_{tampon}$                      & \num{0}       & \num{30}    &  -                                                                   \\
  $T3_{min}$                                   & \num{15}      & \num{40}    &  -                                                                   \\
  \\
  \addlinespace[\defaultaddspace]
  \multicolumn{4}{l}{\textbf{Enveloppe du bâtiment}}                                                                              \\
  \midrule
  Type de vitrage                              & -             & -           &  Voir \tabref{tab:carac_vitrages}                                             \\
  $R$ murs                                     & \num{4}       & \num{7}     &  -                                                                   \\
  $R$ plafond                                  & \num{6}       & \num{10}    &  -                                                                   \\
  $R$ plancher                                 & \num{6}       & \num{10}    &  -                                                                   \\
  Surface vitrée est                           & \num{4.3}     & \num{6.46}  & \multirow{4}{*}{Surface totale du mur \SI{26.4}{\metre\squared}}            \\
  Surface vitrée nord                          & \num{0.46}    & \num{0.684} &                                                                      \\
  Surface vitrée sud                           & \num{5.42}    & \num{8.13}  &                                                                      \\
  Surface vitrée ouest                         & \num{2.6}     & \num{3.89}  &                                                                      \\
  \\
  \addlinespace[\defaultaddspace]
  \multicolumn{4}{l}{\textbf{Production d’électricité}}                                                                     \\
  \midrule
  Nombre de capteurs $PV$                      & \num{8}       &  \num{24}   &  \SIrange{13.1}{39.4}{\metre\squared} \\                                                             \\
  \bottomrule
  \end{tabular}
  \raggedright
  *  $5$ pour Bordeaux et $7$ pour Strasbourg.
\end{table}


% - - - - - - - - - - - - - - - - - - - - - - - - - - - - - - - - - - - - - - -
\subsubsection{Logique de contrôle} % (fold)
\label{ssub:logique_de_controle}
Au cours de l’étude paramétrique, $\Delta T_{sol}$ a été évalué comme impactant et est
donc retenue lors de l’application de la méthodologie d’aide à la décision. Dans l’optique
d’une valorisation maximale de l’énergie solaire pour le chauffage, il est aussi
investigué $3$ variations algorithmiques supplémentaires. Le fonctionnement de base de
l’algorithme admet une priorité sur la production d’\abr{ECS} car les travaux de la
littérature ont montré que l’énergie était mieux valorisée de cette manière. Ainsi,
actuellement, la température au niveau de l’échangeur solaire ($T3$) doit être supérieure
à \SI{30}{\degree} avant de pouvoir charger le ballon tampon grâce à l’énergie solaire. Le
choix de cette borne ($T3_{min}$) est arbitraire et l’impact de sa variation est donc
investigué. L’activation du chauffage solaire ($Solaire_{direct}$ ou $Solaire_{indirect}$)
est aussi contrôlé par une borne fixe \figref{fig:automate_chauffage}. Dans le cas du
$Solaire_{direct}$, la température en sortie des capteurs $T1$ doit être supérieure ou
égale au maximum entre \SI{40}{\degree} et la température de l’eau en sortie de
l’échangeur eau/air ($T7$). Dans le cas du $Solaire_{indirect}$, la température du ballon
tampon ($T5$) doit être supérieure à \SI{40}{\degree}. Afin de chercher à valoriser au
maximum l’énergie solaire pour le chauffage, il est proposé non pas une borne fixe mais un
différentiel de température minimal. Dans les deux modes de chauffage, la consigne
d’activation s’adapte ainsi au rapport entre la production et la demande. Il est alors
introduit deux nouvelles variables~: $\Delta min_{capteur}$ et $\Delta min_{tampon}$. Le
chauffage peut ainsi être en $Solaire_{direct}$ lorsque, la différence de température
entre la sortie des capteurs ($T1$) et l’air après le caisson de mélange $T_{air}^{mix}$
est plus importante que $\Delta min_{capteur}$. De même, le chauffage peut être en
$Solaire_{indirect}$, lorsque la différence de température entre le haut du ballon tampon
et $T_{air}^{mix}$ est supérieure à $\Delta min_{tampon}$. Afin d’éviter des instabilité,
il est ajouté un hystérésis de \SI{5}{\degree} en plus du différentiel de référence.
% subsubsection logique_de_controle (end)



% - - - - - - - - - - - - - - - - - - - - - - - - - - - - - - - - - - - - - - -
\subsubsection{Variables qualitatives} % (fold)
\label{ssub:variables_qualitatives}
Le choix a été fait d’introduire deux capteurs plans et deux capteurs sous-vides avec
des caractéristiques optiques et thermiques hétérogènes. La recherche a été faite
à travers deux bases de données (\fnref{http://www.sunwindenergy.com}{\textit{Sun\&Wind Energy}} et
\fnref{http://www.solar-rating.org/}{\textit{ICC-SRCC}}) et seul des capteurs récents sont considérés.
Les capteurs ont tous une \textbf{surface totale} similaire afin de pouvoir évaluer
aisément leurs performances en fonction du nombre installés. Leurs caractéristiques
sont décrites suivant la surface considérée dans leur certification \enquote{Solar Keymark},
indiquée par un \emph{*} (\tabref{tab:capteurs_specs_optimisation}). Les capteurs
retenus sont ainsi dans la mesure du possible, similaires au niveau de leur  géométrie
mais différent d’un point de vue des caractéristiques de performance. Bien que
la surface totale soit similaire, la largeur et la hauteur de chaque capteur varie, et
l’algorithme de \enquote{packaging} est utilisé pour chaque variation afin d’évaluer le
nombre maximal de capteur pouvant loger sur chaque pan de toiture. Il est observé que
malgré les différences géométriques (capteur de chez \textit{Ritter}), il est possible de
loger le même nombre de capteur sur le pan sud, même sur Bordeaux où la surface disponible
est réduite simplifiant le calcul de la répartition des capteurs \abr{PV}.
Finalement, comme pour le capteur solaire thermique \textit{IDMK\,25-AL} utilisé dans
l’étude paramétrique, une régression linéaire est réalisée afin d’identifier les
modificateurs d’\abr{IAM}~: $b_{0}$ et $b_{1}$ (\figref{fig:correlation_IAM_all}). Il peut
être noté que les capteurs sous-vides (\textit{CPC $14$ Star}, \textit{SKY PRO $12$ CPC
$58$}) sont beaucoup moins sensible à la variation de l’angle d’incidence expliquant que
leurs performances soient constantes sur la journée comme montré dans le premier chapitre
(\ref{fig:compare_static_quasi_dyn})

\begin{figure}
    \centering
    \includegraphics[width=\textwidth]{Ressources/Images/Modelisation/Capteurs/IAM_all.pdf}
    \caption[Évolution de l’\abr{IAM} en fonction de l’angle d’incidence]
    {Évolution de l’\abr{IAM} en fonction de l’angle d’incidence pour les quatre capteurs considérés.}
    \label{fig:correlation_IAM_all}
\end{figure}

\begin{table}
\centering
\caption[Caractéristiques des capteurs solaires thermiques sélectionnés]
        {Caractéristiques des capteurs solaires thermiques sélectionnés. La présence d’un astérisque (*)
         indique la surface qui a été utilisée pour déterminer les coefficients de corrélations.}
\label{tab:capteurs_specs_optimisation}
\begin{tabular}{l c c c c r}
    \toprule
                                 & IDMK\,$25$-AL              & CPC $14$ Star            & SKY PRO $12$ CPC $58$      & Energy + ECO 25         & Unité                       \\
    \midrule
    Fabricant                    & Sonnenkraft                & Ritter                   & Kloben                     & Dima                    & -                           \\
    Type                         & Plan vitrée                & Sous-vide                & Sous-vide                  & Plan vitrée             & -                           \\
    \addlinespace[\defaultaddspace]
    Surface totale               & \num{2.52}                 & \num{2.62}               & \num{2.59}*                & \num{2.53}              & \si{\metre\squared}                  \\
    Surface d’entrée             & \num{2.33}*                & \num{2.33}*              & \num{2.28}                 & \num{2.31}*              & \si{\metre\squared}                  \\
    Longueur totale              & \num{2.061}                & \num{1.622}              & \num{1.927}                & \num{2.008}             & \si{\metre}                  \\
    Largeur totale               & \num{1.225}                & \num{1.616}              & \num{1.342}                & \num{1.258}             & \si{\metre}                  \\
    Poids à vide                 & \num{49}                   & \num{41.2}               & \num{51}                   & \num{34}                & \si{kg}                     \\
    Contenance                   & \num{1.7}                  & \num{2.31}               & \num{1.76}                 & \num{1.9}               & \si{\litre}                 \\
    \addlinespace[\defaultaddspace]
    $\eta_{0}$                   & \num{76.5}                 & \num{0.644}              & \num{64.1}                 & \num{74.0}              & \si{\percent}                     \\
    $a_{1}$                      & \num{3.951}                & \num{0.749}              & \num{0.935}                & \num{5.116}             & \si{W/(m^{2}\period K)}     \\
    $a_{2}$                      & \num{0,011}                & \num{0.005}              & \num{0.004}                & \num{0.023}             & \si{W/(m^{2}\period K^{2})} \\
    $b_{0}$                      & \num{-0.1396}              & \num{-0.0709}            & \num{-0.1050}              & \num{-0.1284}           & -                     \\
    $b_{1}$                      & \num{-0.0004}              & \num{0.0006}             & \num{0.0114}               & \num{-0.0008}           & -                     \\
    $K_{\theta,\, dif}$          & \num{93.0}                 & \num{93.0}               & \num{97.2}                 & \num{93.0}              & \si{\percent}  \\
    \addlinespace[\defaultaddspace]
    Fiche technique              & \figref{fig:caracs_idmk}   & \figref{fig:caracs_star} & \figref{fig:caracs_skypro} & \figref{fig:caracs_eco} & - \\
    \bottomrule
\end{tabular}
\end{table}

Concernant le type de vitrage, $3$ variantes sont proposées dont le détail est disponible
dans le \tabref{tab:carac_vitrages}. Les caractéristiques des vitrages sont ici présentés
avec les coefficient caractéristiques mais les vitrages sont implémentés de manière
détaillés \ref{ssub:deperditions_a_travers_les_fenetres}. Le type de vitrage s’applique
sur l’ensemble des parois verticales, la fenêtre de toit, elle, conserve les mêmes caractéristiques.

\begin{table}
\centering
\caption[Descriptif des caractéristiques des différents vitrages envisagés]
        {Descriptif des caractéristiques (suivant \cite{NFEN410} et \cite{NFEN673})
         des différents vitrages envisagés.}
\label{tab:carac_vitrages}
\begin{tabular}{l c c c r}
  \toprule
                     & Planitherm XN       & Planitherm ONE       & OptiwhiteKGlass       & Unité                        \\
  \midrule
  Fabricant    & \fnref{http://fr.saint-gobain-glass.com/product/2422/sgg-planitherm-xn}{%
                       St Gobain}
               & \fnref{http://eg.saint-gobain-glass.com/product/1659/}{%
                       St Gobain}
               & \fnref{https://www.pilkington.com/en-gb/uk/products/product-categories/thermal-insulation/pilkington-k-glass-range/pilkington-k-glass}{%
                       Pilkington}                                                              & -                             \\
  Construction & \num{4}-\num{16}-\num{4}  & \num{4}-\num{16}-\num{4} & \num{4}-\num{16}-\num{4} & -                             \\
  Gaz          & Argon                     & Argon                    & Argon                    & -                             \\
  $U_{g}$      & \num{1.1}                 & \num{1.0}                & \num{1.5}                & \si{W/(m^{2}\period \kelvin)} \\
  $g$          & \num{82}                  & \num{49}                 & \num{78}                 & \si{\percent}                 \\
  \bottomrule
    \end{tabular}
\end{table}
% subsubsection variables_qualitatives (end)


% - - - - - - - - - - - - - - - - - - - - - - - - - - - - - - - - - - - - - - -
\subsubsection{Autres paramètres} % (fold)
\label{ssub:autres_parametres}
Les bornes inférieures et supérieures retenus pour les isolants représentent respectivement les valeurs typiques observés
pour des bâtiment du niveau \abr{RT\,$2012$} et des bâtiment à énergie positive notamment à
travers le projet français \abr{COMEPOS}.
La plage de variation pour les surfaces vitrées a été fixée arbitrairement à plus ou moins
\SI{20}{\percent} de leur valeur d’origine et les proportions de cadre sont ajustées
en fonction de la nouvelle surface (les caractéristiques du cadre et des ponts thermiques
restent constantes).
Le nombre minimal de $PV$ a été définie à $8$ afin de couvrir \SI{80}{\percent} (\SI{\approx
2408}{\kWh}) de la consommation des équipements internes (\SI{\approx 3082}{\kWh}) et éviter
l’évaluation de solutions qui ne pourront pas obtenir un bilan positif.
Pour les capteurs solaires thermiques, il est considéré au minimum $2$ capteurs et respectivement
$5$ et $7$ pour Bordeaux et Strasbourg. Une distinction est faite entre les deux climats
au regard des résultats obtenus durant l’étude paramétrique.
% subsubsection autres_parametres (end)
% subsection definition_des_parametres_a_priori (end)



% ------------------------------------------------------------------------------
\subsection{Définition des objectifs} % (fold)
\label{sub:definition_des_objectifs}
% - - - - - - - - - - - - - - - - - - - - - - - - - - - - - - - - - - - - - - -
\subsubsection{Approche initiale} % (fold)
\label{ssub:approche_initiale}
\noindent Dans un premier temps, fort de l’expérience acquise à travers l’étude paramétrique,
les objectifs sont formalisés de la manière suivante~:
\begin{itemize}
  \item Maximiser le $F_{sol}^{ECS}$
  \item Maximiser le $F_{sol}^{CH}$
  \item Minimiser la $Conso_{app}$
\end{itemize}

Formulé de cette manière, l’approche comporte plusieurs limites. Il est clair que même si
les \num{3} objectifs ne sont pas impactés par les mêmes paramètres, une tendance commune
existe et l’optimisation risque de converger vers un ensemble de solution très réduit voir
une solution unique.

Concernant les deux premiers objectifs, $F_{sol}^{ECS}$ et $F_{sol}^{CH}$, un autre
problème est soulevé. Comme identifié dans la littérature, ces indicateurs évaluent la
part solaire respectivement pour la production d’\abr{ECS} et le chauffage. Cependant une
augmentation de la part solaire ne signifie pas forcement une diminution de la part de
l’appoint en particulier durant la période estivale où le solaire est abondant. Même si
dans ces travaux, la de simulation est limitée à la période s’étendant du $1^{er}$ octobre
au $30$ avril, il est toujours possible que certains apports solaires soient inutiles, en
particulier les apports passifs au niveau des ballons.

Finalement, le troisième objectif introduit un biais important sur la surface de $PV$
qui limite l’exploitation de l’algorithme. En effet, l’approche retenue dans ces travaux
cherche à caractériser l’ensemble des combinaisons existantes qui permettent d’obtenir
une \abr{MEPOS} solaire. Ainsi deux configurations extrêmes sont identifiées~:
\begin{itemize}
  \item Avoir beaucoup de $PV$ et peu de capteurs thermiques
  \item Avoir beaucoup de capteurs thermiques et suffisamment de $PV$ pour couvrir
        les consommations des équipements internes.
\end{itemize}
Cependant la surface de capteurs \abr{PV} est uniquement prise en compte dans le calcul de
la contrainte. Ainsi, lorsque l’optimisation trouve des solutions minimisant au maximum la
$Conso_{tot}$, la surface de capteurs \abr{PV} ne peut plus augmenter sans que les
solutions violent la borne inférieure de la contrainte. En effet d’après
\defref{def:dominance_de_pareto}, une solution est meilleure qu’une autre si et seulement
si elle est meilleure sur un objectif et au moins aussi bonne sur tous les autres. Il est
donc clair que comme la surface de \abr{PV} n’impacte pas les objectifs, des solutions
avec une surface de capteurs \abr{PV} importantes ne sont pas atteignables. Il pourrait
être considéré la minimisation de la $Conso_{tot}$ en place de la $Conso_{app}$ afin qu’un
objectif tiennent compte des capteurs \abr{PV}. Cependant toujours d’après
\defref{def:dominance_de_pareto}, le problème reste entier. La sous-section qui suit
décrit ainsi une approche permettant de répondre à nos problématiques.
% subsubsection approche_initiale (end)


% - - - - - - - - - - - - - - - - - - - - - - - - - - - - - - - - - - - - - - -
\subsubsection{Approche retenue} % (fold)
\label{ssub:approche_retenue}
\noindent Afin de palier aux difficultés rencontrées, une autre formulation est proposée~:
\begin{itemize}
  \item Maximiser le $F_{sav}^{ECS}$
  \item Maximiser le $F_{sav}^{CH}$
  \item Maximiser la $Prod_{PV}$
  \item Minimiser la $Nombre_{PV}$
\end{itemize}

Dans cette nouvelle formulation, l’ensemble des paramètres influence au minimum un des
objectifs. Il est aussi clair que les différents objectifs sont antinomique.
En effet, réduire la surface de capteur thermique permet d’améliorer la production des
$PV$ (plus de capteurs au sud) mais impacte négativement les autres objectifs.
Inversement, une surface de capteur thermique importante implique une production des $PV$
plus faible. Ainsi l’ajout d’une maximisation de la $Prod_{PV}$, permet d’obtenir des
solutions variées~: avec une surface de $PV$ importante et une surface de capteur
thermique faible et inversement.

De plus en remplaçant les indicateurs $F_{sol}^{CH}$ et $F_{sol}^{ECS}$ par respectivement
$F_{sav}^{CH}$ et $F_{sav}^{ECS}$, l’algorithme favorise non plus les solutions augmentant
la part solaire, mais les solutions réduisant la part de l’appoint au profit d’une part
plus importante de solaire \eqref{eq:taux_economie_opti}. Pour rappel, la $Conso_{app}$
inclut la consommation des auxiliaires tels que les pompes. Cependant ces nouveaux
objectifs nécessitent un modèle de référence afin d’évaluer la réduction de la
consommation induite par l’ajout d’un \abr{SSC}. Comme il est évalué conjointement des
variations sur l’enveloppe et sur le \abr{SSC}, il est nécessaire d’obtenir une
consommation de référence pour chaque niveau d’enveloppe afin d’évaluer uniquement
l’amélioration induite par le \abr{SSC} et non l’amélioration apportée par l’amélioration
de l’enveloppe. Un modèle de référence sans apports solaires (tous électrique) est donc
modélisé. Il admet un unique ballon de \SI{200}{\litre} pour le stockage d’\abr{ECS} et
une batterie électrique en terminal pour le chauffage.

\begin{align}\label{eq:taux_economie_opti}
  F_{sav}^{CH}   &= 1 - \frac{Conso_{app}^{CH}}{Conso_{ref}^{CH}} \\
  F_{sav}^{ECS}  &= 1 - \frac{Conso_{app}^{ECS}}{Conso_{ref}^{ECS}} \\
  F_{sav}        &= 1 - \frac{Conso_{app}}{Conso_{ref}}
\end{align}

Finalement le dernier objectif, minimiser la $Nombre_{PV}$, est introduit afin de proposer
des solutions non-dominées sur toute l’intervalle définie par la contrainte. En effet,
sans cet objectif les solutions retenues sont toutes proches de la borne inférieure car
c’est l’espace de solution maximisant la production des capteurs \abr{PV}. Cependant ces
travaux considèrent que toutes les solutions comprises dans l’intervalle sont admissibles.
Il existe en effet, une forte incertitude sur la performance réelle à la fois du bâtiment
et des équipements. Ces incertitudes introduites par le choix des hypothèses sont
inhérente à toutes simulations numériques. De plus le parti pris de cette approche est de
proposer une large variété de solutions dans le respect de la contrainte de \enquote{bilan
positif}~: proposer des solutions sur l’ensemble de l’intervalle répond à cet objectif.

Le problème est donc maintenant correctement formulé afin de permettre d’explorer l’espace
de décision et répondre au problème initial~: explorer l’ensemble de solutions non-dominées
permettant d’obtenir une \abr{MEPOS} solaire.
% subsubsection approche_retenue (end)
% subsection definition_des_objectifs (end)
% section description_de_l_etude_de_cas (end)



% ..............................................................................
% ..............................................................................
\section{Méthode de criblage de Morris} % (fold)
\label{sec:methode_criblage_de_morris}
Afin de réduire la cardinalité du problème, la méthode de \enquote{screening} de
\textit{Morris} est retenue. L’analyse a été réalisée sur Bordeaux et Strasbourg en
considérant \num{20} trajectoires uniques et \num{4} niveaux. Dans un premier temps les
résultats sont discutés pour les principaux indicateurs du \abr{SSC}. Puis un graphe
d’influence est réalisé sur les indicateurs $F_{sol}^{CH}$, $F_{sol}^{ECS}$,
$Conso_{app}$, et $Conso_{tot}$ permettant de représenter les facteurs influents
considérés dans le reste de l’étude.

\subsection{Hypothèses} % (fold)
\label{sub:hypotheses_morris}
La méthode de \textit{Morris} nécessite que l’ensemble des variables soient continues et
puissent varier indépendamment les unes des autres. Ainsi pour des variables qualitatives
comme le type de capteurs solaires thermiques et le type de vitrage, les caractéristiques
principales ont été considérées comme indépendantes uniquement lors de l’étude de
sensibilité. Ce choix est fait uniquement afin d’évaluer de possibles interactions existantes avec d’autres facteurs.
Bien entendu, dans le reste de l’étude, les capteurs et vitrages sont considérés comme
qualitatif et ne pourront prendre que les caractéristiques complètes de chaque type.

Les capteurs sont ainsi exprimées en fonction des coefficient $a_{1}$,
$a_{2}$, et du rendement optique ($\eta_{0}$). Le capteur \textit{IDMK\,25-AL} est
utilisé comme référence et les variations de chaque coefficient remplace la valeur originale.
Les vitrages sont eux exprimés en fonction de l’émissivité
du verre intérieur ($Émis_{int}$) et du coefficient de transmission solaire ($\tau_{sol}$).
Ainsi il est considéré uniquement le vitrage \textit{Planitherm XN} pour lequel les
caractéristiques varient. Les bornes inférieures et supérieures retenues sont issues des caractéristiques
des différents capteurs et vitrages considérés (\tabref{tab:variabilite_capteur_vitrage}).
Bien que la méthode de \textit{Morris} permette de regrouper les paramètres afin d’évaluer
l’influence de l’ensemble et non de chaque élément le composant, les éléments groupés
nécessitent de pouvoir varier indépendamment. Cet option n’est donc pas applicable à notre cas.


Moins contraignant, l’échantillon créé par la méthode de \textit{Morris} suppose que
chaque variable est continue. Dans notre cas le nombre de capteurs solaires thermiques
et de capteurs \abr{PV} sont cependant obligatoirement des entiers. Afin de contourner
ce problème, la plage de variation de ces deux facteurs a été choisie comme~:
\begin{itemize}
  \item \SIrange{2}{5}{} capteurs solaires thermiques soit une variation de \SI{7}{\metre\squared}
  \item \SIrange{8}{14}{} capteurs \abr{PV} soit une variation de \SI{9}{\metre\squared}
\end{itemize}
Ce choix est fait afin d’obtenir la surface équivalente la plus proche possible, tout
en assurant que les valeurs prises lors de la création de l’échantillon sont des entiers
(vrai seulement lorsque l’échantillonnage considère quatre niveaux).


\begin{table}
\centering
\caption[Variabilité des caractéristiques des capteurs et des vitrages pour l’étude
         de sensibilité]
        {Variabilité des caractéristiques des capteurs et des vitrages pour l’étude
         de sensibilité.}
\label{tab:variabilite_capteur_vitrage}
\begin{minipage}[t][][b]{0.45\linewidth}
\begin{tabular}{l c c}
  \toprule
  \addlinespace
                                               & Borne min     & Borne max    \\
  \addlinespace[\defaultaddspace]
  \multicolumn{2}{l}{\textbf{Vitrages}}                                       \\
  \midrule
  $Émis_{int}$                                 & \num{0.047}   & \num{0.837}  \\
  $\tau_{sol}$                                 & \num{0.643}   & \num{0.849}  \\
  \bottomrule
  \end{tabular}
\end{minipage}%
\begin{minipage}[t][][b]{0.45\linewidth}
\begin{tabular}{l c c}
  \toprule
  \addlinespace
                                               & Borne min     & Borne max    \\
  \addlinespace[\defaultaddspace]
  \multicolumn{3}{l}{\textbf{Capteurs solaires}}                              \\
  \midrule
  $a_{1}$                                      & \num{0.749}   &  \num{5.116} \\
  $a_{2}$                                      & \num{0.004}   &  \num{0.023} \\
  $\eta_{0}$                                   & \num{0.644}   &  \num{0.764} \\
  \bottomrule
  \end{tabular}
\end{minipage}
\end{table}
% subsection hypotheses_morris (end)



% - - - - - - - - - - - - - - - - - - - - - - - - - - - - - - - - - - - - - - -
\subsection{Analyse des résultats} % (fold)
\label{sub:analyse_des_resultats_morris}
Cette section discute les résultats obtenus par la méthode de criblage pour
les principaux indicateurs caractérisant un \abr{SSC}~:
\begin{itemize}
  \item La couverture solaire sur l’eau chaude sanitaire ($F_{sol}^{ECS}$)
  \item La couverture solaire sur le chauffage ($F_{sol}^{CH}$)
  \item La consommation de l’appoint ($Conso_{app}$)
  \item La consommation de l’appoint et des autres usages domestiques ($Conso_{tot}$)
  \item La production solaire ($Prod_{sol}$)
\end{itemize}

Comme l’étude considère deux climats, le nombre de courbes nécessaires pour appuyer
les remarques est trop important pour être ajouté dans le corps du document sans gêner
la lecture. Le choix a donc été de fait de présenter uniquement les courbes les plus
importantes et de fournir le reste en annexe (\ref{cha:analyse_de_sensibilite}).

\begin{figure}
    \centering
    \includegraphics[width=\textwidth]{Ressources/Images/Sensibilite/sigma_mu_star_1.pdf}
    \caption[Résultat de l’analyse de \textit{Morris} pour les indicateurs principaux]
            {Résultat de l’analyse de \textit{Morris} pour les indicateurs principaux
             ($f(\mu^{*}) = \sigma$).}
    \label{fig:objectifs_mu_star}
\end{figure}


% - - - - - - - - - - - - - - - - - - - - - - - - - - - - - - - - - - - - - - -
\subsubsection{Couverture solaire sur l’eau chaude sanitaire} % (fold)
\label{ssub:couverture_solaire_sur_l_ECS}
Les mêmes facteurs influents sont identifiés pour l’indicateur $F_{sol}^{ECS}$ que ce soit
pour Bordeaux ou Strasbourg, cependant les interactions et l’ordre d’importance diffèrent
(\figref{fig:objectifs_mu_star}). Dans les deux cas le nombre de capteurs \abr{TH} et le
volume du ballon \abr{ECS} sont les facteurs les plus influents. Sur Bordeaux, les
facteurs $\Delta T_{sol}$, $\Delta min_{capteur}$, et $\Delta min_{tampon}$ ont des effets
non-linéaires ou avec des interactions alors que les autres ont des impacts linéaires. Sur
Strasbourg, chaque facteur a un impact linéaire à l’exception de $\Delta T_{sol}$ dont
l’influence peut être estimée comme non-linéaire ou avec des interactions. Bien que
n’ayant pas d’action directe sur la production d’\abr{ECS}, la performance thermique des
vitrages est évaluée comme faiblement influente. Le \abr{SSC} répartissant l’énergie
solaire disponible entre chauffage et production d’\abr{ECS}, réduire les besoins de
chauffage permet de valoriser une plus grande part de l’énergie solaire disponible pour la
production d’\abr{ECS}.
% subsubsection couverture_solaire_sur_l_ECS (end)


% - - - - - - - - - - - - - - - - - - - - - - - - - - - - - - - - - - - - - - -
\subsubsection{Couverture solaire sur le chauffage} % (fold)
\label{ssub:couverture_solaire_sur_le_chauffage}
Pour le $F_{sol}^{CH}$, la même tendance est observée~: les facteurs influents ont plus
d’effets non-linéaires ou avec des interactions pour le climat de Bordeaux que sur
Strasbourg. En effet sur Bordeaux, le volume du ballon \abr{ECS}, le $\Delta
min_{capteur}$, et la $Ech_{sol}^{pos}$, ont tous des effets non linéaires ou avec
interactions alors que sur Strasbourg, seul le $\Delta T_{sol}$ est influent. Pour les
deux climats, le volume du ballon tampon et le nombre de capteurs solaires thermiques sont
les plus influents. Il apparaît que les caractéristiques de l’enveloppe, exception faite
des vitrages, ne soit pas parmi les plus influentes même sur le climat strasbourgeois où
les besoins en chauffage sont importants. Il est même complètement exclut des facteurs
influent la résistance thermique du plancher. Sur Bordeaux et sur Strasbourg les
paramètres liés au réglage de l’algorithme de contrôle ($\Delta T_{sol}$, $\Delta
min_{capteur}$, $\Delta min_{tampon}$) sont aussi identifiés comme fortement influents. Il
est aussi noté pour Bordeaux un faible impact de la $Ech_{sol}^{pos}$ et de la surface
vitrée à l’est. L’épaisseur de l’isolant du ballon tampon semble aussi être influente sur
Bordeaux mais pas sur Strasbourg.

Afin de mieux comprendre pourquoi ces paramètres sont influents, l’analyse de
\textit{Morris} est réalisée en distinguant la part active ($Prod_{sol}^{CH}$ active) de
la part passive (pertes des ballons, $Prod_{sol}^{CH}$ passive) (annexes,
\figref{fig:prod_sol_chauffage_mu}). Pour Bordeaux et Strasbourg, il est ainsi mis en
évidence que l’amélioration de l’isolation des ballons à un effet négatif sur la part
passive mais aucuns effets sur la part active. À l’inverse lorsque le paramètre
$Émis_{int}$ augmente ($U_{g}$ diminue) la part solaire active est plus importante mais la
part passive ne varie pas. Il semble donc qu’il existe un compromis entre la qualité de
l’enveloppe et la part des consommations couverte par le solaire. De plus pour Bordeaux,
si seule la part active est considérée il est noté que les facteurs influents sont
principalement, le nombre de capteur, la performance des vitrages, et le $\Delta
min_{tampon}$. En effet, augmenter le volume du ballon \abr{ECS} améliore la part solaire
passive mais diminue la part active, expliquant pourquoi ce facteur est fortement
non-linéaire ou avec des interactions mais ne fait pas parti des principaux facteurs. Par
contre l’augmentation du volume du ballon tampon améliore à la fois la part active et la
part passive même si le facteur est plus influent sur cette dernière.

Il apparaît que l’analyse des facteurs liés à l’algorithme de contrôle soit plus complexe.
Augmenter le $\Delta min_{tampon}$ semble influencer négativement la part active sur
Bordeaux et Strasbourg, mais permet d’améliorer la part passive uniquement sur
Strasbourg. Les conditions d’ensoleillement étant meilleures et les besoins moins
important sur Bordeaux, le ballon tampon est en moyenne à une température plus élevée et
faire varier le $\Delta min_{tampon}$ est donc moins impactant. Aussi il est observé que
augmenter le $\Delta min_{capteur}$ impacte négativement la part solaire active sur
Strasbourg et est fortement non-linéaire ou avec des interactions sur Bordeaux (annexe,
\figref{fig:prod_sol_chauffage_mu_star}). Finalement le $\Delta T_{sol}$ a un impact
non-linéaire ou avec des interactions sur Strasbourg pour la part solaire active mais ne fait
pas parti des facteurs influent sur Bordeaux.

\paragraph{Bilan~:} % (fold)
\label{par:bilan_prod_sol_chauff}
Il est exposer la complexité liée à l’évaluation couplée d’un \abr{SSC} et d’un bâtiment.
Bien que les facteurs identifiés comme influents soit similaires entre Bordeaux et
Strasbourg, l’analyse détaillée considérant séparément la part solaire active et passive
met en évidence que le \abr{SSC} ne se comporte pas de la même manière sur les deux
climats. Ainsi en plus d’une forte interaction identifiée entre la $F_{sol}^{ECS}$ et la
$F_{sol}^{CH}$, il est mis en évidence une forte interaction entre le chauffage solaire
actif et passif. Le choix des objectifs $F_{sav}^{ECS}$ et $F_{sav}^{CH}$ pour
l’optimisation est donc bien plus pertinent. En effet il n’existe pas à la connaissance de
l’auteur de moyen d’identifier la part passive utile de la part inutile. En considérant la
part économisé d’appoint, il est pris indirectement en compte de l’utilité de l’énergie
solaire apportée au bâtiment, que ce soit la part active ou passive. Finalement
l’existence d’un compromis entre qualité de l’enveloppe et performance du \abr{SSC} sur le
chauffage est aussi identifié.
% paragraph bilan_prod_sol_chauff (end)
% subsubsection couverture_solaire_sur_le_chauffage (end)


% - - - - - - - - - - - - - - - - - - - - - - - - - - - - - - - - - - - - - - -
\subsubsection{Consommation de l’appoint} % (fold)
\label{ssub:consommation_de_l_appoint}
Sur la $Conso_{app}$ (\figref{fig:objectifs_mu_star}) les facteurs les plus influents sont
pour les deux climats~: le nombre de capteurs solaires thermiques et l’$Émis_{int}$ des
vitrages. Les autres caractéristiques de l’enveloppe ont aussi des influences linéaires
($R$ plafond, $R$ murs) sur Bordeaux et Strasbourg. La qualité du plancher par contre a un
impact uniquement sur Strasbourg. Les caractéristiques de l’enveloppe sont donc sur
Strasbourg toutes influentes et la performance du vitrage est le facteur prédominant.

L’algorithme semble aussi influencer directement la consommation de l’appoint avec
$\Delta min_{tampon}$ sur Bordeaux et $\Delta min_{capteur}$ sur Strasbourg, les deux
ayant une influence non-linéaire ou avec des interactions. Il est aussi observé que
le volume du ballon \abr{ECS} est un facteur important dans les deux climats alors
que le volume de l’appoint n’est influent que sur Strasbourg.

Afin de mieux comprendre, l’analyse est détaillée en séparant la part attribuée au
chauffage et à la production d’\abr{ECS} (annexe, \figref{fig:conso_app_mu_star}). Si on
considère uniquement les consommations sur le chauffage ($Conso_{app}^{CH}$), la qualité
thermique du vitrage arrive largement devant les autres facteurs que ce soit pour Bordeaux
ou Strasbourg. Sur Bordeaux le nombre de capteurs solaires thermiques est aussi fortement
impactant par rapport aux autres facteurs. Sur Strasbourg, la performance thermique des
murs est aussi influente que le nombre de capteurs solaires car les besoins en chauffage
sont important.
$\Delta min_{tampon}$ et $\Delta min_{capteur}$ sont influents pour les deux climats mais
uniquement sur la $Conso_{app}^{CH}$. Sur la $Conso_{app}^{ECS}$ la $\Delta min_{tampon}$
et le seul facteur impactant pour Bordeaux alors que sur Strasbourg, seul $\Delta
min_{capteur}$ est influent. Cette remarque permet d’expliquer pourquoi uniquement un des
deux facteurs est influent lorsque on considère la consommation totale $Conso_{app}$. La
même observation peut être faite pour le ballon tampon dont le volume impacte uniquement
la $Conso_{app}^{CH}$ sur Bordeaux. Enfin les caractéristiques des capteurs sont
influentes à la fois sur la $Conso_{app}^{CH}$ et la $Conso_{app}^{ECS}$.

\paragraph{Bilan~:} % (fold)
\label{par:bilan_conso_app}
L’analyse met en évidence que certain facteurs influencent soit la $Conso_{app}^{ECS}$,
soit pour la $Conso_{app}^{CH}$. Les besoins en chauffage sur Strasbourg étant plus
important que sur Bordeaux, les facteurs caractérisants l’enveloppe sont tous influent que
ce soit sur la $Conso_{app}^{CH}$ ou la $Conso_{app}$. Sur Bordeaux ou le chauffage
représente une part moindre par rapport aux besoins en \abr{ECS}, les facteurs peu
impactant sur le chauffage n’apparaissent pas lorsque la $Conso_{app}$ est considérée.
Cette tendance est de plus renforcée par l’ensoleillement plus important sur Bordeaux,
permettant au \abr{SSC} de couvrir un part plus importante des besoins et donc de diminuer
la variation de la consommation du chauffage lorsque l’enveloppe est modifiée. Ces
résultats confirment la nécessité de considérer les systèmes et l’enveloppe de concert
afin de proposer une solution adaptée aux contraintes du climat.
% paragraph bilan_conso_app (end)
% subsubsection consommation_de_l_appoint (end)


% - - - - - - - - - - - - - - - - - - - - - - - - - - - - - - - - - - - - - - -
\subsubsection{Consommation totale} % (fold)
\label{ssub:consommation_totale}
Concernant la consommation totale, le nombre de capteur \abr{PV} est sans surprise, le
plus influent. En effet, les consommations propres aux charges internes (électro-domestique
et éclairage) représentent une part importante des consommations. De plus, la
surface de capteurs thermiques minimale permet déjà de couvrir une part importante des
besoins, notamment sur la production d’\abr{ECS}. Il est aussi important de garder à
l’esprit que la surface de capteur \abr{PV} varie de \SI{2}{\metre\squared} de plus que
celle des capteurs solaires thermiques pour les raisons explicitées en introduction de
section. Ces observations permettent d’expliquer pourquoi le nombre de capteur \abr{PV}
est le facteur le plus influent. Ainsi, sur Strasbourg ou les besoins en \abr{ECS} et
en chauffage sont prédominants, la performance des vitrages et le nombre de capteurs
solaires thermiques sont aussi fortement influents.
% subsubsection consommation_totale (end)


% - - - - - - - - - - - - - - - - - - - - - - - - - - - - - - - - - - - - - - -
\subsubsection{Valorisation de l’énergie solaire~:} % (fold)
\label{ssub:valorisation_de_l_energie_solaire}
Cette dernière partie décrit les facteurs influençant la part solaire absorbée au niveau
des capteurs ($Prod_{sol}$), les pertes en ligne ($Pertes_{réseau}$), et la part solaire
valorisée ($Prod_{sol}^{valorisée} = Prod_{sol} - Pertes_{réseau}$) (annexe,
\figref{fig:prod_sol_valorisee_mu_star}). Pour Bordeaux comme Strasbourg le volume des
ballons, le type de capteur, et le nombre de capteurs solaires influencent la
$Prod_{sol}^{valorisée}$. Comme observée pour la $Conso_{app}$, le $\Delta min_{capteur}$
n’est influent que sur le climat de Strasbourg, et le $\Delta min_{tampon}$ ne l’est que
sur Bordeaux. Ces deux facteurs ayant une influence non-linéaire ou avec des interactions.
Sur Bordeaux, il est aussi mis en évidence que la performance des vitrages a un impact non
linéaire ou avec des interaction importantes. Afin de l’expliquer, il est nécessaire de
croiser plusieurs informations. Il a été vu qu’un compromis existe entre qualité de
l’enveloppe et performance du \abr{SSC} sur le chauffage
(\ref{ssub:couverture_solaire_sur_le_chauffage}). De plus, les résultats de l’étude
paramétrique montrent que sur le climat de Bordeaux, les besoins en chauffage sont faibles
et que le solaire peut en couvrir la majeure partie. Avec un climat disposant d’un
ensoleillement important, le \abr{SSC} sur Bordeaux peut fournir l’énergie supplémentaire
nécessaire lorsque les besoins de chauffage augmente (augmentation de l’$Émis_{int}$). Sur
Strasbourg ce facteur n’est donc pas influent car le \abr{SSC} couvre moins bien les
besoins de chauffage qui sont de surcroît plus importants. De plus l’ensoleillement est
moindre et l’augmentation des pertes thermiques induites par une diminution de la
performance des vitrages est plus grande.

Enfin, les $Pertes_{réseau}$ sont fortement impactées par la surface totale de capteur et
l’épaisseur de l’isolant au niveau des canalisations. Le volume du ballon sanitaire a
aussi un impact important. En comparant l’effet élémentaire sur la moyenne
(\figref{fig:prod_sol_valorisee_mu}) et sur la moyenne pondérée, il apparaît que augmenter
le volume du ballon réduise les pertes en ligne. Les variations algorithmiques impacte
aussi les $Pertes_{réseau}$. $\Delta min_{capteur}$ a en effet une influence non-linéaire
ou avec interaction pour les deux climats alors que $\Delta min_{tampon}$ est influent
uniquement sur Bordeaux. La part solaire perdue en ligne ($Pertes_{réseau}$) étant faible
au regard des apports solaires ($Prod_{sol}$), l’isolation des canalisations
($Isolant_{réseau}^{ep}$) n’est pas influent sur la $Prod_{sol}^{valorisée}$. Par
extension, il n’est pas influent non plus sur les indicateurs $F_{sol}^{ECS}$ ou
$F_{sol}^{CH}$.
% subsubsection valorisation_de_l_energie_solaire (end)
% subsection analyse_des_resultats (end)



% - - - - - - - - - - - - - - - - - - - - - - - - - - - - - - - - - - - - - - -
\subsection{Paramètres retenus a posteriori} % (fold)
\label{sub:parametres_retenus_a_posteriori}
L’analyse de sensibilité a permis de réduire le nombre de critères de décision
pour Bordeaux et Strasbourg. Ainsi sur les $22$ facteurs a priori, il en est retenu
$14$ pour Bordeaux et $15$ pour Strasbourg dont les bornes de variation sont décrites
dans \tabref{tab:facteur_retenues}.
Aussi, suite aux observations faites dans l’analyse des résultats, la borne maximale pour le volume des
ballons a été réajustée à \SI{400}{\litre} maximum par ballon contre \SI{500}{\litre}
auparavant.
Les résultats du criblage sont aussi présentés sous une forme synthétique grâce à un
graphe d’influence qui décrit le lien entre chaque facteur, et chaque indicateur
(\figref{fig:graphe_influence_objectifs}). La visualisation permet de mettre en évidence
les facteurs qui sont uniquement influent pour un indicateur (Surfaces vitrées, isolation du ballon tampon\dots) et
ceux qui en impactent plusieurs (type de vitrage, nombre de capteurs \abr{PV}\dots).

L’analyse a aussi permis de mettre en évidence plusieurs remarques~:
\begin{itemize}
  \item Le \abr{SSC} est fortement influencé par des éléments de l’enveloppe, des équipements,
        mais aussi par les variations algorithmiques.
  \item $16$ facteurs différents ont été identifiés comme influents sur les objectifs de
        l’optimisation pour Bordeaux et Strasbourg cumulés.
  \item Augmenter l’isolation des canalisations n’a pas d’intérêts.
  \item Les variations du \abr{SSC} sont plus influente lorsque la $Conso_{app}$
        est plus importante que la consommation des équipements internes.
  \item Le potentiel de réduction de la consommation de l’appoint est plus important
        pour Strasbourg que Bordeaux.
  \item Il existe un compromis entre qualité de l’enveloppe, la performance du \abr{SSC}
        et la surface installée de capteur \abr{PV}.
  \item Le dimensionnement d’un bâtiment doit être le résultat d’une évaluation
        couplée du bâtiment et du système installée, en particulier lorsque le système
        utilise des énergies renouvelables.
\end{itemize}

Malgré une réduction du nombre de paramètres de décisions, la temps de simulation
important est toujours un frein à l’application d’une méthode d’optimisation
par méta-heuristique. De plus, il est nécessaire d’évaluer en parallèle le modèle
de référence pour chaque variation de l’enveloppe afin de pouvoir calculer l’économie
d’appoint permise par le \abr{SSC}~: $F_{sav}^{CH}$ et $F_{sav}^{ECS}$. Il est donc
proposer d’utiliser des modèles de substitutions.


\begin{table}
\small
\centering
\caption[Description des paramètres retenus pour l’optimisation]
         {Description des paramètres retenus pour l’optimisation. Les facteurs uniquement retenus
          sur Bordeaux sont sous un fond crème, et ceux uniquement pour Strasbourg sous un fond gris.}
\label{tab:facteur_retenues}
\begin{tabular}{l c c c c l}
  \toprule
  \addlinespace
                       & Min        & Max         & Catégorie  & Pas        & Remarques                                \\
  \addlinespace
  \multicolumn{5}{l}{\textbf{\abr{SSC}}}         \\
  \midrule
  \rowcolor{SolarizedBrWhite}
  Nombre capteurs \abr{TH}    & \num{2}    & \num{5} & Discrète    & \num{1}    & \SIrange{4.6}{11.6}{\metre\squared}  \\
  \rowcolor{SolarizedBrCyan}
  Nombre capteurs \abr{TH}    & \num{2}    & \num{7} & Discrète    & \num{1}    & \SIrange{4.6}{16.2}{\metre\squared}   \\
  Type capteurs \abr{TH}      & -          &  -      & Qualitative & -          & Voir \tabref{tab:capteurs_specs_optimisation}   \\

  $Ech_{sol}^{pos}$           & \num{0.8}  &  \num{1.3}  & Continue    & -          & Position relative à la hauteur du ballon     \\
  Volume ballon tampon        & \num{100}  &  \num{400}  & Discrète    & \num{50}   & \multirow{2}{*}{Dimensions adaptées proportionnellement}   \\
  Volume ballon $ECS$         & \num{100}  &  \num{400}  & Discrète    & \num{50}   &    \\
  \rowcolor{SolarizedBrWhite}
  $Isolant_{ballon}$ tampon   & \num{0.055} &  \num{0.12} & Discrète    & \num{50}   &  Résistance dépendante du volume du ballon  \\
  $\Delta T_{sol}$            & \num{5}    &  \num{20}   & Continue    & -          &  -      \\
  $\Delta min_{capteur}$      & \num{0}    &  \num{30}   & Continue    & -          &  -      \\
  $\Delta min_{tampon}$       & \num{0}    &  \num{30}   & Continue    & -          &  -      \\
  \\
  \addlinespace[\defaultaddspace]
  \multicolumn{4}{l}{\textbf{Enveloppe du bâtiment}}             \\
  \midrule
  $R$ murs             & \num{4}    &  \num{7}    & Discrète    & \num{0.5}  & -                                  \\
  $R$ plafond          & \num{6}    &  \num{10}   & Discrète    & \num{0.5}  & -                                                                      \\
  \rowcolor{SolarizedBrCyan}
  $R$ plancher         & \num{6}    &  \num{10}   & Discrète    & \num{0.5}  & -                                                                     \\
  \rowcolor{SolarizedBrCyan}
  Surface vitrée sud   & \num{5.42} &  \num{8.13} & Continue    &  -          & -       \\
  Surface vitrée est   & \num{4.3}  &  \num{6.46} & Continue    &  -          & - \\
  Type de vitrage      & -          &  -          & Qualitative &  -         & Voir \tabref{tab:carac_vitrages} \\
  \\
  \addlinespace[\defaultaddspace]
  \multicolumn{5}{l}{\textbf{Production d’électricité}}      \\
  \midrule
  Nombre capteurs \abr{PV}         & \num{14}   &  \num{24}   & Discrète    &  \num{1}   & \SIrange{23.0}{39.4}{\metre\squared}   \\
  \bottomrule
\end{tabular}
\end{table}

\begin{figure}
    \centering
    \includegraphics[width=0.75\textwidth]{Ressources/Images/Sensibilite/graphInfluence.pdf}
    \caption[Graphe d’influence du \abr{SSC} pour Bordeaux et Strasbourg]
            {Graphe d’influence du \abr{SSC} pour Bordeaux (gauche) et Strasbourg (droite). Les
             relations linéaires sont en noires, et les relations non-linéaires ou
             avec interactions en bleues.}
    \label{fig:graphe_influence_objectifs}
\end{figure}


% subsection parametres_retenus_a_posteriori (end)
% section methode_criblage_de_morris (end)




% ..............................................................................
% ..............................................................................
\section{Construction des modèles de substitutions} % (fold)
\label{sec:construction_des_modeles_de_substitutions}
% ------------------------------------------------------------------------------
\subsection{Création de l’échantillon} % (fold)
\label{sub:creation_de_l_echantillon}
Grâce à une approche par criblage, les variables de décisions non influentes ont pu être
écartées. La combinatoire du problème ainsi réduite permet d’éviter l’évaluation de variations non influentes.
Cependant le temps de simulation est toujours un facteur limitant et ne permet pas
de réaliser l’optimisation dans des temps raisonnables. En effet, il est nécessaire de
compter \SIrange{1}{3}{h} pour simuler le \abr{SSC} et \SIrange{10}{30}{min} pour
le système de référence.
Ainsi $8$ modèles de substitutions sont utilisés pour approximer le calcul des indicateurs
sur Bordeaux et Strasbourg ($4$ par climat)~:
\begin{itemize}
  \item La consommation de l’appoint pour le chauffage ($Conso_{app}^{CH}$)
  \item La consommation de référence sur le chauffage ($Conso_{ref}^{CH}$)
  \item La consommation de l’appoint pour la production d’\abr{ECS} ($Conso_{app}^{ECS}$)
  \item La consommation cumulée de l’appoint pour le chauffage et l’\abr{ECS} ($Conso_{app}$)
\end{itemize}

Ces indicateurs ne correspondent pas directement aux objectifs mais permettent de les calculer.
Avec \eqref{eq:taux_economie_opti} il est possible d’obtenir $F_{sav}^{CH}$ et $F_{sav}^{ECS}$,
et avec \eqref{eq:conso_totale}, il est possible de calculer la $Conso_{tot}$.
Ce choix est préféré car il est plus modulaire. En considérant séparément les consommations
du système de référence et du \abr{SSC}, les deux modèles restent indépendant. Il est
ainsi possible de modifier un des modèles sans impacter le second. Pour rappel, le
modèle de référence est estimée pour chaque variations de l’enveloppe afin de considérer
uniquement le gain apportée par le solaire et non par l’enveloppe.
Concernant les autres objectifs, un modèle de substitution n’est pas nécessaire. La
production des capteurs \abr{PV} est pré-calculée en considérant la répartition décrite
dans \eqref{eq:repartition_toiture} et la $Conso_{ref}^{ECS}$ est très peu impactée par
les variations de l’enveloppe. Sur bordeaux elle est pour une année complète de \SI{2588(4)}{kWh}
et de \SI{2866 (2)}{kWh} pour Strasbourg et est donc considérée comme constante
dans le reste du document.

La méthodologie décrite dans \ref{sub:modeles_de_substitution} est appliquée pour
construire les $4$ échantillons nécessaires~: un pour le système de référence et un pour
le \abr{SSC} le tout pour les deux climats ($2 \times 2$ échantillons). Les solutions sont
générées pseudo-aléatoirement à partir d’une suite à faible discrépance qui permet
d’obtenir un échantillon équitablement réparties offrant une bonne représentativité de
l’espace de décision. Une fois l’échantillon simulé, la bibliothèque
\fnref{http://openturns.org/}{\textit{OpenTurns}} et les travaux de \textcite{Rania2013}
sont utilisés pour construire les modèles de substitutions. Étant en phase de conception
et cherchant à explorer l’ensemble du domaine de recherche, une loi uniforme est
considérée pour chaque paramètre. Afin de prendre en compte les variables qualitatives,
une substitution est nécessaire. Pour les capteurs, il apparaît nécessaire de tenir compte
du rendement optique ($\eta_{0}$) mais aussi du coefficient de pertes linéiques ($a_{1}$)
car uniquement considérer l’un des deux réduit la précision du méta-modèle. Pour le
vitrage, le coefficient de pertes thermique ($U_{g}$) est suffisant. La précision des
modèles de substitutions par rapport au modèle original est évaluée sur deux critères~:
l’erreur quadratique moyenne ($RMSE$, Root Mean Square Error) et l’erreur absolue maximale
($MAE$, Maximal Absolute Error). La $RMSE$ permet d’évaluer l’écart moyen entre, les
sorties du méta-modèle ($\mathcal{M}$), et celles du modèle original ($f$)
\eqref{eq:rmse}. La $MAE$ permet d’obtenir l’écart maximal existant entre le méta-modèle
et le modèle de référence \eqref{eq:mae}.

Pour le système de référence, l’échantillon est construit en faisant varier les différents
paramètres de l’enveloppe soit $6$ facteurs. Pour le système solaire l’ensemble des
variables influentes est considéré, soit respectivement $14$ et $15$ facteurs pour
Bordeaux et Strasbourg (\tabref{tab:facteur_retenues}).
La qualité de l’approximation est discutée pour différentes tailles d’échantillons
dans la section qui suit.

\begin{align}
  \label{eq:rmse}
  RMSE &= \sqrt{\frac{1}{N}\sum^{N}_{i=1} \left[ f(\vec{x}_{i}) - \mathcal{M}(\vec{x}_{i}) \right]^{2} } \\
  \label{eq:mae}
  MAE  &= max \left( \abs{f(\vec{x}_{i}) - \mathcal{M}(\vec{x}_{i})}, \: x = 1, 2, \dotsc, N \right)
\end{align}
% subsection creation_de_l_echantillon (end)



% ------------------------------------------------------------------------------
\subsection{Sélection des méta-modèles} % (fold)
\label{sub:selection_des_meta_modeles}
Afin d’offrir une meilleure lisibilité, seuls les principaux résultats sont intégrés dans
le cœur du document. Cependant comme pour l’étude de sensibilité, tous
les résultats sont disponibles en annexes (\ref{cha:creation_des_meta_modeles}).

Pour l’indicateur $Conso_{ref}^{CH}$, peu de simulations ont été nécessaires afin
d’obtenir une bonne approximation (annexes, \figref{fig:mae_rmse_qualite_ref}). En effet,
à partir d’un échantillon de taille $100$, le méta-modèle est d’ordre $3$ et précis à
\SI{+- 1.5}{kWh}. À partir de $400$ simulations l’erreur quadratique moyenne et l’erreur
absolue stagne toutes les deux en dessous de \SI{0.5}{kWh} (\tabref{tab:meta_result_bilan}).
Au regard des résultats, un échantillon de taille $400$ est donc suffisant pour obtenir
une bonne approximation.

Le modèle du \abr{SSC} étant plus complexe, un nombre plus important de simulation est
nécessaire. Premièrement car le nombre de combinaisons est plus important et modifie à la
fois le système et l’enveloppe. Deuxièmement, à cause des fortes interactions identifiées
entre les parts solaires respectives sur le chauffage et l’\abr{ECS}. Ainsi, il est à
minima nécessaire de considérer un échantillon de taille $600$ afin d’approcher
correctement le modèle original (\figref{fig:rmse_mae}). Entre $600$ et $1000$, la qualité
de l’approximation oscille pour les trois indicateurs sur Bordeaux comme sur Strasbourg.
Enfin au delà de $1000$, l’approximation s’améliore lentement même si sur Strasbourg,
l’erreur absolue maximale continue d’osciller. Ainsi, pour les $3$ méta-modèles, un
échantillon de taille $600$ est pertinent. En effet, l’erreur moyenne quadratique est pour
Bordeaux inférieure à \SI{3.5}{\percent} et l’erreur maximale absolue inférieure à \SI{8}{\percent}.
Sur Strasbourg l’erreur moyenne quadratique est inférieure à \SI{2.7}{\percent} et
l’erreur absolue maximale inférieure à \SI{6.8}{\percent}.

\begin{figure}
    \centering
    \includegraphics[width=\textwidth]{Ressources/Images/MetaModele/RMSE.pdf}
    \caption[Évolution de la $RMSE$ en fonction de l’échantillon]
            {Évolution de la $RMSE$ pour les \num{3} méta-modèles
             en fonction de la taille de l’échantillon.}
    \label{fig:rmse_mae}
\end{figure}

Comme explicité dans le chapitre précédent, \SI{90}{\percent} de l’échantillon est utilisé
pour construire le modèle de substitution et \SI{10}{\percent} est uniquement
utilisée pour la validation à travers le calcul de la $RMSE$ et de la $MAE$.
Les résultats obtenus (\figref{fig:validite_meta_ssc}) montrent que les points sont en
effet quasiment tous compris sur la droite d’équation $x = y$ qui indique une approximation
parfaite du modèle original. Les solutions les moins bien approximées sont pour des
consommations importantes alors que l’optimisation cherche à les minimiser.

\begin{figure}
    \centering
    \includegraphics[width=\textwidth]{Ressources/Images/MetaModele/validite_meta_ssc_600.pdf}
    \caption[Évaluation de la précision des méta-modèles pour les solutions de l’échantillon]
            {Évolution des résultats obtenus par les méta-modèles (axes Y) et du modèle
             original (axes X). La ligne noire représente une approximation parfaite.}
    \label{fig:validite_meta_ssc}
\end{figure}

Finalement, \tabref{tab:meta_result_bilan} récapitule les informations concernant les modèles
de substitutions retenus mais aussi la variabilité de chaque indicateurs observée sur les
$2000$ simulations. Sur les $8$ meta-modèles, $7$ sont d’ordre $2$ et un seul est d’ordre
$3$~: le modèle substituant le calcul de la $Conso_{ref}^{CH}$ sur Strasbourg. Il est
aussi observé un écart de consommation important entre Bordeaux et Strasbourg, la
consommation maximale sur Bordeaux et similaire à la consommation minimale sur Strasbourg.
Sur l’\abr{ECS} la variation est un peu plus importante sur Strasbourg alors que les
besoins sont identiques. En effet, les besoins sont les mêmes pour les deux climats, mais
la température de l’eau du réseau est différente. Aussi, la consommation
en chauffage semble secondaire sur Bordeaux alors qu’elle est importante sur Strasbourg.
Finalement l’échantillon met déjà en avant le potentiel du \abr{SSC} en particulier sur
la production d’\abr{ECS} qui est sans solaire de \SI{2588}{kWh} pour Bordeaux
et de \SI{2866}{kWh} pour Strasbourg.

Dans la partie suivante, les méta-modèles sont confrontés aux solutions obtenues
au cours du processus d’optimisation et les résultats sont discutés.

\begin{table}
\centering
\caption[Erreurs caractéristiques des \num{8} méta-modèles retenus]
        {Erreurs caractéristiques ($RMSE$ et $MAE$) des \num{8} méta-modèles retenus (\si{kWh}).}
\label{tab:meta_result_bilan}
\begin{tabular}{l c c c c c c c c c c}
    \toprule
                    & \multicolumn{4}{c}{Bordeaux} & & \multicolumn{4}{c}{Strasbourg} &
                      Taille \\
                    \cmidrule(r){2-5}
                    \cmidrule(r){7-10}
                    & $RMSE$ & $MAE$  & Ordre & Variation  &       & $RMSE$ & $MAE$ & Ordre & Variation & échantillon \\
    \midrule
    $Conso_{ref}^{CH}$  & \num{0.2}  & \num{0.6}  & \num{2} & \numrange{228}{620}&   & \num{0.19}   & \num{0.93}  & \num{3}     & \numrange{1363}{2110}      & \num{400}  \\
    \addlinespace[\defaultaddspace]
    $Conso_{app}^{CH}$  & \num{7.7}  & \num{22.0} & \num{2} & \numrange{84}{440} &   & \num{16.9}   & \num{38.0}  & \num{2}     & \numrange{612}{1861}       & \num{600} \\
    \addlinespace[\defaultaddspace]
    $Conso_{app}^{ECS}$ & \num{15.7} & \num{56.4} & \num{2} & \numrange{156}{1043}&  & \num{21.3}   & \num{70.8}  & \num{2}     & \numrange{397}{1360}       & \num{600} \\
    \addlinespace[\defaultaddspace]
    $Conso_{app}$       & \num{16.6} & \num{52.4} & \num{2} & \numrange{324}{1372}&  & \num{22.3}   & \num{65.1}  & \num{2}     & \numrange{1222}{3117}       & \num{600} \\
    \bottomrule
\end{tabular}
\end{table}
% subsection selection_des_meta_modeles (end)



% ------------------------------------------------------------------------------
\subsection{Qualité de l’approximation sur les front de Pareto} % (fold)
\label{sub:qualite_de_l_approximation_sur_les_front_de_pareto}
Afin de valider l’approche par méta-modèle, les solutions optimales obtenues pour Bordeaux
et Strasbourg sont simulées avec le modèle original (\figref{fig:validite_meta_ssc_optimisation}).
Comme pour les solutions de l’échantillon, les solutions optimales sont correctement
approximées même si la dispersion des points est plus importante. Il est cependant
observée que les modèles de substitutions semblent sous- évaluer la valeur du modèle
original pour tous les indicateurs en particulier sur Strasbourg pour l’indicateur
$Conso_{app}^{CH}$. L’algorithme d’optimisation cherche en effet à minimiser ces $3$
indicateurs afin de pouvoir maximiser les objectifs $F_{sav}^{ECS}$ et $F_{sav}^{CH}$.
Ainsi en cherchant à trouver les solutions minimisant au maximum les indicateurs, les
erreurs \abr{RMSE} et \abr{MAE} sont maximisées.

Le \tabref{tab:meta_vs_optimisation} montre que l’erreur moyenne quadratique augmente pour
tous les indicateurs, en particulier pour l’indicateur $Conso_{app}^{CH}$. À l’exception
de l’indicateur $Conso_{app}^{CH}$, les erreurs absolues varient peu par rapport à celle
observée sur l’échantillon de validation, elles diminuent même sur Bordeaux pour
l’indicateur $Conso_{app}^{ECS}$. Pour l’indicateur $Conso_{app}^{CH}$ sur Strasbourg,
l’erreur maximale est par contre deux fois plus importante. Ce résultat était attendu au
vu de la forme instabilité observée lors de l’évaluation de l’impact de la taille de
l’échantillon (annexe, \figref{fig:rmse_mae}). Il avait en effet été montré que l’erreur
absolue maximale reste importante même pour un échantillon de taille $2000$.

L’erreur relative reste inférieure à \SI{6}{\percent} sur Bordeaux et
\SI{5}{\percent} sur Strasbourg~: les erreurs obtenues sont donc considérées
comme acceptables et les modèles peuvent êtres utilisés pour analyser les résultats de
l’optimisation.


\begin{figure}
    \centering
    \includegraphics[width=\textwidth]{Ressources/Images/MetaModele/meta_optimization.pdf}
    \caption[Évaluation de la précision des méta-modèles pour les solutions optimales]
            {Évolution des résultats obtenus par les méta-modèles (axes Y) et du modèle
             original (axes X) pour les solutions du front de Pareto.
             La ligne noire représente une approximation parfaite.}
    \label{fig:validite_meta_ssc_optimisation}
\end{figure}


\begin{table}
\centering
\caption[Erreurs caractéristiques des méta-modèles sur l’échantillon
         de validation et sur les solutions non-dominées]
        {Erreurs caractéristiques ($RMSE$ et $MAE$) des méta-modèles sur l’échantillon
         de validation (gauche) et sur les solutions non-dominées obtenues par
         l’optimisation (droite).}
\label{tab:meta_vs_optimisation}
\begin{tabular}{l c c c c c C{0.5cm} c c c c c}
    \toprule
                    & \multicolumn{5}{c}{\textbf{Validation}} & & \multicolumn{5}{c}{\textbf{Optimisation}}  \\
                    & \multicolumn{2}{c}{Bordeaux} & & \multicolumn{2}{c}{Strasbourg} & & \multicolumn{2}{c}{Bordeaux} & & \multicolumn{2}{c}{Strasbourg} \\
                    \cmidrule(r){2-3}
                    \cmidrule(r){8-9}
                    \cmidrule(r){5-6}
                    \cmidrule(r){11-12}
                                     & $RMSE$     & $MAE$        &       & $RMSE$       & $MAE$      &   & $RMSE$       & $MAE$        &      & $RMSE$       & $MAE$ \\
    \midrule
    $Conso_{app}^{CH}$               & \num{7.7}  & \num{22.0}   &       & \num{16.9}   & \num{38.0} &   & \num{10.5}   & \num{32.0}   &      & \num{29.7}   & \num{81.0}  \\
    \addlinespace[\defaultaddspace]
    $Conso_{app}^{ECS}$              & \num{15.7} & \num{56.4}   &       & \num{21.3}   & \num{70.8} &   & \num{24.3}   & \num{52.8}   &      & \num{30.5}   & \num{77.5}  \\
    \addlinespace[\defaultaddspace]
    $Conso_{app}$                    & \num{16.6} & \num{52.4}   &       & \num{22.3}   & \num{65.1} &   & \num{25.6}   & \num{67.2}   &      & \num{33.9}   & \num{73.1}  \\
    \bottomrule
\end{tabular}
\end{table}
% subsection qualite_de_l_approximation_sur_les_front_de_pareto (end)
% section construction_des_modeles_de_substitutions (end)



% ..............................................................................
% ..............................................................................
\section{Optimisation multi-objectif} % (fold)
\label{sec:optimisation_multi_objectif}
Dans l’optique de ces travaux, il a été proposé d’évaluer la performance couplée
du système et du bâtiment à travers un processus d’optimisation. Dans un premier temps
un ensemble de paramètres a priori couvrant à la fois des variations
au niveau du système (logique de contrôle et équipements), et des variations de l’enveloppe,
ont été proposés. Puis, grâce à une méthode de criblage le nombre de facteurs a été limité
aux plus impactants puis des méta-modèles ont été construits afin de se substituer
aux modèles originaux dont le temps d’évaluation est trop important.
Cette partie s’intéresse à l’analyse des solutions obtenues à travers l’optimisation.
Dans un premier temps le paramétrage de l’algorithme puis l’évolution de la recherche est discuté.
Puis, l’ensemble des solutions optimales (front de Pareto) sont détaillées de manière globale. Une étude
plus spécifique est ensuite réalisée sur~:
\begin{itemize}
  \item L’influence du nombre de capteurs photovoltaïques.
  \item L’influence du nombre et du type des capteurs solaires thermiques.
  \item L’influence du volume des ballons, tampon et d’\abr{ECS} et de la logique de contrôle.
  \item Les corrélations entre enveloppe et performance du \abr{SSC}.
\end{itemize}
En dernière partie, l’aide à la décision est illustré sur Strasbourg en fonction
de critères fictifs et une sélection de solutions optimales est discuté.



% ------------------------------------------------------------------------------
\subsection{Paramétrage de l’optimisation} % (fold)
\label{sub:parametrage_de_l_optimisation}
L’algorithme de colonie d’abeilles virtuelles (\ref{sub:description_de_l_approche_globale})
est utilisé afin de réaliser l’optimisation multi-objectifs. L’optimisation est réalisée
sur $4$ objectifs et est contraint afin d’obtenir uniquement des solutions à énergie
positives (\ref{ssub:approche_retenue})~:
\begin{itemize}
  \item Maximiser le $F_{sav}^{ECS}$ ($\epsilon = 0.005$)
  \item Maximiser le $F_{sav}^{CH}$ ($\epsilon = 0.005$)
  \item Maximiser la $Prod_{PV}$ ($\epsilon = 5$)
  \item Minimiser la $Nombre_{PV}$ ($\epsilon = 0.1$)
  \item Contraint par $\abs{Conso_{tot}}   \leq  8 \ \si{kWh_{ep}\per\metre\squared}$.
\end{itemize}

Une archive par $\epsilon$-dominance est utilisée pour stocker les solutions optimales
durant le processus afin d’améliorer à la fois l’exploitation et l’exploration, évitant
ainsi de converger vers des optimums locaux. L’archive impose la définition d’un
hyperplan formés par des hypercubes contenant une unique solution. Les dimensions de chaque
hypercube sont définis suivant les valeurs d’$\epsilon$ respectives de chaque objectif.
Il est important de rappeler que pour appartenir au même hypercube, deux solutions
doivent obligatoirement avoir un écart sur chaque objectif inférieur à la valeur d’$\epsilon$.
Cependant l’hyperplan étant fixe, respecter ces conditions n’est pas suffisant, certaines solutions
très similaires peuvent en effet se trouver respectivement sur les bornes inférieures
et supérieures de deux hypercube adjacents. Dans ce cas, les deux solutions sont retenus
car bien que très similaires, elles appartiennent pas au même hypercube et ne violent donc
pas la contrainte d’unicité.
Par contre, si la solution appartient au même hypercube alors seule la solution la plus
optimale est retenue~: la solution minimisant la distance normalisée au point idéal de l’hypercube.
Pour les objectifs $F_{sav}^{ECS}$ et de $F_{sav}^{CH}$, une taille de \SI{0.5}{\percent}
est retenue, indiquant que deux solutions ayant une différence de plus de
\SI{0.5}{\percent} sont forcement dans un hypercube différent. Dans le cas de l’objectif $Prod_{PV}$, la
valeur retenue n’est que peu importante car la variation de la production des capteurs
\abr{PV} est discrète et est directement liée au nombre de capteur \abr{PV} ($Nombre_{PV}$).
Dans les deux cas la valeur de l’$\epsilon$ est alors choisie inférieur au saut discret
mais est arbitraire. Finalement l’algorithme est définie avec les paramètres suivants~:
\begin{itemize}
  \item $IT = 400$
  \item $NP = 90$
  \item $Max_{Echec} = 2 \times \text{nombre de variables de décisions} = 32$
\end{itemize}

Dans cette configuration, l’algorithme réalise au minimum \num{36090} évaluations ($NP *
(IT + 1)$) mais ce nombre est en pratique supérieur car le nombre d’évaluation total est aussi
dépendant de $Max_{Echec}$, le nombre maximal de variations consécutives que peut subir
une source avant d’être abandonnée. Son abandon entraine le tirage d’une nouvelle source
aléatoirement puis de sa position opposée. Ainsi si une source n’est pas améliorée au bout
de $32$ essais alors elle est abandonnée et l’algorithme évalue donc une solution
supplémentaire (position opposée) pour cette itération. Sur Bordeaux, $296$ sources ont
été abandonnées durant le processus d’optimisation contre $292$ sur Strasbourg.

La partie suivante décrit l’évolution de l’optimisation au cours des itérations
afin de montrer qu’il converge bien vers un ensemble uniformément répartie de solutions.
% subsection parametrage_de_l_optimisation (end)



% ------------------------------------------------------------------------------
\subsection{Évaluation de la progression de l’optimisation} % (fold)
\label{sub:evaluation_de_la_progression_de_l_optimisation}
L’évolution de l’optimisation est décrit par \figref{fig:hypervolume_schott_front} à
travers deux métriques~: la \textit{métrique S} aussi appelée hypervolume, et la métrique
de \textit{Schott}. Une augmentation de l’hypervolume traduit une meilleure convergence et
couverture de l’espace des objectifs par l’ensemble des solutions. Une valeur proche de
zéro pour la métrique de \textit{Schott} indique que les solutions sont réparties
uniformément dans le domaine des objectifs. Afin d’éviter les problèmes d’échelles,
l’ensemble des solutions sont normalisés et admettent donc une valeur allant de
\SIrange{0}{1}{}.

Dans les premières itérations (zone encadrée), les sources contiennent de nombreuses
solutions ne respectant pas la contraintes. En effet, la majorité des combinaisons
existantes ne permettent pas d’obtenir un bilan positif. Ainsi après l’initialisation
(itération $0$), le taux de faisabilité est très bas et les abeilles vont alors favoriser
les solutions qui respectent la contrainte. Ce comportement se traduit par une
augmentation très rapide du taux de faisabilité et de la taille de la population de
l’archive dans les premières itérations. Ce comportement est aussi observable à travers la
métrique d’hypervolume qui augmente rapidement. L’algorithme cherchant dans un premier
temps à converger vers le front optimal, les solutions de l’archive ne sont pas
uniformément réparties comme le montre les variations de la métrique de schott. Le front
évoluant rapidement, les solutions sont rapidement remplacées par de nouvelles plus
performantes et le niveau d’uniformité varie en particulier lorsque le
\abr{SSC} est installé sur Bordeaux. Après $300$ itérations, le front ne semble plus
évoluer et l’algorithme a couvert l’espace des objectifs uniformément.

Sur Bordeaux, il apparaît que la métrique d’hypervolume évolue par plateau. Si on observe
l’évolution du taux de faisabilité, un schéma se dégage. Prenons comme point de référence
le plateau qui démarre à la $155^{ème}$ itération. La qualité du front ne progresse que
très peu dans un premier temps, les abeilles commencent alors à accepter des solutions ne
respectant pas les contraintes. Ces solutions ne sont pas admissibles dans l’archive mais
permettent d’offrir des combinaisons alternatives potentiellement prometteuses. Vers la
$230^{ème}$ itération le taux de faisabilité remonte rapidement; les abeilles ont réussis à
profiter des nouvelles informations pour construire des solutions respectant les
contraintes et les sources sont mises à jour. L’algorithme en acceptant d’explorer des
combinaisons a priori non fructueuses a obtenu des informations supplémentaires
qu’il a pu exploiter pour construire de nouvelles solutions admissibles. Ces nouvelles
solutions sont ensuite ajoutées à l’archive améliorant la couverture du front optimal
(amélioration de la métrique hypervolume). L’algorithme à ce stade de l’optimisation a en
effet déjà convergé mais il est encore possible d’améliorer la couverture de l’espace des
objectifs.
L’exemple décrit ci-avant est aussi observable à plus petite échelle au sein même du plateau
formé entre $155$ et $260$ itérations. Chaque petite progressions du front est précédé par
une réduction du taux de faisabilité (exploration) puis son augmentation (apprentissage).
Sur Strasbourg, le schéma est plus difficilement discernable car l’algorithme suit une
progression plus régulière.

Ainsi, l’algorithme converge rapidement en début d’optimisation puis se diversifie
rapidement dans un premier temps, puis plus lentement pour finalement stagner après
l’obtention d’un front de solutions optimales uniformément réparties. Dans la partie
suivante, il est alors discuté des résultats obtenus pour Bordeaux comme pour
Strasbourg.

\begin{figure}
    \centering
    \begin{subfigure}[b]{0.48\textwidth}
        \centering
        \includegraphics[width=\textwidth]{Ressources/Images/EtudeDeCas/Bordeaux_algo.pdf}
        \caption{}
        \label{fig:hypervolume_schott_bor}
    \end{subfigure}
    \quad
    \begin{subfigure}[b]{0.48\textwidth}
        \centering
        \includegraphics[width=\textwidth]{Ressources/Images/EtudeDeCas/Strasbourg_algo.pdf}
        \caption{}
        \label{fig:hypervolume_schottstras}
    \end{subfigure}
    \caption[Évolution de la convergence et de la diversification de l’optimisation]
             {Évolution de l’hypervolume et de la métrique de \textit{Schott}
              durant le processus d’optimisation en fonction du nombre d’itérations pour
              Bordeaux (a) et Strasbourg (b).}
    \label{fig:hypervolume_schott_front}
\end{figure}
% subsection evaluation_de_la_progression_de_l_optimisation (end)



% ------------------------------------------------------------------------------
\subsection{Analyse globale des solutions non-dominées} % (fold)
\label{sub:analyse_globale_des_solutions_non_dominees}
Le front de Pareto obtenu comprend \num{120} solutions non-dominées sur Bordeaux
(\figref{fig:front_pareto_bordeaux}) contre \num{219} sur Strasbourg
(\figref{fig:front_pareto_strasbourg}). Dans les deux cas, il existe une large diversité
sur les différents objectifs. Certaines solutions favorisant le solaire thermique,
d’autres le photovoltaïque ou des combinaisons des deux. De plus, le fait de considérer
la couverture du chauffage et de l’\abr{ECS} de manière distinctes permet de renforcer le
nombre de compromis en proposant un plus grand nombre d’alternatives sur l’espace de décision.
Le front pour le climat de Strasbourg est plus fournit que celui de Bordeaux en raison d’une plage de variation
optimale plus importante sur le nombre de capteurs \abr{PV}, augmentant le nombre de
combinaisons possibles.
La représentation des front est construite sous forme d’une matrice carrée.
Les graphiques sur la parte inférieure représentent les projections deux à deux des objectifs~:
$F_{sav}^{ECS}$ (\si{\percent}), $F_{sav}^{CH}$ (\si{\percent}), et $Prod_{PV}$ (\si{kWh}).
Les graphiques en partie supérieure décrivent eux aussi les projections deux à deux mais
avec les axes x et y inversés permettant de mieux
apprécier la distribution des solutions en fonction des différents objectifs. La couleur des points,
décrit la répartition des solutions en fonction du dernier objectif~: le nombre de capteurs \abr{PV}.
Finalement, chaque histogramme sur la diagonale décrivent l’évolution et la répartition du
nombre de capteurs \abr{PV} en fonction des autres objectifs. Ces histogrammes sont
particulièrement utiles lorsque le nombre de points est important afin de distinguer
les tendances.

\paragraph{} % (fold)
Les solutions devant toutes respecter un bilan positif, il existe un équilibre en
production d’énergie par les capteurs solaires thermiques et par les capteurs \abr{PV}.
Il est ainsi mis en évidence que sur Bordeaux, afin de réduire au maximum le nombre
de capteurs \abr{PV}, il est nécessaire d’avoir~:
$F_{sav}^{CH} > \SI{57}{\percent} \ et \ F_{sav}^{ECS} > \SI{87}{\percent}$.
Améliorer au delà la performance du système, sur le chauffage, ou, sur la production
d’\abr{ECS}, ne permet pas de réduire le nombre de capteurs \abr{PV}. La consommation de
l’appoint est en effet déjà très faible et le production des capteurs \abr{PV} est
principalement utilisée pour couvrir les consommations électriques des équipements
internes ($Conso_{usages}$). Sur Strasbourg une réduction supérieure à \SI{38}{\percent}
pour la consommation de chauffage, et de \SI{84}{\percent} sur la production d’\abr{ECS},
ne permet pas de réduire le nombre de capteurs \abr{PV} nécessaires. Comme pour Bordeaux,
le gain absolue sur la consommation n’est pas suffisant pour permettre de se passer d’un
autre capteur \abr{PV}. La production des capteurs \abr{PV} admet en effet des sauts
discrets car il a été retenue de faire varier non pas la surface de capteurs mais leur
nombre afin de tenir compte de la géométrie de la toiture. Il est cependant aussi noté que
pour un même nombre de capteurs installés, la production peut légèrement varier. La
répartition des capteurs \abr{PV} est en effet dépendante du nombre de capteurs solaires
thermiques installés. Lorsque que le nombre réduit, il est possible de loger plus de
capteurs \abr{PV} sur le pan sud et donc, améliorer la production.

Afin d’apprécier l’étendu du front de Pareto, les bornes de l’espace des objectifs,
correspondant au point nadir et au point idéal sont disponibles dans le
\tabref{tab:bornes_front_pareto} . Les valeurs renseignées sont donc les maximums ou
minimums pour chaque indicateur, chaque ligne ne correspond pas à une solution existante
mais à la combinaison des solutions maximisant ou minimisant un objectif en particulier.
Sur Bordeaux, le \abr{SSC} permet avec $5$ capteurs de réduire de
\SI{90}{\percent} la consommation par rapport au cas de référence, soit une économie de
\SI{2572}{kWh}. À titre de comparaison, la meilleure solution avec
seulement $2$ capteurs solaires thermique permet une économie de \SI{2038}{kWh}. Ainsi
l’écart entre les deux est peu important et $2$ capteurs permettent déjà d’obtenir un taux
d’économie important~: \SI{76}{\percent} sur la production d’\abr{ECS} et
\SI{31}{\percent} sur le chauffage. Sur Strasbourg avec $7$ capteurs, l’économie
sur la consommation de l’appoint  est de \SI{3103}{kWh} ($F_{sav,\, ext} = \SI{74}{\percent}$)
contre \SI{2009}{kWh} avec seulement $2$ capteurs. Il apparaît donc que même sur Strasbourg, le
\abr{SSC} permet avec peu de capteurs solaires d’obtenir de bon taux d’économie~:
\SI{60}{\percent} sur la production d’\abr{ECS} et \SI{22}{\percent} sur le chauffage.
Le \abr{SSC} permet sur Strasbourg, d’économiser une part absolue plus importante, en
particulier sur le chauffage ($Conso_{app}$). Sur Bordeaux la consommation de chauffage
est en effet très faible et par conséquent améliorer le $F_{sav}^{CH}$ impacte plus
faiblement la consommation totale. Finalement, pour les deux climats, la plage de
variation des taux d’économies sont similaires, le \abr{SSC} est une alternative
renouvelable adaptée à une large palette de conditions climatiques~:
\begin{itemize}
    \item $F_{sav,\, ext}$~: \SI{\sim 30}{\percent}
    \item $F_{sav}^{ECS}$~: \SI{\sim 30}{\percent}
    \item $F_{sav}^{CH}$~: \SI{\sim 60}{\percent}
\end{itemize}
En effet bien que le $F_{sav,\, ext}$ soit moins important, l’économie sur l’appoint
est plus important~: le potentiel d’économie énergétique comme économique est donc supérieur
sur Strasbourg.

\begin{table}
\centering
\caption[Performance maximale pouvant être obtenue pour différents indicateurs]
         {Variation de la performance obtenue pour chaque indicateur en fonction du climat}
\label{tab:bornes_front_pareto}
\begin{tabular}{L{2cm} c c c c c c c c c c}
    \toprule
                & $F_{sav,\,ext}$ & $F_{sav}^{ECS}$ & $F_{sav}^{CH}$ & & $Conso_{app}$ & $Conso_{app}^{ECS}$ & $Conso_{app}^{CH}$ & & $Prod_{PV}$ & $Nbr_{PV}$ \\
    \addlinespace
    \multicolumn{11}{l}{\textbf{Bordeaux}} \\
    \midrule
    Max & 90  & 94  & 84  &   & 1379  & 921 & 470 &   & 4743  & 16  \\
    Min & 57  & 64  & 22  &   & 278 & 164 & 48  &   & 3239  & 11  \\
    \addlinespace
    \multicolumn{11}{l}{\textbf{Strasbourg}} \\
    \midrule
    Max & 74  & 89  & 62  &   & 2780  & 1191  & 1746  &   & 6021  &  24 \\
    Min & 42  & 58  & 8 &   & 1118  & 322 & 508 &   & 4014  & 16  \\
    \bottomrule
\end{tabular}
\end{table}


\begin{figure}
% pair_grid_plot
    \centering
    \includegraphics[width=\textwidth]{Ressources/Images/EtudeDeCas/Bordeaux_front.pdf}
    \caption[Front de Pareto sur Bordeaux pour les $4$ objectifs après $400$ itérations]
             {Front de Pareto sur Bordeaux pour les $4$ objectifs après $400$ itérations.
              Chaque graphique est la projection des objectifs deux à deux.
              Le quatrième objectif, le nombre de capteurs \abr{PV}, est décrit par la couleur des points.}
    \label{fig:front_pareto_bordeaux}
\end{figure}

\begin{figure}
% pair_grid_plot
    \centering
    \includegraphics[width=\textwidth]{Ressources/Images/EtudeDeCas/Strasbourg_front.pdf}
    \caption[Front de Pareto sur Strasbourg pour les $4$ objectifs après $400$ itérations]
             {Front de Pareto sur Strasbourg pour les $4$ objectifs après $400$ itérations.
              Chaque graphique est la projection des objectifs deux à deux.
              Le quatrième objectif, le nombre de capteurs \abr{PV}, est décrit par la couleur des points.}
    \label{fig:front_pareto_strasbourg}
\end{figure}

Pour les deux climats, une large diversité de solutions performantes sont obtenues.
De plus, il apparaît que le potentiel du \abr{SSC} soit plus intéressant sur Strasbourg,
car les besoins en chauffage comme pour la production d’\abr{ECS} sont plus importants.
Les parties suivantes analyses plus en détail le comportement du \abr{SSC} avec dans un
premier temps l’influence du nombre de capteurs photovoltaïques.
% subsection analyse_globale_des_solutions_non_dominees (end)



% ------------------------------------------------------------------------------
\subsection{Influence des capteurs photovoltaïques} % (fold)
\label{sub:influence_des_capteurs_photovoltaiques}
Bien que la part relative maximale économisée ($F_{sav,\,ext}$) soit inférieure sur
Strasbourg, l’économie absolue en énergie ($Conso_{ref} - Conso_{app}$), est elle plus
importante. De ce fait, le \abr{SSC} permet de substituer un nombre plus important de
capteurs \abr{PV}. En effet la $Conso_{ref}$ pour le climat de Strasbourg est supérieure,
autant sur le chauffage que sur la production d’\abr{ECS}. De plus la variation de la
performance de l’enveloppe entraine une variation de la $Conso_{ref}^{CH}$ de
\SI{400}{kWh} sur Bordeaux contre \SI{750}{kWh} sur Strasbourg. Par extension, vu que
l’optimisation est multi-objectifs, les solutions maximisant la surface de capteurs
\abr{PV} doivent produire plus sur Strasbourg afin de compenser l’énergie non économisé
grâce au \abr{SSC}. Ainsi la surface de capteurs \abr{PV} maximale nécessaire est plus
importante sur Strasbourg ($24$) que sur Bordeaux ($16$).

La borne inférieure est elle définie en fonction de deux élément. Le premier est la
consommation des usages spécifiques ($Conso_{usages}$), la valeur minimale que doit
couvrir la production des capteurs \abr{PV}. Celle-ci ne permet cependant pas d’expliquer
pourquoi le nombre minimal de capteurs \abr{PV} est plus important sur Strasbourg car la
$Conso_{usages}$ est équivalente sur les deux climats. Le second élément est la
performance du \abr{SSC}, et plus particulièrement, la différence de consommation entre
l’appoint et le système de référence ($Conso_{ref} - Conso_{app}$). Dans notre cas, le
taux d’économie du \abr{SSC} n’atteint pas \SI{100}{\percent}, il est donc nécessaire pour
atteindre un bilan positif, que la production des capteurs \abr{PV} ($Prod_{PV}$) couvrent
aussi une part de la consommation de l’appoint. Ainsi, plus le \abr{SSC} est performant
plus la part à couvrir par les capteurs \abr{PV} diminue. Cependant, comme expliqué au
paragraphe précédent, la consommation de référence est plus importante sur Strasbourg là
où la performance relative du \abr{SSC} est moins importante. Ainsi, le nombre minimal de
capteurs \abr{PV} est plus important sur Strasbourg.

Les résultats montrent aussi que le nombre de capteur \abr{PV} est inversement
proportionnel à la performance du \abr{SSC}. Lorsque elle augmente, la production
photovoltaïque diminue et la production solaire thermique augmente afin de respecter la
contrainte de bilan positif. Ainsi sur Strasbourg, augmenter le nombre de capteur
thermiques de $5$ (\SI{11.6}{m^{2}}) permet de réduire de $8$ (\SI{13.1}{m^{2}}) le nombre de capteurs \abr{PV} nécessaires
(\figref{fig:fsav_pv_stras}). Sur Bordeaux ajouter $3$ (\SI{7.0}{m^{2}}) capteurs thermique permet
de réduire de $5$ (\SI{8.2}{m^{2}}) le nombre de capteurs \abr{PV} (\figref{fig:fsav_pv_bor}). Il
est de plus important de noter que la production des capteurs solaires thermiques est en adéquation
avec la demande alors que le bilan production / consommation est réalisée à l’échelle annuelle
pour le \abr{PV}. Il est aussi mis en évidence que les solutions encadrées sont celles qui
minimisent le nombre de capteurs \abr{PV} sur la toiture. Contrairement aux solutions en
dehors du cadre, elles ne couvrent pas l’ensemble de l’intervalle défini par la contrainte
de bilan à énergie positive. Il n’existe ainsi pas de solution minimisant au maximum la
$Conso_{tot}$ (\SI{-304}{kWh_{ef}}) et en même temps la surface minimale de capteurs
\abr{PV}.

\begin{figure}
% perf_th_vs_nb_pv
    \centering
    \begin{subfigure}[b]{0.48\textwidth}
        \centering
        \includegraphics[width=\textwidth]{Ressources/Images/EtudeDeCas/fsavext_pv_bor.pdf}
        \caption{}
        \label{fig:fsav_pv_bor}
    \end{subfigure}
    \quad
    \begin{subfigure}[b]{0.48\textwidth}
        \centering
        \includegraphics[width=\textwidth]{Ressources/Images/EtudeDeCas/fsavext_pv_stras.pdf}
        \caption{}
        \label{fig:fsav_pv_stras}
    \end{subfigure}
    \caption[Évolution de la performance globale du \abr{SSC} en fonction du nombre de capteurs \abr{PV}]
             {Évolution de la performance globale du \abr{SSC} en fonction du nombre de capteurs \abr{PV}.}
    \label{fig:fsav_pv_bor_stras}
\end{figure}

Les observations faites sur la variation du nombre de capteurs \abr{PV}
mettent en évidence que l’algorithme a bien exploré et trouvé des solutions sur
l’ensemble de l’espace des objectifs. De plus, augmenter la surface de capteurs
solaires thermiques est plus intéressant que augmenter la surface de capteurs \abr{PV}
en particulier sur Strasbourg. La partie suivante s’intéresse alors plus en détail
à l’impact des capteurs solaires thermiques sur la performance du \abr{SSC}.
% subsection influence_des_capteurs_photovoltaiques (end)

% % ------------------------------------------------------------------------------
% \subsection{Adéquation entre besoins et production} % (fold)
% \label{sub:adequation_entre_besoins_et_production}
% \iunsure{Si j’ai le temps.}
% Il est de plus important de rappeler que la production des capteurs \abr{PV} est estimée
% sur l’année (bilan importation / exportation, \ref{ssub:la_methodologie_de_calcul}) et ne
% tient donc pas compte de l’adéquation entre les besoins et la demande comme illustré sur
% Bordeaux avec $11$ et $16$ capteurs \abr{PV}.
% (\figref{fig:evolution_usages_pv}). Ainsi une forte production durant la période estivale
% compense une faible production durant la période hivernale. À l’opposée, l’économie
% engendrée par le \abr{SSC} est calculée par rapport à une solution de référence et tient
% donc implicitement compte de l’adéquation entre besoins et demande (bilan charge /
% production). C’est d’ailleurs pour cette raison que l’indicateur $F_{sav}$ a été préféré
% au taux de couverture ($F_{sol}$) qui permet de tenir compte de l’adéquation entre besoins
% et demande \textbf{uniquement} si il est calculé sur une base mensuelle.

% \begin{figure}
%     \centering
%     \ftodo{Ajouter évolution apports PV sur l’année vs besoins spécifiques}
%     % \includegraphics[width=\textwidth]{Ressources/Images/EtudeDeCas/Strasbourg_front.pdf}
%     \caption[None]
%              {None.}
%     \label{fig:evolution_usages_pv}
% \end{figure}


% Afin d’évaluer les solutions du front de Pareto obtenu en tenant mieux compte de
% l’adéquation entre besoins et demande, un nouvel indicateur, $F_{sol.\, tot}$ est construit
% en se basant sur le taux de couverture \eqref{eq:f_sol_mensuel_total}. Il permet
% de réaliser un bilan mensuel du rapport entre la part solaire utile et la demande en
% énergie nécessaire pour couvrir l’ensemble des besoins de la maisons, usages spécifiques,
% chauffage, et production d’\abr{ECS}. Étant réaliser à une fréquence mensuelle, l’indicateur
% tient compte en parti de l’adéquation avec le réseau, à la fois pour le thermique et
% pour le photovoltaïque.

% \begin{subequations}\label{eq:f_sol_mensuel_total}
%     \begin{align}
%         F_{sol,\, tot} &= \frac{\sum_{i=1}^{12}\left(Gain_{TH}^{i} + Gain_{PV}^{i} \right)}{Conso_{usages} + Conso_{app}} \\
%         \shortintertext{avec~:}
%         Gain_{TH}^{i} &= \min \left(Prod_{TH}^{i},\ Conso_{app} \right) \\
%         Gain_{PV}^{i} &= \min \left(Prod_{PV}^{i},\ Conso_{usages}^{i} + \max\left(0,\ \left[Conso_{app}^{i} - Prod_{TH}^{i} \right]\right) \right)
%     \end{align}
% \end{subequations}

% Le bâtiment et ses équipements est ainsi évaluée en traçant l’évolution de l’indicateur $F_{sol,\, tot}$,
% en fonction du nombre de capteurs \abr{PV} installés. Pour chaque variation du nombre de capteurs \abr{PV},
% il est retenu uniquement la solution ayant la meilleure performance globale sur le \abr{SSC}, la solution
% maximisant $F_{sav,\,ext}$.

% \begin{figure}
%     \centering
%     \ftodo{Ajouter F sol tot en fonction du nombre de capteurs PV}
%     % \includegraphics[width=\textwidth]{Ressources/Images/EtudeDeCas/Strasbourg_front.pdf}
%     \caption[None]
%              {None.}
%     \label{fig:fsoltot_vs_pvnumber}
% \end{figure}
% % subsection adequation_entre_besoins_et_production (end)



% ------------------------------------------------------------------------------
\subsection{Influence des capteurs solaires thermiques} % (fold)
\label{sub:influence_des_capteurs_solaires_thermiques}
Le processus d’aide à la décision retenue implique dans un premier temps, la réalisation
d’une optimisation multi-objectifs, puis grâce à une approche coordonnées paramétriques,
le front est réduit de manière interactive afin de répondre aux critères du ou des
décideurs. Cependant, cette approche comme toutes approches impliquant une optimisation
contraint à retenir uniquement les meilleures solutions. De ce fait, elle ne permet pas
d’obtenir les solutions qui ne sont pas optimales au sens strict du terme mais qui ont
une performance équivalente, peut être pour des combinaisons de paramètres plus
intéressantes au regard d’objectifs non pris en compte dans l’optimisation. Ainsi afin de
pouvoir analyser l’influence du type de capteur sur la performance du système, il est
nécessaire de chercher à évaluer la performance maximale pour chaque capteurs suivant un
processus d’optimisation indépendant. Profitant d’un temps d’évaluation très faible grâce
aux modèles de substitutions, une approche itérative exploratoire est donc retenue.

L’optimisation a été réalisée dans un premier temps en considérant l’ensemble
des paramètres et montre une majorité écrasante de solutions avec un capteur
sous-vide \textit{SkyPro} (\figref{fig:occurence_type_capteur}). La question qui se pose
est donc de savoir si cette solution est retenue car elle apporte une amélioration de la
performance importante, ou bien si son apport est marginal. Dans le premier cas, il est
logique de favoriser ce type de capteur, mais dans le second cas, d’autres indicateurs
sont nécessaires pour définir quel est le choix le plus pertinent.

\begin{figure}
% count_collector_type
    \centering
    \begin{subfigure}[b]{0.45\textwidth}
        \centering
        \includegraphics[width=\textwidth]{Ressources/Images/EtudeDeCas/occurence_type_capteur_bor.pdf}
        \caption{}
        \label{fig:occurence_type_capteur_bor}
    \end{subfigure}
    \quad
    \begin{subfigure}[b]{0.45\textwidth}
        \centering
        \includegraphics[width=\textwidth]{Ressources/Images/EtudeDeCas/occurence_type_capteur_stras.pdf}
        \caption{}
        \label{fig:occurence_type_capteur_stras}
    \end{subfigure}
    \caption[Occurrences de chaque type de capteurs pour les solutions du front de Pareto]
             {Occurrences de chaque type de capteurs pour les solutions du front de Pareto
              en fonction du nombre installés pour Bordeaux (a) et Strasbourg (b).}
    \label{fig:occurence_type_capteur}
\end{figure}

Quatre optimisations successives pour le climat de Strasbourg sont donc réalisées
et les fronts sont regroupés dans un archive commune.
Cette archive ne considère pas de relation de dominance, l’ensemble des solutions sont
conservées. Les résultats (\tabref{tab:capteur_perf_variation}) sont cohérents avec les
proportions obtenues lorsque le type de capteur est une variable de décision. Le capteur
\textit{SkyPro} est en effet le plus performant devant le capteur \textit{IDMK} mais
l’écart de performance impacte que faiblement la $Conso_{app}$. Par contre pour le capteur \textit{EnergyECO},
l’écart de performance est plus important~: \SI{170}{kWh} sur la $Conso_{app}$.
Les valeurs entre parenthèses indiquent les maximums obtenus pour chaque objectif.
Comme observé précédemment, le meilleur taux d’économie est obtenu par un compromis
entre économie sur le chauffage et économie sur la production d’\abr{ECS}. La production d’\abr{ECS} étant
plus importante et prioritaire, les solutions favorisant le $F_{sav,\, ext}$ admettent toutes
un $F_{sav}^{ECS}$ important. De plus peut importe le type de capteur, la
meilleure performance est obtenue pour une consommation de référence similaire.

Comme explicité lors du choix des collecteurs, il apparaît que les capteurs plans récents
ont tous un rendement optique ($\eta_{0}$) important et de ce fait, le coefficient $a_{1}$ est
le principal responsable des différences de performance. Il est aussi nécessaire
de porter une attention particulière à la surface de référence considérée dans la
fiche technique des capteurs. En effet le capteur \textit{SkyPro} qui apporte la
meilleure performance considère le moins bon rendement mais celui-ci est calculé
pour sa surface totale qui est plus importante que la surface d’entrée utilisée
par les autres capteurs~: \num{2.59} contre \num{2.32}.

\begin{table}
\centering
\caption[Performance maximale pouvant être obtenue pour les différents capteurs solaires sur Strasbourg]
         {Évolution de la performance maximale obtenue sur Strasbourg pour l’indicateur
          $F_{sav,\, ext}$ en fonction du type de capteur considéré. La valeur maximale
          de chaque objectif considéré est indiquée entre parenthèses.}
\label{tab:capteur_perf_variation}
\begin{tabular}{l c c c c c c c c c}
    \toprule
              & $F_{sav,\,ext}$ & $F_{sav}^{ECS}$ & $F_{sav}^{CH}$ & &  $Conso_{app}$ & & $Conso_{ref}$ & $Conso_{ref}^{ECS}$ & $Conso_{ref}^{CH}$ \\
    \midrule
    SkyPro    & \num{74} & \num{86}(\num{89}) & \num{54} (\num{62}) & & \num{1133} & & \num{4200} & \num{2866} & \num{1334} \\
    IDMK      & \num{72} & \num{85}(\num{87}) & \num{47} (\num{60}) & & \num{1166} & & \num{4236} & \num{2866} & \num{1360} \\
    CPCStar   & \num{71} & \num{84}(\num{86}) & \num{46} (\num{57}) & & \num{1208} & & \num{4200} & \num{2866} & \num{1334} \\
    EnergyEco & \num{69} & \num{81}(\num{84}) & \num{51} (\num{55}) & & \num{1301} & & \num{4220} & \num{2866} & \num{1354} \\
    \bottomrule
\end{tabular}
\end{table}


\begin{figure}
% type_collector_evolution
    \centering
    \includegraphics[width=\textwidth]{Ressources/Images/EtudeDeCas/capteurs_type_perf_tampon.pdf}
    \includegraphics[width=\textwidth]{Ressources/Images/EtudeDeCas/capteurs_type_perf_ecs.pdf}
    \caption[Évolution de la performance du \abr{SSC} en fonction du type de capteurs sur Strasbourg]
             {Évolution de la performance du \abr{SSC} sur Strasbourg pour l’indicateur $F_{sav,\,ext}$
              en fonction du nombre de capteurs (chaque colonne), et du type de capteur
              (chaque lignes). La couleur indique respectivement, la taille
              du ballon tampon (figure du haut) et du ballon d’\abr{ECS} (figure du bas).}
    \label{fig:evolution_type_capteurs}
\end{figure}

Afin de mieux comprendre le comportement du \abr{SSC} en fonction du type de capteurs, il
est observé simultanément l’évolution de la performance globale ($F_{sav,\, ext}$) en
fonction du nombre de capteur solaires thermiques installés et des tailles respectives
pour le ballon tampon et le ballon \abr{ECS} (\figref{fig:evolution_type_capteurs}).
Chaque point représente une solution optimale et chaque ligne correspond donc au front
optimal obtenu pour chaque type de capteurs. Du bruit sur la verticale est ajouté afin de
permettre d’identifier chaque point dans chaque groupe de solutions formées en fonction du
nombre de capteurs solaires thermiques. Chaque type de capteurs étant une information
qualitative, la variation de la hauteur des points n’apporte pas d’informations
supplémentaires mais offre une meilleure lisibilité en permettant d’identifier les sous-
groupes formés par les différentes couleurs. Une graine aléatoire identique est utilisée
pour les deux graphiques, la position des solutions est donc préservée d’une figure à
l’autre. Finalement, puisque il existe de nombreuses variations possibles pour la taille
des ballons, le choix a été fait de les regrouper en fonction de leurs tailles dans trois
sous-groupes. Ainsi un ballon dont la taille varie de \SIrange{100}{200}{\litre} inclut se
trouvent dans le sous-groupe $100-200$\dots\ Afin de simplifier la description des résultats,
un volume allant de \SIrange{100}{200}{\litre} sera aussi référencé comme étant
\emph{faible ou petit} alors que un volume supérieur à \SI{300}{\litre} sera considéré
comme \emph{important ou gros}.

Les résultats montrent qu’il existe pour Strasbourg, des solutions optimales pour tous les volumes de
ballon tampon, indépendamment du nombre ou du type de capteurs solaires thermiques. De
plus un $F_{sav,\, ext}$ important est obtenu pour des petits comme des gros volumes de
ballon tampon. Il semble donc possible de retenir un volume de ballon tampon faible sans
dégrader la performance globale du \abr{SSC}. Par contre un volume de ballon \abr{ECS}
important permet d’obtenir la meilleure performance globale. Si on considère moins de $5$
capteurs par contre le volume du ballon d’\abr{ECS} est moins impactant. Il est aussi noté
que même si des volumes importants sont favorisés, il existe des solutions très
performantes avec des volumes peu important d’\abr{ECS}. Ces solutions considèrent
cependant un volume de ballon tampon plus important. Il semble donc qu’il soit possible de
favoriser soit le chauffage, soit la production d’\abr{ECS} pour obtenir un taux
d’économie important. Ces résultats confirment les observations précédentes, un compromis
est nécessaire entre la performance sur le chauffage et sur la production d’\abr{ECS}.

Pour chaque nombre de capteurs installés, l’écart d’économie entre la valeur minimale et
maximale augmente. Ainsi lorsque uniquement $2$ capteurs sont installés, $F_{sav,\, ext}$
varie de \SIrange{38}{46}{\percent} alors que pour $7$ capteurs la variation est de
\SIrange{57}{74}{\percent}, soit une augmentation de \SI{9}{\percent}. En effet à mesure
que le nombre de capteurs augmente, la différence de performance entre les capteurs est
accentuée. Ainsi l’écart entre la performance maximale pour le capteur \textit{SkyPro} et
le capteur \textit{EnergyEco} est de \SI{5}{\percent} lorsque $7$ capteurs sont installées
contre \SI{2.5}{\percent} pour seulement $2$.


\paragraph{} % (fold)
En considérant l’ensemble des solutions formées par les fronts optimaux pour chaque
type de capteurs, il est possible d’évaluer l’influence du nombre de capteurs
sur Strasbourg (\figref{fig:front_pareto_nb_th}). Chaque couleur représente un
nombre de capteur différent et chaque marqueurs permet de mieux identifier les
groupes de solutions. Comme pour les projections $2D$ du front précédentes, la diagonale
renseigne sur les proportions et la répartition de chaque nombre de capteurs.

Des solutions existent pour toutes les variations possibles du nombre de capteurs
attestant d’une bonne représentativité de l’espace des objectifs. Il est aussi noté la
variation de la production des capteurs \abr{PV} en fonction du nombre de capteurs
solaires installées. En effet pour un même nombre de capteurs \abr{PV}, la production
varie faiblement (\SI{\sim 100}{kWh}) car la toiture est partagée. La performance du
\abr{SSC} augmente à mesure que l’on augmente le nombre de capteurs et on note clairement
un front pour chaque nombre installés. Par contre à mesure que l’on augmente le nombre de
capteurs, la différence de performance avec le nombre précédent est moindre. En effet, les
premiers ajouts de capteurs améliorent plus la qualité des solutions. Avec uniquement $3$
capteurs, seule la production d’\abr{ECS} est couverte. Il est nécessaire d’installer au
minimum $4$ pour que le \abr{SSC} récupère assez d’énergie solaire pour améliorer le
$F_{sav}^{CH}$. À l’aide des histogrammes, il est aussi identifié que pour une même
surface de capteur il est possible d’obtenir un $F_{sav}^{ECS}$ et un $F_{sav}^{CH}$ très
variable. Sur le chauffage pour un même nombre de capteur le $F_{sav}^{ECS}$ varie de
\SI{50}{\percent} et sur le $F_{sav}^{CH}$ de \SI{60}{\percent}.

\begin{figure}
% pair_grid_plot
    \centering
    \includegraphics[width=\textwidth]{Ressources/Images/EtudeDeCas/Strasbourg_front_nb_capteurs.pdf}
    \caption[Fronts de Pareto pour les différents types de capteurs sur Strasbourg]
             {Projection $2D$ des fronts de Pareto pour les différents types de capteurs sur Strasbourg.
             La couleur permet d’identifier le nombre de capteurs solaires thermiques installés.}
    \label{fig:front_pareto_nb_th}
\end{figure}


La \figref{fig:perf_nb_th} permet de mieux apprécier l’évolution de la
population sur Strasbourg en fonction du nombre de capteurs, pour respectivement, $F_{sav}^{CH}$,
$F_{sav}^{ECS}$, et $F_{sav,\, ext}$. Ajouter un capteur permet d’améliorer le
$F_{sav}^{CH}$ de \SIrange{8}{10}{\percent} jusqu’à $6$ capteurs. Ajouter le $7^{ème}$ ne
semble plus aussi efficace et permet de l’améliorer uniquement de \SI{2.2}{\percent}. Sur
l’indicateur $F_{sav}^{ECS}$ l’amélioration est similaire de $2$ à $5$ capteurs thermiques
(\SIrange{6}{7}{\percent}) alors que au delà, l’impact est moindre (\SI{3}{\percent}).
Dans les deux cas la meilleure progression est enregistrée lorsque on passe de $3$ à $4$
capteurs solaires thermiques sur le chauffage comme la production d’\abr{ECS} et marque
aussi par extension la meilleure progression du $F_{sav,\, ext}$. L’analyse de la
population en fonction de $F_{sav,\, ext}$ permet de montrer que le $7^{ème}$ capteur
n’apporte qu’une faible amélioration (\SI{3}{\percent}) par rapport aux autres variations
(\SIrange{5}{10}{\percent}). Cependant la distribution des solutions est plus importante
pour $6$ capteurs que pour $7$. Il existe ainsi moins de combinaisons permettant au
\abr{SSC} d’atteindre une très bonne performance. Sur le $F_{sav}^{ECS}$ uniquement, la
dispersion augmente jusqu’à $5$ capteurs puis se réduit alors qu’elle progresse sur le
chauffage jusqu’à $6$ capteurs. Ainsi sur Strasbourg il est pertinent de considérer
uniquement $6$ capteurs ce qui permet de réduire la consommation de l’appoint à
\SI{1251}{kWh} contre \SI{4222}{kWh} sans solaire. Sur Bordeaux, installer $4$ capteurs
est suffisant et permet de réduire la consommation de l’appoint à \SI{380}{kWh} pour une
consommation de référence de \SI{2856}{kWh}. Avec $3$ capteurs la consommation
augmenterait de \SI{180}{kWh}, alors que avec un $5^{ème}$ seulement \SI{100}{kWh} serait
économisé.

\begin{figure}
% nb_collector_evolution
    \centering
    \includegraphics[width=\textwidth]{Ressources/Images/EtudeDeCas/evolution_perf_nb_capteurs.pdf}
    \caption[Performance du \abr{SSC} sur Strasbourg en fonction du nombre de capteurs solaires thermiques]
             {Évolution de la performance du \abr{SSC} sur Strasbourg en fonction du nombre de capteurs solaires thermiques.
              Le jeu de couleur est identique à celui utilisé dans \figref{fig:front_pareto_nb_th}.}
    \label{fig:perf_nb_th}
\end{figure}


Les résultats obtenus permettent donc de lever les conclusions suivantes~:
\begin{itemize}
    \item Le \abr{SSC} obtient une bonne performance pour les différents types de
          capteurs.
    \item Il existe bien un écart de performance lorsque différents types de capteurs
          sont considérés mais il n’est pas déterminant.
    \item Le capteur sous-vide \textit{SkyPro} est le plus performant mais le capteurs
          plan vitré \textit{IDMK} obtient des performances similaires.
    \item Le capteur plan \textit{EnergyEco} est le moins performant.
    \item Le \abr{SSC} ne permet pas de maximiser la performance sur le chauffage
          et sur la production d’\abr{ECS}, il est nécessaire de choisir sa préférence
          notamment en favorisant un volume important soit sur le tampon, soit sur le
          ballon d’\abr{ECS}.
    \item Au delà de $4$ capteurs pour Bordeaux et $6$ capteurs pour Strasbourg,
          le gain énergétique est moins intéressant.
\end{itemize}

Au regard des résultats, il apparaît donc intéressant de chercher à évaluer plus en détail
l’impact du volume des ballons sur la performance du \abr{SSC}. Cette analyse est donc
réalisée dans la partie qui suit.
% subsection influence_des_capteurs_solaires_thermiques (end)


% ------------------------------------------------------------------------------
\subsection{Influence du volume des ballons} % (fold)
\label{sub:influence_du_volume_des_ballons}
Au regard des résultats obtenus en faisant varier la taille des ballons
(\figref{fig:occurence_taille_ballons}), il apparaît clairement que un volume important
pour le ballon d’\abr{ECS} est plus souvent retenue sur Strasbourg alors que pour le
ballon tampon aucune préférence n’est observée. À l’inverse sur Bordeaux c’est un ballon
tampon important qui est majoritairement retenu alors que la taille du ballon \abr{ECS}
est plus variable même si un volume supérieur à \SI{300}{\litre} est majoritairement
retenue. Un volume important pour le ballon tampon permet d’améliorer le $F_{sav}^{CH}$
mais n’impacte que faiblement le $F_{sav,\, ext}$. Les consommations nécessaires pour
couvrir les besoins en chauffage sur Bordeaux sont en effet faibles, ainsi améliorer
l’efficacité du \abr{SSC} sur le chauffage n’apporte pas de réduction importante sur la
$Conso_{app}$. Il semble donc pertinent sur Bordeaux de considérer un volume de tampon
inférieur à \SI{100}{\litre} avec $3$ ou $4$ capteurs (annexe, \figref{fig:tanks_variations_bordeaux}).
En effet la $Conso_{ref}$ varient seulement de \SIrange{228}{620}{kWh} et les solutions
optimales observée montre que la $Conso_{app}$ elle varie de \SI{48}{470}{kWh}. Ainsi le
reste de cette section s’intéresse uniquement aux variations observée pour le climat de
Strasbourg où le potentiel du solaire thermique pour le chauffage est plus important.

\begin{figure}
% count_collector_type
    \centering
    \includegraphics[width=\textwidth]{Ressources/Images/EtudeDeCas/tanks_count.pdf}
    \caption[Occurrences de chaque sous-groupe de taille de ballon pour les solutions du front de Pareto]
             {Occurrences de chaque sous-groupe de taille de ballon pour les solutions du front de Pareto
              en fonction du nombre installés pour Bordeaux (vert) et Strasbourg (bleu).}
    \label{fig:occurence_taille_ballons}
\end{figure}

Afin d’évaluer l’impact de la taille des ballons, il est nécessaire comme pour le type
de capteurs, de réaliser une nouvelle optimisation en autorisant uniquement des volumes allant
de \SIrange{100}{200}{\litre}. L’ensemble des solutions optimales des deux front sont ensuite
ajoutées dans une archive commune sans relation de dominance. L’archive contient ainsi les meilleures
solutions avec de faibles volumes, et les meilleures solutions en laissant l’algorithme décider
du volume le plus optimal pour chaque ballon.
Une fois obtenu, le front cumulé est évalué pour les indicateurs principaux du \abr{SSC}
en fonction du nombre de capteurs, du volume du ballon tampon, et du volume du ballon d\abr{ECS}
(\figref{fig:tanks_variations_locales_strasbourg}, annexe \figref{fig:tanks_variations_strasbourg}).
Si on considère séparément le chauffage et la production d’\abr{ECS},
il apparaît que comme sur Bordeaux, le volume du ballon tampon est uniquement influent
sur le $F_{sav}^{CH}$ lorsque le nombre de capteur solaires thermiques est faible ($2$).
Il est aussi clairement mis en évidence que opter pour des volumes faibles pour le
ballon tampon et le ballon d’\abr{ECS} permet d’obtenir des performances similaires.
Par exemple pour $7$ capteurs, la \abr{SSC} permet au maximum, de couvrir
\SI{62}{\percent}, et \SI{89}{\percent} des consommations pour respectivement le chauffage et l’\abr{ECS}.
En limitant la variation du volume des ballons (\SIrange{100}{200}{\litre}) il est possible d’obtenir
une performance similaire avec respectivement
\SI{56}{\percent} et \SI{86}{\percent}. Comme le montre la figure, cette conclusion est valable
peu importe le nombre de capteurs solaires thermiques installés. Si maintenant, on considère
la performance globale, alors l’écart sur la $Conso_{app}$ est de seulement \SI{70}{kWh}.
Le front uniquement composé des solutions obtenues pour de faibles volumes est disponible
en annexe (\figref{fig:tanks_small_variations_strasbourg}) où le volume exact est décrit par
la couleur du point. Les résultats montrent qu’un volume minimal sur le tampon (\SI{100}{\litre})
et un volume de \SIrange{150}{200}{\litre} pour le ballon d’\abr{ECS} est suffisant sur Strasbourg.
Utiliser un volume de ballon sanitaire plus petit ne permet pas en effet d’améliorer le
$F_{sol}^{CH}$ mais dégrade le $F_{sol}^{ECS}$.
Il est aussi noté que lorsque le nombre de capteurs est faible, il est plus intéressant de
considérer un volume faible pour les deux ballons. En effet, si il n’y a pas assez
d’énergie solaire, la température du ballon tampon reste trop basse pour pouvoir être
ensuite valorisée. Finalement, il est aussi observé que même pour une plage de variation
réduite, le \abr{SSC} est toujours capable de proposer des solutions maximisant soit le
$F_{sav}^{CH}$, soit le $F_{sav}^{ECS}$.

\begin{figure}
% volume_collector_factorplot
    \centering
    \includegraphics[width=\textwidth]{Ressources/Images/EtudeDeCas/strasbourg_tanks_variations_local.pdf}
    \caption[Performance du \abr{SSC} sur Strasbourg en fonction du volume des ballons]
             {Évolution de la performance du \abr{SSC} sur Strasbourg pour les indicateurs $F_{sav}^{CH}$
              et $F_{sav}^{CH}$ en fonction du volume du ballon tampon (colonne) et
              du volume du ballon d’\abr{ECS} (couleur).}
    \label{fig:tanks_variations_locales_strasbourg}
\end{figure}

Concernant l’algorithme de contrôle, une large variabilité dans les valeurs des critères de décisions existe.
Pour améliorer le $F_{sav}^{CH}$, il semble plus intéressant de retenir un $\Delta
min_{capteur}$ et un $\Delta min_{tampon}$ faibles (\SIrange{0}{10}{\celsius}). Augmenter
le $\Delta min_{tampon}$ permet d’améliorer le $F_{sav}^{ECS}$ mais dégrade le
$F_{sav}^{CH}$ et inversement. Concernant le $\Delta min_{capteur}$, les meilleures
solutions sur l’indicateur $F_{sav}^{ECS}$ admettent un $\Delta min_{capteur}$ important.
Ce différentiel important permet diriger une part plus importante du solaire vers la
production d’\abr{ECS} mais limite grandement la qualité de couverture du solaire sur le
chauffage. En effet afin de renforcer l’économie sur le chauffage grâce au solaire, il
apparaît que un $\Delta min_{capteur}$ faible est plus intéressant. Au niveau de la
performance globale, du fait des compromis existants entre chauffage et production
d’\abr{ECS}, de nombreuses combinaisons sont possibles. Une tendance se dégage
cependant pour le paramètre $\Delta T_{sol}$.
Obtenir la meilleure performance sur le chauffage nécessite un $\Delta T_{sol}$
très faible (\SIrange{5}{8}{\celsius}) alors que pour maximiser la production
d’\abr{ECS}, il est plus opportun de considérer un $\Delta T_{sol}$ de
\SIrange{13}{15}{\celsius}. Le meilleur compromis entre les deux permet de faire la
meilleure économie sur l’ensemble des consommations (\SIrange{7}{12}{\celsius}).
Il est important de noter que ces conclusions sont extraites de l’observation des
tendances et qu’il existe des solutions performantes en dehors des bornes proposées.

Concernant la position de l’échangeur dans le ballon sanitaire, il n’existe pas de
position idéale. Une position basse tend à favoriser le $F_{sav}^{ECS}$ (\numrange{0.8}{1})
mais il reste possible d’obtenir un $F_{sav}^{ECS}$ important (\SI{\geq 85}{\percent})
lorsque la position est plus haute (\numrange{1}{1.2}). Par contre si on considère uniquement
des ballons de petites tailles, alors il est toujours plus intéressant de favoriser une
position basse (\numrange{0.8}{1}) que ce soit pour le $F_{sav}^{ECS}$ ou pour le $F_{sav}^{CH}$.
Les meilleures solutions sur l’indicateur $F_{sav}^{CH}$ sont en effet pour une position
de l’échangeur autour de \num{0.9}. Il n’est pas identifié de relations entre la position de l’échangeur
et les variables de décisions sur l’algorithme ($\Delta min_{capteur}$, $\Delta min_{tampon}$,
$\Delta T_{sol}$).


Ainsi, au regard des résultats, il est plus intéressant de considérer des ballons
de faible volume~: pour Strasbourg, \SI{100}{\litre} pour le tampon et \SIrange{150}{200}{\litre} pour le ballon d’\abr{ECS}.
Ces résultats sont encouragent et invitent à repenser la conception d’un \abr{SSC} pour
les maisons à énergie positives.
Un volume important présente en effet de nombreux inconvénients qui sont souvent un frein
à l’adoption d’un tel système~:
\begin{itemize}
    \item Le ballon est l’équipement le plus onéreux d’un \abr{SSC}.
    \item Un volume important réduit l’espace de vie disponible pour les occupants.
    \item Le volume étant plus important, les déperditions sont plus importantes~:
    \begin{itemize}
        \item Si dans un local chauffée, ces déperditions apportent des charges supplémentaires
              en période estivale.
        \item Si installée dans un local non chauffée, les pertes sont plus importantes
              durant la période hivernale.
    \end{itemize}
\end{itemize}

Il a été montré que le \abr{SSC} permet d’obtenir une très bonne performance sur les deux
climats mais il est toujours nécessaire de faire une étude plus poussée afin de comprendre
les corrélations existantes entre l’enveloppe et le \abr{SSC}. La section suivante
s’intéresse à cette question.
% subsection influence_du_volume_des_ballons (end)



% ------------------------------------------------------------------------------
\subsection{Corrélation entre performance de l’enveloppe et \abr{SSC}} % (fold)
\label{sub:correlation_entre_performance_de_l_enveloppe_et_ssc}
Cette section s’intéresse aux corrélations existantes entre la performance de l’enveloppe
et la performance du \abr{SSC}. Dans un premier temps l’influence de la qualité de
l’enveloppe est discuté globalement en fonction des indicateurs de performances relatifs,
puis il est exploré les relations pouvant exister entre la $Conso_{ref}^{CH}$ (qui traduit
la performance de l’enveloppe) et la performance du \abr{SSC}.

Comme le montre \figref{fig:pareto_enveloppe_strasbourg}, le meilleur taux d’économie est
obtenue lorsque l’isolation du bâtiment est importante. La performance de l’enveloppe
étant caractérisé par un coefficient d’échange global pour le bâtiment ($U_{bat}$) calculé
en tenant compte de l’ensemble des surfaces déperditives. Pour le chauffage, il apparaît
possible d’obtenir un taux d’économie important sans pour autant fortement isoler le
bâtiment Cependant, les meilleures solutions assument une isolation importante.
Maintenant si on regarde l’évolution couplée du $F_{sav}^{ECS}$ et du $F_{sav,\, ext}$, la
tendance n’est pas la même lorsque $F_{sav,\, ext} < 66$ et lorsque $F_{sav,\, ext} > 66$.
Dans le premier cas, une enveloppe peu performante permet d’obtenir le meilleur $F_{sav,\,ext}$~:
il n’y a pas assez d’énergie solaire pour permettre d’optimiser à la fois le
chauffage et la production d’\abr{ECS} (qui est prioritaire). Dans le second cas, une
enveloppe performante améliore le $F_{sav,\, ext}$. Ces solutions comportent $6$ ou $7$
capteurs solaires thermiques et il est alors possible de couvrir une part importante sur
la production d’\abr{ECS} tout en ayant encore assez d’énergie pour couvrir les besoins sur
le chauffage. Le solaire ne couvrant pas l’ensemble des besoins en
\abr{ECS}, réduire les besoins de chauffage permet de réduire la part solaire attribuée au
chauffage, et donc de couvrir plus efficacement les besoins en \abr{ECS}.
Il est ainsi mis en évidence l’existence de nombreux compromis sur la qualité de l’enveloppe
permettant tous d’obtenir un taux d’économie important favorisant tantôt le chauffage
tantôt la production d’\abr{ECS}.

\begin{figure}
    \centering
    \includegraphics[width=\textwidth]{Ressources/Images/EtudeDeCas/Strasbourg_front_enveloppe_colormap.pdf}
    \caption[Évolution de la performance du \abr{SSC} en fonction de la qualité de l’enveloppe]
             {Évolution de la performance du \abr{SSC} sur Strasbourg pour les indicateurs $F_{sav}^{CH}$
              et $F_{sav}^{CH}$, et $F_{sav,\,ext}$ en fonction de la performance de l’enveloppe (couleur).}
    \label{fig:pareto_enveloppe_strasbourg}
\end{figure}


% Afin de pouvoir évaluer le front optimal, la qualité de l’enveloppe est définie
% de manière qualitative (\figref{fig:pareto_enveloppe_strasbourg}). Il est retenu
% trois valeurs~: \enquote{faible}, \enquote{forte}, et \enquote{moyenne}. Pour obtenir
% ces valeurs, un coefficient de transfert thermique pour l’ensemble du bâtiment est calculé
% en fonction de la performance des murs, du plancher, du plafond et des vitrages (en tenant
% compte de la variation de surface). Les résultats sont ensuite répartis dans les $3$ groupes~:
% \enquote{faible} regroupe les solutions où l’isolation est peu importante et \enquote{forte}
% les solutions ayant la meilleure enveloppe.
% Les résultats montrent que majoritairement, il est préférable d’isoler fortement le bâtiment
% afin d’améliorer la performance du bâtiment. L’écart de consommation entre la solution la plus performante
% pour une enveloppe \enquote{moyenne} et pour une enveloppe \enquote{forte} est de \SI{3}{kWh\per m^{2}}
% Cet écart est relativement peu important au regard de la consommation de référence (\SI{\sim 41}{kWh\per m^{2}})
% et le surcoût dû à une sur-isolation n’est certainement profitable au niveau économique.
% \itodo{Continue}


% \begin{figure}
%     \centering
%     \includegraphics[width=\textwidth]{Ressources/Images/EtudeDeCas/Strasbourg_front_enveloppe.pdf}
%     \caption[Évolution de la performance du \abr{SSC} en fonction de la qualité de l’enveloppe]
%              {Évolution de la performance du \abr{SSC} sur Strasbourg pour les indicateurs $F_{sav}^{CH}$
%               et $F_{sav}^{CH}$, et $F_{sav,\,ext}$ en fonction de la performance de l’enveloppe (couleurs).}
%     \label{fig:pareto_enveloppe_strasbourg}
% \end{figure}


Afin de mieux caractériser les corrélations existantes, $3$ indicateurs sont introduits,
$p$, la probabilité qu’une corrélation existe entre deux variables mais aussi, les
coefficients de corrélation de \textit{Pearson} ($spearmanr$) et de \textit{Spearman}
($pearsonr$). L’indicateur $pearsonr$ permet d’évaluer le niveau de corrélation linéaire
existant entre deux variables et $spearmanr$ si il existe une relation monotone. La valeur
est comprise entre $-1$ et $1$, où les extrêmes indiquent une corrélation totale et $0$
qu’il n’en existe pas. Le signe indique lui le sens d’évolution de la corrélation~: une
valeur positive indique que l’augmentation d’une des variables implique aussi
l’augmentation de la seconde. Le second paramètre, $p$, indique lui la probabilité que le
système ne soit pas corrélé du tout~: $0$ indique que le système est corrélé et $1$ qu’il
ne l’est pas.

L’analyse est uniquement réalisée sur Strasbourg car les besoins de chauffage sur Bordeaux
sont très faibles et l’amélioration de l’indicateur $F_{sav}^{CH}$ n’a que peu d’impact sur
la $Conso_{app}$. La \figref{fig:conso_ref_vs_app} montre qu’il existe une corrélation linéaire forte entre la
consommation du système de référence ($Conso_{ref}^{CH}$) et la consommation de l’appoint sur
le \abr{SSC} ($Conso_{app}^{CH}$). Ainsi améliorer l’isolation du bâtiment semble permettre
de réduire la $Conso_{app}^{CH}$. Maintenant si on cherche à évaluer la performance du
\abr{SSC} de manière globale, il est utile d’évaluer la relation existante entre~:
le taux d’économie en tenant compte de la consommation des pompes ($F_{sav,\,ext}$), et
la qualité de l’enveloppe. Au regard des résultats (\figref{fig:conso_ref_vs_f_sav}),
il apparaît que la corrélation n’est pas linéaire mais l’évolution peut être définie
comme monotone ($spearmanr = \num{-0.63}$). Le système est donc en moyenne plus performant
lorsque la qualité de l’enveloppe augmente. Ainsi, les solutions optimales
considèrent majoritairement une enveloppe performante qui permet à la fois de réduire
la $Conso_{app}^{CH}$ et augmenter le $F_{sav,\,ext}$. En améliorant l’enveloppe, les besoins
sur le chauffage sont réduits et donc une part solaire plus importante peut être
utilisée pour couvrir les besoins en \abr{ECS}.

Il serait alors tentant de conclure que améliorer l’isolation au maximum est
toujours une bonne idée, pourtant la \figref{fig:conso_ref_vs_diffconso} permet
de modérer cette conclusion. Il est en effet clairement mis en évidence
qu’il n’existe pas de corrélation entre l’économie sur la consommation
($Conso_{ref}^{CH} - Conso_{app}^{CH}$) et la $Conso_{ref}^{CH}$. Ainsi sur le front de Pareto, il existe des solutions diverses
et variées pour la même performance d’enveloppe. De plus il existe un saut (\SI{110}{kWh})
net lorsque l’économie est inférieur à \SI{1550}{kWh}. Renforcer l’isolation
au delà impacte ainsi négativement la performance du \abr{SSC} au regard de l’économie
absolue. Il semble donc qu’il existe un compromis entre la performance de l’enveloppe,
la performance du \abr{SSC} et l’économie absolue sur l’appoint .
Cependant, retenir $Conso_{ref}^{CH} - Conso_{app}^{CH}$ comme objectif
pour l’optimisation n’est pas non plus judicieux. Formulé ainsi, le front de Pareto
comportera des solutions favorisant une enveloppe peu performante afin que le gain
absolue potentiel soit plus important. On obtiendra donc des solutions
maximisant $F_{sav}^{ECS}$ (car $Conso_{ref}^{ECS}$ est fixe) mais aucunes ne
favorisant $F_{sav}^{CH}$. Il est en effet plus intéressant de dégrader la performance
de l’enveloppe si on cherche à augmente l’économie d’énergie en valeur absolue.

En conclusion, le choix du niveau d’isolation doit bien être déterminé parallèlement au choix des
équipements du \abr{SSC}. Il est de plus nécessaire de considérer un autre indicateur afin
d’arbitrer ce choix. Ce nouvel indicateur peut par exemple être le coût de l’installation,
ou le temps de retour sur investissement. Cependant comme le montre la littérature,
l’évaluation du coût est délicat car de nombreux facteurs doivent être pris en compte. La
condition essentielle pour estimer correctement le coût de l’installation est de
travailler en collaboration avec des industriels pour avoir une idée réaliste du coût des
équipements, et avec des installateurs pour obtenir une indication quand au coût
d’installation, en particulier le coût de la main d’œuvre. Celle-ci est en effet difficile
à estimer et est souvent exprimée de manière approximative comme un simple ratio horaire.
De même sur l’enveloppe, une intervention d’un expert est nécessaire afin d’évaluer le
coût réel. Par exemple, il est souvent considéré pour l’isolant, un coût dépendant
uniquement de son épaisseur. Cette approximation ne permet cependant pas de considérer le
bâtiment dans son ensemble. En effet en fonction de la structure du bâtiment, des surcoûts
\emph{importants} peuvent survenir lorsque on passe d’une épaisseur à une autre de part
les contraintes de support, de hauteur\dots\ Finalement, afin de calculer un
temps de retour sur investissement, un nombre conséquent de facteur supplémentaires entre en jeu, le
taux d’intérêt, l’inflation, l’évolution du coût des énergies, la valeur ajoutée au
bâtiment ou valeur verte, la durée de vie des équipements, le coût de maintenance\dots\
Ainsi l’évaluation économique nécessite un large panel de connaissances
afin de fournir des résultats pertinents.
La difficulté inhérente à l’évaluation du coût est mis en avant par les résultats
disparates obtenus dans la littérature amenant à des temps de retour variant de entre moins de
\num{5} ans à plus de \num{100} ans.
Dans le cadre d’une \abr{MEPOS}, il est aussi possible de diriger son choix en tenant
compte des résultats d’une Analyse de Cycle de Vie (\abr{ACV}) ou bien du confort des occupants.
L’utilisation de modèle de substitution permet en effet de simuler très rapidement
un ensemble de solution et ne nécessite pas de connaissances sur le modèle, ni le
logiciel utilisé pour construire le modèle original. Ainsi de part sa modularité
et bien que non traité dans ces travaux, l’ajout de nouveaux objectifs est simplifié.

\begin{figure}
% conso_chauffage_joinplot
    \centering
    \begin{subfigure}[b]{0.313\textwidth}
        \centering
        \includegraphics[width=\textwidth]{Ressources/Images/EtudeDeCas/correlation/conso_ref_vs_app.pdf}
        \caption{}
        \label{fig:conso_ref_vs_app}
    \end{subfigure}
    \quad
    \begin{subfigure}[b]{0.313\textwidth}
        \centering
        \includegraphics[width=\textwidth]{Ressources/Images/EtudeDeCas/correlation/conso_ref_vs_f_sav.pdf}
        \caption{}
        \label{fig:conso_ref_vs_f_sav}
    \end{subfigure}
    \quad
    \begin{subfigure}[b]{0.313\textwidth}
        \centering
        \includegraphics[width=\textwidth]{Ressources/Images/EtudeDeCas/correlation/conso_ref_vs_diffconso.pdf}
        \caption{}
        \label{fig:conso_ref_vs_diffconso}
    \end{subfigure}
    \caption[Corrélation entre la qualité de l’enveloppe et la performance du \abr{SSC} sur Strasbourg]
             {Corrélations entre la qualité de l’enveloppe et la performance du \abr{SSC}
              sur Strasbourg en fonction des indicateurs suivants~: (a) $Conso_{app}^{CH}$, (b) $F_{sav,\, ext}$,
              (c) $Conso_{ref}^{CH} - Conso_{app}^{CH}$.}
    \label{fig:conso_ref_vs_app_f_sav}
\end{figure}
% subsection correlation_entre_performance_de_l_enveloppe_et_ssc (end)



% ..............................................................................
% ..............................................................................
\subsection{Exploration du front de Pareto} % (fold)
\label{sub:exploration_du_front_de_pareto}
Le processus d’aide à la décision interactif est la dernière étape de la méthodologie
développée dans le chapitre III et permet d’aider à l’identification et au choix de
solutions parmi celles du front de Pareto. Durant ce processus les objectifs ne sont pas
forcement les mêmes que durant l’optimisation. En effet l’avantage d’une optimisation
multi-objectifs réside dans sa capacité à explorer l’espace des objectifs afin de proposer
de multiples combinaisons dans l’espace de décision sans introduire en amont de
préférence. De cette manière, de nouvelles connaissances sur le problème peuvent émerger
comme ce fut le cas dans ces travaux. Par contre durant le processus d’aide à la décision,
les préférences du décideur doivent être introduites afin de réduire le nombre de
solutions optimales. Ainsi l’aide à la décision revient à explorer de manière simultanée
l’espace de décision et l’espace des objectifs. Cette section illustre l’utilisation du
logiciel \textit{Xdat}, un outil d’analyse par coordonnées parallèles qui permet comme
décrit dans le chapitre précédent de bouger des curseurs pour chaque critère afin de
réduire le nombre de solutions optimales. L’application de la méthode est illustrée
sur Strasbourg en considérant le front commun formé par les optimisations avec et sans
contraintes sur le volume des ballons.

Comme explicité ci-avant, les critères pouvant être retenus pour réaliser l’aide à la
décision couvrent les critères de décisions, les objectifs, ou encore de nouvelles
contraintes. Par exemple, il est possible de préférer certains matériaux afin de faire
travailler des professionnels de la région. Dans le cas d’un constructeur de maisons
individuelles, l’approche peut permettre d’aider  les clients à sélectionner une solution
parmi en catalogue de solutions (géométrie, qualité de l’enveloppe, type de système de chauffage\dots).
L’optimisation est réalisée en amont permettant au constructeur de proposée une palette de
solutions. Puis, grâce à une visualisation par coordonnées parallèles, le client participe
de manière active aux choix constructifs comme aux choix des systèmes, sans avoir besoin
d’une expertise importante. Le client devient acteur de son choix et évalue mieux
les contraintes et atouts de chaque proposition. Le constructeur lui, est mieux à même de
présenter et défendre ces offres. Il existe bien sûr de nombreuses autres applications où
l’approche par coordonnées parallèles est adaptée.

\paragraph{} % (fold)
Dans cette dernière partie, la sélection des solutions optimales par application de la
préférence de l’utilisateur est illustrée à travers \figref{fig:xdat_exemple} où chaque
ajout de préférence fabrique un nouveau cluster identifiable par une couleur spécifique.
Les solutions grises respectent aucunes des contraintes et les autres couleurs répondent à
un certain nombre des préférences comme explicité ci-après. Considérons le cas où le
client souhaite minimiser la place prise par l’installation dans sa maison. En réduisant
l’intervalle autorisée pour le volume des ballons à l’aide de leur curseurs respectifs, le
front peut être réduit aux solution ayant un volume inférieur à \SI{200}{\litre} (courbes
bleues). Maintenant, il souhaite faire des économies sur l’isolation tout en limitant
l’augmentation de la consommation électrique. Dans ce cas il est nécessaire de faire
varier les curseurs sur les paramètres caractérisant l’enveloppe comme par exemple la
résistance thermique des murs, puis, d’observer la performance des solutions restantes
(courbes roses). À ce stade, l’ensemble des solutions nécessitent une consommation
électrique de \SIrange{14}{17}{kWh\per\m^{2}\per an} et une variation d’économie de
\SI{300}{kWh \per an} est observée. Il est aussi noté que l’espace de décision est
toujours diverse en particulier sur la qualité de l’enveloppe. Plusieurs choix sont ainsi
possibles~:
\begin{itemize}
  \item Sélection d’une solution minimisant le nombre de capteurs (courbes rouges)
  \item Sélection d’une solution minimisant la qualité de l’enveloppe (courbes vertes)
  \item Sélection d’une solution résultant d’un compromis (courbes noires)
\end{itemize}

\begin{figure}
    \centering
    \includegraphics[width=0.92\textwidth, clip=true, trim=0mm 20mm 0mm 0mm]{Ressources/Images/EtudeDeCas/AideDecision/base.png}
    \\
    \includegraphics[width=0.92\textwidth, clip=true, trim=0mm 20mm 0mm 0mm]{Ressources/Images/EtudeDeCas/AideDecision/volume_ballons.png}
    \\
    \includegraphics[width=0.92\textwidth, clip=true, trim=0mm 20mm 0mm 0mm]{Ressources/Images/EtudeDeCas/AideDecision/final.png}
    \caption[Évolution de l’aide à la décision en fonction de choix fictif d’un décideur]
             {Évolution de l’aide à la décision en fonction de choix fictifs d’un décideur. Les courbes grisées ne répondent à
             aucunes contraintes du décideur. Chaque autre couleur indique par contre le respect d’une ou plusieurs contraintes.}
    \label{fig:xdat_exemple}
\end{figure}

Afin d’évaluer les différences entre les différentes solutions, il est sélectionnée
une solution dans chaque groupe, auxquelles il est ajouté les solutions maximisant les indicateurs
principaux de l’étude ($F_{sav,\,ext}$, $F_{sav}^{ECS}$, $F_{sav}^{CH}$, $Prod_{sol}$),
soit cinq autres solutions (deux maximisant $F_{sav,\,ext}$).
Le détail des paramètres de décisions pour chaque solution est rapporté dans le \tabref{tab:detail_opti_stras}
et chaque solution est présentée ci-dessous~:
\begin{blockdescription}{Solution 1 (\abr{S1})~:}
  \item[Solution 1 (\abr{S1})~:] Cette solution maximise le $F_{sav,\,ext}$. Elle représente
                                 le meilleur compromis entre la couverture du chauffage
                                 et de la production d’\abr{ECS} tout en minimisant
                                 le nombre de capteurs \abr{PV}. Elle admet une taille
                                 importante pour les deux ballons et le maximum de capteurs
                                 thermiques possibles. L’isolation est très importante
                                 afin de réduire au maximum les besoins sur le chauffage mais
                                 ne retient pas le meilleur vitrage existant.
  \item[Solution 2 (\abr{S2})~:] Cette solution maximise le $F_{sav}^{ECS}$ et est très
                                 similaire à la solution $1$. En cherchant à maximiser $F_{sav}^{ECS}$,
                                 la solution minimise $F_{sav}^{CH}$ notamment avec l’utilisation
                                 d’un petit ballon tampon.
  \item[Solution 3 (\abr{S3})~:] Cette solution maximise le $F_{sav}^{CH}$. $F_{sav}^{ECS}$ reste
                                 important pour deux raisons~: la production d’\abr{ECS} est prioritaire,
                                 et les besoins en \abr{ECS} sont similaires durant l’année entière.
                                 À l’inverse de la solution $2$, c’est cette fois le ballon sanitaire qui
                                 est réduit (\SI{200}{\litre}). Cependant le volume minimal n’est pas
                                 atteint car le gain sur le chauffage est nul alors que la perte de
                                 performance sur l’\abr{ECS} est par contre importante. C’est aussi la seule solution où
                                 l’échangeur est plus haut dans le ballon.
                                 En augmentant sa position, la production solaire sur l’\abr{ECS} est réduite
                                 et une part plus importante peut donc être utilisée pour le chauffage.
  \item[Solution 4 (\abr{S4})~:] Cette solution correspond à la solution maximisant le $F_{sav,\,ext}$ lorsque
                                 la $Prod_{sol}$ est maximale. Il existe en effet de nombreuses solutions ayant exactement
                                 la même production par les capteurs \abr{PV} car le nombre de capteurs est discret.
                                 Le \abr{SSC} permet de couvrir près de \SI{43}{\percent} avec seulement $2$ capteurs
                                 solaires thermiques mais l’économie de $5$ capteurs thermiques nécessite l’ajout
                                 de $8$ capteurs \abr{PV} supplémentaires. Il est aussi important de noter que
                                 la production des capteurs n’est pas forcement en adéquation avec la demande
                                 contrairement à l’appoint économisé par le \abr{SSC} qui tient parfaitement compte
                                 de la temporalité des besoins.
  \item[Solution 5 (\abr{S5})~:] Cette solution maximise le $F_{sav,\,ext}$ lorsque le volume de chaque
                                 ballon est inférieur à \SI{250}{\litre}. Mis en exergue dans
                                 \ref{sub:influence_du_volume_des_ballons},
                                 le volume des ballons peuvent être réduit tout en permettant de
                                 conserver une bonne performance globale.
  \item[Solution 6 (\abr{S6})~:] Représente une solution répondant aux préférences du décideur
                                 et qui admet uniquement $5$ capteurs solaires thermiques.
  \item[Solution 7 (\abr{S7})~:] Représente une solution répondant aux préférences du décideur
                                 et qui admet une isolation minimale. Contrairement aux autres
                                 solutions la résistance thermique du plafond est fortement réduite.
  \item[Solution 8 (\abr{S8})~:] Représente une solution répondant aux préférences du décideur
                                 pour laquelle, un compromis entre performance de l’enveloppe
                                 et nombre de capteurs thermiques a été fait.
\end{blockdescription}

\begin{table}
\small
\centering
\caption[Détail des solutions optimales sélectionnée sur Strasbourg]
         {Détail des solutions optimales sélectionnée sur Strasbourg.}
\label{tab:detail_opti_stras}
\begin{tabular}{l c c c c c c c c}
  \toprule
  \addlinespace
                       & S1        & S2         & S3  & S4       & S5   & S6 & S7  & S8                       \\
  \addlinespace
  \multicolumn{9}{l}{\textbf{\abr{SSC}}}         \\
  \midrule
  Nombre capteurs \abr{TH}     & \num{7} & \num{7} & \num{7} & \num{2} & \num{7} & \num{5} & \num{7} & \num{6} \\
  Type capteurs \abr{TH}       & \multicolumn{8}{c}{\textit{SkyPro}} \\
  $Ech_{sol}^{pos}$            & \num{0.8} & \num{0.96}  & \num{1.23}  & \num{0.95}  & \num{0.94}  & \num{0.92}  & \num{0.89}  & \num{0.96}  \\
  Volume ballon tampon         & \num{400} & \num{100} & \num{400} & \num{250} & \num{200} & \num{150} & \num{150} & \num{100} \\
  Volume ballon $ECS$          & \num{400} & \num{400} & \num{200} & \num{350} & \num{200} & \num{200} & \num{200} & \num{200} \\
  $\Delta T_{sol}$             & \num{5} & \num{10}  & \num{8} & \num{10}  & \num{10}  & \num{26}  & \num{18}  & \num{11}  \\
  $\Delta min_{capteur}$       & \num{30}  & \num{28}  & \num{0} & \num{8} & \num{16}  & \num{0} & \num{2} & \num{29}  \\
  $\Delta min_{tampon}$        & \num{25}  & \num{30}  & \num{0} & \num{10}  & \num{21}  & \num{0} & \num{2} & \num{29}  \\

  \\
  \addlinespace[\defaultaddspace]
  \multicolumn{9}{l}{\textbf{Enveloppe du bâtiment}}             \\
  \midrule
  $R$ murs              & \num{7} & \num{7} & \num{7} & \num{4} & \num{7} & \num{6} & \num{6} & \num{6} \\
  $R$ plafond           & \num{10}  & \num{10}  & \num{10}  & \num{9.5} & \num{10}  & \num{10}  & \num{6} & \num{10}  \\
  $R$ plancher          & \num{8.5} & \num{8.5} & \num{10}  & \num{10}  & \num{9} & \num{6} & \num{6.5} & \num{7.5} \\
  Surface vitrée sud    & \num{5.8} & \num{6.6} & \num{6,2} & \num{5.4} & \num{6.2} & \num{7.3} & \num{6.6} & \num{7.3} \\
  Surface vitrée est    & \num{4.3} & \num{4.7} & \num{4,3} & \num{4.7} & \num{4.4} & \num{4.5} & \num{4.8} & \num{4.3} \\
  Type de vitrage       & \multicolumn{3}{c}{\textit{PlanithermXN}} & \textit{PlanithermOne}  & \multicolumn{2}{l}{\textit{PlanithermXN}} & \textit{PlanithermOne}  & \textit{PlanithermXN} \\

  \\
  \addlinespace[\defaultaddspace]
  \multicolumn{9}{l}{\textbf{Production d’électricité}}      \\
  \midrule
  Nombre capteurs \abr{PV}  & \num{16}  & \num{17}  & \num{18}  & \num{24}  & \num{16}  & \num{20}  & \num{20}  & \num{18}  \\
  \\
  \addlinespace[\defaultaddspace]
  \multicolumn{5}{l}{\textbf{Indicateurs}}      \\
  \midrule
  $Conso_{app}$ & \num{1118}  & \num{1208}  & \num{1233}  & \num{2667}  & \num{1217}  & \num{1645}  & \num{1629}  & \num{1473}  \\
  \addlinespace[\defaultaddspace]
  $Conso_{ref}$ & \num{4221}  & \num{4219}  & \num{4187}  & \num{4726}  & \num{4208}  & \num{4350}  & \num{4722}  & \num{4308}  \\
  \addlinespace[\defaultaddspace]
  $Conso_{app,\, diff}$ & \num{3103}  & \num{3011}  & \num{2954}  & \num{2059}  & \num{2990}  & \num{2704}  & \num{3092}  & \num{2835}  \\
  \addlinespace[\defaultaddspace]
  $F_{sav,\, ext}$  & \num{74}  & \num{71}  & \num{70}  & \num{43}  & \num{71}  & \num{62}  & \num{65}  & \num{66}  \\
  \addlinespace[\defaultaddspace]
  $F_{sav}^{ECS}$ & \num{87}  & \num{89}  & \num{80}  & \num{61}  & \num{84}  & \num{79}  & \num{82}  & \num{83}  \\
  \addlinespace[\defaultaddspace]
  $F_{sav}^{CH}$  & \num{44}  & \num{34}  & \num{62}  & \num{18}  & \num{45}  & \num{33}  & \num{43}  & \num{34}  \\
  \addlinespace[\defaultaddspace]
  $Prod_{PV}$ & \num{4014}  & \num{4252}  & \num{4490}  & \num{6021}  & \num{4014}  & \num{5005}  & \num{4966}  & \num{4516}  \\
  \bottomrule
\end{tabular}
\end{table}
% subsection exploration_du_front_de_pareto (end)




% ..............................................................................
% ..............................................................................
\section{Bilan} % (fold)
\label{sec:bilan_aide_decision}
Une approche exploratoire par optimisation multi-objectifs s’est montrée être pertinente à
la fois pour évaluer les interactions existantes mais aussi pour proposer un large choix
de solutions optimales. Il est en effet mis en exergue une large diversité des paramètres
de décision traduisant une bonne exploration de l’algorithme. Cependant l’approche
introduit un élitisme important sur certains critères de décisions en partie car tous les
objectifs ne sont pas antinomiques. Il a donc été proposé un processus itératif d’aide à
la décision. Dans un premier temps l’optimisation est réalisée en considérant l’ensemble
des critères de décisions, puis en fonction des résultats de la première optimisation,
certains critères de décisions sont fixés ou la plage de variation réduite. Le processus
d’optimisation est alors relancé et les solutions optimales ajoutées dans une archive
commune. Cette méthode a permis d’identifier des compromis similaires aux solutions
optimales mais proposant de nouvelles combinaisons sur l’espace de décisions. Ce processus
permet ainsi d’affiner la recherche afin de proposer un ensemble de solutions optimales à
la fois par rapports aux objectifs mais aussi aux contraintes propres du ou des décideurs.
Elle reste de plus adaptée pour des applications où les objectifs sont tous antinomiques.
En effet la complexité augmente rapidement avec le nombre d’objectifs considérés et la
littérature montre que les l’efficacité des méta-heuristiques diminue à mesure que le
nombre d’objectifs augmente. L’aide à la décision par coordonnées parallèles s’est aussi
montrée pertinentes et simple d’utilisation. Elle permet de rapidement sélectionner une
solution sans forcement avoir une conséquence approfondie du modèle. Elle est ainsi
adaptée aux secteur du bâtiment afin de permettre à chaque acteur de participer aux
décisions et de regrouper l’expertise des différents acteurs du bâtiment. De plus l’ajout d’un objectif
ou d’une contraintes étant trivial, elle permet aussi de tenir compte des contraintes de chaque corps de métiers.

Des résultats encourageant pour le développement de \abr{SSC} pour les maisons à énergie
positives ont aussi été identifiés. Afin d’obtenir la meilleure performance globale
($F_{sav,\, ext}$), il est nécessaire de trouver le bon équilibre en performance sur le
chauffage et sur la production d’\abr{ECS}, car maximiser l’un des deux ne permet jamais
d’obtenir la meilleure économie. Il est aussi possible d’obtenir une performance
équivalente pour différentes tailles de ballons. En plus de libérer une place conséquente,
l’utilisation de ballons de faible volume permet de réduire les coût d’installations comme
de maintenance tout en limitant les risques de surchauffes estivales. Ces résultats
marquent ainsi une avancée importante pour le développement de
\abr{SSC} pour les maisons individuelles où la place disponible est limitée tant sur le
plan énergétique que économique.
Finalement, il est aussi identifié l’existence d’un compromis entre qualité de l’enveloppe et
performance du \abr{SSC}, le choix restant propre aux contraintes du décideur~: sur-isoler
et réduire le nombre de capteurs, ou moins isoler et augmenter le nombre de capteurs.
Concernant les capteurs, le choix du type reste peu influent tant que le capteur
est récent. Finalement, au delà de $4$ capteurs sur Bordeaux et de $6$ sur Strasbourg le
gain potentiel est moins important.

\paragraph{} % (fold)
La méthodologie développée a ainsi mis en évidence l’existence de nombreuses solutions
prometteuses et invitent à repenser les choix de conception lors du développement de
\abr{SSC} couplés à des \abr{MEPOS}.
% section bilan_aide_decision (end)
