%!TEX root = ../main.tex
% Chapitres/Chap4-OptimisationSystemeSolaire.tex


\iunsure{Description commune de Bordeaux et Strasbourg pour éviter répétition (section 2, 3)}
\iunsure{Relancer les simulations Pareto optimales avec Dymola pour comparaison}
\iunsure{Nécessaire de garder des facteurs avec $sigma = \mu^{*}$ même si distance faible}


% ..............................................................................
% ..............................................................................
\section{Description de l’étude de cas} % (fold)
\label{sec:description_de_l_etude_de_cas}
L’étude de cas est réalisée sur le bâtiment décrit dans le chapitre II et les scénarios de
charges internes (équipements, éclairage, et occupants) sont issues de la simulation de
référence (\ref{sub:scenarios_de_reference}). De même, le scénario de référence est
retenue pour la consigne de chauffage ($19$-$18$-$16$) comme pour la ventilation
($90-20$). Pour le puisage en $ECS$, le profil $Réaliste$ est retenue tout en tenant
compte des variations hebdomadaires et mensuelles
(\ref{ssub:puisage_en_eau_chaude_sanitaire}). En effet, comme il a été montré, il est
important de tenir compte de la variation du profil de puisage au cours de l’année afin de
ne pas sur-estimer la performance du $SSC$. Enfin, l’inclinaison des capteurs est définie
suivant la pente de la toiture du bâtiment. Pour Bordeaux la pente du bâtiment existant
est de \SI{33}{\percent} soit \SI{18.9}{\degree}. Pour Strasbourg, il est retenu une pente
de \SI{54}{\degree} typique d’une architecture alsacienne.

L’optimisation est réalisée dans un premier temps sur Bordeaux, dont le climat
est doux et l’ensoleillement important. Dans un second temps, sur Strasbourg caractérisé
par un climat rude et un ensoleillement limité durant la période de chauffage bien que
important durant la période estival.
\iunsure{Ajouter comparatif de l’évolution mensuelle de l’énergie disponible pour les deux.}

L’optimisation multi-objectifs est réalisée à la fois sur l’enveloppe, le \abr{SSC},
et sur l’algorithme de contrôle. Dans un premier temps, les hypothèses
et paramètres a priori sont explicités, puis les contraintes et objectifs retenus
discutés. Afin de limiter la
cardinalité du problème, une étude de sensibilité est réalisée en utilisant la méthode de
\textit{Morris}. Comme explicité précédemment elle permet d’identifier les facteurs les
plus influents de manière qualitative. Le but étant de réduire le nombre de variables de
décision pour l’optimisation. Une fois les variables influentes
identifiées, un échantillon représentatif pour chaque climat est construit grâce à une
méthode de \textit{quasi-Monte-Carlo} afin que les solutions soient uniformément réparties.
Ces échantillons sont ensuite utilisés afin de construire un modèle de substitution pour chaque objectif ou
contraintes. Ensuite l’optimisation par colonie d’abeilles virtuelles est réalisée
grâce à la bibliothèque \textit{pyMayBee}.
Une fois le front de Pareto obtenu, deux approches sont explorées~:
\begin{itemize}
  \item L’utilisation d’une méthode interactive d’aide à la décision avec le
        logiciel \textit{Xdat} afin de guider le choix d’une solution finale.
  \item L’utilisation de l’indicateur $FSC$ afin d’évaluer la capacité du \abr{SSC}
        à valoriser l’énergie solaire disponible.
\end{itemize}
Les deux approches sont complémentaires et ne visent pas le même public. La première
est plus utile pour faire le choix lors de la réalisation d’une construction, alors que
la seconde permet d’évaluer la capacité du \abr{SSC} à valoriser l’énergie solaire
disponible.
L’utilisation de modèles de substitution introduisant toujours des approximations,
les solutions non-dominées sont simulées avec le modèle original. Les résultats obtenues
avec les deux approches sont alors discutés.



% ------------------------------------------------------------------------------
\subsection{Hypothèses retenus} % (fold)
\label{sub:hypotheses_retenus}
L’objectif de ces travaux est de déterminer les configurations existantes qui permettent
de garantir l’obtention d’une \abr{MEPOS} dont les besoins sont majoritairement couverts
par l’énergie solaire. Le terme \abr{MEPOS} considère ici quatre usages, les consommation
propres à la climatisation n’est ainsi pas discutées~:
\begin{itemize}
  \item La chauffage
  \item La production d’\abr{ECS}
  \item L’éclairage
  \item Les équipements électro-domestiques
\end{itemize}



% - - - - - - - - - - - - - - - - - - - - - - - - - - - - - - - - - - - - - - -
\subsubsection{Répartition de la toiture} % (fold)
\label{ssub:repartition_de_la_toiture}
Afin de pouvoir couvrir les consommations électriques, éclairage ($Conso_{éclairage}$) et
électro-domestique ($Conso_{électroménager}$), une production photovoltaïque ($Prod_{PV}$)
est retenue. La toiture doit donc être partagée avec d’une part les capteurs thermiques, et
d’autre part les capteurs photovoltaïques (\abr{PV}). Comme le montre la littérature et les résultats
de l’étude paramétrique l’orientation est pour le \abr{SSC} un des facteurs les plus
important. Le choix est donc fait de favoriser en priorité les capteurs solaires
thermiques sur le pan sud. Afin d’être le plus représentatif possible la géométrie
de la toiture et des capteurs est prise en compte.
En effet la maison comporte une toiture quatre pans, chaque pan est donc triangulaire~; par
contre les capteurs sont eux rectangulaires. Il apparaît donc clairement que la surface
de toiture réellement disponible ne peut pas être approximée par sa surface totale.
De plus, chaque capteur admet une surface totale supérieure à sa surface d’entrée,
les proportions variants fortement en fonction du capteur considéré.
Ainsi afin d’estimer la surface réellement disponible un algorithme de \enquote{packaging}
a été développé (annexe \ref{cha:repartition_des_capteurs}). À partir des dimensions
du capteur (largeur et longueur de la surface totale), l’algorithme identifie le
nombre maximal pouvant loger sur chaque pan de toiture sans superpositions.
Ainsi la surface de capteurs photovoltaïques ($Surf_{PV}$) est définie suivant
\eqref{eq:repartition_toiture} en tenant compte de leur géométrie propre.

\begin{equation}\label{eq:repartition_toiture}
  \begin{aligned}
    Surf_{PV}^{sud}   &= \max\left[0,\quad \min \left(Surf_{PV}^{tot},\quad Surf_{PV}^{capteur} \times
                                                      \Bigg\lfloor\frac{Surf_{dispo}^{sud} - Surf_{TH}^{tot}}{Surf_{PV}^{capteur}}\Bigg\rfloor\right)
                                   \right] \\[10pt]
    Surf_{PV}^{ouest} &= \max \left[0,\quad \min\left(Surf_{dispo}^{ouest},\quad
                                                      (Surf_{PV}^{tot} - Surf_{PV}^{sud}\right)
                                    \right] \\[10pt]
    Surf_{PV}^{est} &= \max \left[0,\quad \min\left(Surf_{dispo}^{est},\quad
                                                    (Surf_{PV}^{tot} - Surf_{PV}^{sud} - Surf_{PV}^{ouest}) \right) \right] \\[10pt]
  \end{aligned}
\end{equation}

Avec $Surf_{PV}^{tot}$ la surface totale de capteur \abr{PV} a installée,
$Surf_{PV}^{capteur}$ la surface extérieure d’un capteur \abr{PV}, $Surf_{dispo}$ la
surface totale disponible sur le pan de toiture considéré, et $Surf_{TH}^{tot}$ la surface
installée de capteurs solaires thermiques. Sur le pan sud, il est admis que
la $Surf_{dispo}$ est calculée en fonction de la géométrie des capteurs solaires
thermiques. Les capteurs thermiques étant prioritaires sur ce pan de toiture, la surface
disponible pour les capteurs \abr{PV} est définie par $Surf_{dispo} - Surf_{TH}^{tot}$.
Sur les autres pans où il n’y a pas de capteurs thermiques, la $Surf_{dispo}$ est
définie en fonction de la géométrie (largeur et hauteur) des capteurs \abr{PV}. En effet,
les capteurs solaires thermiques et photovoltaïques n’ont pas les mêmes dimensions.
Dans tout les cas, si un capteur ne loge pas complètement sur le pan de toiture,
il est ajouté au pan suivant. Afin de favoriser au maximum la production solaire
les capteurs sont en priorité mis sur le pan sud, puis ouest, puis est.
% subsubsection repartition_de_la_toiture (end)


% - - - - - - - - - - - - - - - - - - - - - - - - - - - - - - - - - - - - - - -
\subsubsection{Conditions limites} % (fold)
\label{ssub:conditions_limites}
Comme pour l’étude paramétrique le coût important d’une simulation impose de simuler
sur une \textbf{période s’étendant du $\bm{1^{er}}$ octobre au $\bm{30}$ avril}. Le \abr{SSC}
couvre en effet les besoins sur le reste de l’année comme explicité dans le chapitre II.
Il est donc important de garder à l’esprit que la performance du \abr{SSC} sur l’année
complète est plus importante que celle obtenue sur la période la plus défavorable en
particulier sur la couverture des besoins en \abr{ECS} qui restent important durant
l’été.

Ainsi l’ensemble de solution non dominées obtenu au cours du processus d’optimisation
seront simulées sur une année complète afin d’obtenir la performance annuelle du \abr{SSC}.

La production d’énergie par les capteurs photovoltaïques est elle pré-calculée pour
les différentes combinaisons existantes. En effet les capteurs solaires thermiques
étant prioritaires sur le pan sud, la surface de capteurs \abr{PV} n’influence
pas la performance du \abr{SSC}. De la même manière, les charges internes propres
aux occupants sont fixes et suivent un scénario prédéfini. Ainsi la production
\abr{PV} et les charges internes sont évaluées sur \textbf{l’année complète}.

En couplant les résultats des deux simulations, il est donc possible d’évaluer
la consommation totale sur les quatre usages considérés. Le \abr{SSC} couvrant en
totalité les besoins sur la période non simulée, la consommation de l’appoint
est équivalente à la consommation des pompes qui au regard des résultats de l’étude
paramétrique sont négligeables. Il est donc possible d’approximer $Conso_{app}$
par la consommation de l’appoint sur la période simulée. Ainsi il peut être
obtenue la consommation totale suivant \eqref{eq:conso_totale}.

\begin{equation} \label{eq:conso_totale}
  Conso_{tot} = Conso_{app} + Conso_{électroménager} + Conso_{éclairage} - Prod_{PV} \\
\end{equation}
% subsubsection conditions_limites (end)


% - - - - - - - - - - - - - - - - - - - - - - - - - - - - - - - - - - - - - - -
\subsubsection{Définition des contraintes} % (fold)
\label{ssub:definition_des_contraintes}
Afin de pouvoir proposer un ensemble d’alternatives, la maison est considérée
à énergie positive si sa consommation totale \eqref{eq:conso_totale} ($Conso_{tot}$)
est proche de $0$~:
\iunsure{Mettre un plage différente pour Strasbourg comme la RT2012}
\begin{blockdescription}{Strasbourg}
  \item[Bordeaux~:]   \qquad $\abs{Conso_{tot}}   \leq  8 \quad (\si{kWh_{ep}\per\metre\squared})$
  \item[Strasbourg~:]   \qquad $\abs{Conso_{tot}}   \leq  8 \quad (\si{kWh_{ep}\per\metre\squared})$
  % \item[Strasbourg~:] $-3 \leq  Conso_{tot} \leq  13 \quad (\si{kWh_{ep}\per\metre\squared})$
\end{blockdescription}

Ainsi les solutions produisant trop d’énergie comme celle produisant trop peu d’énergie
seront écartées.
% subsubsection definition_des_contraintes (end)
% subsection hypotheses_retenus (end)



% ------------------------------------------------------------------------------
\subsection{Description des paramètres a priori} % (fold)
\label{sub:description_des_parametres_a_priori}
Comme décrit dans le chapitre précédent, une analyse de sensibilité est nécessaire
afin de réduire la cardinalité du problème et mieux guider l’exploration durant l’optimisation.
Couvrant à la fois les caractéristiques de l’enveloppe et du \abr{SSC}, l’ensemble des
paramètres retenus a priori sont disponibles dans le \tabref{tab:parametre_a_priori}.

\begin{table}
\centering
\caption{Liste des paramètres a priori utilisés pour l’analyse de sensibilité.}
\label{tab:parametre_a_priori}
\begin{tabular}{l c c l}
  \toprule
  \addlinespace
                                               & Borne min     & Borne max   & Remarques                                                            \\
  \addlinespace
  \multicolumn{4}{l}{\textbf{\abr{SSC}}}                                                                           \\
  \midrule
  Nombre capteurs                              & \num{2}       & \num{5} ou \num{7} & \SIrange{4.6}{11.6}{\metre\squared} ou \SI{16.2}{\metre\squared}                            \\
  Type de capteurs                             & -             & -           & Voir \tabref{tab:capteurs_specs_optimisation}   \\                                                                   \\
  $Ech_{sol}^{pos}$                            & \num{0.8}     & \num{1.3}   & Position relative à la hauteur du ballon                              \\
  Volume ballon tampon                         & \num{100}     & \num{400}   & \multirow{2}{*}{Géométrie adaptée proportionnellement}             \\
  Volume ballon $ECS$                          & \num{100}     & \num{400}   &                                                                      \\
  $Isolation_{ballon, ép}$ tampon        & \num{0.055}   & \num{0.12}  &  \multirow{2}{*}{Résistance dépend du volume du ballon}   \\
  $Isolation_{ballon, ép}$ \abr{ECS}     & \num{0.055}   & \num{0.12}  &                                                           \\
  $Isolation_{réseau, ép}$               & \num{0.013}   & \num{0.04}  & Résistance dépendante du nombre de capteurs                           \\
  $\Delta T_{sol}$                             & \num{5}       & \num{20}    &  -                                                                  \\
  $\Delta T Mini_{capteur}$                      & \num{0}       & \num{30}    &  -                                                                   \\
  $\Delta T Mini_{tampon}$                       & \num{0}       & \num{30}    &  -                                                                   \\
  $T3_{mini}$                                  & \num{15}      & \num{40}    &  -                                                                   \\
  \\
  \addlinespace[\defaultaddspace]
  \multicolumn{4}{l}{\textbf{Enveloppe du bâtiment}}                                                                              \\
  \midrule
  Type de vitrage                              & -             & -           &  Voir \tabref{tab:carac_vitrages}                                             \\
  $R$ murs                                     & \num{4}       & \num{7}     &  -                                                                   \\
  $R$ plafond                                  & \num{6}       & \num{10}    &  -                                                                   \\
  $R$ plancher                                 & \num{6}       & \num{10}    &  -                                                                   \\
  Surface vitrée Est                           & \num{4.3}     & \num{6.46}  & \multirow{4}{*}{Surface totale du mur \SI{26.4}{\metre\squared}}            \\
  Surface vitrée Nord                          & \num{0.46}    & \num{0.684} &                                                                      \\
  Surface vitrée Sud                           & \num{5.42}    & \num{8.13}  &                                                                      \\
  Surface vitrée Ouest                         & \num{2.6}     & \num{3.89}  &                                                                      \\
  \\
  \addlinespace[\defaultaddspace]
  \multicolumn{4}{l}{\textbf{Production d’électricité}}                                                                     \\
  \midrule
  Nombre de capteurs $PV$                      & \num{8}       &  \num{30}   &  Capteurs thermiques prioritaires sur le pan Sud \\                                                             \\
  \bottomrule
  \end{tabular}
\end{table}


% - - - - - - - - - - - - - - - - - - - - - - - - - - - - - - - - - - - - - - -
\subsubsection{Logique de contrôle} % (fold)
\label{ssub:logique_de_controle}
Au cours de l’étude paramétrique, $\Delta T_{sol}$ a été évalué comme impactant et
est donc retenue lors de l’application de la méthodologie d’aide à la décision.
Dans l’optique d’une valorisation maximale de l’énergie solaire pour le chauffage, il est aussi
investigué $3$ variations algorithmiques supplémentaires.
Le fonctionnement de base de l’algorithme admet une priorité sur la production d’\abr{ECS}
car les travaux de la littérature ont montré que l’énergie était mieux valorisée de cette
manière. Pour rappel, la température au niveau de l’échangeur solaire ($T3$) doit être
supérieure à \SI{30}{\degree} afin de pouvoir charger le ballon tampon. Le choix de cette
borne ($T3_{mini}$) est arbitraire et l’impact de sa variation est donc investigué.
L’activation du chauffage solaire ($Solaire_{direct}$ ou $Solaire_{indirect}$) est aussi
contrôlé par une borne fixe \figref{fig:automate_chauffage}. Dans le cas du
$Solaire_{direct}$, la température en sortie des capteurs $T1$ doit être supérieure ou
égale au maximum entre \SI{40}{\degree} et la température de l’eau en sortie de
l’échangeur eau/air ($T7$). Dans le cas du $Solaire_{indirect}$, la température du ballon
tampon ($T5$) doit être supérieure à \SI{40}{\degree}. Afin de chercher à valoriser au
maximum l’énergie solaire pour le chauffage, il est proposé non pas une borne fixe mais un
différentiel de température minimal. Dans les deux modes de chauffage, la consigne
d’activation s’adapte ainsi au rapport entre la production et la demande. Il est
alors introduit deux nouvelles variables~: $\Delta Mini_{capteur}$ et $\Delta Mini_{tampon}$.
Le chauffage peut ainsi être en $Solaire_{direct}$ lorsque, la différence de température entre la sortie des capteurs
($T1$) et l’air après le caisson de mélange $T_{air}^{mix}$ est plus importante que
$\Delta Mini_{capteur}$. De même, le chauffage peut être en $Solaire_{indirect}$, lorsque
la différence de température entre le haut du ballon tampon et $T_{air}^{mix}$ est
supérieure à $\Delta Mini_{tampon}$. Afin d’éviter des instabilité, il est de plus ajouté
un hystérésis de \SI{5}{\degree} sur le différentiel de référence.
% subsubsection logique_de_controle (end)



% - - - - - - - - - - - - - - - - - - - - - - - - - - - - - - - - - - - - - - -
\subsubsection{Variables qualitatives} % (fold)
\label{ssub:variables_qualitatives}
Le choix a aussi été fait d’introduire deux capteurs plans et deux capteurs sous-vides avec
des caractéristiques optiques et thermiques hétérogènes. La recherche a été faite
à travers deux bases de données (\fnref{http://www.sunwindenergy.com}{\textit{Sun\&Wind Energy}} et
\fnref{http://www.solar-rating.org/}{\textit{ICC-SRCC}}) et seul des capteurs récents sont considérés.
Les capteurs ont tous une \textbf{surface totale} similaire afin de pouvoir évaluer
aisément leurs performances en fonction de la surface installée. Par contre, les caractéristiques
de chaque capteur sont décrites suivant la surface considérée dans leur certification \enquote{Solar Keymark},
indiquée par un \emph{*} (\tabref{tab:capteurs_specs_optimisation}). Les capteurs
retenus sont ainsi dans la mesure du possible, similaires d’un point de vue géométrique
mais offre une diversité d’un point de vue des caractéristiques de performance. Bien que
la surface totale soit similaire, la largeur et la hauteur de chaque capteur varie, et
l’algorithme de \enquote{packaging} est utilisé pour chaque variation afin d’évaluer le
nombre maximal de capteur pouvant loger sur chaque pan de toiture. Il est observée que
malgré les différences géométriques (capteur de chez \textit{Ritter}), il est possible de
loger le même nombre de capteur sur le pan sud, même sur Bordeaux où la surface disponible
est réduite. Comme pour le capteur solaire thermique \textit{IDMK\,25-AL}, une régression
linéaire est réalisée afin d’identifier les coefficients d’\abr{IAM} $b_{0}$ et $b_{1}$
(\figref{fig:correlation_IAM_all}).

\begin{figure}
    \centering
    \includegraphics{Ressources/Images/Modelisation/Capteurs/IAM_all.pdf}
    \caption[Évolution des \abr{IAM}s en fonction de l’angle d’incidence]
    {Évolution des \abr{IAM}s respectifs des quatre capteurs considérés en fonction
     de l’angle d’incidence.}
    \label{fig:correlation_IAM_all}
\end{figure}

Finalement la performance des vitrages est aussi variable et $3$ variantes sont proposées
dont le détail est disponible dans le \tabref{tab:carac_vitrages}. Les caractéristiques
des vitrages sont ici présentés avec les coefficient caractéristiques mais les vitrages
sont implémentés de manière détaillés \ref{ssub:deperditions_a_travers_les_fenetres}.
Le type de vitrage s’applique sur l’ensemble des parois verticales, la fenêtre verticale
conserve ses caractéristiques propres.

Dans la suite du document l’optimisation est réalisée pour chaque type de capteur solaire
mais le type de vitrage est un paramètre variable. Ainsi le front de \textit{Pareto}
obtenu pour chaque capteur est discuté.

\begin{table}
\centering
\caption{Caractéristiques des capteurs solaires thermiques sélectionnés. La présence d’un astérisque (*)
         indique la surface utilisée pour les corrélations.}
\label{tab:capteurs_specs_optimisation}
\begin{tabular}{l c c c c r}
    \toprule
                                 & IDMK\,25-AL                & CPC 14 Star              & SKY PRO 12 CPC 58          & Energy + ECO 25         & Unité                       \\
    \midrule
    Fabricant                    & Sonnenkraft                & Ritter                   & Kloben                     & Dima                    & -                           \\
    Type                         & Plan vitrée                & Sous-vide                & Sous-vide                  & Plan vitrée             & -                           \\
    \addlinespace[\defaultaddspace]
    Surface totale               & \num{2.52}                 & \num{2.62}               & \num{2.59}*                & \num{2.53}              & \si{\metre\squared}                  \\
    Surface d’entrée             & \num{2.33}*                & \num{2.33}*              & \num{2.28}                 & \num{2.31}*              & \si{\metre\squared}                  \\
    Longueur totale              & \num{2.061}                & \num{1.622}              & \num{1.927}                & \num{2.008}             & \si{\metre}                  \\
    Largeur totale               & \num{1.225}                & \num{1.616}              & \num{1.342}                & \num{1.258}             & \si{\metre}                  \\
    Poids à vide                 & \num{49}                   & \num{41.2}               & \num{51}                   & \num{34}                & \si{kg}                     \\
    Contenance                   & \num{1.7}                  & \num{2.31}               & \num{1.76}                 & \num{1.9}               & \si{\litre}                 \\
    \addlinespace[\defaultaddspace]
    $\eta_{0}$                   & \num{76.5}                 & \num{0.644}              & \num{64.1}                 & \num{74.0}              & \si{\percent}                     \\
    $a_{1}$                      & \num{3.951}                & \num{0.749}              & \num{0.935}                & \num{5.116}             & \si{W/(m^{2}\period K)}     \\
    $a_{2}$                      & \num{0,011}                & \num{0.005}              & \num{0.004}                & \num{0.023}             & \si{W/(m^{2}\period K^{2})} \\
    $b_{0}$                      & \num{-0.1396}              & \num{-0.0709}            & \num{-0.1050}              & \num{-0.1284}           & -                     \\
    $b_{1}$                      & \num{-0.0004}              & \num{0.0006}             & \num{0.0114}               & \num{-0.0008}           & -                     \\
    $K_{\theta,\, dif}$          & \num{0.93}                 & \num{0.93}               & \num{97.2}                 & \num{0.93}              & \si{\percent}  \\
    \addlinespace[\defaultaddspace]
    Fiche technique              & \figref{fig:caracs_idmk}   & \figref{fig:caracs_star} & \figref{fig:caracs_skypro} & \figref{fig:caracs_eco} & - \\
    \bottomrule
\end{tabular}
\end{table}

\begin{table}
\centering
\caption{Descriptif des caractéristiques (suivant \cite{NFEN410} et \cite{NFEN673})
         des différents vitrages envisagés.}
\label{tab:carac_vitrages}
\begin{tabular}{l c c c r}
  \toprule
                     & Planitherm XN       & Planitherm ONE       & OptiwhiteKGlass       & Unité                        \\
  \midrule
  Fabricant    & \fnref{http://fr.saint-gobain-glass.com/product/2422/sgg-planitherm-xn}{%
                       St Gobain}
               & \fnref{http://eg.saint-gobain-glass.com/product/1659/}{%
                       St Gobain}
               & \fnref{https://www.pilkington.com/en-gb/uk/products/product-categories/thermal-insulation/pilkington-k-glass-range/pilkington-k-glass}{%
                       Pilkington}                                                              & -                             \\
  Construction & \num{4}-16-4              & \num{4}-16-4            & \num{4}-16-4             & -                             \\
  Gaz          & Argon                     & Argon                   & Argon                    & -                             \\
  $U_{g}$      & \num{1}.1                 & \num{1}.0               & \num{1}.5                & \si{W/(m^{2}\period \kelvin)} \\
  $g$          & \num{82}                  & \num{49}                & \num{78}                 & \si{\percent}                 \\
  \bottomrule
    \end{tabular}
\end{table}
% subsubsection variables_qualitatives (end)


% - - - - - - - - - - - - - - - - - - - - - - - - - - - - - - - - - - - - - - -
\subsubsection{Autres paramètres} % (fold)
\label{ssub:autres_parametres}
Les bornes inférieures et supérieures retenus pour les isolants représentent respectivement les valeurs typiques observés
pour des bâtiment du niveau \abr{RT\,$2012$} et des bâtiment à énergie positive.
La plage de variation pour les surfaces vitrées a été fixée arbitrairement à plus ou moins
\SI{20}{\percent} de leur valeur d’origine et les proportions de cadre sont ajustées
en fonction de la nouvelle surface (les caractéristiques du cadre et des ponts thermiques
restent constantes).
Le nombre minimal de $PV$ a été définie à $8$ afin de couvrir \SI{80}{\percent} (\SI{\approx
2408}{\kWh}) de la consommation des équipements internes (\SI{\approx 3082}{\kWh}) à minima.
Pour les capteurs solaires thermiques, il est considéré au minimum $2$ capteurs et respectivement
$5$ et $7$ pour Bordeaux et Strasbourg. Une distinction est faite entre les deux climats afin
de tenir compte des spécificités de chaque climats.
% subsubsection autres_parametres (end)
% subsection description_des_parametres_a_priori (end)



% ------------------------------------------------------------------------------
\subsection{Définition des objectifs} % (fold)
\label{sub:definition_des_objectifs}
% - - - - - - - - - - - - - - - - - - - - - - - - - - - - - - - - - - - - - - -
\subsubsection{Approche initiale} % (fold)
\label{ssub:approche_initiale}
\noindent Dans un premier temps, fort de l’expérience acquise à travers l’étude paramétrique,
les objectifs sont formalisés de la manière suivante~:
\begin{itemize}
  \item Maximiser le $F_{sol}^{ECS}$
  \item Maximiser le $F_{sol}^{CH}$
  \item Minimiser la $Conso_{app}$
\end{itemize}

Formulé de cette manière, l’approche comporte plusieurs limites. Il est clair que même si
les \num{3} objectifs ne sont pas impactés par les mêmes paramètres, une tendance commune
existe et l’optimisation risque de converger vers un ensemble de solution très réduit voir
une solution unique.

Concernant les deux premiers objectifs, $F_{sol}^{ECS}$ et $F_{sol}^{CH}$, un autre
problème est soulevé. Comme identifié dans la littérature, ces indicateurs évaluent la
part solaire respectivement pour la production d’\abr{ECS} et le chauffage. Cependant une
augmentation de la part solaire ne signifie pas forcement une diminution de la part de
l’appoint en particulier durant la période estivale où le solaire est abondant. Même si
dans ces travaux, la de simulation est limitée à la période s’étendant du $1^{er}$ octobre
au $30$ avril, il est toujours possible que certains apports solaires soient inutiles, en
particulier les apports passifs au niveau des ballons.

Finalement, le troisième objectif introduit un biais important sur la surface de $PV$
limitant l’exploratoire de l’algorithme. En effet, l’approche retenue dans ces travaux
cherche à caractériser l’ensemble des combinaisons existantes permettant d’obtenir
une \abr{MEPOS} solaire. Ainsi deux configurations extrêmes sont identifiés~:
\begin{itemize}
  \item Avoir beaucoup de $PV$ et peu de capteurs thermiques
  \item Avoir beaucoup de capteurs thermiques et suffisamment de $PV$ pour couvrir
        les consommations des équipements internes.
\end{itemize}
Cependant la surface de capteurs \abr{PV} est uniquement prise en compte que dans le
calcul de la contrainte. Ainsi, lorsque l’optimisation trouve des solutions minimisant au
maximum la $Conso_{tot}$, la surface de capteurs \abr{PV} ne peut plus augmenter sans que
les solutions violent la borne inférieure de la contrainte. En effet d’après
\defref{def:dominance_de_pareto}, une solution est meilleure qu’une autre si et seulement
si elle est meilleure sur un objectif et au moins aussi bonne sur tous les autres. Il est
donc clair que comme la surface de \abr{PV} n’impacte pas les objectifs, des solutions
avec une surface de capteurs \abr{PV} importantes ne sont pas atteignables. Il pourrait
être considéré la minimisation de la $Conso_{tot}$ en place de la $Conso_{app}$ afin qu’un
objectif tiennent compte des capteurs \abr{PV}. Cependant toujours d’après
\defref{def:dominance_de_pareto}, le problème reste entier.
% subsubsection approche_initiale (end)


% - - - - - - - - - - - - - - - - - - - - - - - - - - - - - - - - - - - - - - -
\subsubsection{Approche retenue} % (fold)
\label{ssub:approche_retenue}
\noindent Afin de palier aux difficultés rencontrées, une autre formulation est proposée~:
\begin{itemize}
  \item Maximiser le $F_{sav}^{ECS}$
  \item Maximiser le $F_{sav}^{CH}$
  \item Maximiser la $Prod_{PV}$
  \item Minimiser la $Surf_{PV}$
\end{itemize}

Dans cette nouvelle formulation, l’ensemble des paramètres influence au minimum un des
objectifs. Il est aussi clair que les différents objectifs sont antinomique.
En effet, réduire la surface de capteur thermique permet d’améliorer la production des
$PV$ (plus de capteurs au sud) mais impacte négativement les autres objectifs.
Inversement, une surface de capteur thermique importante implique une production des $PV$
plus faible. Ainsi l’ajout d’une maximisation de la $Prod_{PV}$, permet d’obtenir des
solutions variées~: avec une surface de $PV$ importante et une surface de capteur
thermique faible et inversement.

De plus en remplaçant les indicateurs $F_{sol}^{CH}$ et $F_{sol}^{ECS}$ par respectivement
$F_{sav}^{CH}$ et $F_{sav}^{ECS}$, l’algorithme favorise non plus les solutions
augmentant la part solaire, mais les solutions réduisant la part de l’appoint au
profit d’une part plus importante de solaire. Cependant ces nouveaux objectifs
nécessitent un modèle de référence afin d’évaluer la réduction de la consommation
induite par l’ajout d’un \abr{SSC}. Comme il est évalué conjointement des variations
sur l’enveloppe et sur le \abr{SSC}, il est nécessaire d’obtenir une consommation
de référence pour chaque performance d’enveloppe afin d’évaluer uniquement
l’amélioration induite par le \abr{SSC}. Un modèle de référence sans apports solaires
(tous électrique) est donc introduit. Il admet un unique ballon de \SI{200}{\litre} pour le stockage
d’\abr{ECS} et seul une batterie électrique en terminal est utilisée.

Finalement le dernier objectif, minimiser la $Surf_{PV}$ , est introduit afin de proposer des solutions non-dominées
sur toute l’intervalle définie par la contrainte. En effet, sans cet objectif les solutions
retenues sont toutes proches de la borne inférieure (maximisation de la surface de capteurs \abr{PV})
car c’est l’espace de solution maximisant la production des capteurs \abr{PV}. Cependant ces travaux considèrent que
toutes les solutions comprises dans l’intervalle sont admissibles. Il existe en effet,
une forte incertitude sur la performance réelle à la fois du bâtiment et des équipements. Ces
incertitudes introduites par le choix des hypothèses sont inhérente à toutes simulations numériques.
De plus le parti pris de cette approche est de proposer une large variété de solutions dans le
respect de la contrainte de \enquote{bilan positif}~: proposer des solutions
sur l’ensemble de l’intervalle répond à cet objectif.

Le problème est maintenant correctement formulé afin de permettre d’explorer l’espace de décision et
répondre au problème initial~: explorer l’ensemble de solutions non-dominées permettant
d’obtenir une \abr{MEPOS} solaire.
% subsubsection approche_retenue (end)
% subsection definition_des_objectifs (end)
% section description_de_l_etude_de_cas (end)



% ..............................................................................
% ..............................................................................
\section{Réduction de la cardinalité} % (fold)
\label{sec:reduction_de_la_cardinalite}

Afin de réduire la cardinalité du problème, la méthode de \enquote{screening} de
\textit{Morris} est retenue. L’analyse a été réalisée sur Bordeaux et Strasbourg en
considérant \num{20} trajectoires uniques et \num{4} niveaux. Dans un premier temps les
résultats sont discutés pour les principaux indicateurs d’un \abr{SSC}. Puis un graphe
d’influence est réalisé sur les indicateurs $F_{sol}^{CH}$, $F_{sol}^{ECS}$,
$Conso_{app}$, et $Conso_{tot}$ permettant de représenter les facteurs influents
considérés dans le reste de l’étude.

\subsection{Hypothèses} % (fold)
\label{sub:hypotheses}
La méthode de Morris nécessite que l’ensemble des variables soient continues
et puissent varier indépendamment les unes des autres. Ainsi pour des variables
qualitatives comme le type de capteurs solaires thermiques et le type de vitrage,
les caractéristiques principales ont été considérées comme indépendantes uniquement
lors de l’étude de sensibilité.
Les capteurs sont ainsi exprimées en fonction des coefficient $a_{1}$, $a_{2}$, et
du rendement optique ($\eta_{0}$). Ainsi, il est uniquement considéré le capteur
\textit{IDMK\,25-AL} auquel les variations sont appliquées.
Les vitrages sont eux exprimés en fonction de l’émissivité
du verre intérieur ($Émis_{int}$) et de son coefficient de transmission solaire ($\tau_{sol}$).
Ainsi il est considéré uniquement le vitrage \textit{Planitherm XN} auquel les variations
sont assignées. Les bornes inférieures et supérieures retenues sont issues des caractéristiques
des différents capteurs et vitrages considérés (\tabref{tab:variabilite_capteur_vitrage}).
Bien que la méthode permette de regrouper les paramètres afin d’évaluer
l’influence de l’ensemble et non de chaque élément le composant, les éléments groupés
nécessitent de pouvoir varier indépendamment. Cet option n’est donc pas applicable.
Ce choix est fait afin d’évaluer de possibles interactions existantes avec d’autres facteurs.
Bien entendu, dans le reste de l’étude (optimisation), les variables sont
considérées comme qualitatives et admettent uniquement un des choix existants.

Moins contraignant, l’échantillon créé par la méthode de \textit{Morris} suppose que
la variable est continue. Dans notre cas le nombre de capteurs solaires thermiques
et de capteurs \abr{PV} sont cependant obligatoirement des entiers. Afin de contourner
ce problème, la plage de variation de ces deux facteurs a été choisie comme~:
\begin{itemize}
  \item \SIrange{2}{5}{} capteurs solaires thermiques soit une variation de \SI{7}{\metre\squared}
  \item \SIrange{8}{14}{} capteurs \abr{PV} soit une variation de \SI{9}{\metre\squared}
\end{itemize}
Ce choix est fait afin d’obtenir la surface équivalente la plus proche possible, tout
en assurant que les valeurs prises lors de la création de l’échantillon sont des entiers
lorsque la méthode considère quatre niveaux.


\begin{table}
\centering
\caption{Variabilité des caractéristiques des capteurs et des vitrages pour l’étude
         de sensibilité.}
\label{tab:variabilite_capteur_vitrage}
\begin{tabular}{l c c}
  \toprule
  \addlinespace
                                               & Borne min     & Borne max    \\
  \addlinespace[\defaultaddspace]
  \multicolumn{2}{l}{\textbf{Vitrages}}                                       \\
  \midrule
  $Émis_{int}$                                 & \num{0.047}   & \num{0.837}  \\
  $\tau_{sol}$                                 & \num{0.643}   & \num{0.849}  \\
  \\
  \addlinespace[\defaultaddspace]
  \multicolumn{3}{l}{\textbf{Capteurs solaires}}                              \\
  \midrule
  $a_{1}$                                      & \num{0.749}   &  \num{5.116} \\
  $a_{2}$                                      & \num{0.004}   &  \num{0.023} \\
  $\eta_{0}$                                   & \num{0.644}   &  \num{0.764} \\
  \bottomrule
  \end{tabular}
\end{table}
% subsection hypotheses (end)



% % - - - - - - - - - - - - - - - - - - - - - - - - - - - - - - - - - - - - - - -
% \subsection{Analyse des résultats} % (fold)
% \label{sub:analyse_des_resultats_morris}
% L’analyse de Morris a été réalisée en considérant \num{15} trajectoires uniques à travers
% \num{4} niveaux. Les résultats sont analysés sur les indicateurs caractéristiques d’un $SSC$
% (le $F_{sol}^{CH}$, le $F_{sol}^{ECS}$, et la $Conso_{app}$) mais aussi sur les parts
% actives et passives du chauffage solaire, respectivement notées $Prod_{sol}^{active}$ et
% $Prod_{sol}^{passive}$. En effet, les ballons étant dans le bâtiment, une partie de
% l’énergie solaire est fournie par leurs déperditions. Il est alors intéressant de
% mieux comprendre les interactions existantes.


% % - - - - - - - - - - - - - - - - - - - - - - - - - - - - - - - - - - - - - - -
% \subsubsection{Couverture solaire sur l’eau chaude sanitaire} % (fold)
% \label{ssub:couverture_solaire_sur_l_ECS}
% Sur l’indicateur $F_{sol}^{ECS}$ (\figref{fig:objectifs_mu_star}) les facteurs ayant une influence linéaire sont
% peu nombreux~: la nombre de capteur et ses caractéristiques ($a_{1}$, $a_{2}$, $\eta_{0}$)
% ainsi que le volume du ballon sanitaire. La $Ech_{sol}^{pos}$ quand à elle a une influence
% non-linéaire ou avec des interactions. L’ordre des facteurs influents reste sensiblement le
% même pour Bordeaux et Strasbourg.
% % subsubsection couverture_solaire_sur_l_ECS (end)


% % - - - - - - - - - - - - - - - - - - - - - - - - - - - - - - - - - - - - - - -
% \subsubsection{Couverture solaire sur le chauffage} % (fold)
% \label{ssub:couverture_solaire_sur_le_chauffage}
% Sur l’indicateur $F_{sol}^{CH}$ (\figref{fig:objectifs_mu_star}), le nombre de capteur, ces caractéristiques ($a_{1}$,
% $a_{2}$, et $\eta_{0}$), le volume des deux ballons et la performance thermique des
% vitrages ($Émis_{ext}$) sont tous influents. Pour Bordeaux comme Strasbourg le volume du
% ballon $ECS$ a un effet non linéaire ou avec des interactions. Cependant, il peut aussi
% être noté que l’$Émis_{ext}$ a un effet non linéaire sur Strasbourg alors qu’il est
% fortement linéaire sur Bordeaux. Aussi, le $DeltaT_{sol}$ et la $Ech_{sol}^{pos}$ ont
% aussi tous deux des effets non linéaire ou avec des interactions mais l’impact est
% important uniquement sur le climat Bordelais. De même la résistance thermique de
% l’enveloppe ($R$ murs et $R$ plafond) a un impact linéaire uniquement sur Bordeaux.

% Afin de mieux comprendre l’influence de chaque facteur, il est intéressant d’analyser
% l’impact sur la $Prod_{sol}^{CH}$ passive et la $Prod_{sol}^{CH}$ active qui forment à eux
% deux la $Prod_{sol}^{CH}$ (\figref{fig:prod_sol_chauffage_mu_star}, \figref{fig:prod_sol_chauffage_mu}).
% Pour les deux climats le volume du ballon tampon influence principalement la
% $Prod_{sol}^{CH}$ passive. Sur Strasbourg, ce facteur est d’ailleurs le plus influent
% devant le nombre de capteurs. À l’opposé l’$DeltaT_{sol}$ et $Émis_{ext}$ influence
% principalement la $Prod_{sol}^{CH}$ active sur Bordeaux comme Strasbourg. Il est aussi
% clair que le $DeltaT_{sol}$ a un effet fortement non-linéaire ou avec de fortes
% interactions sur Bordeaux. Bien que non influent si on considère la $Prod_{sol}^{CH}$, la
% performance de l’isolation des ballons est un facteur impactant sur la part passive.
% D’autre part, il apparaît que la surface vitrée Sud est influente (non- linéaire ou avec
% des interactions) sur la $Prod_{sol}^{CH}$ mais pas sur le $F_{sol}^{CH}$. Enfin le nombre
% de capteurs et leur performance sont influents autant sur la part passive que la part
% active.

% Il est donc mis en exergue la complexité liée à l’évaluation couplée d’un système
% et du bâtiment. Il existe de nombreux facteurs influents autant sur l’enveloppe que
% sur le système. Certains influence principalement la part passive, d’autres la part
% active, et enfin d’autres les deux. Enfin, de nombreux facteurs ont un impact non-linéaire
% ou avec des interactions en particulier pour le climat Bordelais.
% % subsubsection couverture_solaire_sur_le_chauffage (end)

% % - - - - - - - - - - - - - - - - - - - - - - - - - - - - - - - - - - - - - - -
% \subsubsection{Consommation de l’appoint} % (fold)
% \label{ssub:consommation_de_l_appoint}
% Les résultats sur les objectifs principaux (\figref{fig:objectifs_mu_star}) montrent que
% les paramètres $a_{1}$, $a_{2}$, $\eta_{0}$, $Émis_{ext}$, le volume du ballon $ECS$, la
% résistance thermique des murs et du plafond, ainsi que le nombre de capteurs thermiques,
% sont tous influents. Seul le volume du ballon $ECS$ a une influence non-linéaire ou avec
% des interactions. Il apparaît aussi que le niveau d’isolation des ballons comme des
% canalisations ne soit pas très important. Pour Strasbourg, contrairement à Bordeaux, de
% nombreux facteurs liés à l’enveloppe sont impactant. La résistance du plancher et le
% $\tau_{sol}$, ont en effet un impact linéaire, et les surfaces vitrées au Sud comme à
% l’Est un impact non-linéaire ou avec interactions. Aussi l’isolation des vitrages
% ($Émis_{ext}$) est le facteur le plus influent sur Strasbourg, devant le nombre de
% capteur.

% Lorsque seule la $Conso_{app}^{CH}$ (\figref{fig:conso_app_mu_star}, \figref{fig:conso_app_mu})
% est considérée, il est clair que l’enveloppe joue un rôle important même pour Bordeaux où
% l’$Émis_{ext}$ est le facteur le plus influent suivit par le nombre de capteur et la
% résistance des murs. Pour Bordeaux, il peut aussi être noté que la $Conso_{app}^{CH}$ est
% influencée par la surface de vitrage au Sud mais les besoins en chauffage étant moins
% important que les besoins en $ECS$ l’indicateur n’est pas parmi les plus influents sur la
% $Conso_{app}$. Enfin pour Bordeaux comme Strasbourg, le volume du ballon tampon comme la
% $Ech_{sol}^{pos}$ sont influent sur la $Conso_{app}^{ECS}$. Ainsi au regard de résultats
% il apparaît que la performance du $SSC$ est plus impactée par les caractéristiques de
% l’enveloppe lorsque les conditions extérieures sont plus rudes.
% % subsubsection consommation_de_l_appoint (end)


% % - - - - - - - - - - - - - - - - - - - - - - - - - - - - - - - - - - - - - - -
% \subsubsection{Consommation totale} % (fold)
% \label{ssub:consommation_totale}
% Sur la $Conso_{totale}$, il est observé une très forte influence linéaire de la surface de
% $PV$. En comparaison, les autres paramètres ont une influence modérée sur Bordeaux~:
% nombre des capteurs et ses caractéristiques, volume du ballon sanitaire, isolation des
% vitrages, des murs, et du plafond. La surface de $PV$ est le seul facteur permettant de
% couvrir la consommation électrique des équipements internes. Ces équipements représentant
% la majorité des consommations pour Bordeaux, il est normal que ce facteur soit très
% influents. Par contre sur Strasbourg, où la $Conso_{app}$ est importante, et plus
% particulièrement la $Conso_{app}^{CH}$ les facteurs caractérisant l’enveloppe ont une plus
% grande importance. Ainsi sur Strasbourg, le second facteur le plus influent est
% l’$Émis_{ext}$ alors que c’est le nombre de capteurs pour Bordeaux. Enfin, le volume du
% ballon sanitaire a un impact non-linéaire ou avec des interactions et linéaire
% respectivement pour Bordeaux et Strasbourg.
% % subsubsection consommation_totale (end)


% % - - - - - - - - - - - - - - - - - - - - - - - - - - - - - - - - - - - - - - -
% \subsubsection{Valorisation de l’énergie solaire~:} % (fold)
% \label{ssub:valorisation_de_l_energie_solaire_}
% L’analyse de l’influence des facteurs (\figref{fig:prod_sol_valorisee_mu_star}) montre que
% les $Pertes_{réseau}$ sont fortement impactées par la surface totale de capteur comme de
% l’épaisseur de l’isolant au niveau des canalisations. Plus intéressant, les résultats
% montrent que le volume du ballon sanitaire a aussi un impact important. Afin de mieux
% comprendre le signe de l’influence, il est nécessaire de regarder l’évolution de la
% moyenne ($\mu$) (\figref{fig:prod_sol_valorisee_mu}). En comparant l’évolution de
% $\mu^{*}$ et de $\mu$ il est clair que augmenter le volume influe négativement les
% $Pertes_{réseau}$. Il est aussi observé que pour Bordeaux comme Strasbourg, l’impact est
% avec interaction ou non linéaire même si l’effet est plus important sur Bordeaux. La
% $Prod_{sol}$ étant plus importante sur Bordeaux que Strasbourg, les $Pertes_{réseau}$ sont
% aussi augmentées. Dans une moindre mesure le $DeltaT_{sol}$ est aussi influent et a des
% interaction non linéaires ou avec des interactions. Contrairement aux observations
% réalisés sur les indicateurs sélectionnées comme des objectifs, le facteur semble plus
% influent sur Strasbourg.

% Maintenant si on s’intéresse à l’effet de ces pertes sur la $Prod_{sol}^{valorisée}$, la
% performance thermiques des canalisations ne sont plus importantes. Cependant le volume du
% ballon sanitaire est toujours influent mais les effets sont cette fois linéaires. Il est
% aussi possible d’identifier un nouveau facteur influent de manière linéaire~: le volume du
% ballon tampon. L’explication est simple~: la proportion des $Pertes_{réseau}$ est moins
% importante que la $Prod_{sol}$, ainsi les facteurs les plus influents pour la
% $Prod_{sol}^{valorisée}$ sont ceux ayant une influence sur la $Prod_{sol}$.

% Il est aussi intéressant de voir que la performance des vitrages ($Émis_{ext}$) est un
% facteur impactant avec des effets non linéaires ou des interactions. Il semble en effet
% que la dégradation de la performance des fenêtre permette de valoriser plus d’énergie
% solaire. Pour les deux climats, l’influence est similaire mais il a été noté que la
% dégradation de l’$Émis_{ext}$ des vitrages impacte fortement la $Conso_{app}$
% particulièrement pour Strasbourg. Le solaire ne semble donc capable de couvrir que une
% partie de l’augmentation des besoins.
% % subsubsection valorisation_de_l_energie_solaire_ (end)


\begin{figure}
    \centering
    \includegraphics{Ressources/Images/Sensibilite/sigma_mu_star_objectifs.pdf}
    \caption{Résultat de l’analyse de Morris pour les objectifs principaux
             ($f(\mu^{*}) = \sigma$).}
    \label{fig:objectifs_mu_star}
\end{figure}


\begin{landscape}
    \begin{figure}
        \centering
        \includegraphics{Ressources/Images/Sensibilite/graphInfluence.pdf}
        \caption[Graphe d’influence du $SSC$ pour Bordeaux et Strasbourg]
                {Graphe d’influence du $SSC$ pour Bordeaux et Strasbourg avec les
                 relations linéaires (noir), non-linéaires (bleu), et indirectes (pointillées).}
        \label{fig:graphe_influence_objectifs}
    \end{figure}
\end{landscape}
% subsection analyse_des_resultats (end)



% - - - - - - - - - - - - - - - - - - - - - - - - - - - - - - - - - - - - - - -
\subsection{Paramètres retenus} % (fold)
\label{sub:parametres_retenus}
Au regard de cette analyse, un récapitulatif des paramètres conservés permet de voir que
des facteurs à la fois sur l’enveloppe, et le système ont été retenus. Au cours de
l’optimisation, des capteurs et vitrages existants sont utilisé afin d’être cohérent avec
les performances existantes sur le marché (\tabref{tab:capteurs_specs_optimisation},
\tabref{tab:carac_vitrages}). Il peut aussi être noté que l’enveloppe impacte de manière
plus importante le $SSC$ sur le climat Strasbourgeois où le climat est plus rude. Il
existe ainsi des compromis intéressants entre la qualité de l’enveloppe, la surface de
$PV$, et les caractéristiques du $SSC$. Enfin le criblage a permis de passer de
\num{19} paramètres à \num{11} pour Bordeaux et \num{12} pour Strasbourg. Le
nombre de combinaisons à évaluer est ainsi fortement réduit.


\begin{table}
\centering
\caption{Liste des paramètres retenus pour l’optimisation.}
\label{tab:facteur_retenues}
\begin{tabular}{l c c c c l}
  \toprule
  \addlinespace
                       & Min        & Max         & Catégorie  & Pas        & Remarques                                \\
  \addlinespace
  \multicolumn{5}{l}{\bm{$SSC$}}         \\
  \midrule
  Nombre capteurs      & \num{2}    & \num{5}     & Discrète    & \num{1}    & \num{4.64} -- \SI{11.6}{\metre\squared}   \\
  Type de capteur      & -          &  -          & Qualitative & -          & Voir \tabref{tab:capteurs_specs_optimisation}   \\
  $Ech_{sol}^{pos}$    & \num{0.8}  &  \num{1.3}  & Continue    & -          & Position relative à la taille du ballon     \\
  Volume ballon tampon & \num{100}  &  \num{500}  & Discrète    & \num{50}   & \multirow{2}{*}{Dimensions adaptées proportionnellement}   \\
  Volume ballon $ECS$  & \num{100}  &  \num{500}  & Discrète    & \num{50}   &    \\
  $DeltaT_{sol}$       & \num{5}    &  \num{15}   & Continue    & -          &  \emph{Uniquement pour Bordeaux}      \\
  \\
  \addlinespace[\defaultaddspace]
  \multicolumn{4}{l}{\textbf{Enveloppe du bâtiment}}             \\
  \midrule
  $R$ murs             & \num{4}    &  \num{7}    & Discrète    & \num{0.5}  & -                                  \\
  $R$ plafond          & \num{6}    &  \num{10}   & Discrète    & \num{0.5}  & -                                                                      \\
  $R$ plancher         & \num{6}    &  \num{10}   & Discrète    & \num{0.5}  & \emph{Uniquement pour Strasbourg}                                                                     \\
  Surface vitrée Sud  & \num{5.42} &  \num{8.13} & Continue    &  -         & -       \\
  Surface vitrée Est  & \num{4.3}  &  \num{6.46} & Continue    &  -         &  \emph{Uniquement pour Strasbourg} \\
  Type de vitrage      & -          &  -          & Qualitative &  -         & Voir \tabref{tab:carac_vitrages} \\
  \\
  \addlinespace[\defaultaddspace]
  \multicolumn{5}{l}{\textbf{Production d’électricité}}      \\
  \midrule
  Surface PV           & \num{14}   &  \num{30}   & Discret    &  \num{1}   & Capteurs thermiques prioritaires sur le pan Sud   \\
  \bottomrule
\end{tabular}
\end{table}

\itodo{Présenter les résultats de la régression + ref équations chapitre 2}


% \begin{table}
% \centering
% \itodo{Remplacer avec les bons capteurs et expliciter les paramètres}
% \caption{Caractéristiques des panneaux solaires.
% \label{tab:capteurs_specs_optimisation}}
% \begin{tabular}{l c c c c r}
%     \toprule
%                                  & IDMK\,25             & 308C\,HP             & 12\,CPC58      & ECO 25        & Unité                       \\
%     \midrule
%     Fabricant                    & Sonnenkraft          & Radco                & Sky Pro        & Dima          & -                           \\
%     Type                         & Plan vitrée          & Plan vitrée          & Tubulaire      & Plan vitrée   & -                           \\
%     Surface nette                & \num{2.32}           & \num{2.193}          & \num{2.28}     & \num{2.312}   & \si{m^{2}}                  \\
%     Poids à vide                 & \num{54}             & \num{36}             & \num{53}       & \num{41}      & \si{kg}                     \\
%     Contenance                   & \num{1.35}           & \num{3.5}            & \num{1.83}     & \num{1.9}     & \si{\litre}                 \\
%     $\eta_{0}$                   & \num{78}             & \num{83.4}           & \num{63}       & \num{66.4}    & \si{\percent}                     \\
%     $a_{1}$                      & \num{3.796}          & \num{1.4539}         & \num{0.9249}   & \num{4.9510}  & \si{W/(m^{2}\period K)}     \\
%     $a_{2}$                      & \num{0,013}          & \num{0.0589}         & \num{0.00069}  & \num{0.01527} & \si{W/(m^{2}\period K^{2})} \\
%     $IMDiff$                     & \num{100}            & \num{96}             & \num{102}      & \num{93}      & \si{\percent}                     \\
%     \bottomrule
% \end{tabular}
% \end{table}

% \begin{table}
% \centering
% \caption{Descriptif des caractéristiques (suivant \cite{NFEN410} et \cite{NFEN673})
%          des différents vitrages envisagés.}
% \label{tab:carac_vitrages}
% \begin{tabular}{l c c c r}
%   \toprule
%                      & Planitherm XN       & Planitherm ONE       & OptiwhiteKGlass       & Unité                        \\
%   \midrule
%   Fabricant    & \href{http://fr.saint-gobain-glass.com/product/2422/sgg-planitherm-xn}{%
%                        St Gobain}
%                & \href{http://eg.saint-gobain-glass.com/product/1659/}{%
%                        St Gobain}
%                & \href{https://www.pilkington.com/en-gb/uk/products/product-categories/thermal-insulation/pilkington-k-glass-range/pilkington-k-glass}{%
%                        Pilkington}                                                              & -                             \\
%   Construction & \num{4}-16-4              & \num{4}-16-4            & \num{4}-16-4             & -                             \\
%   Gaz          & Argon                     & Argon                   & Argon                    & -                             \\
%   $U_{g}$      & \num{1}.1                 & \num{1}.0               & \num{1}.5                & \si{W/(m^{2}\period \kelvin)} \\
%   $g$          & \num{82}                  & \num{49}                & \num{78}                 & \si{\percent}                 \\
%   \bottomrule
%     \end{tabular}
% \end{table}
% subsection parametres_retenus (end)
% section reduction_de_la_cardinalite (end)




% ..............................................................................
% ..............................................................................
\section{Construction d’un modèle de substitution} % (fold)
\label{sub:construction_d_un_modele_de_substitution}
\subsection{Création de l’échantillon} % (fold)
\label{sub:creation_de_l_echantillon}
Afin de réduire la durée de simulation le modèle détaillée peut être substitué à un méta-
modèle. Comme décrit dans le chapitre précédent un ensemble représentatif des combinaisons
possibles est nécessaire afin de construire un modèle valide. Grâce à une méthode
de criblage réalisée en amont la taille de l’échantillon nécessaire pour construire le
modèle est réduit. Il a été construit grâce à une méthode de pseudo-Monte-Carlo
en considérant la suite de Halton comme générateur pseudo-aléatoire. Les solutions construites
sont donc équitablement réparties, assurant une bonne représentativité de l’espace de décision.
% subsection creation_de_l_echantillon (end)

\subsection{Création des modèles de substitution} % (fold)
\label{sub:creation_des_modeles_de_substitution}
Une fois l’échantillon simulé, la bibliothèque \textit{OpenTurns} a été utilisée
afin de construire les \num{3} méta-modèles. La $Prod_{sol}$ étant pré-calculée,
un modèle est uniquement nécessaire pour le $F_{sol}^{ECS}$, le $F_{sol}^{CH}$, et la $Conso_{app}$.
Bien que la $Conso_{app}$ ne soit pas un objectif de l’optimisation, sa connaissance est
nécessaire afin de déterminer la contrainte~: la $Conso_{totale}$.

Étant en phase de conception et cherchant à explorer l’ensemble du domaine de recherche,
une loi uniforme est utilisée pour chaque paramètre durant la construction des méta-modèles.
Afin d’évaluer leurs précisions par rapport au modèle de référence, l’erreur
quadratique moyenne ($RMSE$, Root Mean Square Error) et l’erreur absolue maximale ($MAE$,
Maximal Absolute Error) sont utilisées. La $RMSE$ permet d’évaluer l’écart moyen entre
les sorties du méta-modèle ($\mathcal{M}$) et du modèle de référence ($f$)
\eqref{eq:rmse}. Le second, la $MAE$, permet d’obtenir l’écart maximal existant entre le
méta-modèle et le modèle de référence \eqref{eq:mae}. Ainsi ces deux variables statistiques
permettent d’évaluer la qualité de l’approximation du méta-modèle.

\begin{align}
  \label{eq:rmse}
  RMSE &= \sqrt{\frac{1}{N}\sum^{N}_{i=1} \left[ f(\vec{x}_{i}) - \mathcal{M}(\vec{x}_{i}) \right]^{2} } \\
  \label{eq:mae}
  MAE  &= max \left( \abs{f(\vec{x}_{i}) - \mathcal{M}(\vec{x}_{i})}, \: x = 1, 2, \dotsc, N \right)
\end{align}

\itodo{Décrire choix variable pour capteurs et vitrages}
Afin de représenter les variables qualitatives, vitrages et capteurs, leurs variables
caractéristiques sont utilisés en substitution. Au regard des résultats il apparaît que
les capteurs sont mieux approximer si on le substitue par sa valeur pour le coefficient
$a_{1}$. Pour les vitrages il semble que l’approximation est meilleur si on utilise en
substitut la valeur de l’$Émis_{ext}$  (\mtodo{ref tab annexe}).

\itodo{Ajouter analyse des courbes une fois étude finies}
Une étude paramétrique est ensuite réalisée afin de déterminer la taille minimale
nécessaire pour construire un méta-modèle approximant au mieux le modèle de référence.
L’évolution de la $RMSE$ (\figref{fig:rmse}) et de la $MAE$ (\figref{fig:rmse}) est
alors analysé.


\begin{figure}
    \centering
    \includegraphics{Ressources/Images/MetaModele/RMSE.pdf}
    \caption[Évolution de la $RMSE$ en fonction de l’échantillon]
            {Évolution de la $RMSE$ pour les \num{3} méta-modèles
             en fonction de la taille de l’échantillon.}
    \label{fig:rmse}
\end{figure}

\begin{figure}
    \centering
    \includegraphics{Ressources/Images/MetaModele/MAE.pdf}
    \caption[Évolution de la $MAE$ en fonction de l’échantillon]
            {Évolution de la $MAE$ pour les \num{3} méta-modèles
             en fonction de la taille de l’échantillon.}
    \label{fig:mae}
\end{figure}

Un méta-modèle d’ordre \num{3} est donc retenu pour chaque indicateur, le $F_{sol}^{ECS}$, le $F_{sol}^{CH}$,
et la $Conso_{app}$ dont les erreurs statistiques sont fournies à travers le \tabref{tab:meta_retenus}.
La comparaison entre les méta-modèles et le modèle de référence est aussi illustré pour chaque
point de l’échantillon non utilisés pour sa construction (\figref{fig:validite_meta}).

\begin{figure}
    \centering
    \includegraphics{Ressources/Images/MetaModele/validite_meta.pdf}
    \caption[Validité des méta-modèles]
            {Évaluation de la validité des méta-modèles pour un échantillon
             de taille \num{1000}.}
    \label{fig:validite_meta}
\end{figure}

\begin{table}
\centering
\caption{Erreurs caractéristiques ($RMSE$ et $MAE$) obtenues pour les \num{3} méta-modèles retenus
         (taille de l’échantillon de \num{1000})
\label{tab:meta_retenus}}
\begin{tabular}{l c c c c r}
    \toprule
                    & \multicolumn{2}{c}{Bordeaux} & \multicolumn{2}{c}{Strasbourg} & \multirow{2}{*}{Unité} \\
                    \cmidrule(r){2-3}
                    \cmidrule(r){4-5}
                    & $RMSE$ & $MAE$               & $RMSE$ & $MAE$                 &                        \\
    \midrule
    $F_{sol}^{ECS}$ &  & & & & \si{\percent} \\
    \addlinespace[\defaultaddspace]
    $F_{sol}^{ECS}$ &  & & & & \si{\percent} \\
    \addlinespace[\defaultaddspace]
    $Conso_{app}$   &  & & & & \si{kWh}      \\
    \bottomrule
\end{tabular}
\end{table}


% subsection creation_des_modeles_de_substitution (end)
% section construction_d_un_modele_de_substitution (end)




% ..............................................................................
% ..............................................................................
\section{Vers une solution adaptée} % (fold)
\label{sec:vers_une_solution_adaptee}
% ------------------------------------------------------------------------------
\subsection{Optimisation multi-objectif} % (fold)
\label{sub:optimisation_multi_objectif}
~
\itodo{Décrire l’évolution du front et le panel de solution obtenue}
\itodo{Décrire l’exploration de l’espace}
\ftodo{Représentation 3D}
\ftodo{Représentation 2D}
\ftodo{Évolution du front en fonction des itérations}
\ftodo{Identification de sous-groupe par couleur}
% ------------------------------------------------------------------------------
\subsection{Aide à la décision a posteriori} % (fold)
\label{sub:aide_a_la_decision_a_posteriori}
~
\itodo{Description des critères qui rentre en jeu~: coût, Suface disponible,
       fournisseurs locaux, type de clientèle, réglementation, ...}
\itodo{Ajouter un exemple avec XDAT}
\itodo{Tableau ou figure}
\ftodo{Ajouter screenshot de XDAT avec la/les solutions retenues}
\ttodo{Ajouter tableau avec caractéristiques des solutions retenues}

% subsection aide_a_la_decision_a_posteriori (end)
% subsection optimisation_multi_objectif (end)
% section vers_une_solution_adaptee (end)
