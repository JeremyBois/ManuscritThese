%!TEX root = ../main.tex
% Chapitres/Chap4-OptimisationSystemeSolaire.tex

\iunsure{Considérer l’électroménager dans le bilan énergétique}

\iunsure{Description commune de Bordeaux et Strasbourg pour éviter répétition (section 2, 3)}
\iunsure{Relancer les simulations Pareto optimales avec Dymola pour comparaison}
\iunsure{Nécessaire de garder des facteurs avec $sigma = \mu^{*}$ même si distance faible}

\ifix{Surface vitrée Est pas influente pour Bordeaux}

% Pour optimisation
% Par exemple un pas de \num{0.1} avec des bornes min de \num{0} et max de \num{0.5} donnera
% les variations possibles suivantes~: (\num{0} - \num{0.1} - \num{0.2} - \num{0.3} - \num{0.4} - \num{0.5}).



% ..............................................................................
% ..............................................................................
\section{Description de l’étude de cas} % (fold)
\label{sec:description_de_l_etude_de_cas}
L’étude de cas est réalisée sur le bâtiment utilisée pour l’étude paramétrique.
Les scénarios pour les charges internes (équipements, éclairage, et occupants) sont
issues de la simulation de référence (\ref{sub:scenarios_de_reference}). De même,
le scénario de référence est retenue pour la consigne de chauffage ($19$-$18$-$16$)
comme pour la ventilation ($90-20$).
Pour le puisage en $ECS$, le profil $Réaliste$ est retenue tout en tenant compte des
variations hebdomadaires et mensuelles (\ref{ssub:puisage_en_eau_chaude_sanitaire}).
En effet, comme il a été montré, il est important de tenir compte de la variation
du profil de puisage au cours de l’année afin de ne pas sur-estimer la performance
du $SSC$. Enfin, l’inclinaison des capteurs est imposée par la géométrie du bâtiment et
elle est donc fixée à \SI{33}{\percent} soit \SI{18.9}{\degree}.

Deux climats sont ici considérés afin d’évaluer la performance du $SSC$. Dans un premier temps
dans la région Bordelaise, avec un climat doux et un ensoleillement important. Puis,
au niveau de Strasbourg qui est caractérisé par un climat rude et un ensoleillement
limité durant la période de chauffage.
\iunsure{Ajouter comparatif de l’évolution mensuelle de l’énergie disponible pour les deux.}

Un ensemble de paramètres a priori sont définis et une analyse de sensibilité est conduite
pour les deux conditions climatiques considérés. Les résultats sont ensuite utilisés afin
de construire les modèles de substitutions employés pour réaliser l’optimisation multi-
critère. Une fois le front de Pareto obtenu, le processus d’aide à la décision est
illustré à travers une approche paramétrique. Afin de rendre la comparaison plus efficace,
le processus sera mené de manière parallèle pour Bordeaux et Strasbourg. Finalement, la
fidélité des méta-modèles pour l’ensemble des solutions optimales est aussi discuté.



% ------------------------------------------------------------------------------
\subsection{Approche naïve} % (fold)
\label{sub:approche_naive}
\noindent Dans un premier temps, fort de l’expérience acquise à travers l’étude paramétrique,
les objectifs sont formalisés de la manière suivante~:
\begin{itemize}
  \item Maximiser le $F_{sol}^{ECS}$
  \item Maximiser le $F_{sol}^{CH}$
  \item Minimiser la $Conso_{app}$
\end{itemize}
L’étude de cas cherchant à obtenir un bâtiment ayant une consommation très faible, il est
nécessaire d’introduire une production locale d’électricité. Cette production permettra de
couvrir l’énergie consommée par les équipements internes (électroménager et éclairage)
ainsi que la $Conso_{app}$. Ainsi une contrainte est introduite afin de ne retenir que
les solutions proposant une maison faiblement consommatrice \eqref{eq:conso_totale} fixé arbitrairement à~:
\begin{itemize}
   \item $\abs{Conso_{totale}} \leq \SI{300}{kWh}$ pour Bordeaux
   \item $\abs{Conso_{totale}} \leq \SI{600}{kWh}$ pour Strasbourg
 \end{itemize}
De cette manière, les bâtiments produisant localement respectivement trop ou pas assez d’énergie sont
écartés.

\begin{equation} \label{eq:conso_totale}
  Conso_{totale} = Conso_{app} + Conso_{équipements} + Conso_{éclairage} - Prod_{PV} \\
\end{equation}

Cependant formulé de cette manière, l’approche comporte plusieurs failles. Premièrement,
il est clair que même si les \num{3} objectifs ne sont pas impactés par les mêmes
paramètres, une tendance commune est identifiable. Ainsi, sans objectifs évoluant de
manière \enquote{contradictoires} l’optimisation risque de converger vers un ensemble
de solution très réduit voir une solution unique.
Deuxièmement, l’approche introduit un biais important sur la surface de $PV$. Afin de
respecter la contrainte, deux configurations extrêmes sont possibles~:
\begin{itemize}
  \item Avoir beaucoup de $PV$ et peu de capteurs thermiques
  \item Avoir beaucoup de capteurs thermiques et suffisamment de $PV$ pour couvrir
        les consommations des équipements internes.
\end{itemize}
Cependant, une fois que la $Conso_{totale}$
a atteinte la limite basse imposée par la contrainte (par exemple \SI{-298}{kWH} pour Bordeaux),
la surface de $PV$ ne peut plus augmenter sans que la contrainte soit violée (production trop importante d’électricité).
La surface de $PV$ n’a en effet aucun impact sur les objectifs résultant et les solutions avec
une surface plus importante seront donc rejetées.
Afin que la surface de $PV$ soit influente sur au moins un objectif il est possible
de minimiser la $Conso_{totale}$ au lieu de minimiser la $Conso_{app}$.
Cependant, il a été vu au chapitre précédent que la dominance de Pareto implique que une solution
est meilleure qu’une autre si et seulement si elle est meilleure sur un objectif et au moins
aussi bonne sur tous les autres (\defref{def:dominance_de_pareto}). Il est donc clair que
cette modification ne permet pas de corriger le biais sur la surface de $PV$.
% subsection approche_naive (end)


% ------------------------------------------------------------------------------
\subsection{Approche retenue} % (fold)
\label{sub:approche_retenue_optimisation}
\noindent Afin de palier aux difficultés rencontrées, une autre formulation est proposée où les objectifs
sont~:
\begin{itemize}
  \item Maximiser le $F_{sol}^{ECS}$
  \item Maximiser le $F_{sol}^{CH}$
  \item Minimiser la $Prod_{PV}$
\end{itemize}
La contrainte elle n’est pas modifiée~:
\begin{itemize}
   \item $\abs{Conso_{totale}} \leq \SI{300}{kWh}$ pour Bordeaux
   \item $\abs{Conso_{totale}} \leq \SI{600}{kWh}$ pour Strasbourg
\end{itemize}

Dans cette nouvelle formulation, l’ensemble des paramètres influence au minimum un des
objectifs. Il est aussi clair que les deux premiers objectifs sont \enquote{contradictoires} avec
le dernier. En effet, réduire la surface de capteur thermique permet d’améliorer la
production des $PV$ (Plus de capteurs au Sud) mais impacte négativement les autres
objectifs. Inversement, une surface de capteur thermique importante implique une
production des $PV$ plus faible. Il est donc maintenant possible d’obtenir des
solutions avec une surface de $PV$ importante et une surface de capteur thermique
faible et inversement.
Le problème est maintenant correctement formulé afin de permettre d’explorer l’espace de décision et
répondre au problème initial~: réaliser une maison passive dont les besoins sont couverts
par l’énergie solaire.
% subsubsection approche_retenue (end)

% ------------------------------------------------------------------------------
\subsection{Détail des objectifs} % (fold)
\label{sub:detail_des_objectifs}
~
\iunsure{Parler de la variabilité de $F_{sol}$ si on considère l’été}
De la même manière que lors de l’étude paramétrique, les simulations sont réalisées
uniquement durant la période où le système solaire n’est pas autonome afin de réduire la
durée de simulation~: \emph{période s’étendant du $1^{er}$ octobre au $30$ avril}. Ainsi
les deux premiers objectifs, $F_{sol}^{ECS}$ et $F_{sol}^{CH}$ ne sont évalués que durant
la période la plus défavorable. En effet durant le reste de l’année, les besoins en $ECS$
et en chauffage sont couverts intégralement par le $SSC$~: $F_{sol}^{ECS} = F_{sol}^{CH} = 1$.

À l’inverse, la $Conso_{app}$ durant la période de simulation est quasiment équivalente à
la $Conso_{app}$ durant l’année complète. En effet, durant le reste de l’année la
$Conso_{app}$ est égale à la $Conso_{pompes}$ qui comme l’a montrée l’analyse paramétrique
est négligeable. C’est pour cette raison qu’il est possible de calculer la
$Conso_{totale}$ sur l’année complète sans avoir à simuler l’année complète.

La $Prod_{PV}$ est donc calculée en amont pour l’ensemble des combinaisons existantes entre
capteurs thermiques et capteur $PV$ pour l’année complète. Afin de l’évaluer correctement,
la surface disponible en toiture doit cependant être partagée. Comme il
a été vu au chapitre 2, la $Prod_{sol}$ dépend fortement de l’orientation des capteurs
solaires thermiques. Ils sont donc prioritaire sur le pan Sud. De plus, il est aussi
nécessaire de tenir compte de la géométrie à la fois de la toiture et des capteurs. La
maison comporte une toiture quatre pans, chaque pan est donc triangulaire. Les capteurs
eux sont rectangulaires. Il apparaît donc clairement que l’approche naïve qui consiste à
considérer la surface totale de la toiture comme disponible n’est pas valable~: seule une
partie est réellement utilisable. Un algorithme de \enquote{packaging} a donc été
développé afin d’évaluer le nombre de capteurs pouvant loger sur chaque pan de toiture et
le détail du calcul est disponible en annexe (\ref{cha:repartition_des_capteurs}).

Il est donc important de garder à l’esprit que la $F_{sol}^{ECS}$ et $F_{sol}^{CH}$
sont plus importantes si on considère une année complète. À l’inverse la $Conso_{app}$
sur l’année est équivalente à celle de la période simulée.
% subsection detail_des_objectifs (end)
% section description_de_l_etude_de_cas (end)





% ..............................................................................
% ..............................................................................
\subsection{Réduction de la cardinalité} % (fold)
\label{sec:reduction_de_la_cardinalite}
% ------------------------------------------------------------------------------
\subsection{Paramètres a priori} % (fold)
\label{sub:parametres_a_priori}
~
\iunsure{Ajouter explication variables discrètes}
\iunsure{Relancer sensibilité avec les vraies bornes des capteurs retenus}
Comme décrit dans le chapitre précédent, une analyse de sensibilité est nécessaire
afin de réduire la cardinalité du problème, évitant ainsi de simuler des variations
non influentes relativement aux autres paramètres.
L’étude de cas est réalisée pour deux climat différents, Bordeaux et Strasbourg,
et un ensemble de \num{22} paramètres a priori est retenue (\tabref{tab:facteur_sensibilite})
en faisant varier à la fois la performance de l’enveloppe et les caractéristiques du $SSC$.
\itodo{Vérifier sources}
Au niveau des isolants, les bornes inférieures et supérieures représentent respectivement
le niveau recommandé pour obtenir le niveau \textit{RT\,2005} et le \textit{MEPOS}.
La plage de variation pour les surfaces vitrées a été fixée arbitrairement à plus ou moins \SI{20}{\percent}
de leur valeur d’origine.
La surface minimale de $PV$ a elle été définie afin de couvrir \SI{75}{\percent} (\SI{\approx
2330}{\kWh}) de la consommation des équipements internes sur Bordeaux (\SI{\approx 3082}{\kWh}).
De plus la même variation surfacique est considérée pour les capteurs $PV$ et
thermiques~: \SI{\approx 7}{\metre\squared} afin de pouvoir comparer leurs influences respectives.


\paragraph{Variables qualitatives} % (fold)
\label{par:variables_qualitatives}
La méthode de Morris nécessite que l’ensemble des variables soient quantitatives et chaque
paramètre varie indépendamment. Il n’est donc pas possible d’évaluer directement
l’influence des capteurs solaires ou des vitrages qui sont des variables qualitatives.

Afin de tenir compte de ces paramètres, leurs caractéristiques principales sont considérés
comme des paramètres indépendants. Dans le cas des capteurs solaires, le rendement optique
$\eta_{0}$ et les coefficients $a_{1}$ et $a_{2}$ sont retenues. Dans le cas des vitrages
les caractéristiques principales souvent utilisées sont le $U_{g}$ et le $g$ permettant
respectivement d’évaluer la résistance au transfert thermique et au passage de l’énergie
solaire. Les vitrages étant définies de manière détaillé, les paramètres $Émis_{ext}$
et $\tau_{sol}$ sont retenues. Bien que la méthode de Morris a été adapté afin de permettre de
regrouper un jeu de paramètre en un paramètre unique (création d’un groupe), ils restent
indépendants et l’approche n’est donc pas mise en place. Ainsi, les combinaisons réalisées
représentent des capteurs / vitrages hypothétiques dont la faisabilité technique n’est pas
garantie mais dont la diversité est représentative des capteurs existants. L’approche permet
aussi de mettre en exergue l’importance relative de chaque
caractéristique sur la performance du capteur ou du vitrage et son impact global au niveau
du $SSC$. De plus, les autres paramètres définies a priori peuvent tenir compte de la
forte variabilité des capteurs et des vitrages. En effet, il est possible qu’il existe des
interactions entre les vitrages ou capteurs solaires et les autres facteurs quantitatifs.
% paragraph variables_qualitatives (end)

\begin{table}
\centering
\caption{Liste des paramètres a priori utilisés pour l’analyse de sensibilité.}
\label{tab:facteur_sensibilite}
\begin{tabular}{l c c l}
  \toprule
  \addlinespace
                                               & Borne min     & Borne max   & Remarques                                                            \\
  \addlinespace
  \multicolumn{4}{l}{\bm{$SSC$}}                                                                           \\
  \midrule
  Nombre capteurs                              & \num{2}       & \num{5}     & \num{4.64} -- \SI{11.6}{\metre\squared}                              \\
  $\eta_{0}$                                   & \num{0.63}    & \num{0.84}  & \multirow{3}{*}{Diversité issue de \href{www.solar-rating.org}{ICC-SRCC}}   \\
  $a_{1}$                                      & \num{0.65}    & \num{6.7}   &                                                                      \\
  $a_{2}$                                      & \num{0.00069} & \num{0.29}  &                                                                      \\
  $Ech_{sol}^{pos}$                            & \num{0.8}     & \num{1.3}   & Position relative à la taille du ballon                              \\
  Volume ballon tampon                         & \num{100}     & \num{500}   & \multirow{2}{*}{Dimensions adaptées proportionnellement}             \\
  Volume ballon $ECS$                          & \num{100}     & \num{500}   &                                                                      \\
  $R$ ballon sanitaire                         & \num{7}       & \num{10}    & \multirow{2}{*}{Variation uniquement de l’épaisseur de l’isolant}    \\
  $R$ ballon tampon                            & \num{7}       & \num{10}    &                                                                      \\
  $Isolant_{réseau}^{épaisseur}$               & \num{0.013}   & \num{0.04}  & Résistance dépendant du nombre de capteurs                           \\
  $DeltaT_{sol}$                               & \num{5}       & \num{15}    &  -                                                                   \\
  \\
  \addlinespace[\defaultaddspace]
  \multicolumn{4}{l}{\textbf{Enveloppe du bâtiment}}                                                                              \\
  \midrule
  $R$ plancher                                 & \num{6}       & \num{10}    &  -                                                                   \\
  $R$ murs                                     & \num{4}       & \num{7}     &  -                                                                   \\
  $R$ plafond                                  & \num{6}       & \num{10}    &  -                                                                   \\
  $\tau_{sol}$                                 & \num{0.643}   & \num{0.849} & \multirow{2}{*}{Variation des vitrages Nord et Ouest uniquement}     \\
  $Émis_{ext}$                                 & \num{0.037}   & \num{0.837} &                                                                      \\
  Surface vitrée Est                           & \num{4.3}     & \num{6.46}  & \multirow{4}{*}{Surface totale \SI{26.4}{\metre\squared}}            \\
  Surface vitrée Nord                          & \num{0.46}    & \num{0.684} &                                                                      \\
  Surface vitrée Sud                           & \num{5.42}    & \num{8.13}  &                                                                      \\
  Surface vitrée Ouest                         & \num{2.3}     & \num{3.89}  &                                                                      \\
  \\
  \addlinespace[\defaultaddspace]
  \multicolumn{4}{l}{\textbf{Production d’électricité}}                                                                     \\
  \midrule
  Surface $PV$                                 & \num{12}       &  \num{19}   &  Capteurs thermiques prioritaires sur le pan Sud                                                             \\
  \bottomrule
  \end{tabular}
\end{table}
% subsection parametres_a_priori (end)



% - - - - - - - - - - - - - - - - - - - - - - - - - - - - - - - - - - - - - - -
\subsection{Analyse des résultats} % (fold)
\label{sub:analyse_des_resultats_morris}
L’analyse de Morris a été réalisée en considérant \num{15} trajectoires uniques à travers
\num{4} niveaux. Les résultats sont analysés sur les indicateurs caractéristiques d’un $SSC$
(le $F_{sol}^{CH}$, le $F_{sol}^{ECS}$, et la $Conso_{app}$) mais aussi sur les parts
actives et passives du chauffage solaire, respectivement notées $Prod_{sol}^{active}$ et
$Prod_{sol}^{passive}$. En effet, les ballons étant dans le bâtiment, une partie de
l’énergie solaire est fournie par leurs déperditions. Il est alors intéressant de
mieux comprendre les interactions existantes.


% - - - - - - - - - - - - - - - - - - - - - - - - - - - - - - - - - - - - - - -
\subsubsection{Couverture solaire sur l’eau chaude sanitaire} % (fold)
\label{ssub:couverture_solaire_sur_l_ECS}
Sur l’indicateur $F_{sol}^{ECS}$ (\figref{fig:objectifs_mu_star}) les facteurs ayant une influence linéaire sont
peu nombreux~: la nombre de capteur et ses caractéristiques ($a_{1}$, $a_{2}$, $\eta_{0}$)
ainsi que le volume du ballon sanitaire. La $Ech_{sol}^{pos}$ quand à elle a une influence
non-linéaire ou avec des interactions. L’ordre des facteurs influents reste sensiblement le
même pour Bordeaux et Strasbourg.
% subsubsection couverture_solaire_sur_l_ECS (end)


% - - - - - - - - - - - - - - - - - - - - - - - - - - - - - - - - - - - - - - -
\subsubsection{Couverture solaire sur le chauffage} % (fold)
\label{ssub:couverture_solaire_sur_le_chauffage}
Sur l’indicateur $F_{sol}^{CH}$ (\figref{fig:objectifs_mu_star}), le nombre de capteur, ces caractéristiques ($a_{1}$,
$a_{2}$, et $\eta_{0}$), le volume des deux ballons et la performance thermique des
vitrages ($Émis_{ext}$) sont tous influents. Pour Bordeaux comme Strasbourg le volume du
ballon $ECS$ a un effet non linéaire ou avec des interactions. Cependant, il peut aussi
être noté que l’$Émis_{ext}$ a un effet non linéaire sur Strasbourg alors qu’il est
fortement linéaire sur Bordeaux. Aussi, le $DeltaT_{sol}$ et la $Ech_{sol}^{pos}$ ont
aussi tous deux des effets non linéaire ou avec des interactions mais l’impact est
important uniquement sur le climat Bordelais. De même la résistance thermique de
l’enveloppe ($R$ murs et $R$ plafond) a un impact linéaire uniquement sur Bordeaux.

Afin de mieux comprendre l’influence de chaque facteur, il est intéressant d’analyser
l’impact sur la $Prod_{sol}^{CH}$ passive et la $Prod_{sol}^{CH}$ active qui forment à eux
deux la $Prod_{sol}^{CH}$ (\figref{fig:prod_sol_chauffage_mu_star}, \figref{fig:prod_sol_chauffage_mu}).
Pour les deux climats le volume du ballon tampon influence principalement la
$Prod_{sol}^{CH}$ passive. Sur Strasbourg, ce facteur est d’ailleurs le plus influent
devant le nombre de capteurs. À l’opposé l’$DeltaT_{sol}$ et $Émis_{ext}$ influence
principalement la $Prod_{sol}^{CH}$ active sur Bordeaux comme Strasbourg. Il est aussi
clair que le $DeltaT_{sol}$ a un effet fortement non-linéaire ou avec de fortes
interactions sur Bordeaux. Bien que non influent si on considère la $Prod_{sol}^{CH}$, la
performance de l’isolation des ballons est un facteur impactant sur la part passive.
D’autre part, il apparaît que la surface vitrée Sud est influente (non- linéaire ou avec
des interactions) sur la $Prod_{sol}^{CH}$ mais pas sur le $F_{sol}^{CH}$. Enfin le nombre
de capteurs et leur performance sont influents autant sur la part passive que la part
active.

Il est donc mis en exergue la complexité liée à l’évaluation couplée d’un système
et du bâtiment. Il existe de nombreux facteurs influents autant sur l’enveloppe que
sur le système. Certains influence principalement la part passive, d’autres la part
active, et enfin d’autres les deux. Enfin, de nombreux facteurs ont un impact non-linéaire
ou avec des interactions en particulier pour le climat Bordelais.
% subsubsection couverture_solaire_sur_le_chauffage (end)

% - - - - - - - - - - - - - - - - - - - - - - - - - - - - - - - - - - - - - - -
\subsubsection{Consommation de l’appoint} % (fold)
\label{ssub:consommation_de_l_appoint}
Les résultats sur les objectifs principaux (\figref{fig:objectifs_mu_star}) montrent que
les paramètres $a_{1}$, $a_{2}$, $\eta_{0}$, $Émis_{ext}$, le volume du ballon $ECS$, la
résistance thermique des murs et du plafond, ainsi que le nombre de capteurs thermiques,
sont tous influents. Seul le volume du ballon $ECS$ a une influence non-linéaire ou avec
des interactions. Il apparaît aussi que le niveau d’isolation des ballons comme des
canalisations ne soit pas très important. Pour Strasbourg, contrairement à Bordeaux, de
nombreux facteurs liés à l’enveloppe sont impactant. La résistance du plancher et le
$\tau_{sol}$, ont en effet un impact linéaire, et les surfaces vitrées au Sud comme à
l’Est un impact non-linéaire ou avec interactions. Aussi l’isolation des vitrages
($Émis_{ext}$) est le facteur le plus influent sur Strasbourg, devant le nombre de
capteur.

Lorsque seule la $Conso_{app}^{CH}$ (\figref{fig:conso_app_mu_star}, \figref{fig:conso_app_mu})
est considérée, il est clair que l’enveloppe joue un rôle important même pour Bordeaux où
l’$Émis_{ext}$ est le facteur le plus influent suivit par le nombre de capteur et la
résistance des murs. Pour Bordeaux, il peut aussi être noté que la $Conso_{app}^{CH}$ est
influencée par la surface de vitrage au Sud mais les besoins en chauffage étant moins
important que les besoins en $ECS$ l’indicateur n’est pas parmi les plus influents sur la
$Conso_{app}$. Enfin pour Bordeaux comme Strasbourg, le volume du ballon tampon comme la
$Ech_{sol}^{pos}$ sont influent sur la $Conso_{app}^{ECS}$. Ainsi au regard de résultats
il apparaît que la performance du $SSC$ est plus impactée par les caractéristiques de
l’enveloppe lorsque les conditions extérieures sont plus rudes.
% subsubsection consommation_de_l_appoint (end)


% - - - - - - - - - - - - - - - - - - - - - - - - - - - - - - - - - - - - - - -
\subsubsection{Consommation totale} % (fold)
\label{ssub:consommation_totale}
Sur la $Conso_{totale}$, il est observé une très forte influence linéaire de la surface de
$PV$. En comparaison, les autres paramètres ont une influence modérée sur Bordeaux~:
nombre des capteurs et ses caractéristiques, volume du ballon sanitaire, isolation des
vitrages, des murs, et du plafond. La surface de $PV$ est le seul facteur permettant de
couvrir la consommation électrique des équipements internes. Ces équipements représentant
la majorité des consommations pour Bordeaux, il est normal que ce facteur soit très
influents. Par contre sur Strasbourg, où la $Conso_{app}$ est importante, et plus
particulièrement la $Conso_{app}^{CH}$ les facteurs caractérisant l’enveloppe ont une plus
grande importance. Ainsi sur Strasbourg, le second facteur le plus influent est
l’$Émis_{ext}$ alors que c’est le nombre de capteurs pour Bordeaux. Enfin, le volume du
ballon sanitaire a un impact non-linéaire ou avec des interactions et linéaire
respectivement pour Bordeaux et Strasbourg.
% subsubsection consommation_totale (end)


% - - - - - - - - - - - - - - - - - - - - - - - - - - - - - - - - - - - - - - -
\subsubsection{Valorisation de l’énergie solaire~:} % (fold)
\label{ssub:valorisation_de_l_energie_solaire_}
L’analyse de l’influence des facteurs (\figref{fig:prod_sol_valorisee_mu_star}) montre que
les $Pertes_{réseau}$ sont fortement impactées par la surface totale de capteur comme de
l’épaisseur de l’isolant au niveau des canalisations. Plus intéressant, les résultats
montrent que le volume du ballon sanitaire a aussi un impact important. Afin de mieux
comprendre le signe de l’influence, il est nécessaire de regarder l’évolution de la
moyenne ($\mu$) (\figref{fig:prod_sol_valorisee_mu}). En comparant l’évolution de
$\mu^{*}$ et de $\mu$ il est clair que augmenter le volume influe négativement les
$Pertes_{réseau}$. Il est aussi observé que pour Bordeaux comme Strasbourg, l’impact est
avec interaction ou non linéaire même si l’effet est plus important sur Bordeaux. La
$Prod_{sol}$ étant plus importante sur Bordeaux que Strasbourg, les $Pertes_{réseau}$ sont
aussi augmentées. Dans une moindre mesure le $DeltaT_{sol}$ est aussi influent et a des
interaction non linéaires ou avec des interactions. Contrairement aux observations
réalisés sur les indicateurs sélectionnées comme des objectifs, le facteur semble plus
influent sur Strasbourg.

Maintenant si on s’intéresse à l’effet de ces pertes sur la $Prod_{sol}^{valorisée}$, la
performance thermiques des canalisations ne sont plus importantes. Cependant le volume du
ballon sanitaire est toujours influent mais les effets sont cette fois linéaires. Il est
aussi possible d’identifier un nouveau facteur influent de manière linéaire~: le volume du
ballon tampon. L’explication est simple~: la proportion des $Pertes_{réseau}$ est moins
importante que la $Prod_{sol}$, ainsi les facteurs les plus influents pour la
$Prod_{sol}^{valorisée}$ sont ceux ayant une influence sur la $Prod_{sol}$.

Il est aussi intéressant de voir que la performance des vitrages ($Émis_{ext}$) est un
facteur impactant avec des effets non linéaires ou des interactions. Il semble en effet
que la dégradation de la performance des fenêtre permette de valoriser plus d’énergie
solaire. Pour les deux climats, l’influence est similaire mais il a été noté que la
dégradation de l’$Émis_{ext}$ des vitrages impacte fortement la $Conso_{app}$
particulièrement pour Strasbourg. Le solaire ne semble donc capable de couvrir que une
partie de l’augmentation des besoins.
% subsubsection valorisation_de_l_energie_solaire_ (end)


\begin{figure}
    \centering
    \includegraphics{Ressources/Images/Sensibilite/sigma_mu_star_objectifs.pdf}
    \caption{Résultat de l’analyse de Morris pour les objectifs principaux
             ($f(\mu^{*}) = \sigma$).}
    \label{fig:objectifs_mu_star}
\end{figure}


\begin{landscape}
    \begin{figure}
        \centering
        \includegraphics{Ressources/Images/Sensibilite/graphInfluence.pdf}
        \caption[Graphe d’influence du $SSC$ pour Bordeaux et Strasbourg]
                {Graphe d’influence du $SSC$ pour Bordeaux et Strasbourg avec les
                 relations linéaires (noir), non-linéaires (bleu), et indirectes (pointillées).}
        \label{fig:graphe_influence_objectifs}
    \end{figure}
\end{landscape}
% subsection analyse_des_resultats (end)



% - - - - - - - - - - - - - - - - - - - - - - - - - - - - - - - - - - - - - - -
\subsection{Paramètres retenus} % (fold)
\label{sub:parametres_retenus}
Au regard de cette analyse, un récapitulatif des paramètres conservés permet de voir que
des facteurs à la fois sur l’enveloppe, et le système ont été retenus. Au cours de
l’optimisation, des capteurs et vitrages existants sont utilisé afin d’être cohérent avec
les performances existantes sur le marché (\tabref{tab:capteurs_specs_optimisation},
\tabref{tab:carac_vitrages}). Il peut aussi être noté que l’enveloppe impacte de manière
plus importante le $SSC$ sur le climat Strasbourgeois où le climat est plus rude. Il
existe ainsi des compromis intéressants entre la qualité de l’enveloppe, la surface de
$PV$, et les caractéristiques du $SSC$. Enfin le criblage a permis de passer de
\num{19} paramètres à \num{11} pour Bordeaux et \num{12} pour Strasbourg. Le
nombre de combinaisons à évaluer est ainsi fortement réduit.


\begin{table}
\centering
\caption{Liste des paramètres retenus pour l’optimisation.}
\label{tab:facteur_retenues}
\begin{tabular}{l c c c c l}
  \toprule
  \addlinespace
                       & Min        & Max         & Catégorie  & Pas        & Remarques                                \\
  \addlinespace
  \multicolumn{5}{l}{\bm{$SSC$}}         \\
  \midrule
  Nombre capteurs      & \num{2}    & \num{5}     & Discrète    & \num{1}    & \num{4.64} -- \SI{11.6}{\metre\squared}   \\
  Type de capteur      & -          &  -          & Qualitative & -          & Voir \tabref{tab:capteurs_specs_optimisation}   \\
  $Ech_{sol}^{pos}$    & \num{0.8}  &  \num{1.3}  & Continue    & -          & Position relative à la taille du ballon     \\
  Volume ballon tampon & \num{100}  &  \num{500}  & Discrète    & \num{50}   & \multirow{2}{*}{Dimensions adaptées proportionnellement}   \\
  Volume ballon $ECS$  & \num{100}  &  \num{500}  & Discrète    & \num{50}   &    \\
  $DeltaT_{sol}$       & \num{5}    &  \num{15}   & Continue    & -          &  \emph{Uniquement pour Bordeaux}      \\
  \\
  \addlinespace[\defaultaddspace]
  \multicolumn{4}{l}{\textbf{Enveloppe du bâtiment}}             \\
  \midrule
  $R$ murs             & \num{4}    &  \num{7}    & Discrète    & \num{0.5}  & -                                  \\
  $R$ plafond          & \num{6}    &  \num{10}   & Discrète    & \num{0.5}  & -                                                                      \\
  $R$ plancher         & \num{6}    &  \num{10}   & Discrète    & \num{0.5}  & \emph{Uniquement pour Strasbourg}                                                                     \\
  Surface vitrée Sud  & \num{5.42} &  \num{8.13} & Continue    &  -         & -       \\
  Surface vitrée Est  & \num{4.3}  &  \num{6.46} & Continue    &  -         &  \emph{Uniquement pour Strasbourg} \\
  Type de vitrage      & -          &  -          & Qualitative &  -         & Voir \tabref{tab:carac_vitrages} \\
  \\
  \addlinespace[\defaultaddspace]
  \multicolumn{5}{l}{\textbf{Production d’électricité}}      \\
  \midrule
  Surface PV           & \num{14}   &  \num{30}   & Discret    &  \num{1}   & Capteurs thermiques prioritaires sur le pan Sud   \\
  \bottomrule
\end{tabular}
\end{table}

\begin{table}
\centering
\itodo{Corriger les surfaces en fonction de la méthode de calcul}
\caption{Caractéristiques des panneaux solaires.
\label{tab:capteurs_specs_optimisation}}
\begin{tabular}{l c c c c r}
    \toprule
                                 & IDMK\,25             & 308C\,HP             & 12\,CPC58      & ECO 25        & Unité                       \\
    \midrule
    Fabricant                    & Sonnenkraft          & Radco                & Sky Pro        & Dima          & -                           \\
    Type                         & Plan vitrée          & Plan vitrée          & Tubulaire      & Plan vitrée   & -                           \\
    Surface nette                & \num{2.32}           & \num{2.193}          & \num{2.28}     & \num{2.312}   & \si{m^{2}}                  \\
    Poids à vide                 & \num{54}             & \num{36}             & \num{53}       & \num{41}      & \si{kg}                     \\
    Contenance                   & \num{1.35}           & \num{3.5}            & \num{1.83}     & \num{1.9}     & \si{\litre}                 \\
    $\eta_{0}$                   & \num{78}             & \num{83.4}           & \num{63}       & \num{66.4}    & \si{\percent}                     \\
    $a_{1}$                      & \num{3.796}          & \num{1.4539}         & \num{0.9249}   & \num{4.9510}  & \si{W/(m^{2}\period K)}     \\
    $a_{2}$                      & \num{0,013}          & \num{0.0589}         & \num{0.00069}  & \num{0.01527} & \si{W/(m^{2}\period K^{2})} \\
    $IMDiff$                     & \num{100}            & \num{96}             & \num{102}      & \num{93}      & \si{\percent}                     \\
    \bottomrule
\end{tabular}
\end{table}

\begin{table}
\centering
\caption{Descriptif des caractéristiques (suivant \cite{NFEN410} et \cite{NFEN673})
         des différents vitrages envisagés.}
\label{tab:carac_vitrages}
\begin{tabular}{l c c c r}
  \toprule
                     & Planitherm XN       & Planitherm ONE       & OptiwhiteKGlass       & Unité                        \\
  \midrule
  Fabricant    & \href{http://fr.saint-gobain-glass.com/product/2422/sgg-planitherm-xn}{%
                       St Gobain}
               & \href{http://eg.saint-gobain-glass.com/product/1659/}{%
                       St Gobain}
               & \href{https://www.pilkington.com/en-gb/uk/products/product-categories/thermal-insulation/pilkington-k-glass-range/pilkington-k-glass}{%
                       Pilkington}                                                              & -                             \\
  Construction & \num{4}-16-4              & \num{4}-16-4            & \num{4}-16-4             & -                             \\
  Gaz          & Argon                     & Argon                   & Argon                    & -                             \\
  $U_{g}$      & \num{1}.1                 & \num{1}.0               & \num{1}.5                & \si{W/(m^{2}\period \kelvin)} \\
  $g$          & \num{82}                  & \num{49}                & \num{78}                 & \si{\percent}                 \\
  \bottomrule
    \end{tabular}
\end{table}
% subsection parametres_retenus (end)
% section reduction_de_la_cardinalite (end)




% ..............................................................................
% ..............................................................................
\section{Construction d’un modèle de substitution} % (fold)
\label{sub:construction_d_un_modele_de_substitution}
\subsection{Création de l’échantillon} % (fold)
\label{sub:creation_de_l_echantillon}
Afin de réduire la durée de simulation le modèle détaillée peut être substitué à un méta-
modèle. Comme décrit dans le chapitre précédent un ensemble représentatif des combinaisons
possibles est nécessaire afin de construire un modèle valide. Grâce à une méthode
de criblage réalisée en amont la taille de l’échantillon nécessaire pour construire le
modèle est réduit. Il a été construit grâce à une méthode de pseudo-Monte-Carlo
en considérant la suite de Halton comme générateur pseudo-aléatoire. Les solutions construites
sont donc équitablement réparties, assurant une bonne représentativité de l’espace de décision.
% subsection creation_de_l_echantillon (end)

\subsection{Création des modèles de substitution} % (fold)
\label{sub:creation_des_modeles_de_substitution}
Une fois l’échantillon simulé, la bibliothèque \textit{OpenTurns} a été utilisée
afin de construire les \num{3} méta-modèles. La $Prod_{sol}$ étant pré-calculée,
un modèle est uniquement nécessaire pour le $F_{sol}^{ECS}$, le $F_{sol}^{CH}$, et la $Conso_{app}$.
Bien que la $Conso_{app}$ ne soit pas un objectif de l’optimisation, sa connaissance est
nécessaire afin de déterminer la contrainte~: la $Conso_{totale}$.

Étant en phase de conception et cherchant à explorer l’ensemble du domaine de recherche,
une loi uniforme est utilisée pour chaque paramètre durant la construction des méta-modèles.
Afin d’évaluer leurs précisions par rapport au modèle de référence, l’erreur
quadratique moyenne ($RMSE$, Root Mean Square Error) et l’erreur absolue maximale ($MAE$,
Maximal Absolute Error) sont utilisées. La $RMSE$ permet d’évaluer l’écart moyen entre
les sorties du méta-modèle ($\mathcal{M}$) et du modèle de référence ($f$)
\eqref{eq:rmse}. Le second, la $MAE$, permet d’obtenir l’écart maximal existant entre le
méta-modèle et le modèle de référence \eqref{eq:mae}. Ainsi ces deux variables statistiques
permettent d’évaluer la qualité de l’approximation du méta-modèle.

\begin{align}
  \label{eq:rmse}
  RMSE &= \sqrt{\frac{1}{N}\sum^{N}_{i=1} \left[ f(\vec{x}_{i}) - \mathcal{M}(\vec{x}_{i}) \right]^{2} } \\
  \label{eq:mae}
  MAE  &= max \left( \abs{f(\vec{x}_{i}) - \mathcal{M}(\vec{x}_{i})}, \: x = 1, 2, \dotsc, N \right)
\end{align}

\itodo{Décrire choix variable pour capteurs et vitrages}
Afin de représenter les variables qualitatives, vitrages et capteurs, leurs variables
caractéristiques sont utilisés en substitution. Au regard des résultats il apparaît que
les capteurs sont mieux approximer si on le substitue par sa valeur pour le coefficient
$a_{1}$. Pour les vitrages il semble que l’approximation est meilleur si on utilise en
substitut la valeur de l’$Émis_{ext}$  (\mtodo{ref tab annexe}).

\itodo{Ajouter analyse des courbes une fois étude finies}
Une étude paramétrique est ensuite réalisée afin de déterminer la taille minimale
nécessaire pour construire un méta-modèle approximant au mieux le modèle de référence.
L’évolution de la $RMSE$ (\figref{fig:rmse}) et de la $MAE$ (\figref{fig:rmse}) est
alors analysé.


\begin{figure}
    \centering
    \includegraphics{Ressources/Images/MetaModele/RMSE.pdf}
    \caption[Évolution de la $RMSE$ en fonction de l’échantillon]
            {Évolution de la $RMSE$ pour les \num{3} méta-modèles
             en fonction de la taille de l’échantillon.}
    \label{fig:rmse}
\end{figure}

\begin{figure}
    \centering
    \includegraphics{Ressources/Images/MetaModele/MAE.pdf}
    \caption[Évolution de la $MAE$ en fonction de l’échantillon]
            {Évolution de la $MAE$ pour les \num{3} méta-modèles
             en fonction de la taille de l’échantillon.}
    \label{fig:mae}
\end{figure}

Un méta-modèle d’ordre \num{3} est donc retenu pour chaque indicateur, le $F_{sol}^{ECS}$, le $F_{sol}^{CH}$,
et la $Conso_{app}$ dont les erreurs statistiques sont fournies à travers le \tabref{tab:meta_retenus}.
La comparaison entre les méta-modèles et le modèle de référence est aussi illustré pour chaque
point de l’échantillon non utilisés pour sa construction (\figref{fig:validite_meta}).

\begin{figure}
    \centering
    \includegraphics{Ressources/Images/MetaModele/validite_meta.pdf}
    \caption[Validité des méta-modèles]
            {Évaluation de la validité des méta-modèles pour un échantillon
             de taille \num{1000}.}
    \label{fig:validite_meta}
\end{figure}

\begin{table}
\centering
\caption{Erreurs caractéristiques ($RMSE$ et $MAE$) obtenues pour les \num{3} méta-modèles retenus
         (taille de l’échantillon de \num{1000})
\label{tab:meta_retenus}}
\begin{tabular}{l c c c c r}
    \toprule
                    & \multicolumn{2}{c}{Bordeaux} & \multicolumn{2}{c}{Strasbourg} & \multirow{2}{*}{Unité} \\
                    \cmidrule(r){2-3}
                    \cmidrule(r){4-5}
                    & $RMSE$ & $MAE$               & $RMSE$ & $MAE$                 &                        \\
    \midrule
    $F_{sol}^{ECS}$ &  & & & & \si{\percent} \\
    \addlinespace[\defaultaddspace]
    $F_{sol}^{ECS}$ &  & & & & \si{\percent} \\
    \addlinespace[\defaultaddspace]
    $Conso_{app}$   &  & & & & \si{kWh}      \\
    \bottomrule
\end{tabular}
\end{table}


% subsection creation_des_modeles_de_substitution (end)
% section construction_d_un_modele_de_substitution (end)




% ..............................................................................
% ..............................................................................
\section{Vers une solution adaptée} % (fold)
\label{sec:vers_une_solution_adaptee}
% ------------------------------------------------------------------------------
\subsection{Optimisation multi-objectif} % (fold)
\label{sub:optimisation_multi_objectif}
~
\itodo{Décrire l’évolution du front et le panel de solution obtenue}
\itodo{Décrire l’exploration de l’espace}
\ftodo{Représentation 3D}
\ftodo{Représentation 2D}
\ftodo{Évolution du front en fonction des itérations}
\ftodo{Identification de sous-groupe par couleur}
% ------------------------------------------------------------------------------
\subsection{Aide à la décision a posteriori} % (fold)
\label{sub:aide_a_la_decision_a_posteriori}
~
\itodo{Description des critères qui rentre en jeu~: coût, Suface disponible,
       fournisseurs locaux, type de clientèle, réglementation, ...}
\itodo{Ajouter un exemple avec XDAT}
\itodo{Tableau ou figure}
\ftodo{Ajouter screenshot de XDAT avec la/les solutions retenues}
\ttodo{Ajouter tableau avec caractéristiques des solutions retenues}

% subsection aide_a_la_decision_a_posteriori (end)
% subsection optimisation_multi_objectif (end)
% section vers_une_solution_adaptee (end)
