%!TEX root = ../main.tex


\itodo{Vérifier définition maître oeuvre et ouvrage}
\itodo{Vérifier définition des phases de construction APD, ...}
\itodo{Renseigner plus sur PAC CO2 et absorption}




\itodo{Conclusion}
Cette thèse s’est intéressée aux \abr{MEPOS} solaires et en particulier au développement
d’une méthodologie multi-critères et multi-objectifs permettant d’accompagner les concepteurs dans leur prise de décision
pour le dimensionnement couplée du \abr{SSC} et du bâtiment. Le choix a en
effet été fait de considérer le bâtiment, le \abr{SSC} et son algorithme de contrôle comme
un ensemble cohérent et fonctionnant de concert. S’inspirant de la norme
\abr{PREN\,$15603$} et basée sur une d’aide à la décision a posteriori, la méthodologie
s’est montrée pertinente à la fois pour évaluer les interactions existantes mais
aussi pour proposer un large choix de solutions optimales. Parmi ces solutions, des
compromis intéressants ont été identifiés, invitant à repenser les choix de conception
lors du développement de \abr{SSC} couplés à des \abr{MEPOS}.


** chapter 1
Il a été vu que la construction d’un \abr{MEPOS} est un problème multi-objectif complexe
faisant intervenir de nombreux facteurs. De plus il n’existe pas encore à l’heure actuelle
de définition précise en France même si un cadre européen se construit à travers la norme
\abr{PREN\,$15603$}. Cette norme invitent à favoriser l’utilisation des énergies renouvelables
sur site afin de favoriser une part importante d’auto-consommation et ainsi limiter le
déséquilibre des réseaux. De part sa simplicité, le solaire thermique et par extension
le \abr{SSC} est un candidat pertinent pour couvrir une part importante des besoins
en adéquation avec la demande, favorisant ainsi l’auto-consommation. Pourtant cette
technologie n’est pas utilisée pour la constructions de \abr{BEPOS} et il n’existe
pas de travaux cherchant à caractériser le potentiel d’un tel couplage en tenant
compte à la fois, de l’enveloppe, du système, et de l’algorithme de contrôle.
Il est pourtant nécessaire avec l’augmentation de la complexité de tenir compte
des interactions existantes mais il est aussi important de tenir compte des contraintes
de temps disponibles pour la conception d’un bâtiment.
C’est pour répondre à ces trois contraintes principales que la méthodologie développée
dans ces travaux a été pensée.



** chapter 2
Parallèlement au développement de la méthodologie, un bâtiment couplé à un \abr{SSC} a été modélisé.
Le langage \textit{Modelica} et le logiciel \textit{Dymola} ont été retenus car
ils sont particulièrement adaptés au processus de développement itératif nécessaire
dans cette application. Dans
un premier temps, il a été décrit les scénarios et l’enveloppe du bâtiment de référence
et la comparaison entre un modèle multi-zone et mono-zone a montré que l’approche simplifiée
est suffisante dans le contexte de ces travaux.





Le bâtiment est un système complexe
importance algorithme


** chapter 3
Favorisant le temps machine à travers un processus d’optimisation multi-objectifs
le concepteur

la recherche de solutions optimales, proposer intéractivement une aide à la décision,
modélisation dé



** chapter 4
Mis en pratique sur deux cas d’études, la méthodologie s’est montrée efficace
et a permis d’identifier de nombreuses combinaisons pertinentes.
L’approche a permis d’explorer un domaine très large et


\itodo{Perspectives}
Dans la continuité de ces travaux, il est identifié trois axes principaux~: la validation
expérimentale, une évaluation économique détaillée, et la caractérisation du potentiel
d’un \abr{SSC} en période hivernale comme estivale.

Comme explicité dès le second chapitre, les réponses et observations faites au cours de
cette thèse nécessitent d’être confrontées aux résultats d’une expérimentation à échelle
$1$. En effet même si les composants du modèle ont fait pour la plupart l’objet d’une
vérification analytique, le modèle dans son ensemble nécessitent d’être validés
expérimentalement afin de corroborer les conclusions de ces travaux. La méthodologie
développée par contre reste valable et elle a d’ailleurs montrée sa capacité à explorer le
domaine de définition pour proposer de nombreuses combinaisons pertinentes. Il serait
alors uniquement nécessaire de vérifier que le modèle traduit correctement le comportement
du réel du \abr{SSC}. En plus de la validation, il serait intéressant d’évaluer les
risques de surchauffes et de caractériser les méthodes existantes pour évacuer la chaleur.
soit en la valorisant pour une autre application comme le chauffage d’une piscine, soit
en déchargeant le système par exemple en faisant circuler l’eau du \abr{SSC} dans les
capteurs solaires en période nocturne.

Afin de compléter l’étude, il semble pertinent d’intégrer la prise en compte de manière
quantitative du critère économique à travers le coût d’investissement et le temps de
retour. Il est cependant nécessaire de rester très prudent quand à leur implémentation en
particulier pour le temps de retour sur investissement qui doit tenir compte de la
complexité inhérente à ce type d’analyse comme explicité à travers le chapitre précédent.
De plus, comme noté à travers le premier chapitre, les résultats obtenus sont fortement
dépendant des données météos utilisées. Le rayonnement solaire en particulier est très
changeant d’une année à l’autre, il serait alors pertinent d’évaluer la variation de la
performance du \abr{SSC} pour une période étendue de simulation. Cette analyse peut être
réalisée avec l’aide de données météorologiques existantes mais une étude prospective sur
l’évolution du climat futur est plus profitable. En effet afin d’être complète, l’analyse
du retour sur investissement doit tenir compte de l’évolution du coût des énergies mais
aussi des besoins du bâtiment en partie liée à l’évolution du climat. Il apparaît ainsi
judicieux de coupler la construction de fichiers météorologiques prospectifs à la
construction de scénarios économiques afin de pouvoir quantifier le potentiel énergétique
et économique du \abr{SSC} selon différents scénarios tout en tenant compte du
vieillissement des équipements. Afin de répondre à ces perspectives, il peut être envisagé
un couplage entre la méthodologie développée et une approche par clustering afin de
réduire chaque année à un jeu de semaines représentatives et donc limiter le temps de
simulation nécessaire. Ainsi les résultats issus de la comparaison entre les deux
approches pourrait ainsi amener à revoir les moyens à mettre en œuvre pour évaluer la
performance d’un \abr{SSC} de manière plus complète.

Ces travaux ont permis d’identifier des solutions techniques permettant d’obtenir un taux
d’économie important en réduisant la surface de capteurs mais surtout la taille des
ballons, levant ainsi les principales contraintes qui freinent l’adoption de tels
systèmes. Cependant avec l’amélioration de l’enveloppe des bâtiments et l’augmentation des
consommations dues aux charges internes, le confort estival est aussi un facteur clé. Le
solaire étant abondant durant l’été, un travail approfondie sur le dimensionnement d’un
\abr{SSC} à la fois pour couvrir les besoins hivernaux et estivaux apparaît comme une suite
logique à ces travaux et viendrait en complément à la tâche $53$ actuellement en cours.
Parmi les technologies potentielles, il serait intéressant d’investiguer le couplage entre
un système solaire et une \abr{PAC} à $CO_{2}$ plus respectueuse de l’environnement que
que les \abr{PAC} plus classiques qui nécessitent l’utilisation d’un fluide frigorigène
polluant. Ce couplage présente aussi un avantage énergétique, grâce à l’énergie solaire,
le coefficient de performance (\abr{COP} pour le chauffage ou \abr{EER} pour le
refroidissement) d’une \abr{PAC} peut être maintenue à un niveau important durant
l’ensemble de l’année. Ainsi le potentiel solaire est double et permet à la fois de
réduire la consommation en se substituant à l’énergie électrique mais aussi en améliorant
la performance de la \abr{PAC}~: le gain énergétique est donc potentiellement plus
important que avec une batterie électrique. Une autre alternative possible serait
l’utilisation d’une pompe à absorption. Le principe est similaire à celui d’une \abr{PAC}
mais le compresseur est remplacé par un bouilleur limitant ainsi la consommation
électrique, le bruit, et améliorant la robustesse du système. Il n’existe cependant pas
actuellement sur le marché de pompe à absorption de puissance assez faible pour convenir à
une application pour une \abr{MEPOS}. De plus, le bouilleur nécessite une source chaude à
une température plus élevée que une \abr{PAC} classique et des capteurs solaires
cylindro-paraboliques pourraient être plus adaptés. La méthodologie développée dans ces travaux
pourrait ainsi être étendue afin de caractériser le potentiel du couplage avec une
\abr{PAC} ou d’une pompe à absorption. Bien sur dans cet optique, il apparaît pertinent de
tenir compte du confort d’été en s’orientant vers une modélisation du bâtiment plus
détaillée.



Conclusion:
Ces travaux représentent une avancée nécessaire sur la compréhension des $SSC$ pour
les bâtiment de demain. L’influence des différents paramètres, les limites et atouts
de ces systèmes a ainsi pu être détaillée à travers une étude paramétrique, un criblage
global et les résultats d’une optimisation multi-objectif.

Il est clair que la connaissance acquise doit être mise à profis afin de développer
les $SSC$ pour la maison individuelle. Plusieurs axes peuvent ainsi être proposées.
Dans l’optique d’une aide à la décision, il est possible d’évaluer la pertinence
des autres méthodes d’aide à la décision a posteriori même si elle me semble dans
l’optique d’une interaction entre maître d’ouvrage et d’oeuvre plus complexe à mettre
en place.

