%!TEX root = ../main.tex


\itodo{Vérifier définition maître oeuvre et ouvrage}
\itodo{Vérifier définition des phases de construction APD, ...}
\itodo{Renseigner plus sur PAC CO2 et absorption}



Cette thèse s’est intéressée aux \abr{MEPOS} solaires et en particulier au développement
d’une méthodologie multi-critères et multi-objectifs permettant d’accompagner les concepteurs dans leur prise de décision
pour le dimensionnement couplée du \abr{SSC} et du bâtiment. Le choix a en
effet été fait de considérer le bâtiment, le \abr{SSC} et son algorithme de contrôle comme
un ensemble cohérent et fonctionnant de concert. S’inspirant de la norme
\abr{PREN\,$15603$} et basée sur une d’aide à la décision a posteriori, la méthodologie
s’est montrée pertinente à la fois pour évaluer les interactions existantes mais
aussi pour proposer un large choix de solutions optimales. Parmi ces solutions, des
compromis intéressants ont été identifiés, invitant à repenser les choix de conception
lors du développement de \abr{SSC} couplés à des \abr{MEPOS}.

\paragraph{} % (fold)
Il a été vu que la construction d’un \abr{MEPOS} est un problème multi-objectif complexe
faisant intervenir de nombreux facteurs. De plus il n’existe pas encore à l’heure actuelle
de définition précise en France même si un cadre européen se construit à travers la norme
\abr{PREN\,$15603$}. Cette norme invitent à favoriser l’utilisation des énergies renouvelables
sur site afin de favoriser une part importante d’auto-consommation et ainsi limiter le
déséquilibre des réseaux. De part sa simplicité, le solaire thermique et par extension
le \abr{SSC} est un candidat pertinent pour couvrir une part importante des besoins
en adéquation avec la demande, favorisant ainsi l’auto-consommation. Pourtant cette
technologie n’est pas utilisée pour la constructions de \abr{BEPOS} et il n’existe
pas de travaux cherchant à caractériser le potentiel d’un tel couplage en tenant
compte à la fois, de l’enveloppe, du système, et de l’algorithme de contrôle.
Il est pourtant nécessaire avec l’augmentation de la complexité de tenir compte
des interactions existantes mais il est aussi important de tenir compte des contraintes
de temps disponibles pour la conception d’un bâtiment. Il est alors mis en évidence les
points suivants~:
\begin{itemize}
    \item La logique de contrôle doit faire partie du processus de décision.
    \item La conception d’une \abr{MEPOS} solaire doit être le compromis entre qualité
          de l’enveloppe et performance du \abr{SSC}.
    \item La taille des équipements du \abr{SSC} doit être réduite afin d’être viable
          économiquement.
    \item Un regard neuf est nécessaire pour repenser la conception des \abr{SSC}
          afin de les revaloriser dans le cadre des \abr{MEPOS}.
\end{itemize}

Ainsi fort de l’expérience acquise à travers l’analyse de l’état de l’art et des tendances
au niveau industriel, un bâtiment couplé à un \abr{SSC} a été dans un premier temps
modélisé les principaux composants validés analytiquement. Le langage \textit{Modelica} et le logiciel \textit{Dymola} ont été retenus car
ils sont particulièrement adaptés au processus de développement itératif nécessaire pour
cette application. Pour le bâtiment, un modèle mono-zone a été retenu après une
comparaison avec une approche multi-zone plus précise sous \textit{EnergyPlus}.
Concernant le \abr{SSC}, le système retenu utilise l’air comme vecteur de chaleur afin
d’améliorer la réactivité du bâtiment et éviter l’installation d’un système de chauffage
conventionnel. De plus, le choix a été fait de retenir une modélisation détaillée afin de
pouvoir contrôler l’ensemble des facteurs influents que ce soit sur les caractéristiques
du système ou sur les paramètres de la logique de contrôle et ainsi pouvoir affiner la
logique de contrôle pour favoriser la production solaire. Les premiers résultats sont
obtenus en suivant la méthodologie couramment utilisée dans les bureaux d’études, à savoir
une approche paramétrique~: chaque évaluation est réalisée à partir de l’expérience
acquise en amont et l’enveloppe du bâtiment est fixée. Bien que les résultats aient permis
d’identifier le potentiel du couplage d’une \abr{MEPOS} à un \abr{SSC}, l’approche ne
permet pas d’évaluer de manière complète le domaine de définition. Ainsi ces premiers
résultats confirment en majeure partie les observations déjà documentées par la
littérature comme la nécessité d’opter pour des volumes importants ou de favoriser en
priorité la production d’\abr{ECS}. Il est cependant aussi mis en avant un comportement
caractéristique du système étudié. Le \abr{SSC} développé est capable de répartir
l’énergie solaire disponible en fonction des caractéristiques des équipements retenus,
soit pour réduire la consommation sur la production d’\abr{ECS}, soit sur la consommation
du chauffage.

Une fois le potentiel du \abr{SSC} développé et les limites de l’approche paramétrique
mis en exergue, il est proposer une méthodologie plus adaptée permettant d’accompagner
le concepteur dans son choix. La méthodologie favorise
le temps machine afin de couvrir l’ensemble du domaine de définition tout
en libérant le temps humain et est définie suivant trois étapes principales~:
\begin{itemize}
    \item Réduire la complexité du problème et le temps nécessaire pour une évaluation.
    \item Explorer l’espace de décision grâce à processus automatisée afin d’obtenir
          un large choix de solutions optimales pour un temps humain très court.
    \item Accompagner le concepteur dans le choix de la solution la plus adaptée
          de manière simple et interactive.
\end{itemize}
Pour la première étape, la méthode de \textit{Morris} permet de sélectionner uniquement
les critères de décisions les plus influents grâce à un tri qualitatif. Suit la
création d’un modèle de substitution afin de réduire fortement le temps d’évaluation
nécessaire pour chaque évaluation. Cette approche s’est montrée particulièrement adaptée
pour le modèle développée dans ces travaux pour lequel le temps de simulation initial
était un facteur limitant.
Afin d’explorer le domaine de décision, une approche d’aide à la décision a posteriori
est retenu afin de ne introduire un caractère préférentiel. Celle-ci nécessite dans un premier temps la réalisation d’une optimisation
multi-objectifs pour obtenir un ensemble de solutions optimales. Après une
analyse de la littérature et toujours dans l’optique de proposer une approche simplifiée,
l’optimisation est réalisée par méta-heuristique basé sur le comportement des abeilles
mellifères principalement car il ne nécessite la détermination de peu de paramètres. Tenant compte
des résultats de la littérature, l’approche est améliorée afin d’offrir un bon équilibre entre
exploration et exploitation sans sacrifier pour autant la simplicité. Finalement, la dernière
étape consiste à assister le ou les décideurs dans le choix de la solution la plus adaptée.
Dans ces travaux, le choix a été fait d’utiliser une approche interactive simple d’utilisation
afin que chaque intervenant puisse participer et être acteur de leur décision.

La méthodologie développée est finalement illustrée à travers l’étude de deux cas
d’études, Bordeaux avec un climat propice et Strasbourg où le climat est plus rude durant
la période hivernale. L’évaluation est réalisé en faisant varier de manière concomitante
les paramètres caractéristiques de l’enveloppe, du \abr{SSC} et de sa logique de contrôle.
La définition de \abr{MEPOS} intègre ici les consommations internes (équipements
domestiques et éclairage) et les consommation propres à la couverture des besoins pour la
production d’\abr{ECS} et du chauffage. Ainsi, la toiture est partagée entre des capteurs
photovoltaïques et des capteurs solaires thermiques en tenant compte de la géométrie
triangulaire caractéristique d’une toiture quatre pans. L’approche s’est montrée
particulièrement pertinente à la fois pour évaluer les interactions existantes mais aussi
pour proposer un large choix de solutions optimales. Il est en effet trouvé des solutions
diverses permettant toutes d’obtenir un taux d’économie important. Grâce à une approche
itérative à partir des solutions du premier front de \textit{Pareto}, il est aussi identifié des
solutions similaires pour des volumes de ballons très faibles, ou encore pour une surface de
capteur réduite. Ainsi contrairement aux premiers résultats obtenus par une approche
paramétrique, l’exploration sans préférence a priori a permis d’identifier des solutions
encourageant pour le développement de \abr{MEPOS} solaires. En effet en plus de libérer une place conséquente, réduire la taille des
ballons permet de limiter les risques de surchauffes mais surtout de limiter le coût d’investissement.
Finalement, au regard des résultats, il semble que l’utilisation d’un \abr{SSC} soit plus pertinent
dans un climat rude où les besoins en chauffage restent important malgré une isolation
importante comme sur Strasbourg.


\paragraph{} % (fold)
Dans la continuité de ces travaux, il est identifié trois axes principaux~: la validation
expérimentale, une évaluation économique détaillée, et la caractérisation du potentiel
d’un \abr{SSC} en période hivernale comme estivale.

Comme explicité dès le second chapitre, les réponses et observations faites au cours de
cette thèse nécessitent d’être confrontées aux résultats d’une expérimentation à échelle
$1$. En effet même si les composants du modèle ont fait pour la plupart l’objet d’une
vérification analytique, le modèle dans son ensemble nécessitent d’être validés
expérimentalement afin de corroborer les conclusions de ces travaux. La méthodologie
développée par contre reste valable et elle a d’ailleurs montrée sa capacité à explorer le
domaine de définition pour proposer de nombreuses combinaisons pertinentes. Il serait
alors uniquement nécessaire de vérifier que le modèle traduit correctement le comportement
du réel du \abr{SSC}. En plus de la validation, il serait intéressant d’évaluer les
risques de surchauffes et de caractériser les méthodes existantes pour évacuer la chaleur.
soit en la valorisant pour une autre application comme le chauffage d’une piscine, soit
en déchargeant le système par exemple en faisant circuler l’eau du \abr{SSC} dans les
capteurs solaires en période nocturne.

Afin de compléter l’étude, il semble pertinent d’intégrer la prise en compte de manière
quantitative du critère économique à travers le coût d’investissement et le temps de
retour. Il est cependant nécessaire de rester très prudent quand à leur implémentation en
particulier pour le temps de retour sur investissement qui doit tenir compte de la
complexité inhérente à ce type d’analyse comme explicité à travers le chapitre précédent.
De plus, comme noté à travers le premier chapitre, les résultats obtenus sont fortement
dépendant des données météos utilisées. Le rayonnement solaire en particulier est très
changeant d’une année à l’autre, il serait alors pertinent d’évaluer la variation de la
performance du \abr{SSC} pour une période étendue de simulation. Cette analyse peut être
réalisée avec l’aide de données météorologiques existantes mais une étude prospective sur
l’évolution du climat futur est plus profitable. En effet afin d’être complète, l’analyse
du retour sur investissement doit tenir compte de l’évolution du coût des énergies mais
aussi des besoins du bâtiment en partie liée à l’évolution du climat. Il apparaît ainsi
judicieux de coupler la construction de fichiers météorologiques prospectifs à la
construction de scénarios économiques afin de pouvoir quantifier le potentiel énergétique
et économique du \abr{SSC} selon différents scénarios tout en tenant compte du
vieillissement des équipements. Afin de répondre à ces perspectives, il peut être envisagé
un couplage entre la méthodologie développée et une approche par clustering afin de
réduire chaque année à un jeu de semaines représentatives et donc limiter le temps de
simulation nécessaire. Ainsi les résultats issus de la comparaison entre les deux
approches pourrait ainsi amener à revoir les moyens à mettre en œuvre pour évaluer la
performance d’un \abr{SSC} de manière plus complète.

Ces travaux ont permis d’identifier des solutions techniques permettant d’obtenir un taux
d’économie important en réduisant la surface de capteurs mais surtout la taille des
ballons, levant ainsi les principales contraintes qui freinent l’adoption de tels
systèmes. Cependant avec l’amélioration de l’enveloppe des bâtiments et l’augmentation des
consommations dues aux charges internes, le confort estival est aussi un facteur clé. Le
solaire étant abondant durant l’été, un travail approfondie sur le dimensionnement d’un
\abr{SSC} à la fois pour couvrir les besoins hivernaux et estivaux apparaît comme une suite
logique à ces travaux et viendrait en complément à la tâche $53$ actuellement en cours.
Parmi les technologies potentielles, il serait intéressant d’investiguer le couplage entre
un système solaire et une \abr{PAC} à $CO_{2}$ plus respectueuse de l’environnement que
que les \abr{PAC} plus classiques qui nécessitent l’utilisation d’un fluide frigorigène
polluant. Ce couplage présente aussi un avantage énergétique, grâce à l’énergie solaire,
le coefficient de performance (\abr{COP} pour le chauffage ou \abr{EER} pour le
refroidissement) d’une \abr{PAC} peut être maintenue à un niveau important durant
l’ensemble de l’année. Ainsi le potentiel solaire est double et permet à la fois de
réduire la consommation en se substituant à l’énergie électrique mais aussi en améliorant
la performance de la \abr{PAC}~: le gain énergétique est donc potentiellement plus
important que avec une batterie électrique. Une autre alternative possible serait
l’utilisation d’une pompe à absorption. Le principe est similaire à celui d’une \abr{PAC}
mais le compresseur est remplacé par un bouilleur limitant ainsi la consommation
électrique, le bruit, et améliorant la robustesse du système. Il n’existe cependant pas
actuellement sur le marché de pompe à absorption de puissance assez faible pour convenir à
une application pour une \abr{MEPOS}. De plus, le bouilleur nécessite une source chaude à
une température plus élevée que une \abr{PAC} classique et des capteurs solaires
cylindro-paraboliques pourraient être plus adaptés. La méthodologie développée dans ces travaux
pourrait ainsi être étendue afin de caractériser le potentiel du couplage avec une
\abr{PAC} ou d’une pompe à absorption. Bien sur dans cet optique, il apparaît pertinent de
tenir compte du confort d’été en s’orientant vers une modélisation du bâtiment plus
détaillée.
