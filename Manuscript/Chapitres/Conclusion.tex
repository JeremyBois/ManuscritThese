%!TEX root = ../main.tex


\itodo{Vérifier définition maître oeuvre et ouvrage}
\itodo{Vérifier définition des phases de construction APD, ...}

Oui c’est la fin, le grand final !!

Ouverture:
Évaluer la pertinence des fichiers météos pour un dimensionnement solaire
Aide à la décision par des méthodes plus poussées
Méthode de **clustering** pour évaluer le comportement hivernal et estival sur une
longue période.

Conclusion:
Ces travaux représentent une avancée nécessaire sur la compréhension des $SSC$ pour
les bâtiment de demain. L’influence des différents paramètres, les limites et atouts
de ces systèmes a ainsi pu être détaillée à travers une étude paramétrique, un criblage
global et les résultats d’une optimisation multi-objectif.

Il est clair que la connaissance acquise doit être mise à profis afin de développer
les $SSC$ pour la maison individuelle. Plusieurs axes peuvent ainsi être proposées.
Dans l’optique d’une aide à la décision, il est possible d’évaluer la pertinence
des autres méthodes d’aide à la décision a posteriori même si elle me semble dans
l’optique d’une interaction entre maître d’ouvrage et d’oeuvre plus complexe à mettre
en place.

Experimentation~:
Fort de l’expérience acquise, il serait aussi très intéressant de s’orienter vers
l’expérimentation à échelle 1 afin dans un premier temps valider le modèle mais surtout
de mieux comprendre la dynamique du système en fonctionnement réelle.
Il est aussi évident au regard des résultats que deux ballons, un tampon comme un sanitaire
représente une perte d’espace et un risque potentiel d’inconfort estival. Ce qui m’amène à
un autre point important, le comportement d’un $SSC$ dans une maison performante, en période
estivale. Il serait alors intéressant de comprendre ou trouver des moyens permettant de
décharger l’énergie des capteurs ou fort l’influence des ballons sur le confort des occupants.


Numérique~:
Coût~:
Finalement au niveau numérique, une approche par clustering ou un modèle réduit pourrait
permettre d’évaluer le $SSC$ sur le temps en tenant compte du viellissement et du coût
de l’entretien. C’est en effet un autre point qui n’a pas été pris en compte dans ces travaux
car c’est un paramètre très complexe nécessitant je pense un travail doctoral à lui seul.
Alors bien sûr l’ajout d’une composante coût à l’optimisation a été discuté mais le
temps disponible n’est pas extensible et des choix ont été faits.
L’évaluation du coût doit en effet tenir compte de nombreux paramètres~: le coût initial,
les aides existantes, le retour sur investissement, la valeur verte du bâtiment, l’évolution
du coût des énergies, l’entretien et le remplacement des composant, le coût de la main d’oeurvre,
le temps d’occupation, ...
Afin d’être pertinent d’un point de vue technique et économique, il est donc nécessaire
de travailler avec un partenaire ayant accès à ces informations au tout du moins un
échantillon.

Impact environnemental~:
Il a aussi été discuté l’intégration d’une analyse de l’impact environnemental
d’un tel système mais faute de temps, n’a pas vu le jour.

Fichiers météos~:
Enfin l’analyse de la pertinence des fichiers météo est aussi une perspective intéressant.
Il a été vu sur un petit échantillon que l’ensoleillement issue des fichiers météos
sous-estime fortement la réalité. En effet ces fichiers ont été construit dans l’optique
de simuler la performance d’un bâtiment et il et peut être opportun de discuter d’une
autre méthode pour le dimensionnement des systèmes solaires.


Climatisation~:
Potentiel solaire en été
