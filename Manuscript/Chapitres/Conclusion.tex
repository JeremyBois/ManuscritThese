%!TEX root = ../main.tex

Cette thèse s’est intéressée aux \abr{MEPOS} solaires et en particulier au développement
d’une méthodologie  multi-objectifs permettant d’accompagner les concepteurs pour le
dimensionnement de \abr{SSC} couplés à des \abr{MEPOS}. Le choix de
considérer le bâtiment, le \abr{SSC} et son algorithme de contrôle comme un ensemble
cohérent et fonctionnant de concert a été fait. S’inspirant de la norme \abr{PREN\,$15603$} et
basée sur une aide à la décision a posteriori, la méthodologie s’est montrée pertinente
à la fois pour évaluer les interactions existantes mais aussi pour proposer un large choix
de solutions optimales. Parmi ces solutions, des compromis intéressants ont été
identifiés invitant à repenser les choix de conception lors du développement de \abr{SSC}
couplés à des \abr{MEPOS}.


\paragraph{} % (fold)
À travers ces travaux, il a été mis en avant que la construction d’un \abr{MEPOS} est un
problème multi-objectifs complexe faisant intervenir de nombreux critères. Il n’existe
cependant pas à l’heure actuelle une définition précise en France, même si un cadre
européen se construit à travers la norme \abr{PREN\,$15603$}. Cette norme invite à
la sobriété énergétique en favorisant l’utilisation des énergies renouvelables sur site
pour de l’auto-consommation. Le but étant de limiter la production électrique directement
injectée dans le réseau afin d’éviter des aberrations écologiques.
De part sa simplicité, le solaire thermique et par extension le \abr{SSC} est un
candidat approprié pour couvrir une part importante des besoins en adéquation avec la
demande. Pourtant cette technologie n’est que peu utilisée pour la constructions de
\abr{BEPOS}, et il n’existe pas de travaux cherchant à caractériser le potentiel d’un tel
couplage en tenant compte à la fois, de l’enveloppe, du système, et de l’algorithme de
contrôle. Partant de ce constat, un regard neuf est proposé dans ce document et invite à
repenser la conception des \abr{SSC} pour les \abr{MEPOS} avec une attention
particulière sur les points suivants~:
\begin{itemize}
    \item La conception est un processus complexe, la méthodologie se doit d’être
          relativement simple à mettre en place afin d’être applicable.
    \item La logique de contrôle doit faire partie du processus de décision.
    \item La conception d’une \abr{MEPOS} solaire doit être le résultat d’un compromis
          entre qualité de l’enveloppe et performance du \abr{SSC}.
    \item L’évaluation du \abr{SSC} doit tenir compte de l’adéquation entre production
          et demande suivant un bilan importation / exportation.
    \item La taille des équipements du \abr{SSC} doit être réduite afin d’être viable
          économiquement.
\end{itemize}

Ainsi fort de l’expérience acquise à travers l’analyse de l’état de l’art et des tendances
observées au niveau industriel, un bâtiment couplé à un \abr{SSC} a été dans un premier temps
modélisé et les principaux composants validés analytiquement. Le langage \textit{Modelica}
et le logiciel \textit{Dymola} ont été retenus car ils sont particulièrement adaptés au
processus de développement itératif nécessaire pour cette application. Pour le bâtiment,
un modèle mono-zone a été sélectionné après une comparaison avec une approche multi-zone
sous \textit{EnergyPlus}. Concernant le \abr{SSC}, le système retenu utilise l’air
comme vecteur de chaleur afin d’améliorer la réactivité du bâtiment et éviter
l’installation d’un système de chauffage conventionnel. De plus, une modélisation détaillée
est retenue afin de pouvoir contrôler l’ensemble des facteurs
influents que ce soit sur les caractéristiques du système ou sur les paramètres de la
logique de contrôle. Ce choix a permis d’affiner l’algorithme de contrôle afin de
favoriser la part solaire. Les
premiers résultats sont obtenus en suivant la méthodologie couramment utilisée dans les
bureaux d’études, à savoir une approche paramétrique où chaque évaluation est réalisée
à partir de l’expérience acquise en amont et où la performance de l’enveloppe est fixée. Bien que les
résultats aient permis d’identifier le potentiel du couplage d’une \abr{MEPOS} à un
\abr{SSC}, l’approche ne permet pas d’évaluer de manière complète le domaine de
définition et donc de caractériser la performance pouvant être attendue.
% Ainsi ces premiers résultats confirment en majeure partie les observations
% déjà documentées par la littérature comme la nécessité d’opter pour des volumes importants
% ou de favoriser en priorité la production d’\abr{ECS}.
Il est cependant mis en avant
un comportement caractéristique du système étudié~: le \abr{SSC} développé est capable de
répartir l’énergie solaire disponible en fonction des caractéristiques des équipements
retenus pour réduire la consommation de l’appoint, soit sur l’\abr{ECS}, soit sur le chauffage,
sans modifier la couverture solaire au niveau global.

Une fois le potentiel du \abr{SSC} identifié et les limites de l’approche paramétrique
mises en exergue, il est proposé une méthodologie plus adaptée permettant d’accompagner
le concepteur dans son choix à travers trois étapes~:
\begin{itemize}
    \item Réduire la complexité du problème et le temps nécessaire pour une évaluation.
    \item Explorer l’espace de décision grâce à processus automatisé afin d’obtenir
          un large choix de solutions optimales pour un temps humain court.
    \item Accompagner le concepteur dans le choix de la solution la plus adaptée
          de manière simple et interactive.
\end{itemize}
Au cours de la première étape, la méthode de \textit{Morris} permet de sélectionner uniquement
les critères de décisions les plus influents grâce à un tri qualitatif. Suit la création
d’un modèle de substitution par la méthode du chaos polynomiale afin de réduire fortement le temps d’évaluation nécessaire
pour chaque évaluation et permettre de créer un modèle faisant abstraction de la complexité du modèle original.
Cette approche s’est montrée particulièrement adaptée pour le
modèle développé dans ces travaux où le temps de simulation initial était un facteur
limitant. Afin d’explorer le domaine de décision, une approche d’aide à la décision a
posteriori est choisie afin de ne pas introduire de caractères préférentiels. Celle-ci
nécessite dans un premier temps la réalisation d’une optimisation multi-objectifs pour
obtenir un ensemble de solutions optimales. Après une analyse de la littérature et
toujours dans l’optique de proposer une approche simplifiée, un méta-heuristique basé sur
le comportement des abeilles mellifères (\abr{ABC}) est retenu pour l’optimisation principalement car
il nécessite la détermination de seulement trois paramètres. Puis en tenant compte des observations
de la littérature, l’approche est améliorée afin d’offrir un bon équilibre entre exploration
et exploitation sans sacrifier pour autant la simplicité d’utilisation. L’implémentation est
réalisée en \textit{Python}, disponible en open source sous le nom \textit{pyMayBee}.
Finalement, la dernière étape consiste à assister le ou
les décideurs dans le choix de la solution la plus adaptée. Dans ces travaux, le choix a
été fait d’utiliser une approche interactive simple afin que chaque intervenant puisse
participer et être acteur de leur décision.

La méthodologie développée est finalement illustrée à travers l’étude de deux cas
d’études, Bordeaux avec un climat propice, et Strasbourg où le climat est plus rude durant
la période hivernale. L’évaluation est réalisée en faisant varier de manière concomitante
les paramètres caractéristiques de l’enveloppe du \abr{SSC} et de sa logique de contrôle.
La définition de la \abr{MEPOS} intègre ici les consommations internes (équipements
domestiques et éclairage) et les consommation propres à la couverture des besoins pour la
production d’\abr{ECS} et du chauffage. Ainsi, la toiture est partagée entre des capteurs
photovoltaïques et des capteurs solaires thermiques en tenant compte de la géométrie
triangulaire caractéristique d’une toiture quatre pans afin de couvrir les besoins
électriques et thermiques.
L’approche s’est montrée particulièrement pertinente à la fois pour évaluer
les interactions existantes mais aussi pour proposer un large choix de solutions
optimales. Des solutions diverses permettant toutes d’obtenir un
bilan nul ou positif ont été trouvées, soit en favorisant une surface importante de capteurs \abr{PV}, soit
en favorisant la production du \abr{SSC}. Au regard des résultats, l’utilisation d’un
\abr{SSC} est plus pertinente dans un climat rude comme sur Strasbourg où les besoins en chauffage restent
élevés malgré une isolation importante. Pour des conditions
similaires à celles de Bordeaux, les besoins en chauffage sont en effet limités et même si
l’économie relative est plus importante, l’économie absolue reste plus faible. Aussi,
l’identification d’un compromis entre qualité de l’enveloppe et investissement sur le
\abr{SSC} confirme l’hypothèse initiée dans le premier chapitre~: la
construction d’une \abr{MEPOS} solaire doit être le fruit du dimensionnement couplé du
bâtiment avec le système solaire afin de tenir compte des interactions. Une fois les
résultats de la première optimisation analysés, l’introduction
de préférence est illustrée dans un premier temps à travers une optimisation itérative qui
consiste à relancer le processus d’optimisation en ajoutant une ou plusieurs contraintes
supplémentaires. Cette approche a permis d’identifier des solutions similaires pour des
volumes de ballons très faibles, ou encore pour une surface de capteur réduite. Ainsi
contrairement aux premiers résultats obtenus par une approche paramétrique, l’exploration
sans préférence a priori a permis d’identifier des solutions encourageante pour le
développement de \abr{MEPOS} solaires. En effet, en plus de libérer une place conséquente,
réduire la taille des ballons permet de limiter les risques de surchauffes mais surtout de
limiter le coût d’investissement. Finalement l’ajout du caractère préférentiel est illustré à travers
une représentation des solutions par coordonnées parallèles permettant de mettre en avant
la praticité de la méthodologie développée.


\paragraph{} % (fold)
Dans la continuité de ces travaux, trois axes principaux sont identifiés~: la validation
expérimentale, une évaluation économique détaillée, et la caractérisation du potentiel
d’un \abr{SSC} en période hivernale comme estivale.

Comme explicité dès le second chapitre, les réponses et observations faites au cours de
cette thèse nécessitent d’être confrontées aux résultats d’une expérimentation à échelle
$1$. En effet même si les composants du modèle ont fait, pour la plupart, l’objet d’une
vérification analytique, le modèle dans son ensemble nécessite d’être validé
expérimentalement afin de corroborer les conclusions de ces travaux. Il est alors
nécessaire de vérifier que le modèle traduit correctement le comportement du
\abr{SSC} dans des conditions réelles.
En plus de la validation, il serait aussi intéressant d’évaluer les risques de surchauffe
et de caractériser les méthodes existantes pour évacuer la chaleur, soit en la valorisant
pour une autre application comme pour le chauffage d’une piscine, soit en déchargeant
l’énergie excédentaire, par exemple, en faisant circuler l’eau du \abr{SSC} dans les
capteurs solaires en période nocturne.

Afin de compléter l’étude, il semble aussi pertinent d’intégrer la prise en compte de manière
quantitative du critère économique à travers le coût d’investissement et le temps de
retour. Il est cependant nécessaire de rester très prudent quand à leur implémentation en
particulier pour le temps de retour sur investissement qui doit tenir compte de la
complexité inhérente à ce type d’analyses comme explicité à travers le chapitre précédent.
De plus, comme noté à travers le premier chapitre, les résultats obtenus sont fortement
dépendants des données météos utilisées. Le rayonnement solaire en particulier est très
changeant d’une année à l’autre, il serait alors pertinent d’évaluer la variation de la
performance du \abr{SSC} pour une période étendue de simulation. Cette analyse peut être
réalisée avec l’aide de données météorologiques existantes mais une étude prospective sur
l’évolution du climat futur est plus profitable. En effet afin d’être complète, l’analyse
du retour sur investissement doit tenir compte de l’évolution du coût des énergies, mais
aussi, des besoins du bâtiment qui sont en partie liés à l’évolution du climat. Il apparaît ainsi
judicieux de coupler la construction de fichiers météorologiques prospectifs à la
construction de scénarios économiques. Il est alors possible de quantifier le potentiel énergétique
et économique du \abr{SSC} pour différents scénarios, tout en tenant compte du
vieillissement des équipements. Afin de répondre à ces perspectives, il peut être envisagé
un couplage entre la méthodologie développée dans ce manuscrit et une approche par clustering. Chaque
année serait alors représentée par un jeu de semaines typiques permettant de limiter le temps de
simulation nécessaire.
% Les résultats issus de la comparaison entre les deux
% approches pourraient amener à revoir les moyens à mettre en œuvre pour évaluer la
% performance d’un \abr{SSC} de manière plus complète.

Enfin, ces travaux ont permis d’identifier des solutions techniques permettant d’obtenir un taux
d’économie important en réduisant la surface de capteurs mais surtout la taille des
ballons, levant ainsi les principales contraintes qui freinent l’adoption de tels
systèmes. Cependant avec l’amélioration de l’enveloppe des bâtiments et l’augmentation des
consommations dues aux charges internes, le confort estival est aussi un facteur clé. Le
solaire étant abondant durant l’été, un travail approfondi sur le dimensionnement d’un
\abr{SSC} pour couvrir à la fois les besoins hivernaux et estivaux apparaît comme une suite
logique à ces travaux et viendrait en complément à la tâche $53$ actuellement en cours.
Parmi les technologies potentielles, il serait intéressant d’investiguer le couplage entre
un système solaire et une \abr{PAC} respectueuse de l’environnement en substituant
l’utilisation d’un fluide frigorigène par l’utilisation du $CO_{2}$. De plus, ce couplage présente aussi un
avantage énergétique. Grâce à l’énergie solaire,
le coefficient de performance (\abr{COP} pour le chauffage) d’une \abr{PAC} peut être maintenu à un niveau important durant
l’ensemble de l’année. Ainsi, le potentiel solaire est double et permet à la fois de
réduire la consommation en se substituant à l’énergie électrique mais peut aussi permettre d’améliorer
la performance de la \abr{PAC}~: le gain énergétique est donc potentiellement plus
important qu’avec une batterie électrique. Une autre alternative possible serait
l’utilisation d’une pompe à absorption. Le principe est similaire à celui d’une \abr{PAC}
mais le compresseur est remplacé par un bouilleur limitant ainsi la consommation
électrique, le bruit, et améliorant la robustesse du système. Il n’existe cependant pas
actuellement sur le marché de pompe à absorption de puissance assez faible pour convenir à
une application pour une \abr{MEPOS}. De plus, le bouilleur nécessite une source chaude à
une température plus élevée qu’une \abr{PAC} classique et des capteurs solaires
cylindro-paraboliques pourraient alors être plus adaptés. La méthodologie développée dans ces travaux
pourrait ainsi être étendue afin de caractériser le potentiel du couplage avec une
\abr{PAC} ou d’une pompe à absorption. Bien sûr dans cet optique, il apparaît pertinent
d’également tenir compte du confort d’été en s’orientant vers une modélisation du bâtiment plus
détaillée et donc vers un modèle multi-zone.
