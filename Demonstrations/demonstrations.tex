% \documentclass[11pt, draft, twoside]{report}
% \documentclass[12pt, twoside]{report}
\documentclass[12pt, oneside]{report}

% ------------------------------------------------------------------------------
% Packages import
% ------------------------------------------------------------------------------
% Text font pack
\usepackage[OT1, T1]{fontenc}              % .pdf Encoding spec
\usepackage[utf8]{inputenc}                % .tex Encoding spec
\usepackage[greek, french]{babel}          % language used
\usepackage{csquotes}                      % Ensure correct quote text
\usepackage[dvipsnames, svgnames]{xcolor}  % Color support
\frenchspacing                             % French typo
\frenchbsetup{StandardLists=true}          % Turn off french bulleted lists
\usepackage{./Template/graphics_support}   % Image support
\usepackage[scaled]{beramono}

% Add support for annotations with transparent color
\usepackage{pdfcomment}
\definecolor{transparent}{cmyk}{1, 1, 1, 1}

% Template packages
\usepackage[top=2cm,
            bottom=2cm,
            left=2cm,
            right=2cm,
            a4paper]{geometry}             % Page layout support
\usepackage{bookmark}                      % load hyperref internaly
\usepackage{appendix}                      % Appendix package to add sub-annex | annex to main table

\hypersetup{
    pdfkeywords={Passive,
                 Optimisation,
                 Solaire,
                 Système combiné},         % list of keywords
    unicode=true,                          % non-Latin characters in Acrobat’s bookmarks
    pdftoolbar=true,                       % show Acrobat’s toolbar?
    pdfmenubar=true,                       % show Acrobat’s menu?
    pdffitwindow=false,                    % window fit to page when opened
    pdfstartview={FitH},                   % fits the width of the page to the window
    pdftitle={Outil d’aide à la décision
              pour la construction de
              maison passives
              100\% solaires},             % title
    pdfauthor={Jérémy Bois},               % author
    pdfsubject={Optimisation,
                Maison passive},           % subject of the document
    pdfcreator={Jérémy Bois},              % creator of the document
    pdfproducer={Jérémy Bois},             % producer of the document
    pdfnewwindow=true,                     % links in new PDF window
    colorlinks=true,                       % false: boxed links; true: colored links
    linkcolor=black!30!orange!70,          % color of internal links (change box color with linkbordercolor)
    citecolor=magenta,                     % color of links to bibliography
    filecolor=magenta,                     % color of file links
    urlcolor=cyan                          % color of external links
}


% Scientist support
\usepackage{siunitx}                                % Unit package
\usepackage{amsmath, amssymb}                       % Allow adding complex equations
\usepackage{nicefrac}                               % for \nicefrac macro
% \usepackage{listings}                             % Code block support
\usepackage[algoruled,
            linesnumbered,
            algosection]{algorithm2e}               % Algorithm support
\usepackage{letltxmacro}                            % Support for optional argument in redefinition
% Change the ways square root appears
\makeatletter
    \let\oldr@@t\r@@t
    \def\r@@t#1#2{%
    \setbox0=\hbox{$\oldr@@t#1{#2\,}$}\dimen0=\ht0
    \advance\dimen0-0.2\ht0
    \setbox2=\hbox{\vrule height\ht0 depth -\dimen0}%
    {\box0\lower0.4pt\box2}}
    \LetLtxMacro{\oldsqrt}{\sqrt}
    \renewcommand*{\sqrt}[2][\ ]{\oldsqrt[#1]{#2}}
\makeatother

% Bibliography support
\usepackage[style=authoryear,
            sorting=none,
            backend=biber]{biblatex}                % Bibliography support
\addbibresource{../Bibliographie/references.bib}    % Where to find bibliography


\usepackage{mathtools}
\DeclarePairedDelimiter\ceil{\lceil}{\rceil}
\DeclarePairedDelimiter\floor{\lfloor}{\rfloor}

\begin{document}

\chapter*{Expression analytique de la distribution de Lévy stable et symétrique}
La distribution est le plus souvent exprimée sous la forme d’une transformée de Fourier:
\begin{equation}\label{eq:fourier_levy}
    \mathcal{F}(k) = \exp\left(-\alpha|k|^{\beta}\right), \qquad  0 < \beta \leq 2
\end{equation}
Il existe des cas spéciaux où la transformée inverse de Fourier correspond à une distribution
normale ($\beta = 2$), ou une distribution de Gauchy ($\beta = 1$).

La forme analytique de la \emph{distribution de Lévy stable et symétrique} (symmetrical Lévy stable distribution with index $\beta$) peut
s’exprimer sous la forme suivante (\cite{Gutowski2001}):
\begin{equation}\label{eq:dist_levy}
    L(s) = \frac{1}{\pi} \int_{0}^{\infty} \cos(k s)\exp\left(-\alpha k^{\beta}\right) dk, \qquad  0 < \beta \leq 2, \quad \alpha > 0
\end{equation}

Démonstration du passage à l’expression analytique avec $\mathcal{F}(k) = \hat{L}(k)$:
\begin{equation}
    \begin{split}
        L(s) &= \frac{1}{2\pi}\int_{-\infty}^{+\infty}\hat{L}(k)e^{+iks} ~dk\\
             &= \frac{1}{2\pi}\int_{-\infty}^{+\infty}\hat{L}(k) [cos(ks) + isin(ks)] ~dk\\
             &= \frac{1}{2\pi}\int_{-\infty}^{+\infty}\exp\left(-\alpha k^{\beta}\right)  [cos(ks) + isin(ks)] ~dk\\
             &= \frac{1}{2\pi}\int_{-\infty}^{+\infty}\exp\left(-\alpha k^{\beta}\right)  cos(ks) ~dk,
                \qquad \pdfmarkupcomment[mathstyle=\displaystyle, color=transparent] {\text{(Partie complexe = 0)}}%
                {On conserve alors uniquement la partie entière nous permettant de faire la simplification ci-après}\\
             &= \frac{1}{\pi}\int_{0}^{+\infty} \exp(-\alpha k^{\beta})  cos(ks) ~dk
    \end{split}
\end{equation}


\chapter*{Estimation de la variance pour une marche aléatoire}
Ce chapitre montre que les notations de \cite{Yang201445} et \cite{Gutowski2001}
sont toutes les deux valables. L’un utilisant la notion de \emph{variance} et
l’autre de distance moyenne quadratique, \emph{RMS}.\\
Soit $\mathbf{X}$ suivant une loi de distribution pour construire une marche aléatoire:
\begin{equation}
    \begin{split}
        Var(\mathbf{X}) &= \frac{1}{n^{2}} \sum_{i=1}^{n} \sum_{j=1}^{n} \frac{1}{2}(x_{i} - x_{j})^{2}\\
        Var(\mathbf{X}) &= \frac{1}{n^{2}} \sum_{i=1}^{n} \sum_{j>i}^{n} (x_{i} - x_{j})^{2}\\
        Var(\mathbf{X}) &= \frac{1}{n^{2}} \sum_{i=1}^{n} (x_{i} - x_{i+1})^{2}\\
    \end{split}
\end{equation}
Posons:  $x_{i} = x_{i} - x_{i+1}$:
\begin{equation}
    \begin{split}
        \langle RMS\rangle = \bar{x}_{2} &= \sqrt{\frac{1}{n}} \times \sqrt{\sum_{i=1}^{n}x_{i}^{2}}\\
                          &= \sqrt{\frac{1}{n}} \times \sqrt{\sum_{i=1}^{n}(x_{i} - x_{i+1})^{2}}\\
    \end{split}
\end{equation}
On a alors:
\begin{align*}
    & \langle RMS^{2}\rangle = \frac{1}{n} \sum_{i=1}^{n}(x_{i} - x_{i+1})^{2} \\\\
    & Var(\mathbf{X}) \sim \langle RMS^{2}\rangle\\
\end{align*}


\chapter*{Évaluation du nombre de rectangle dans un triangle quelconque}
Considérons un rectangle de dimensions $l \times h$ et un triangle ABC.
Combien de rectangle (sans rotation) peut contenir le triangle ABC (Fig.~\ref{fig:rects_in_triangle}).

\begin{figure}
    \begin{center}
        \includegraphics{Demonstrations/rects_in_triangle.png}
    \end{center}
    \caption{Représentation schématique du problème : Combien de rectangle
             (sans rotation) peut contenir le triangle ABC
             \label{fig:rects_in_triangle}}
\end{figure}

D’après la \textbf{loi des cosinus} (Théorême de Al Kashi) il est possible d’écrire:
\begin{equation}
        \alpha = \arccos \left[\frac{AC^{2} + BC^{2} - AB^{2}}{2 \times AC \times BC}\right]
    \label{eq:loi_cosinus}
\end{equation}
On peut alors définir la hauteur du triangle ABC en utilisant la trigonométrie et
\autoref{eq:loi_cosinus}:
\begin{equation}
    \begin{split}
        H &= \sin (\arccos[\alpha]) \times BC\\
        \\
          &= \sin \left(\arccos \left[\frac{AC^{2} + BC^{2} - AB^{2}}{2 \times AC \times BC}\right]\right) \times BC
    \end{split}
    \label{fig:height_triangle}
\end{equation}

En utilisant le \textbf{théorême de Thalès} dans ABC et A’BC on peut définir la longueur DE
en fonction de $H$.
\begin{equation}
    \begin{split}
        DE &= \frac{BE \times AC}{BC} \\
        \\
           &= \frac{(H - h) \times AC}{H}
    \end{split}
\end{equation}

Le nombre total de rectangle sans chevauchement peut alors être définie par \autoref{eq:nb_rects_in_triangle}:
\begin{equation}
    N_{rect} = \sum_{n = 1}^{\floor*{\nicefrac{H}{h}}} \floor*{\frac{(H - n \times h) \times AC}{l \times H}}
    \label{eq:nb_rects_in_triangle}
\end{equation}


\chapter*{Autres notes}

L’équation généralisé de la moyenne peux être défini de la manière suivante:
\begin{equation}\label{eq:moyenne_generale}
    \bar{x}_{m} = \left[ \frac{1}{n} \sum_{i=1}^{n}(x_{i}^{m})\right]^{\nicefrac{1}{m}}\\
\end{equation}
\begin{align*}\label{eq:parametre_m}
    & m \to +\infty & \text{maximum de } x_{i}\\
    & m \to -\infty & \text{minimum de } x_{i}\\
    & m = 2         & \text{moyenne quadratique (RMS)}\\
    & m = 1         & \text{moyenne arithmétique}\\
    & m = -1        & \text{moyenne harmonique}\\
    & m \to 0       & \text{moyenne géométrique}\\
\end{align*}

La variance pour un cas discret peut être écrite de la façon suivante:
\begin{equation}
    V(\mathbf{X}) = \frac{1}{n} \sum_{i=1}^{n} (x_{i} - \mu)^{2}, \qquad \mu \text{ étant la moyenne}
\end{equation}

La déviation est alors définie par:
\begin{equation}
    \sigma = \sqrt{V(\mathbf{X})}
\end{equation}


Remarque sur la fonction $\mathbf{\Gamma}$:
\begin{equation}
    \mathbf{\Gamma}(1+\beta) = \beta \times \mathbf{\Gamma}(\beta)
\end{equation}


% Add bibliography
\printbibliography

\end{document}
